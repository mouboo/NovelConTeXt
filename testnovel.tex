\starttext
WALDEN




and



ON THE DUTY OF CIVIL DISOBEDIENCE



by Henry David Thoreau


cover


Contents


 WALDEN

 Economy
 Where I Lived, and What I Lived For
 Reading
 Sounds
 Solitude
 Visitors
 The Bean-Field
 The Village
 The Ponds
 Baker Farm
 Higher Laws
 Brute Neighbors
 House-Warming
 Former Inhabitants and Winter Visitors
 Winter Animals
 The Pond in Winter
 Spring
 Conclusion

 ON THE DUTY OF CIVIL DISOBEDIENCE



WALDEN

Economy

When I wrote the following pages, or rather the bulk of them, I lived
alone, in the woods, a mile from any neighbor, in a house which I had
built myself, on the shore of Walden Pond, in Concord, Massachusetts,
and earned my living by the labor of my hands only. I lived there two
years and two months. At present I am a sojourner in civilized life
again.

I should not obtrude my affairs so much on the notice of my readers if
very particular inquiries had not been made by my townsmen concerning
my mode of life, which some would call impertinent, though they do not
appear to me at all impertinent, but, considering the circumstances,
very natural and pertinent. Some have asked what I got to eat; if I did
not feel lonesome; if I was not afraid; and the like. Others have been
curious to learn what portion of my income I devoted to charitable
purposes; and some, who have large families, how many poor children I
maintained. I will therefore ask those of my readers who feel no
particular interest in me to pardon me if I undertake to answer some of
these questions in this book. In most books, the _I_, or first person,
is omitted; in this it will be retained; that, in respect to egotism,
is the main difference. We commonly do not remember that it is, after
all, always the first person that is speaking. I should not talk so
much about myself if there were anybody else whom I knew as well.
Unfortunately, I am confined to this theme by the narrowness of my
experience. Moreover, I, on my side, require of every writer, first or
last, a simple and sincere account of his own life, and not merely what
he has heard of other men’s lives; some such account as he would send
to his kindred from a distant land; for if he has lived sincerely, it
must have been in a distant land to me. Perhaps these pages are more
particularly addressed to poor students. As for the rest of my readers,
they will accept such portions as apply to them. I trust that none will
stretch the seams in putting on the coat, for it may do good service to
him whom it fits.

I would fain say something, not so much concerning the Chinese and
Sandwich Islanders as you who read these pages, who are said to live in
New England; something about your condition, especially your outward
condition or circumstances in this world, in this town, what it is,
whether it is necessary that it be as bad as it is, whether it cannot
be improved as well as not. I have travelled a good deal in Concord;
and everywhere, in shops, and offices, and fields, the inhabitants have
appeared to me to be doing penance in a thousand remarkable ways. What
I have heard of Brahmins sitting exposed to four fires and looking in
the face of the sun; or hanging suspended, with their heads downward,
over flames; or looking at the heavens over their shoulders “until it
becomes impossible for them to resume their natural position, while
from the twist of the neck nothing but liquids can pass into the
stomach;” or dwelling, chained for life, at the foot of a tree; or
measuring with their bodies, like caterpillars, the breadth of vast
empires; or standing on one leg on the tops of pillars,—even these
forms of conscious penance are hardly more incredible and astonishing
than the scenes which I daily witness. The twelve labors of Hercules
were trifling in comparison with those which my neighbors have
undertaken; for they were only twelve, and had an end; but I could
never see that these men slew or captured any monster or finished any
labor. They have no friend Iolas to burn with a hot iron the root of
the hydra’s head, but as soon as one head is crushed, two spring up.

I see young men, my townsmen, whose misfortune it is to have inherited
farms, houses, barns, cattle, and farming tools; for these are more
easily acquired than got rid of. Better if they had been born in the
open pasture and suckled by a wolf, that they might have seen with
clearer eyes what field they were called to labor in. Who made them
serfs of the soil? Why should they eat their sixty acres, when man is
condemned to eat only his peck of dirt? Why should they begin digging
their graves as soon as they are born? They have got to live a man’s
life, pushing all these things before them, and get on as well as they
can. How many a poor immortal soul have I met well nigh crushed and
smothered under its load, creeping down the road of life, pushing
before it a barn seventy-five feet by forty, its Augean stables never
cleansed, and one hundred acres of land, tillage, mowing, pasture, and
wood-lot! The portionless, who struggle with no such unnecessary
inherited encumbrances, find it labor enough to subdue and cultivate a
few cubic feet of flesh.

But men labor under a mistake. The better part of the man is soon
plowed into the soil for compost. By a seeming fate, commonly called
necessity, they are employed, as it says in an old book, laying up
treasures which moth and rust will corrupt and thieves break through
and steal. It is a fool’s life, as they will find when they get to the
end of it, if not before. It is said that Deucalion and Pyrrha created
men by throwing stones over their heads behind them:—

     Inde genus durum sumus, experiensque laborum,
     Et documenta damus quâ simus origine nati.

Or, as Raleigh rhymes it in his sonorous way,—

     “From thence our kind hard-hearted is, enduring pain and care,
     Approving that our bodies of a stony nature are.”

So much for a blind obedience to a blundering oracle, throwing the
stones over their heads behind them, and not seeing where they fell.

Most men, even in this comparatively free country, through mere
ignorance and mistake, are so occupied with the factitious cares and
superfluously coarse labors of life that its finer fruits cannot be
plucked by them. Their fingers, from excessive toil, are too clumsy and
tremble too much for that. Actually, the laboring man has not leisure
for a true integrity day by day; he cannot afford to sustain the
manliest relations to men; his labor would be depreciated in the
market. He has no time to be anything but a machine. How can he
remember well his ignorance—which his growth requires—who has so often
to use his knowledge? We should feed and clothe him gratuitously
sometimes, and recruit him with our cordials, before we judge of him.
The finest qualities of our nature, like the bloom on fruits, can be
preserved only by the most delicate handling. Yet we do not treat
ourselves nor one another thus tenderly.

Some of you, we all know, are poor, find it hard to live, are
sometimes, as it were, gasping for breath. I have no doubt that some of
you who read this book are unable to pay for all the dinners which you
have actually eaten, or for the coats and shoes which are fast wearing
or are already worn out, and have come to this page to spend borrowed
or stolen time, robbing your creditors of an hour. It is very evident
what mean and sneaking lives many of you live, for my sight has been
whetted by experience; always on the limits, trying to get into
business and trying to get out of debt, a very ancient slough, called
by the Latins _æs alienum_, another’s brass, for some of their coins
were made of brass; still living, and dying, and buried by this other’s
brass; always promising to pay, promising to pay, tomorrow, and dying
today, insolvent; seeking to curry favor, to get custom, by how many
modes, only not state-prison offences; lying, flattering, voting,
contracting yourselves into a nutshell of civility or dilating into an
atmosphere of thin and vaporous generosity, that you may persuade your
neighbor to let you make his shoes, or his hat, or his coat, or his
carriage, or import his groceries for him; making yourselves sick, that
you may lay up something against a sick day, something to be tucked
away in an old chest, or in a stocking behind the plastering, or, more
safely, in the brick bank; no matter where, no matter how much or how
little.

I sometimes wonder that we can be so frivolous, I may almost say, as to
attend to the gross but somewhat foreign form of servitude called Negro
Slavery, there are so many keen and subtle masters that enslave both
north and south. It is hard to have a southern overseer; it is worse to
have a northern one; but worst of all when you are the slave-driver of
yourself. Talk of a divinity in man! Look at the teamster on the
highway, wending to market by day or night; does any divinity stir
within him? His highest duty to fodder and water his horses! What is
his destiny to him compared with the shipping interests? Does not he
drive for Squire Make-a-stir? How godlike, how immortal, is he? See how
he cowers and sneaks, how vaguely all the day he fears, not being
immortal nor divine, but the slave and prisoner of his own opinion of
himself, a fame won by his own deeds. Public opinion is a weak tyrant
compared with our own private opinion. What a man thinks of himself,
that it is which determines, or rather indicates, his fate.
Self-emancipation even in the West Indian provinces of the fancy and
imagination,—what Wilberforce is there to bring that about? Think,
also, of the ladies of the land weaving toilet cushions against the
last day, not to betray too green an interest in their fates! As if you
could kill time without injuring eternity.

The mass of men lead lives of quiet desperation. What is called
resignation is confirmed desperation. From the desperate city you go
into the desperate country, and have to console yourself with the
bravery of minks and muskrats. A stereotyped but unconscious despair is
concealed even under what are called the games and amusements of
mankind. There is no play in them, for this comes after work. But it is
a characteristic of wisdom not to do desperate things.

When we consider what, to use the words of the catechism, is the chief
end of man, and what are the true necessaries and means of life, it
appears as if men had deliberately chosen the common mode of living
because they preferred it to any other. Yet they honestly think there
is no choice left. But alert and healthy natures remember that the sun
rose clear. It is never too late to give up our prejudices. No way of
thinking or doing, however ancient, can be trusted without proof. What
everybody echoes or in silence passes by as true to-day may turn out to
be falsehood to-morrow, mere smoke of opinion, which some had trusted
for a cloud that would sprinkle fertilizing rain on their fields. What
old people say you cannot do you try and find that you can. Old deeds
for old people, and new deeds for new. Old people did not know enough
once, perchance, to fetch fresh fuel to keep the fire a-going; new
people put a little dry wood under a pot, and are whirled round the
globe with the speed of birds, in a way to kill old people, as the
phrase is. Age is no better, hardly so well, qualified for an
instructor as youth, for it has not profited so much as it has lost.
One may almost doubt if the wisest man has learned any thing of
absolute value by living. Practically, the old have no very important
advice to give the young, their own experience has been so partial, and
their lives have been such miserable failures, for private reasons, as
they must believe; and it may be that they have some faith left which
belies that experience, and they are only less young than they were. I
have lived some thirty years on this planet, and I have yet to hear the
first syllable of valuable or even earnest advice from my seniors. They
have told me nothing, and probably cannot tell me any thing to the
purpose. Here is life, an experiment to a great extent untried by me;
but it does not avail me that they have tried it. If I have any
experience which I think valuable, I am sure to reflect that this my
Mentors said nothing about.

One farmer says to me, “You cannot live on vegetable food solely, for
it furnishes nothing to make bones with;” and so he religiously devotes
a part of his day to supplying his system with the raw material of
bones; walking all the while he talks behind his oxen, which, with
vegetable-made bones, jerk him and his lumbering plough along in spite
of every obstacle. Some things are really necessaries of life in some
circles, the most helpless and diseased, which in others are luxuries
merely, and in others still are entirely unknown.

The whole ground of human life seems to some to have been gone over by
their predecessors, both the heights and the valleys, and all things to
have been cared for. According to Evelyn, “the wise Solomon prescribed
ordinances for the very distances of trees; and the Roman prætors have
decided how often you may go into your neighbor’s land to gather the
acorns which fall on it without trespass, and what share belongs to
that neighbor.” Hippocrates has even left directions how we should cut
our nails; that is, even with the ends of the fingers, neither shorter
nor longer. Undoubtedly the very tedium and ennui which presume to have
exhausted the variety and the joys of life are as old as Adam. But
man’s capacities have never been measured; nor are we to judge of what
he can do by any precedents, so little has been tried. Whatever have
been thy failures hitherto, “be not afflicted, my child, for who shall
assign to thee what thou hast left undone?”

We might try our lives by a thousand simple tests; as, for instance,
that the same sun which ripens my beans illumines at once a system of
earths like ours. If I had remembered this it would have prevented some
mistakes. This was not the light in which I hoed them. The stars are
the apexes of what wonderful triangles! What distant and different
beings in the various mansions of the universe are contemplating the
same one at the same moment! Nature and human life are as various as
our several constitutions. Who shall say what prospect life offers to
another? Could a greater miracle take place than for us to look through
each other’s eyes for an instant? We should live in all the ages of the
world in an hour; ay, in all the worlds of the ages. History, Poetry,
Mythology!—I know of no reading of another’s experience so startling
and informing as this would be.

The greater part of what my neighbors call good I believe in my soul to
be bad, and if I repent of anything, it is very likely to be my good
behavior. What demon possessed me that I behaved so well? You may say
the wisest thing you can, old man,—you who have lived seventy years,
not without honor of a kind,—I hear an irresistible voice which invites
me away from all that. One generation abandons the enterprises of
another like stranded vessels.

I think that we may safely trust a good deal more than we do. We may
waive just so much care of ourselves as we honestly bestow elsewhere.
Nature is as well adapted to our weakness as to our strength. The
incessant anxiety and strain of some is a well nigh incurable form of
disease. We are made to exaggerate the importance of what work we do;
and yet how much is not done by us! or, what if we had been taken sick?
How vigilant we are! determined not to live by faith if we can avoid
it; all the day long on the alert, at night we unwillingly say our
prayers and commit ourselves to uncertainties. So thoroughly and
sincerely are we compelled to live, reverencing our life, and denying
the possibility of change. This is the only way, we say; but there are
as many ways as there can be drawn radii from one centre. All change is
a miracle to contemplate; but it is a miracle which is taking place
every instant. Confucius said, “To know that we know what we know, and
that we do not know what we do not know, that is true knowledge.” When
one man has reduced a fact of the imagination to be a fact to his
understanding, I foresee that all men at length establish their lives
on that basis.



Let us consider for a moment what most of the trouble and anxiety which
I have referred to is about, and how much it is necessary that we be
troubled, or, at least, careful. It would be some advantage to live a
primitive and frontier life, though in the midst of an outward
civilization, if only to learn what are the gross necessaries of life
and what methods have been taken to obtain them; or even to look over
the old day-books of the merchants, to see what it was that men most
commonly bought at the stores, what they stored, that is, what are the
grossest groceries. For the improvements of ages have had but little
influence on the essential laws of man’s existence; as our skeletons,
probably, are not to be distinguished from those of our ancestors.

By the words, _necessary of life_, I mean whatever, of all that man
obtains by his own exertions, has been from the first, or from long use
has become, so important to human life that few, if any, whether from
savageness, or poverty, or philosophy, ever attempt to do without it.
To many creatures there is in this sense but one necessary of life,
Food. To the bison of the prairie it is a few inches of palatable
grass, with water to drink; unless he seeks the Shelter of the forest
or the mountain’s shadow. None of the brute creation requires more than
Food and Shelter. The necessaries of life for man in this climate may,
accurately enough, be distributed under the several heads of Food,
Shelter, Clothing, and Fuel; for not till we have secured these are we
prepared to entertain the true problems of life with freedom and a
prospect of success. Man has invented, not only houses, but clothes and
cooked food; and possibly from the accidental discovery of the warmth
of fire, and the consequent use of it, at first a luxury, arose the
present necessity to sit by it. We observe cats and dogs acquiring the
same second nature. By proper Shelter and Clothing we legitimately
retain our own internal heat; but with an excess of these, or of Fuel,
that is, with an external heat greater than our own internal, may not
cookery properly be said to begin? Darwin, the naturalist, says of the
inhabitants of Tierra del Fuego, that while his own party, who were
well clothed and sitting close to a fire, were far from too warm, these
naked savages, who were farther off, were observed, to his great
surprise, “to be streaming with perspiration at undergoing such a
roasting.” So, we are told, the New Hollander goes naked with impunity,
while the European shivers in his clothes. Is it impossible to combine
the hardiness of these savages with the intellectualness of the
civilized man? According to Liebig, man’s body is a stove, and food the
fuel which keeps up the internal combustion in the lungs. In cold
weather we eat more, in warm less. The animal heat is the result of a
slow combustion, and disease and death take place when this is too
rapid; or for want of fuel, or from some defect in the draught, the
fire goes out. Of course the vital heat is not to be confounded with
fire; but so much for analogy. It appears, therefore, from the above
list, that the expression, _animal life_, is nearly synonymous with the
expression, _animal heat_; for while Food may be regarded as the Fuel
which keeps up the fire within us,—and Fuel serves only to prepare that
Food or to increase the warmth of our bodies by addition from
without,—Shelter and Clothing also serve only to retain the _heat_ thus
generated and absorbed.

The grand necessity, then, for our bodies, is to keep warm, to keep the
vital heat in us. What pains we accordingly take, not only with our
Food, and Clothing, and Shelter, but with our beds, which are our
night-clothes, robbing the nests and breasts of birds to prepare this
shelter within a shelter, as the mole has its bed of grass and leaves
at the end of its burrow! The poor man is wont to complain that this is
a cold world; and to cold, no less physical than social, we refer
directly a great part of our ails. The summer, in some climates, makes
possible to man a sort of Elysian life. Fuel, except to cook his Food,
is then unnecessary; the sun is his fire, and many of the fruits are
sufficiently cooked by its rays; while Food generally is more various,
and more easily obtained, and Clothing and Shelter are wholly or half
unnecessary. At the present day, and in this country, as I find by my
own experience, a few implements, a knife, an axe, a spade, a
wheelbarrow, &c., and for the studious, lamplight, stationery, and
access to a few books, rank next to necessaries, and can all be
obtained at a trifling cost. Yet some, not wise, go to the other side
of the globe, to barbarous and unhealthy regions, and devote themselves
to trade for ten or twenty years, in order that they may live,—that is,
keep comfortably warm,—and die in New England at last. The luxuriously
rich are not simply kept comfortably warm, but unnaturally hot; as I
implied before, they are cooked, of course _à la mode_.

Most of the luxuries, and many of the so called comforts of life, are
not only not indispensable, but positive hindrances to the elevation of
mankind. With respect to luxuries and comforts, the wisest have ever
lived a more simple and meagre life than the poor. The ancient
philosophers, Chinese, Hindoo, Persian, and Greek, were a class than
which none has been poorer in outward riches, none so rich in inward.
We know not much about them. It is remarkable that _we_ know so much of
them as we do. The same is true of the more modern reformers and
benefactors of their race. None can be an impartial or wise observer of
human life but from the vantage ground of what we should call voluntary
poverty. Of a life of luxury the fruit is luxury, whether in
agriculture, or commerce, or literature, or art. There are nowadays
professors of philosophy, but not philosophers. Yet it is admirable to
profess because it was once admirable to live. To be a philosopher is
not merely to have subtle thoughts, nor even to found a school, but so
to love wisdom as to live according to its dictates, a life of
simplicity, independence, magnanimity, and trust. It is to solve some
of the problems of life, not only theoretically, but practically. The
success of great scholars and thinkers is commonly a courtier-like
success, not kingly, not manly. They make shift to live merely by
conformity, practically as their fathers did, and are in no sense the
progenitors of a nobler race of men. But why do men degenerate ever?
What makes families run out? What is the nature of the luxury which
enervates and destroys nations? Are we sure that there is none of it in
our own lives? The philosopher is in advance of his age even in the
outward form of his life. He is not fed, sheltered, clothed, warmed,
like his contemporaries. How can a man be a philosopher and not
maintain his vital heat by better methods than other men?

When a man is warmed by the several modes which I have described, what
does he want next? Surely not more warmth of the same kind, as more and
richer food, larger and more splendid houses, finer and more abundant
clothing, more numerous incessant and hotter fires, and the like. When
he has obtained those things which are necessary to life, there is
another alternative than to obtain the superfluities; and that is, to
adventure on life now, his vacation from humbler toil having commenced.
The soil, it appears, is suited to the seed, for it has sent its
radicle downward, and it may now send its shoot upward also with
confidence. Why has man rooted himself thus firmly in the earth, but
that he may rise in the same proportion into the heavens above?—for the
nobler plants are valued for the fruit they bear at last in the air and
light, far from the ground, and are not treated like the humbler
esculents, which, though they may be biennials, are cultivated only
till they have perfected their root, and often cut down at top for this
purpose, so that most would not know them in their flowering season.

I do not mean to prescribe rules to strong and valiant natures, who
will mind their own affairs whether in heaven or hell, and perchance
build more magnificently and spend more lavishly than the richest,
without ever impoverishing themselves, not knowing how they live,—if,
indeed, there are any such, as has been dreamed; nor to those who find
their encouragement and inspiration in precisely the present condition
of things, and cherish it with the fondness and enthusiasm of
lovers,—and, to some extent, I reckon myself in this number; I do not
speak to those who are well employed, in whatever circumstances, and
they know whether they are well employed or not;—but mainly to the mass
of men who are discontented, and idly complaining of the hardness of
their lot or of the times, when they might improve them. There are some
who complain most energetically and inconsolably of any, because they
are, as they say, doing their duty. I also have in my mind that
seemingly wealthy, but most terribly impoverished class of all, who
have accumulated dross, but know not how to use it, or get rid of it,
and thus have forged their own golden or silver fetters.



If I should attempt to tell how I have desired to spend my life in
years past, it would probably surprise those of my readers who are
somewhat acquainted with its actual history; it would certainly
astonish those who know nothing about it. I will only hint at some of
the enterprises which I have cherished.

In any weather, at any hour of the day or night, I have been anxious to
improve the nick of time, and notch it on my stick too; to stand on the
meeting of two eternities, the past and future, which is precisely the
present moment; to toe that line. You will pardon some obscurities, for
there are more secrets in my trade than in most men’s, and yet not
voluntarily kept, but inseparable from its very nature. I would gladly
tell all that I know about it, and never paint “No Admittance” on my
gate.

I long ago lost a hound, a bay horse, and a turtle-dove, and am still
on their trail. Many are the travellers I have spoken concerning them,
describing their tracks and what calls they answered to. I have met one
or two who had heard the hound, and the tramp of the horse, and even
seen the dove disappear behind a cloud, and they seemed as anxious to
recover them as if they had lost them themselves.

To anticipate, not the sunrise and the dawn merely, but, if possible,
Nature herself! How many mornings, summer and winter, before yet any
neighbor was stirring about his business, have I been about mine! No
doubt, many of my townsmen have met me returning from this enterprise,
farmers starting for Boston in the twilight, or woodchoppers going to
their work. It is true, I never assisted the sun materially in his
rising, but, doubt not, it was of the last importance only to be
present at it.

So many autumn, ay, and winter days, spent outside the town, trying to
hear what was in the wind, to hear and carry it express! I well-nigh
sunk all my capital in it, and lost my own breath into the bargain,
running in the face of it. If it had concerned either of the political
parties, depend upon it, it would have appeared in the Gazette with the
earliest intelligence. At other times watching from the observatory of
some cliff or tree, to telegraph any new arrival; or waiting at evening
on the hill-tops for the sky to fall, that I might catch something,
though I never caught much, and that, manna-wise, would dissolve again
in the sun.

For a long time I was reporter to a journal, of no very wide
circulation, whose editor has never yet seen fit to print the bulk of
my contributions, and, as is too common with writers, I got only my
labor for my pains. However, in this case my pains were their own
reward.

For many years I was self-appointed inspector of snow storms and rain
storms, and did my duty faithfully; surveyor, if not of highways, then
of forest paths and all across-lot routes, keeping them open, and
ravines bridged and passable at all seasons, where the public heel had
testified to their utility.

I have looked after the wild stock of the town, which give a faithful
herdsman a good deal of trouble by leaping fences; and I have had an
eye to the unfrequented nooks and corners of the farm; though I did not
always know whether Jonas or Solomon worked in a particular field
to-day; that was none of my business. I have watered the red
huckleberry, the sand cherry and the nettle tree, the red pine and the
black ash, the white grape and the yellow violet, which might have
withered else in dry seasons.

In short, I went on thus for a long time, I may say it without
boasting, faithfully minding my business, till it became more and more
evident that my townsmen would not after all admit me into the list of
town officers, nor make my place a sinecure with a moderate allowance.
My accounts, which I can swear to have kept faithfully, I have, indeed,
never got audited, still less accepted, still less paid and settled.
However, I have not set my heart on that.

Not long since, a strolling Indian went to sell baskets at the house of
a well-known lawyer in my neighborhood. “Do you wish to buy any
baskets?” he asked. “No, we do not want any,” was the reply. “What!”
exclaimed the Indian as he went out the gate, “do you mean to starve
us?” Having seen his industrious white neighbors so well off,—that the
lawyer had only to weave arguments, and by some magic, wealth and
standing followed, he had said to himself; I will go into business; I
will weave baskets; it is a thing which I can do. Thinking that when he
had made the baskets he would have done his part, and then it would be
the white man’s to buy them. He had not discovered that it was
necessary for him to make it worth the other’s while to buy them, or at
least make him think that it was so, or to make something else which it
would be worth his while to buy. I too had woven a kind of basket of a
delicate texture, but I had not made it worth any one’s while to buy
them. Yet not the less, in my case, did I think it worth my while to
weave them, and instead of studying how to make it worth men’s while to
buy my baskets, I studied rather how to avoid the necessity of selling
them. The life which men praise and regard as successful is but one
kind. Why should we exaggerate any one kind at the expense of the
others?

Finding that my fellow-citizens were not likely to offer me any room in
the court house, or any curacy or living any where else, but I must
shift for myself, I turned my face more exclusively than ever to the
woods, where I was better known. I determined to go into business at
once, and not wait to acquire the usual capital, using such slender
means as I had already got. My purpose in going to Walden Pond was not
to live cheaply nor to live dearly there, but to transact some private
business with the fewest obstacles; to be hindered from accomplishing
which for want of a little common sense, a little enterprise and
business talent, appeared not so sad as foolish.

I have always endeavored to acquire strict business habits; they are
indispensable to every man. If your trade is with the Celestial Empire,
then some small counting house on the coast, in some Salem harbor, will
be fixture enough. You will export such articles as the country
affords, purely native products, much ice and pine timber and a little
granite, always in native bottoms. These will be good ventures. To
oversee all the details yourself in person; to be at once pilot and
captain, and owner and underwriter; to buy and sell and keep the
accounts; to read every letter received, and write or read every letter
sent; to superintend the discharge of imports night and day; to be upon
many parts of the coast almost at the same time;—often the richest
freight will be discharged upon a Jersey shore;—to be your own
telegraph, unweariedly sweeping the horizon, speaking all passing
vessels bound coastwise; to keep up a steady despatch of commodities,
for the supply of such a distant and exorbitant market; to keep
yourself informed of the state of the markets, prospects of war and
peace every where, and anticipate the tendencies of trade and
civilization,—taking advantage of the results of all exploring
expeditions, using new passages and all improvements in
navigation;—charts to be studied, the position of reefs and new lights
and buoys to be ascertained, and ever, and ever, the logarithmic tables
to be corrected, for by the error of some calculator the vessel often
splits upon a rock that should have reached a friendly pier,—there is
the untold fate of La Perouse;—universal science to be kept pace with,
studying the lives of all great discoverers and navigators, great
adventurers and merchants, from Hanno and the Phœnicians down to our
day; in fine, account of stock to be taken from time to time, to know
how you stand. It is a labor to task the faculties of a man,—such
problems of profit and loss, of interest, of tare and tret, and gauging
of all kinds in it, as demand a universal knowledge.

I have thought that Walden Pond would be a good place for business, not
solely on account of the railroad and the ice trade; it offers
advantages which it may not be good policy to divulge; it is a good
port and a good foundation. No Neva marshes to be filled; though you
must every where build on piles of your own driving. It is said that a
flood-tide, with a westerly wind, and ice in the Neva, would sweep St.
Petersburg from the face of the earth.



As this business was to be entered into without the usual capital, it
may not be easy to conjecture where those means, that will still be
indispensable to every such undertaking, were to be obtained. As for
Clothing, to come at once to the practical part of the question,
perhaps we are led oftener by the love of novelty, and a regard for the
opinions of men, in procuring it, than by a true utility. Let him who
has work to do recollect that the object of clothing is, first, to
retain the vital heat, and secondly, in this state of society, to cover
nakedness, and he may judge how much of any necessary or important work
may be accomplished without adding to his wardrobe. Kings and queens
who wear a suit but once, though made by some tailor or dressmaker to
their majesties, cannot know the comfort of wearing a suit that fits.
They are no better than wooden horses to hang the clean clothes on.
Every day our garments become more assimilated to ourselves, receiving
the impress of the wearer’s character, until we hesitate to lay them
aside, without such delay and medical appliances and some such
solemnity even as our bodies. No man ever stood the lower in my
estimation for having a patch in his clothes; yet I am sure that there
is greater anxiety, commonly, to have fashionable, or at least clean
and unpatched clothes, than to have a sound conscience. But even if the
rent is not mended, perhaps the worst vice betrayed is improvidence. I
sometimes try my acquaintances by such tests as this;—who could wear a
patch, or two extra seams only, over the knee? Most behave as if they
believed that their prospects for life would be ruined if they should
do it. It would be easier for them to hobble to town with a broken leg
than with a broken pantaloon. Often if an accident happens to a
gentleman’s legs, they can be mended; but if a similar accident happens
to the legs of his pantaloons, there is no help for it; for he
considers, not what is truly respectable, but what is respected. We
know but few men, a great many coats and breeches. Dress a scarecrow in
your last shift, you standing shiftless by, who would not soonest
salute the scarecrow? Passing a cornfield the other day, close by a hat
and coat on a stake, I recognized the owner of the farm. He was only a
little more weather-beaten than when I saw him last. I have heard of a
dog that barked at every stranger who approached his master’s premises
with clothes on, but was easily quieted by a naked thief. It is an
interesting question how far men would retain their relative rank if
they were divested of their clothes. Could you, in such a case, tell
surely of any company of civilized men, which belonged to the most
respected class? When Madam Pfeiffer, in her adventurous travels round
the world, from east to west, had got so near home as Asiatic Russia,
she says that she felt the necessity of wearing other than a travelling
dress, when she went to meet the authorities, for she “was now in a
civilized country, where —— — people are judged of by their clothes.”
Even in our democratic New England towns the accidental possession of
wealth, and its manifestation in dress and equipage alone, obtain for
the possessor almost universal respect. But they yield such respect,
numerous as they are, are so far heathen, and need to have a missionary
sent to them. Beside, clothes introduced sewing, a kind of work which
you may call endless; a woman’s dress, at least, is never done.

A man who has at length found something to do will not need to get a
new suit to do it in; for him the old will do, that has lain dusty in
the garret for an indeterminate period. Old shoes will serve a hero
longer than they have served his valet,—if a hero ever has a
valet,—bare feet are older than shoes, and he can make them do. Only
they who go to soirées and legislative halls must have new coats, coats
to change as often as the man changes in them. But if my jacket and
trousers, my hat and shoes, are fit to worship God in, they will do;
will they not? Who ever saw his old clothes,—his old coat, actually
worn out, resolved into its primitive elements, so that it was not a
deed of charity to bestow it on some poor boy, by him perchance to be
bestowed on some poorer still, or shall we say richer, who could do
with less? I say, beware of all enterprises that require new clothes,
and not rather a new wearer of clothes. If there is not a new man, how
can the new clothes be made to fit? If you have any enterprise before
you, try it in your old clothes. All men want, not something to _do
with_, but something to _do_, or rather something to _be_. Perhaps we
should never procure a new suit, however ragged or dirty the old, until
we have so conducted, so enterprised or sailed in some way, that we
feel like new men in the old, and that to retain it would be like
keeping new wine in old bottles. Our moulting season, like that of the
fowls, must be a crisis in our lives. The loon retires to solitary
ponds to spend it. Thus also the snake casts its slough, and the
caterpillar its wormy coat, by an internal industry and expansion; for
clothes are but our outmost cuticle and mortal coil. Otherwise we shall
be found sailing under false colors, and be inevitably cashiered at
last by our own opinion, as well as that of mankind.

We don garment after garment, as if we grew like exogenous plants by
addition without. Our outside and often thin and fanciful clothes are
our epidermis, or false skin, which partakes not of our life, and may
be stripped off here and there without fatal injury; our thicker
garments, constantly worn, are our cellular integument, or cortex; but
our shirts are our liber or true bark, which cannot be removed without
girdling and so destroying the man. I believe that all races at some
seasons wear something equivalent to the shirt. It is desirable that a
man be clad so simply that he can lay his hands on himself in the dark,
and that he live in all respects so compactly and preparedly, that, if
an enemy take the town, he can, like the old philosopher, walk out the
gate empty-handed without anxiety. While one thick garment is, for most
purposes, as good as three thin ones, and cheap clothing can be
obtained at prices really to suit customers; while a thick coat can be
bought for five dollars, which will last as many years, thick
pantaloons for two dollars, cowhide boots for a dollar and a half a
pair, a summer hat for a quarter of a dollar, and a winter cap for
sixty-two and a half cents, or a better be made at home at a nominal
cost, where is he so poor that, clad in such a suit, of _his own
earning_, there will not be found wise men to do him reverence?

When I ask for a garment of a particular form, my tailoress tells me
gravely, “They do not make them so now,” not emphasizing the “They” at
all, as if she quoted an authority as impersonal as the Fates, and I
find it difficult to get made what I want, simply because she cannot
believe that I mean what I say, that I am so rash. When I hear this
oracular sentence, I am for a moment absorbed in thought, emphasizing
to myself each word separately that I may come at the meaning of it,
that I may find out by what degree of consanguinity _They_ are related
to _me_, and what authority they may have in an affair which affects me
so nearly; and, finally, I am inclined to answer her with equal
mystery, and without any more emphasis of the “they,”—“It is true, they
did not make them so recently, but they do now.” Of what use this
measuring of me if she does not measure my character, but only the
breadth of my shoulders, as it were a peg to hang the coat on? We
worship not the Graces, nor the Parcæ, but Fashion. She spins and
weaves and cuts with full authority. The head monkey at Paris puts on a
traveller’s cap, and all the monkeys in America do the same. I
sometimes despair of getting anything quite simple and honest done in
this world by the help of men. They would have to be passed through a
powerful press first, to squeeze their old notions out of them, so that
they would not soon get upon their legs again, and then there would be
some one in the company with a maggot in his head, hatched from an egg
deposited there nobody knows when, for not even fire kills these
things, and you would have lost your labor. Nevertheless, we will not
forget that some Egyptian wheat was handed down to us by a mummy.

On the whole, I think that it cannot be maintained that dressing has in
this or any country risen to the dignity of an art. At present men make
shift to wear what they can get. Like shipwrecked sailors, they put on
what they can find on the beach, and at a little distance, whether of
space or time, laugh at each other’s masquerade. Every generation
laughs at the old fashions, but follows religiously the new. We are
amused at beholding the costume of Henry VIII., or Queen Elizabeth, as
much as if it was that of the King and Queen of the Cannibal Islands.
All costume off a man is pitiful or grotesque. It is only the serious
eye peering from and the sincere life passed within it, which restrain
laughter and consecrate the costume of any people. Let Harlequin be
taken with a fit of the colic and his trappings will have to serve that
mood too. When the soldier is hit by a cannon ball rags are as becoming
as purple.

The childish and savage taste of men and women for new patterns keeps
how many shaking and squinting through kaleidoscopes that they may
discover the particular figure which this generation requires today.
The manufacturers have learned that this taste is merely whimsical. Of
two patterns which differ only by a few threads more or less of a
particular color, the one will be sold readily, the other lie on the
shelf, though it frequently happens that after the lapse of a season
the latter becomes the most fashionable. Comparatively, tattooing is
not the hideous custom which it is called. It is not barbarous merely
because the printing is skin-deep and unalterable.

I cannot believe that our factory system is the best mode by which men
may get clothing. The condition of the operatives is becoming every day
more like that of the English; and it cannot be wondered at, since, as
far as I have heard or observed, the principal object is, not that
mankind may be well and honestly clad, but, unquestionably, that
corporations may be enriched. In the long run men hit only what they
aim at. Therefore, though they should fail immediately, they had better
aim at something high.

\stoptext

As for a Shelter, I will not deny that this is now a necessary of life,
though there are instances of men having done without it for long
periods in colder countries than this. Samuel Laing says that “the
Laplander in his skin dress, and in a skin bag which he puts over his
head and shoulders, will sleep night after night on the snow—in a
degree of cold which would extinguish the life of one exposed to it in
any woollen clothing.” He had seen them asleep thus. Yet he adds, “They
are not hardier than other people.” But, probably, man did not live
long on the earth without discovering the convenience which there is in
a house, the domestic comforts, which phrase may have originally
signified the satisfactions of the house more than of the family;
though these must be extremely partial and occasional in those climates
where the house is associated in our thoughts with winter or the rainy
season chiefly, and two thirds of the year, except for a parasol, is
unnecessary. In our climate, in the summer, it was formerly almost
solely a covering at night. In the Indian gazettes a wigwam was the
symbol of a day’s march, and a row of them cut or painted on the bark
of a tree signified that so many times they had camped. Man was not
made so large limbed and robust but that he must seek to narrow his
world, and wall in a space such as fitted him. He was at first bare and
out of doors; but though this was pleasant enough in serene and warm
weather, by daylight, the rainy season and the winter, to say nothing
of the torrid sun, would perhaps have nipped his race in the bud if he
had not made haste to clothe himself with the shelter of a house. Adam
and Eve, according to the fable, wore the bower before other clothes.
Man wanted a home, a place of warmth, or comfort, first of physical
warmth, then the warmth of the affections.

We may imagine a time when, in the infancy of the human race, some
enterprising mortal crept into a hollow in a rock for shelter. Every
child begins the world again, to some extent, and loves to stay out
doors, even in wet and cold. It plays house, as well as horse, having
an instinct for it. Who does not remember the interest with which when
young he looked at shelving rocks, or any approach to a cave? It was
the natural yearning of that portion of our most primitive ancestor
which still survived in us. From the cave we have advanced to roofs of
palm leaves, of bark and boughs, of linen woven and stretched, of grass
and straw, of boards and shingles, of stones and tiles. At last, we
know not what it is to live in the open air, and our lives are domestic
in more senses than we think. From the hearth to the field is a great
distance. It would be well perhaps if we were to spend more of our days
and nights without any obstruction between us and the celestial bodies,
if the poet did not speak so much from under a roof, or the saint dwell
there so long. Birds do not sing in caves, nor do doves cherish their
innocence in dovecots.

However, if one designs to construct a dwelling house, it behooves him
to exercise a little Yankee shrewdness, lest after all he find himself
in a workhouse, a labyrinth without a clue, a museum, an almshouse, a
prison, or a splendid mausoleum instead. Consider first how slight a
shelter is absolutely necessary. I have seen Penobscot Indians, in this
town, living in tents of thin cotton cloth, while the snow was nearly a
foot deep around them, and I thought that they would be glad to have it
deeper to keep out the wind. Formerly, when how to get my living
honestly, with freedom left for my proper pursuits, was a question
which vexed me even more than it does now, for unfortunately I am
become somewhat callous, I used to see a large box by the railroad, six
feet long by three wide, in which the laborers locked up their tools at
night, and it suggested to me that every man who was hard pushed might
get such a one for a dollar, and, having bored a few auger holes in it,
to admit the air at least, get into it when it rained and at night, and
hook down the lid, and so have freedom in his love, and in his soul be
free. This did not appear the worst, nor by any means a despicable
alternative. You could sit up as late as you pleased, and, whenever you
got up, go abroad without any landlord or house-lord dogging you for
rent. Many a man is harassed to death to pay the rent of a larger and
more luxurious box who would not have frozen to death in such a box as
this. I am far from jesting. Economy is a subject which admits of being
treated with levity, but it cannot so be disposed of. A comfortable
house for a rude and hardy race, that lived mostly out of doors, was
once made here almost entirely of such materials as Nature furnished
ready to their hands. Gookin, who was superintendent of the Indians
subject to the Massachusetts Colony, writing in 1674, says, “The best
of their houses are covered very neatly, tight and warm, with barks of
trees, slipped from their bodies at those seasons when the sap is up,
and made into great flakes, with pressure of weighty timber, when they
are green.... The meaner sort are covered with mats which they make of
a kind of bulrush, and are also indifferently tight and warm, but not
so good as the former.... Some I have seen, sixty or a hundred feet
long and thirty feet broad.... I have often lodged in their wigwams,
and found them as warm as the best English houses.” He adds, that they
were commonly carpeted and lined within with well-wrought embroidered
mats, and were furnished with various utensils. The Indians had
advanced so far as to regulate the effect of the wind by a mat
suspended over the hole in the roof and moved by a string. Such a lodge
was in the first instance constructed in a day or two at most, and
taken down and put up in a few hours; and every family owned one, or
its apartment in one.

In the savage state every family owns a shelter as good as the best,
and sufficient for its coarser and simpler wants; but I think that I
speak within bounds when I say that, though the birds of the air have
their nests, and the foxes their holes, and the savages their wigwams,
in modern civilized society not more than one half the families own a
shelter. In the large towns and cities, where civilization especially
prevails, the number of those who own a shelter is a very small
fraction of the whole. The rest pay an annual tax for this outside
garment of all, become indispensable summer and winter, which would buy
a village of Indian wigwams, but now helps to keep them poor as long as
they live. I do not mean to insist here on the disadvantage of hiring
compared with owning, but it is evident that the savage owns his
shelter because it costs so little, while the civilized man hires his
commonly because he cannot afford to own it; nor can he, in the long
run, any better afford to hire. But, answers one, by merely paying this
tax the poor civilized man secures an abode which is a palace compared
with the savage’s. An annual rent of from twenty-five to a hundred
dollars, these are the country rates, entitles him to the benefit of
the improvements of centuries, spacious apartments, clean paint and
paper, Rumford fireplace, back plastering, Venetian blinds, copper
pump, spring lock, a commodious cellar, and many other things. But how
happens it that he who is said to enjoy these things is so commonly a
_poor_ civilized man, while the savage, who has them not, is rich as a
savage? If it is asserted that civilization is a real advance in the
condition of man,—and I think that it is, though only the wise improve
their advantages,—it must be shown that it has produced better
dwellings without making them more costly; and the cost of a thing is
the amount of what I will call life which is required to be exchanged
for it, immediately or in the long run. An average house in this
neighborhood costs perhaps eight hundred dollars, and to lay up this
sum will take from ten to fifteen years of the laborer’s life, even if
he is not encumbered with a family;—estimating the pecuniary value of
every man’s labor at one dollar a day, for if some receive more, others
receive less;—so that he must have spent more than half his life
commonly before _his_ wigwam will be earned. If we suppose him to pay a
rent instead, this is but a doubtful choice of evils. Would the savage
have been wise to exchange his wigwam for a palace on these terms?

It may be guessed that I reduce almost the whole advantage of holding
this superfluous property as a fund in store against the future, so far
as the individual is concerned, mainly to the defraying of funeral
expenses. But perhaps a man is not required to bury himself.
Nevertheless this points to an important distinction between the
civilized man and the savage; and, no doubt, they have designs on us
for our benefit, in making the life of a civilized people an
_institution_, in which the life of the individual is to a great extent
absorbed, in order to preserve and perfect that of the race. But I wish
to show at what a sacrifice this advantage is at present obtained, and
to suggest that we may possibly so live as to secure all the advantage
without suffering any of the disadvantage. What mean ye by saying that
the poor ye have always with you, or that the fathers have eaten sour
grapes, and the children’s teeth are set on edge?

“As I live, saith the Lord God, ye shall not have occasion any more to
use this proverb in Israel.”

“Behold all souls are mine; as the soul of the father, so also the soul
of the son is mine: the soul that sinneth, it shall die.”

When I consider my neighbors, the farmers of Concord, who are at least
as well off as the other classes, I find that for the most part they
have been toiling twenty, thirty, or forty years, that they may become
the real owners of their farms, which commonly they have inherited with
encumbrances, or else bought with hired money,—and we may regard one
third of that toil as the cost of their houses,—but commonly they have
not paid for them yet. It is true, the encumbrances sometimes outweigh
the value of the farm, so that the farm itself becomes one great
encumbrance, and still a man is found to inherit it, being well
acquainted with it, as he says. On applying to the assessors, I am
surprised to learn that they cannot at once name a dozen in the town
who own their farms free and clear. If you would know the history of
these homesteads, inquire at the bank where they are mortgaged. The man
who has actually paid for his farm with labor on it is so rare that
every neighbor can point to him. I doubt if there are three such men in
Concord. What has been said of the merchants, that a very large
majority, even ninety-seven in a hundred, are sure to fail, is equally
true of the farmers. With regard to the merchants, however, one of them
says pertinently that a great part of their failures are not genuine
pecuniary failures, but merely failures to fulfil their engagements,
because it is inconvenient; that is, it is the moral character that
breaks down. But this puts an infinitely worse face on the matter, and
suggests, beside, that probably not even the other three succeed in
saving their souls, but are perchance bankrupt in a worse sense than
they who fail honestly. Bankruptcy and repudiation are the springboards
from which much of our civilization vaults and turns its somersets, but
the savage stands on the unelastic plank of famine. Yet the Middlesex
Cattle Show goes off here with _éclat_ annually, as if all the joints
of the agricultural machine were suent.

The farmer is endeavoring to solve the problem of a livelihood by a
formula more complicated than the problem itself. To get his
shoestrings he speculates in herds of cattle. With consummate skill he
has set his trap with a hair spring to catch comfort and independence,
and then, as he turned away, got his own leg into it. This is the
reason he is poor; and for a similar reason we are all poor in respect
to a thousand savage comforts, though surrounded by luxuries. As
Chapman sings,—

     “The false society of men—
             —for earthly greatness
     All heavenly comforts rarefies to air.”

And when the farmer has got his house, he may not be the richer but the
poorer for it, and it be the house that has got him. As I understand
it, that was a valid objection urged by Momus against the house which
Minerva made, that she “had not made it movable, by which means a bad
neighborhood might be avoided;” and it may still be urged, for our
houses are such unwieldy property that we are often imprisoned rather
than housed in them; and the bad neighborhood to be avoided is our own
scurvy selves. I know one or two families, at least, in this town, who,
for nearly a generation, have been wishing to sell their houses in the
outskirts and move into the village, but have not been able to
accomplish it, and only death will set them free.

Granted that the _majority_ are able at last either to own or hire the
modern house with all its improvements. While civilization has been
improving our houses, it has not equally improved the men who are to
inhabit them. It has created palaces, but it was not so easy to create
noblemen and kings. And _if the civilized man’s pursuits are no
worthier than the savage’s, if he is employed the greater part of his
life in obtaining gross necessaries and comforts merely, why should he
have a better dwelling than the former?_

But how do the poor minority fare? Perhaps it will be found, that just
in proportion as some have been placed in outward circumstances above
the savage, others have been degraded below him. The luxury of one
class is counterbalanced by the indigence of another. On the one side
is the palace, on the other are the almshouse and “silent poor.” The
myriads who built the pyramids to be the tombs of the Pharaohs were fed
on garlic, and it may be were not decently buried themselves. The mason
who finishes the cornice of the palace returns at night perchance to a
hut not so good as a wigwam. It is a mistake to suppose that, in a
country where the usual evidences of civilization exist, the condition
of a very large body of the inhabitants may not be as degraded as that
of savages. I refer to the degraded poor, not now to the degraded rich.
To know this I should not need to look farther than to the shanties
which every where border our railroads, that last improvement in
civilization; where I see in my daily walks human beings living in
sties, and all winter with an open door, for the sake of light, without
any visible, often imaginable, wood pile, and the forms of both old and
young are permanently contracted by the long habit of shrinking from
cold and misery, and the development of all their limbs and faculties
is checked. It certainly is fair to look at that class by whose labor
the works which distinguish this generation are accomplished. Such too,
to a greater or less extent, is the condition of the operatives of
every denomination in England, which is the great workhouse of the
world. Or I could refer you to Ireland, which is marked as one of the
white or enlightened spots on the map. Contrast the physical condition
of the Irish with that of the North American Indian, or the South Sea
Islander, or any other savage race before it was degraded by contact
with the civilized man. Yet I have no doubt that that people’s rulers
are as wise as the average of civilized rulers. Their condition only
proves what squalidness may consist with civilization. I hardly need
refer now to the laborers in our Southern States who produce the staple
exports of this country, and are themselves a staple production of the
South. But to confine myself to those who are said to be in _moderate_
circumstances.

Most men appear never to have considered what a house is, and are
actually though needlessly poor all their lives because they think that
they must have such a one as their neighbors have. As if one were to
wear any sort of coat which the tailor might cut out for him, or,
gradually leaving off palmleaf hat or cap of woodchuck skin, complain
of hard times because he could not afford to buy him a crown! It is
possible to invent a house still more convenient and luxurious than we
have, which yet all would admit that man could not afford to pay for.
Shall we always study to obtain more of these things, and not sometimes
to be content with less? Shall the respectable citizen thus gravely
teach, by precept and example, the necessity of the young man’s
providing a certain number of superfluous glow-shoes, and umbrellas,
and empty guest chambers for empty guests, before he dies? Why should
not our furniture be as simple as the Arab’s or the Indian’s? When I
think of the benefactors of the race, whom we have apotheosized as
messengers from heaven, bearers of divine gifts to man, I do not see in
my mind any retinue at their heels, any car-load of fashionable
furniture. Or what if I were to allow—would it not be a singular
allowance?—that our furniture should be more complex than the Arab’s,
in proportion as we are morally and intellectually his superiors! At
present our houses are cluttered and defiled with it, and a good
housewife would sweep out the greater part into the dust hole, and not
leave her morning’s work undone. Morning work! By the blushes of Aurora
and the music of Memnon, what should be man’s _morning work_ in this
world? I had three pieces of limestone on my desk, but I was terrified
to find that they required to be dusted daily, when the furniture of my
mind was all undusted still, and I threw them out the window in
disgust. How, then, could I have a furnished house? I would rather sit
in the open air, for no dust gathers on the grass, unless where man has
broken ground.

It is the luxurious and dissipated who set the fashions which the herd
so diligently follow. The traveller who stops at the best houses, so
called, soon discovers this, for the publicans presume him to be a
Sardanapalus, and if he resigned himself to their tender mercies he
would soon be completely emasculated. I think that in the railroad car
we are inclined to spend more on luxury than on safety and convenience,
and it threatens without attaining these to become no better than a
modern drawing room, with its divans, and ottomans, and sun-shades, and
a hundred other oriental things, which we are taking west with us,
invented for the ladies of the harem and the effeminate natives of the
Celestial Empire, which Jonathan should be ashamed to know the names
of. I would rather sit on a pumpkin and have it all to myself than be
crowded on a velvet cushion. I would rather ride on earth in an ox cart
with a free circulation, than go to heaven in the fancy car of an
excursion train and breathe a _malaria_ all the way.

The very simplicity and nakedness of man’s life in the primitive ages
imply this advantage at least, that they left him still but a sojourner
in nature. When he was refreshed with food and sleep he contemplated
his journey again. He dwelt, as it were, in a tent in this world, and
was either threading the valleys, or crossing the plains, or climbing
the mountain tops. But lo! men have become the tools of their tools.
The man who independently plucked the fruits when he was hungry is
become a farmer; and he who stood under a tree for shelter, a
housekeeper. We now no longer camp as for a night, but have settled
down on earth and forgotten heaven. We have adopted Christianity merely
as an improved method of _agri_-culture. We have built for this world a
family mansion, and for the next a family tomb. The best works of art
are the expression of man’s struggle to free himself from this
condition, but the effect of our art is merely to make this low state
comfortable and that higher state to be forgotten. There is actually no
place in this village for a work of _fine_ art, if any had come down to
us, to stand, for our lives, our houses and streets, furnish no proper
pedestal for it. There is not a nail to hang a picture on, nor a shelf
to receive the bust of a hero or a saint. When I consider how our
houses are built and paid for, or not paid for, and their internal
economy managed and sustained, I wonder that the floor does not give
way under the visitor while he is admiring the gewgaws upon the
mantel-piece, and let him through into the cellar, to some solid and
honest though earthy foundation. I cannot but perceive that this so
called rich and refined life is a thing jumped at, and I do not get on
in the enjoyment of the _fine_ arts which adorn it, my attention being
wholly occupied with the jump; for I remember that the greatest genuine
leap, due to human muscles alone, on record, is that of certain
wandering Arabs, who are said to have cleared twenty-five feet on level
ground. Without factitious support, man is sure to come to earth again
beyond that distance. The first question which I am tempted to put to
the proprietor of such great impropriety is, Who bolsters you? Are you
one of the ninety-seven who fail, or of the three who succeed? Answer
me these questions, and then perhaps I may look at your bawbles and
find them ornamental. The cart before the horse is neither beautiful
nor useful. Before we can adorn our houses with beautiful objects the
walls must be stripped, and our lives must be stripped, and beautiful
housekeeping and beautiful living be laid for a foundation: now, a
taste for the beautiful is most cultivated out of doors, where there is
no house and no housekeeper.

Old Johnson, in his “Wonder-Working Providence,” speaking of the first
settlers of this town, with whom he was contemporary, tells us that
“they burrow themselves in the earth for their first shelter under some
hillside, and, casting the soil aloft upon timber, they make a smoky
fire against the earth, at the highest side.” They did not “provide
them houses,” says he, “till the earth, by the Lord’s blessing, brought
forth bread to feed them,” and the first year’s crop was so light that
“they were forced to cut their bread very thin for a long season.” The
secretary of the Province of New Netherland, writing in Dutch, in 1650,
for the information of those who wished to take up land there, states
more particularly that “those in New Netherland, and especially in New
England, who have no means to build farmhouses at first according to
their wishes, dig a square pit in the ground, cellar fashion, six or
seven feet deep, as long and as broad as they think proper, case the
earth inside with wood all round the wall, and line the wood with the
bark of trees or something else to prevent the caving in of the earth;
floor this cellar with plank, and wainscot it overhead for a ceiling,
raise a roof of spars clear up, and cover the spars with bark or green
sods, so that they can live dry and warm in these houses with their
entire families for two, three, and four years, it being understood
that partitions are run through those cellars which are adapted to the
size of the family. The wealthy and principal men in New England, in
the beginning of the colonies, commenced their first dwelling houses in
this fashion for two reasons; firstly, in order not to waste time in
building, and not to want food the next season; secondly, in order not
to discourage poor laboring people whom they brought over in numbers
from Fatherland. In the course of three or four years, when the country
became adapted to agriculture, they built themselves handsome houses,
spending on them several thousands.”

In this course which our ancestors took there was a show of prudence at
least, as if their principle were to satisfy the more pressing wants
first. But are the more pressing wants satisfied now? When I think of
acquiring for myself one of our luxurious dwellings, I am deterred,
for, so to speak, the country is not yet adapted to _human_ culture,
and we are still forced to cut our _spiritual_ bread far thinner than
our forefathers did their wheaten. Not that all architectural ornament
is to be neglected even in the rudest periods; but let our houses first
be lined with beauty, where they come in contact with our lives, like
the tenement of the shellfish, and not overlaid with it. But, alas! I
have been inside one or two of them, and know what they are lined with.

Though we are not so degenerate but that we might possibly live in a
cave or a wigwam or wear skins today, it certainly is better to accept
the advantages, though so dearly bought, which the invention and
industry of mankind offer. In such a neighborhood as this, boards and
shingles, lime and bricks, are cheaper and more easily obtained than
suitable caves, or whole logs, or bark in sufficient quantities, or
even well-tempered clay or flat stones. I speak understandingly on this
subject, for I have made myself acquainted with it both theoretically
and practically. With a little more wit we might use these materials so
as to become richer than the richest now are, and make our civilization
a blessing. The civilized man is a more experienced and wiser savage.
But to make haste to my own experiment.



Near the end of March, 1845, I borrowed an axe and went down to the
woods by Walden Pond, nearest to where I intended to build my house,
and began to cut down some tall, arrowy white pines, still in their
youth, for timber. It is difficult to begin without borrowing, but
perhaps it is the most generous course thus to permit your fellow-men
to have an interest in your enterprise. The owner of the axe, as he
released his hold on it, said that it was the apple of his eye; but I
returned it sharper than I received it. It was a pleasant hillside
where I worked, covered with pine woods, through which I looked out on
the pond, and a small open field in the woods where pines and hickories
were springing up. The ice in the pond was not yet dissolved, though
there were some open spaces, and it was all dark colored and saturated
with water. There were some slight flurries of snow during the days
that I worked there; but for the most part when I came out on to the
railroad, on my way home, its yellow sand heap stretched away gleaming
in the hazy atmosphere, and the rails shone in the spring sun, and I
heard the lark and pewee and other birds already come to commence
another year with us. They were pleasant spring days, in which the
winter of man’s discontent was thawing as well as the earth, and the
life that had lain torpid began to stretch itself. One day, when my axe
had come off and I had cut a green hickory for a wedge, driving it with
a stone, and had placed the whole to soak in a pond hole in order to
swell the wood, I saw a striped snake run into the water, and he lay on
the bottom, apparently without inconvenience, as long as I stayed
there, or more than a quarter of an hour; perhaps because he had not
yet fairly come out of the torpid state. It appeared to me that for a
like reason men remain in their present low and primitive condition;
but if they should feel the influence of the spring of springs arousing
them, they would of necessity rise to a higher and more ethereal life.
I had previously seen the snakes in frosty mornings in my path with
portions of their bodies still numb and inflexible, waiting for the sun
to thaw them. On the 1st of April it rained and melted the ice, and in
the early part of the day, which was very foggy, I heard a stray goose
groping about over the pond and cackling as if lost, or like the spirit
of the fog.

So I went on for some days cutting and hewing timber, and also studs
and rafters, all with my narrow axe, not having many communicable or
scholar-like thoughts, singing to myself,—

     Men say they know many things;
     But lo! they have taken wings,—
     The arts and sciences,
     And a thousand appliances;
     The wind that blows
     Is all that any body knows.

I hewed the main timbers six inches square, most of the studs on two
sides only, and the rafters and floor timbers on one side, leaving the
rest of the bark on, so that they were just as straight and much
stronger than sawed ones. Each stick was carefully mortised or tenoned
by its stump, for I had borrowed other tools by this time. My days in
the woods were not very long ones; yet I usually carried my dinner of
bread and butter, and read the newspaper in which it was wrapped, at
noon, sitting amid the green pine boughs which I had cut off, and to my
bread was imparted some of their fragrance, for my hands were covered
with a thick coat of pitch. Before I had done I was more the friend
than the foe of the pine tree, though I had cut down some of them,
having become better acquainted with it. Sometimes a rambler in the
wood was attracted by the sound of my axe, and we chatted pleasantly
over the chips which I had made.

By the middle of April, for I made no haste in my work, but rather made
the most of it, my house was framed and ready for the raising. I had
already bought the shanty of James Collins, an Irishman who worked on
the Fitchburg Railroad, for boards. James Collins’ shanty was
considered an uncommonly fine one. When I called to see it he was not
at home. I walked about the outside, at first unobserved from within,
the window was so deep and high. It was of small dimensions, with a
peaked cottage roof, and not much else to be seen, the dirt being
raised five feet all around as if it were a compost heap. The roof was
the soundest part, though a good deal warped and made brittle by the
sun. Door-sill there was none, but a perennial passage for the hens
under the door board. Mrs. C. came to the door and asked me to view it
from the inside. The hens were driven in by my approach. It was dark,
and had a dirt floor for the most part, dank, clammy, and aguish, only
here a board and there a board which would not bear removal. She
lighted a lamp to show me the inside of the roof and the walls, and
also that the board floor extended under the bed, warning me not to
step into the cellar, a sort of dust hole two feet deep. In her own
words, they were “good boards overhead, good boards all around, and a
good window,”—of two whole squares originally, only the cat had passed
out that way lately. There was a stove, a bed, and a place to sit, an
infant in the house where it was born, a silk parasol, gilt-framed
looking-glass, and a patent new coffee mill nailed to an oak sapling,
all told. The bargain was soon concluded, for James had in the
meanwhile returned. I to pay four dollars and twenty-five cents
to-night, he to vacate at five to-morrow morning, selling to nobody
else meanwhile: I to take possession at six. It were well, he said, to
be there early, and anticipate certain indistinct but wholly unjust
claims on the score of ground rent and fuel. This he assured me was the
only encumbrance. At six I passed him and his family on the road. One
large bundle held their all,—bed, coffee-mill, looking-glass, hens,—all
but the cat, she took to the woods and became a wild cat, and, as I
learned afterward, trod in a trap set for woodchucks, and so became a
dead cat at last.

I took down this dwelling the same morning, drawing the nails, and
removed it to the pond side by small cartloads, spreading the boards on
the grass there to bleach and warp back again in the sun. One early
thrush gave me a note or two as I drove along the woodland path. I was
informed treacherously by a young Patrick that neighbor Seeley, an
Irishman, in the intervals of the carting, transferred the still
tolerable, straight, and drivable nails, staples, and spikes to his
pocket, and then stood when I came back to pass the time of day, and
look freshly up, unconcerned, with spring thoughts, at the devastation;
there being a dearth of work, as he said. He was there to represent
spectatordom, and help make this seemingly insignificant event one with
the removal of the gods of Troy.

I dug my cellar in the side of a hill sloping to the south, where a
woodchuck had formerly dug his burrow, down through sumach and
blackberry roots, and the lowest stain of vegetation, six feet square
by seven deep, to a fine sand where potatoes would not freeze in any
winter. The sides were left shelving, and not stoned; but the sun
having never shone on them, the sand still keeps its place. It was but
two hours’ work. I took particular pleasure in this breaking of ground,
for in almost all latitudes men dig into the earth for an equable
temperature. Under the most splendid house in the city is still to be
found the cellar where they store their roots as of old, and long after
the superstructure has disappeared posterity remark its dent in the
earth. The house is still but a sort of porch at the entrance of a
burrow.

At length, in the beginning of May, with the help of some of my
acquaintances, rather to improve so good an occasion for neighborliness
than from any necessity, I set up the frame of my house. No man was
ever more honored in the character of his raisers than I. They are
destined, I trust, to assist at the raising of loftier structures one
day. I began to occupy my house on the 4th of July, as soon as it was
boarded and roofed, for the boards were carefully feather-edged and
lapped, so that it was perfectly impervious to rain; but before
boarding I laid the foundation of a chimney at one end, bringing two
cartloads of stones up the hill from the pond in my arms. I built the
chimney after my hoeing in the fall, before a fire became necessary for
warmth, doing my cooking in the mean while out of doors on the ground,
early in the morning: which mode I still think is in some respects more
convenient and agreeable than the usual one. When it stormed before my
bread was baked, I fixed a few boards over the fire, and sat under them
to watch my loaf, and passed some pleasant hours in that way. In those
days, when my hands were much employed, I read but little, but the
least scraps of paper which lay on the ground, my holder, or
tablecloth, afforded me as much entertainment, in fact answered the
same purpose as the Iliad.



It would be worth the while to build still more deliberately than I
did, considering, for instance, what foundation a door, a window, a
cellar, a garret, have in the nature of man, and perchance never
raising any superstructure until we found a better reason for it than
our temporal necessities even. There is some of the same fitness in a
man’s building his own house that there is in a bird’s building its own
nest. Who knows but if men constructed their dwellings with their own
hands, and provided food for themselves and families simply and
honestly enough, the poetic faculty would be universally developed, as
birds universally sing when they are so engaged? But alas! we do like
cowbirds and cuckoos, which lay their eggs in nests which other birds
have built, and cheer no traveller with their chattering and unmusical
notes. Shall we forever resign the pleasure of construction to the
carpenter? What does architecture amount to in the experience of the
mass of men? I never in all my walks came across a man engaged in so
simple and natural an occupation as building his house. We belong to
the community. It is not the tailor alone who is the ninth part of a
man; it is as much the preacher, and the merchant, and the farmer.
Where is this division of labor to end? and what object does it finally
serve? No doubt another _may_ also think for me; but it is not
therefore desirable that he should do so to the exclusion of my
thinking for myself.

True, there are architects so called in this country, and I have heard
of one at least possessed with the idea of making architectural
ornaments have a core of truth, a necessity, and hence a beauty, as if
it were a revelation to him. All very well perhaps from his point of
view, but only a little better than the common dilettantism. A
sentimental reformer in architecture, he began at the cornice, not at
the foundation. It was only how to put a core of truth within the
ornaments, that every sugar plum in fact might have an almond or
caraway seed in it,—though I hold that almonds are most wholesome
without the sugar,—and not how the inhabitant, the indweller, might
build truly within and without, and let the ornaments take care of
themselves. What reasonable man ever supposed that ornaments were
something outward and in the skin merely,—that the tortoise got his
spotted shell, or the shellfish its mother-o’-pearl tints, by such a
contract as the inhabitants of Broadway their Trinity Church? But a man
has no more to do with the style of architecture of his house than a
tortoise with that of its shell: nor need the soldier be so idle as to
try to paint the precise color of his virtue on his standard. The enemy
will find it out. He may turn pale when the trial comes. This man
seemed to me to lean over the cornice, and timidly whisper his half
truth to the rude occupants who really knew it better than he. What of
architectural beauty I now see, I know has gradually grown from within
outward, out of the necessities and character of the indweller, who is
the only builder,—out of some unconscious truthfulness, and nobleness,
without ever a thought for the appearance and whatever additional
beauty of this kind is destined to be produced will be preceded by a
like unconscious beauty of life. The most interesting dwellings in this
country, as the painter knows, are the most unpretending, humble log
huts and cottages of the poor commonly; it is the life of the
inhabitants whose shells they are, and not any peculiarity in their
surfaces merely, which makes them _picturesque;_ and equally
interesting will be the citizen’s suburban box, when his life shall be
as simple and as agreeable to the imagination, and there is as little
straining after effect in the style of his dwelling. A great proportion
of architectural ornaments are literally hollow, and a September gale
would strip them off, like borrowed plumes, without injury to the
substantials. They can do without _architecture_ who have no olives nor
wines in the cellar. What if an equal ado were made about the ornaments
of style in literature, and the architects of our bibles spent as much
time about their cornices as the architects of our churches do? So are
made the _belles-lettres_ and the _beaux-arts_ and their professors.
Much it concerns a man, forsooth, how a few sticks are slanted over him
or under him, and what colors are daubed upon his box. It would signify
somewhat, if, in any earnest sense, _he_ slanted them and daubed it;
but the spirit having departed out of the tenant, it is of a piece with
constructing his own coffin,—the architecture of the grave, and
“carpenter” is but another name for “coffin-maker.” One man says, in
his despair or indifference to life, take up a handful of the earth at
your feet, and paint your house that color. Is he thinking of his last
and narrow house? Toss up a copper for it as well. What an abundance of
leisure he must have! Why do you take up a handful of dirt? Better
paint your house your own complexion; let it turn pale or blush for
you. An enterprise to improve the style of cottage architecture! When
you have got my ornaments ready I will wear them.

Before winter I built a chimney, and shingled the sides of my house,
which were already impervious to rain, with imperfect and sappy
shingles made of the first slice of the log, whose edges I was obliged
to straighten with a plane.

I have thus a tight shingled and plastered house, ten feet wide by
fifteen long, and eight-feet posts, with a garret and a closet, a large
window on each side, two trap doors, one door at the end, and a brick
fireplace opposite. The exact cost of my house, paying the usual price
for such materials as I used, but not counting the work, all of which
was done by myself, was as follows; and I give the details because very
few are able to tell exactly what their houses cost, and fewer still,
if any, the separate cost of the various materials which compose them:—


    Boards.......................... $ 8.03½, mostly shanty boards.
    Refuse shingles for roof sides,..  4.00
    Laths,...........................  1.25
    Two second-hand windows
       with glass,...................  2.43
    One thousand old brick,..........  4.00
    Two casks of lime,...............  2.40  That was high.
    Hair,............................  0.31  More than I needed.
    Mantle-tree iron,................  0.15
    Nails,...........................  3.90
    Hinges and screws,...............  0.14
    Latch,...........................  0.10
    Chalk,...........................  0.01
    Transportation,..................  1.40  I carried a good part
                                     ———— on my back.
        In all,..................... \$28.12½


These are all the materials excepting the timber stones and sand, which
I claimed by squatter’s right. I have also a small wood-shed adjoining,
made chiefly of the stuff which was left after building the house.

I intend to build me a house which will surpass any on the main street
in Concord in grandeur and luxury, as soon as it pleases me as much and
will cost me no more than my present one.

I thus found that the student who wishes for a shelter can obtain one
for a lifetime at an expense not greater than the rent which he now
pays annually. If I seem to boast more than is becoming, my excuse is
that I brag for humanity rather than for myself; and my shortcomings
and inconsistencies do not affect the truth of my statement.
Notwithstanding much cant and hypocrisy,—chaff which I find it
difficult to separate from my wheat, but for which I am as sorry as any
man,—I will breathe freely and stretch myself in this respect, it is
such a relief to both the moral and physical system; and I am resolved
that I will not through humility become the devil’s attorney. I will
endeavor to speak a good word for the truth. At Cambridge College the
mere rent of a student’s room, which is only a little larger than my
own, is thirty dollars each year, though the corporation had the
advantage of building thirty-two side by side and under one roof, and
the occupant suffers the inconvenience of many and noisy neighbors, and
perhaps a residence in the fourth story. I cannot but think that if we
had more true wisdom in these respects, not only less education would
be needed, because, forsooth, more would already have been acquired,
but the pecuniary expense of getting an education would in a great
measure vanish. Those conveniences which the student requires at
Cambridge or elsewhere cost him or somebody else ten times as great a
sacrifice of life as they would with proper management on both sides.
Those things for which the most money is demanded are never the things
which the student most wants. Tuition, for instance, is an important
item in the term bill, while for the far more valuable education which
he gets by associating with the most cultivated of his contemporaries
no charge is made. The mode of founding a college is, commonly, to get
up a subscription of dollars and cents, and then following blindly the
principles of a division of labor to its extreme, a principle which
should never be followed but with circumspection,—to call in a
contractor who makes this a subject of speculation, and he employs
Irishmen or other operatives actually to lay the foundations, while the
students that are to be are said to be fitting themselves for it; and
for these oversights successive generations have to pay. I think that
it would be _better than this_, for the students, or those who desire
to be benefited by it, even to lay the foundation themselves. The
student who secures his coveted leisure and retirement by
systematically shirking any labor necessary to man obtains but an
ignoble and unprofitable leisure, defrauding himself of the experience
which alone can make leisure fruitful. “But,” says one, “you do not
mean that the students should go to work with their hands instead of
their heads?” I do not mean that exactly, but I mean something which he
might think a good deal like that; I mean that they should not _play_
life, or _study_ it merely, while the community supports them at this
expensive game, but earnestly _live_ it from beginning to end. How
could youths better learn to live than by at once trying the experiment
of living? Methinks this would exercise their minds as much as
mathematics. If I wished a boy to know something about the arts and
sciences, for instance, I would not pursue the common course, which is
merely to send him into the neighborhood of some professor, where any
thing is professed and practised but the art of life;—to survey the
world through a telescope or a microscope, and never with his natural
eye; to study chemistry, and not learn how his bread is made, or
mechanics, and not learn how it is earned; to discover new satellites
to Neptune, and not detect the motes in his eyes, or to what vagabond
he is a satellite himself; or to be devoured by the monsters that swarm
all around him, while contemplating the monsters in a drop of vinegar.
Which would have advanced the most at the end of a month,—the boy who
had made his own jackknife from the ore which he had dug and smelted,
reading as much as would be necessary for this,—or the boy who had
attended the lectures on metallurgy at the Institute in the mean while,
and had received a Rodgers’ penknife from his father? Which would be
most likely to cut his fingers?... To my astonishment I was informed on
leaving college that I had studied navigation!—why, if I had taken one
turn down the harbor I should have known more about it. Even the _poor_
student studies and is taught only _political_ economy, while that
economy of living which is synonymous with philosophy is not even
sincerely professed in our colleges. The consequence is, that while he
is reading Adam Smith, Ricardo, and Say, he runs his father in debt
irretrievably.

As with our colleges, so with a hundred “modern improvements”; there is
an illusion about them; there is not always a positive advance. The
devil goes on exacting compound interest to the last for his early
share and numerous succeeding investments in them. Our inventions are
wont to be pretty toys, which distract our attention from serious
things. They are but improved means to an unimproved end, an end which
it was already but too easy to arrive at; as railroads lead to Boston
or New York. We are in great haste to construct a magnetic telegraph
from Maine to Texas; but Maine and Texas, it may be, have nothing
important to communicate. Either is in such a predicament as the man
who was earnest to be introduced to a distinguished deaf woman, but
when he was presented, and one end of her ear trumpet was put into his
hand, had nothing to say. As if the main object were to talk fast and
not to talk sensibly. We are eager to tunnel under the Atlantic and
bring the old world some weeks nearer to the new; but perchance the
first news that will leak through into the broad, flapping American ear
will be that the Princess Adelaide has the whooping cough. After all,
the man whose horse trots a mile in a minute does not carry the most
important messages; he is not an evangelist, nor does he come round
eating locusts and wild honey. I doubt if Flying Childers ever carried
a peck of corn to mill.

One says to me, “I wonder that you do not lay up money; you love to
travel; you might take the cars and go to Fitchburg to-day and see the
country.” But I am wiser than that. I have learned that the swiftest
traveller is he that goes afoot. I say to my friend, Suppose we try who
will get there first. The distance is thirty miles; the fare ninety
cents. That is almost a day’s wages. I remember when wages were sixty
cents a day for laborers on this very road. Well, I start now on foot,
and get there before night; I have travelled at that rate by the week
together. You will in the mean while have earned your fare, and arrive
there some time to-morrow, or possibly this evening, if you are lucky
enough to get a job in season. Instead of going to Fitchburg, you will
be working here the greater part of the day. And so, if the railroad
reached round the world, I think that I should keep ahead of you; and
as for seeing the country and getting experience of that kind, I should
have to cut your acquaintance altogether.

Such is the universal law, which no man can ever outwit, and with
regard to the railroad even we may say it is as broad as it is long. To
make a railroad round the world available to all mankind is equivalent
to grading the whole surface of the planet. Men have an indistinct
notion that if they keep up this activity of joint stocks and spades
long enough all will at length ride somewhere, in next to no time, and
for nothing; but though a crowd rushes to the depot, and the conductor
shouts “All aboard!” when the smoke is blown away and the vapor
condensed, it will be perceived that a few are riding, but the rest are
run over,—and it will be called, and will be, “A melancholy accident.”
No doubt they can ride at last who shall have earned their fare, that
is, if they survive so long, but they will probably have lost their
elasticity and desire to travel by that time. This spending of the best
part of one’s life earning money in order to enjoy a questionable
liberty during the least valuable part of it, reminds me of the
Englishman who went to India to make a fortune first, in order that he
might return to England and live the life of a poet. He should have
gone up garret at once. “What!” exclaim a million Irishmen starting up
from all the shanties in the land, “is not this railroad which we have
built a good thing?” Yes, I answer, _comparatively_ good, that is, you
might have done worse; but I wish, as you are brothers of mine, that
you could have spent your time better than digging in this dirt.



Before I finished my house, wishing to earn ten or twelve dollars by
some honest and agreeable method, in order to meet my unusual expenses,
I planted about two acres and a half of light and sandy soil near it
chiefly with beans, but also a small part with potatoes, corn, peas,
and turnips. The whole lot contains eleven acres, mostly growing up to
pines and hickories, and was sold the preceding season for eight
dollars and eight cents an acre. One farmer said that it was “good for
nothing but to raise cheeping squirrels on.” I put no manure whatever
on this land, not being the owner, but merely a squatter, and not
expecting to cultivate so much again, and I did not quite hoe it all
once. I got out several cords of stumps in ploughing, which supplied me
with fuel for a long time, and left small circles of virgin mould,
easily distinguishable through the summer by the greater luxuriance of
the beans there. The dead and for the most part unmerchantable wood
behind my house, and the driftwood from the pond, have supplied the
remainder of my fuel. I was obliged to hire a team and a man for the
ploughing, though I held the plough myself. My farm outgoes for the
first season were, for implements, seed, work, &c., $14.72½. The seed
corn was given me. This never costs anything to speak of, unless you
plant more than enough. I got twelve bushels of beans, and eighteen
bushels of potatoes, beside some peas and sweet corn. The yellow corn
and turnips were too late to come to any thing. My whole income from
the farm was

                                       $ 23.44
      Deducting the outgoes,...........  14.72½
                                         ————
      There are left,................. $  8.71½,

beside produce consumed and on hand at the time this estimate was made
of the value of $4.50,—the amount on hand much more than balancing a
little grass which I did not raise. All things considered, that is,
considering the importance of a man’s soul and of to-day,
notwithstanding the short time occupied by my experiment, nay, partly
even because of its transient character, I believe that that was doing
better than any farmer in Concord did that year.

The next year I did better still, for I spaded up all the land which I
required, about a third of an acre, and I learned from the experience
of both years, not being in the least awed by many celebrated works on
husbandry, Arthur Young among the rest, that if one would live simply
and eat only the crop which he raised, and raise no more than he ate,
and not exchange it for an insufficient quantity of more luxurious and
expensive things, he would need to cultivate only a few rods of ground,
and that it would be cheaper to spade up that than to use oxen to
plough it, and to select a fresh spot from time to time than to manure
the old, and he could do all his necessary farm work as it were with
his left hand at odd hours in the summer; and thus he would not be tied
to an ox, or horse, or cow, or pig, as at present. I desire to speak
impartially on this point, and as one not interested in the success or
failure of the present economical and social arrangements. I was more
independent than any farmer in Concord, for I was not anchored to a
house or farm, but could follow the bent of my genius, which is a very
crooked one, every moment. Beside being better off than they already,
if my house had been burned or my crops had failed, I should have been
nearly as well off as before.

I am wont to think that men are not so much the keepers of herds as
herds are the keepers of men, the former are so much the freer. Men and
oxen exchange work; but if we consider necessary work only, the oxen
will be seen to have greatly the advantage, their farm is so much the
larger. Man does some of his part of the exchange work in his six weeks
of haying, and it is no boy’s play. Certainly no nation that lived
simply in all respects, that is, no nation of philosophers, would
commit so great a blunder as to use the labor of animals. True, there
never was and is not likely soon to be a nation of philosophers, nor am
I certain it is desirable that there should be. However, _I_ should
never have broken a horse or bull and taken him to board for any work
he might do for me, for fear I should become a horse-man or a herds-man
merely; and if society seems to be the gainer by so doing, are we
certain that what is one man’s gain is not another’s loss, and that the
stable-boy has equal cause with his master to be satisfied? Granted
that some public works would not have been constructed without this
aid, and let man share the glory of such with the ox and horse; does it
follow that he could not have accomplished works yet more worthy of
himself in that case? When men begin to do, not merely unnecessary or
artistic, but luxurious and idle work, with their assistance, it is
inevitable that a few do all the exchange work with the oxen, or, in
other words, become the slaves of the strongest. Man thus not only
works for the animal within him, but, for a symbol of this, he works
for the animal without him. Though we have many substantial houses of
brick or stone, the prosperity of the farmer is still measured by the
degree to which the barn overshadows the house. This town is said to
have the largest houses for oxen, cows, and horses hereabouts, and it
is not behindhand in its public buildings; but there are very few halls
for free worship or free speech in this county. It should not be by
their architecture, but why not even by their power of abstract
thought, that nations should seek to commemorate themselves? How much
more admirable the Bhagvat-Geeta than all the ruins of the East! Towers
and temples are the luxury of princes. A simple and independent mind
does not toil at the bidding of any prince. Genius is not a retainer to
any emperor, nor is its material silver, or gold, or marble, except to
a trifling extent. To what end, pray, is so much stone hammered? In
Arcadia, when I was there, I did not see any hammering stone. Nations
are possessed with an insane ambition to perpetuate the memory of
themselves by the amount of hammered stone they leave. What if equal
pains were taken to smooth and polish their manners? One piece of good
sense would be more memorable than a monument as high as the moon. I
love better to see stones in place. The grandeur of Thebes was a vulgar
grandeur. More sensible is a rod of stone wall that bounds an honest
man’s field than a hundred-gated Thebes that has wandered farther from
the true end of life. The religion and civilization which are barbaric
and heathenish build splendid temples; but what you might call
Christianity does not. Most of the stone a nation hammers goes toward
its tomb only. It buries itself alive. As for the Pyramids, there is
nothing to wonder at in them so much as the fact that so many men could
be found degraded enough to spend their lives constructing a tomb for
some ambitious booby, whom it would have been wiser and manlier to have
drowned in the Nile, and then given his body to the dogs. I might
possibly invent some excuse for them and him, but I have no time for
it. As for the religion and love of art of the builders, it is much the
same all the world over, whether the building be an Egyptian temple or
the United States Bank. It costs more than it comes to. The mainspring
is vanity, assisted by the love of garlic and bread and butter. Mr.
Balcom, a promising young architect, designs it on the back of his
Vitruvius, with hard pencil and ruler, and the job is let out to Dobson
& Sons, stonecutters. When the thirty centuries begin to look down on
it, mankind begin to look up at it. As for your high towers and
monuments, there was a crazy fellow once in this town who undertook to
dig through to China, and he got so far that, as he said, he heard the
Chinese pots and kettles rattle; but I think that I shall not go out of
my way to admire the hole which he made. Many are concerned about the
monuments of the West and the East,—to know who built them. For my
part, I should like to know who in those days did not build them,—who
were above such trifling. But to proceed with my statistics.

By surveying, carpentry, and day-labor of various other kinds in the
village in the mean while, for I have as many trades as fingers, I had
earned $13.34. The expense of food for eight months, namely, from July
4th to March 1st, the time when these estimates were made, though I
lived there more than two years,—not counting potatoes, a little green
corn, and some peas, which I had raised, nor considering the value of
what was on hand at the last date, was


    Rice,................... $ 1.73½
    Molasses,................  1.73     Cheapest form of the
                                         saccharine.
    Rye meal,................  1.04¾
    Indian meal,.............  0.99¾     Cheaper than rye.
    Pork,....................  0.22

    All experiments which failed:
    Flour,...................  0.88  Costs more than Indian meal,
                                      both money and trouble.
    Sugar,...................  0.80
    Lard,....................  0.65
    Apples,..................  0.25
    Dried apple,.............  0.22
    Sweet potatoes,..........  0.10
    One pumpkin,.............  0.06
    One watermelon,..........  0.02
    Salt,....................  0.03

Yes, I did eat $8.74, all told; but I should not thus unblushingly
publish my guilt, if I did not know that most of my readers were
equally guilty with myself, and that their deeds would look no better
in print. The next year I sometimes caught a mess of fish for my
dinner, and once I went so far as to slaughter a woodchuck which
ravaged my bean-field,—effect his transmigration, as a Tartar would
say,—and devour him, partly for experiment’s sake; but though it
afforded me a momentary enjoyment, notwithstanding a musky flavor, I
saw that the longest use would not make that a good practice, however
it might seem to have your woodchucks ready dressed by the village
butcher.

Clothing and some incidental expenses within the same dates, though
little can be inferred from this item, amounted to

                                            $8.40¾
    Oil and some household utensils,.......  2.00

So that all the pecuniary outgoes, excepting for washing and mending,
which for the most part were done out of the house, and their bills
have not yet been received,—and these are all and more than all the
ways by which money necessarily goes out in this part of the
world,—were

    House,................................ $ 28.12½
    Farm one year,.......................... 14.72½
    Food eight months,......................  8.74
    Clothing, etc., eight months,...........  8.40¾
    Oil, &c., eight months,.................  2.00
                                           ——————
        In all,........................... $ 61.99¾

I address myself now to those of my readers who have a living to get.
And to meet this I have for farm produce sold

                                            $23.44
    Earned by day-labor,...................  13.34
                                           ——————
        In all,............................ $36.78,

which subtracted from the sum of the outgoes leaves a balance of
$25.21¾ on the one side,—this being very nearly the means with which I
started, and the measure of expenses to be incurred,—and on the other,
beside the leisure and independence and health thus secured, a
comfortable house for me as long as I choose to occupy it.

These statistics, however accidental and therefore uninstructive they
may appear, as they have a certain completeness, have a certain value
also. Nothing was given me of which I have not rendered some account.
It appears from the above estimate, that my food alone cost me in money
about twenty-seven cents a week. It was, for nearly two years after
this, rye and Indian meal without yeast, potatoes, rice, a very little
salt pork, molasses, and salt, and my drink water. It was fit that I
should live on rice, mainly, who loved so well the philosophy of India.
To meet the objections of some inveterate cavillers, I may as well
state, that if I dined out occasionally, as I always had done, and I
trust shall have opportunities to do again, it was frequently to the
detriment of my domestic arrangements. But the dining out, being, as I
have stated, a constant element, does not in the least affect a
comparative statement like this.

I learned from my two years’ experience that it would cost incredibly
little trouble to obtain one’s necessary food, even in this latitude;
that a man may use as simple a diet as the animals, and yet retain
health and strength. I have made a satisfactory dinner, satisfactory on
several accounts, simply off a dish of purslane (_Portulaca oleracea_)
which I gathered in my cornfield, boiled and salted. I give the Latin
on account of the savoriness of the trivial name. And pray what more
can a reasonable man desire, in peaceful times, in ordinary noons, than
a sufficient number of ears of green sweet-corn boiled, with the
addition of salt? Even the little variety which I used was a yielding
to the demands of appetite, and not of health. Yet men have come to
such a pass that they frequently starve, not for want of necessaries,
but for want of luxuries; and I know a good woman who thinks that her
son lost his life because he took to drinking water only.

The reader will perceive that I am treating the subject rather from an
economic than a dietetic point of view, and he will not venture to put
my abstemiousness to the test unless he has a well-stocked larder.

Bread I at first made of pure Indian meal and salt, genuine hoe-cakes,
which I baked before my fire out of doors on a shingle or the end of a
stick of timber sawed off in building my house; but it was wont to get
smoked and to have a piny flavor. I tried flour also; but have at last
found a mixture of rye and Indian meal most convenient and agreeable.
In cold weather it was no little amusement to bake several small loaves
of this in succession, tending and turning them as carefully as an
Egyptian his hatching eggs. They were a real cereal fruit which I
ripened, and they had to my senses a fragrance like that of other noble
fruits, which I kept in as long as possible by wrapping them in cloths.
I made a study of the ancient and indispensable art of bread-making,
consulting such authorities as offered, going back to the primitive
days and first invention of the unleavened kind, when from the wildness
of nuts and meats men first reached the mildness and refinement of this
diet, and travelling gradually down in my studies through that
accidental souring of the dough which, it is supposed, taught the
leavening process, and through the various fermentations thereafter,
till I came to “good, sweet, wholesome bread,” the staff of life.
Leaven, which some deem the soul of bread, the _spiritus_ which fills
its cellular tissue, which is religiously preserved like the vestal
fire,—some precious bottle-full, I suppose, first brought over in the
Mayflower, did the business for America, and its influence is still
rising, swelling, spreading, in cerealian billows over the land,—this
seed I regularly and faithfully procured from the village, till at
length one morning I forgot the rules, and scalded my yeast; by which
accident I discovered that even this was not indispensable,—for my
discoveries were not by the synthetic but analytic process,—and I have
gladly omitted it since, though most housewives earnestly assured me
that safe and wholesome bread without yeast might not be, and elderly
people prophesied a speedy decay of the vital forces. Yet I find it not
to be an essential ingredient, and after going without it for a year am
still in the land of the living; and I am glad to escape the
trivialness of carrying a bottle-full in my pocket, which would
sometimes pop and discharge its contents to my discomfiture. It is
simpler and more respectable to omit it. Man is an animal who more than
any other can adapt himself to all climates and circumstances. Neither
did I put any sal soda, or other acid or alkali, into my bread. It
would seem that I made it according to the recipe which Marcus Porcius
Cato gave about two centuries before Christ. “Panem depsticium sic
facito. Manus mortariumque bene lavato. Farinam in mortarium indito,
aquæ paulatim addito, subigitoque pulchre. Ubi bene subegeris,
defingito, coquitoque sub testu.” Which I take to mean—“Make kneaded
bread thus. Wash your hands and trough well. Put the meal into the
trough, add water gradually, and knead it thoroughly. When you have
kneaded it well, mould it, and bake it under a cover,” that is, in a
baking-kettle. Not a word about leaven. But I did not always use this
staff of life. At one time, owing to the emptiness of my purse, I saw
none of it for more than a month.

Every New Englander might easily raise all his own breadstuffs in this
land of rye and Indian corn, and not depend on distant and fluctuating
markets for them. Yet so far are we from simplicity and independence
that, in Concord, fresh and sweet meal is rarely sold in the shops, and
hominy and corn in a still coarser form are hardly used by any. For the
most part the farmer gives to his cattle and hogs the grain of his own
producing, and buys flour, which is at least no more wholesome, at a
greater cost, at the store. I saw that I could easily raise my bushel
or two of rye and Indian corn, for the former will grow on the poorest
land, and the latter does not require the best, and grind them in a
hand-mill, and so do without rice and pork; and if I must have some
concentrated sweet, I found by experiment that I could make a very good
molasses either of pumpkins or beets, and I knew that I needed only to
set out a few maples to obtain it more easily still, and while these
were growing I could use various substitutes beside those which I have
named. “For,” as the Forefathers sang,—

     “we can make liquor to sweeten our lips
     Of pumpkins and parsnips and walnut-tree chips.”

Finally, as for salt, that grossest of groceries, to obtain this might
be a fit occasion for a visit to the seashore, or, if I did without it
altogether, I should probably drink the less water. I do not learn that
the Indians ever troubled themselves to go after it.

Thus I could avoid all trade and barter, so far as my food was
concerned, and having a shelter already, it would only remain to get
clothing and fuel. The pantaloons which I now wear were woven in a
farmer’s family,—thank Heaven there is so much virtue still in man; for
I think the fall from the farmer to the operative as great and
memorable as that from the man to the farmer;—and in a new country,
fuel is an encumbrance. As for a habitat, if I were not permitted still
to squat, I might purchase one acre at the same price for which the
land I cultivated was sold—namely, eight dollars and eight cents. But
as it was, I considered that I enhanced the value of the land by
squatting on it.

There is a certain class of unbelievers who sometimes ask me such
questions as, if I think that I can live on vegetable food alone; and
to strike at the root of the matter at once,—for the root is faith,—I
am accustomed to answer such, that I can live on board nails. If they
cannot understand that, they cannot understand much that I have to say.
For my part, I am glad to hear of experiments of this kind being tried;
as that a young man tried for a fortnight to live on hard, raw corn on
the ear, using his teeth for all mortar. The squirrel tribe tried the
same and succeeded. The human race is interested in these experiments,
though a few old women who are incapacitated for them, or who own their
thirds in mills, may be alarmed.



My furniture, part of which I made myself, and the rest cost me nothing
of which I have not rendered an account, consisted of a bed, a table, a
desk, three chairs, a looking-glass three inches in diameter, a pair of
tongs and andirons, a kettle, a skillet, and a frying-pan, a dipper, a
wash-bowl, two knives and forks, three plates, one cup, one spoon, a
jug for oil, a jug for molasses, and a japanned lamp. None is so poor
that he need sit on a pumpkin. That is shiftlessness. There is a plenty
of such chairs as I like best in the village garrets to be had for
taking them away. Furniture! Thank God, I can sit and I can stand
without the aid of a furniture warehouse. What man but a philosopher
would not be ashamed to see his furniture packed in a cart and going up
country exposed to the light of heaven and the eyes of men, a beggarly
account of empty boxes? That is Spaulding’s furniture. I could never
tell from inspecting such a load whether it belonged to a so called
rich man or a poor one; the owner always seemed poverty-stricken.
Indeed, the more you have of such things the poorer you are. Each load
looks as if it contained the contents of a dozen shanties; and if one
shanty is poor, this is a dozen times as poor. Pray, for what do we
_move_ ever but to get rid of our furniture, our _exuviæ_; at last to
go from this world to another newly furnished, and leave this to be
burned? It is the same as if all these traps were buckled to a man’s
belt, and he could not move over the rough country where our lines are
cast without dragging them,—dragging his trap. He was a lucky fox that
left his tail in the trap. The muskrat will gnaw his third leg off to
be free. No wonder man has lost his elasticity. How often he is at a
dead set! “Sir, if I may be so bold, what do you mean by a dead set?”
If you are a seer, whenever you meet a man you will see all that he
owns, ay, and much that he pretends to disown, behind him, even to his
kitchen furniture and all the trumpery which he saves and will not
burn, and he will appear to be harnessed to it and making what headway
he can. I think that the man is at a dead set who has got through a
knot hole or gateway where his sledge load of furniture cannot follow
him. I cannot but feel compassion when I hear some trig,
compact-looking man, seemingly free, all girded and ready, speak of his
“furniture,” as whether it is insured or not. “But what shall I do with
my furniture?” My gay butterfly is entangled in a spider’s web then.
Even those who seem for a long while not to have any, if you inquire
more narrowly you will find have some stored in somebody’s barn. I look
upon England to-day as an old gentleman who is travelling with a great
deal of baggage, trumpery which has accumulated from long housekeeping,
which he has not the courage to burn; great trunk, little trunk,
bandbox and bundle. Throw away the first three at least. It would
surpass the powers of a well man nowadays to take up his bed and walk,
and I should certainly advise a sick one to lay down his bed and run.
When I have met an immigrant tottering under a bundle which contained
his all—looking like an enormous wen which had grown out of the nape of
his neck—I have pitied him, not because that was his all, but because
he had all _that_ to carry. If I have got to drag my trap, I will take
care that it be a light one and do not nip me in a vital part. But
perchance it would be wisest never to put one’s paw into it.

I would observe, by the way, that it costs me nothing for curtains, for
I have no gazers to shut out but the sun and moon, and I am willing
that they should look in. The moon will not sour milk nor taint meat of
mine, nor will the sun injure my furniture or fade my carpet, and if he
is sometimes too warm a friend, I find it still better economy to
retreat behind some curtain which nature has provided, than to add a
single item to the details of housekeeping. A lady once offered me a
mat, but as I had no room to spare within the house, nor time to spare
within or without to shake it, I declined it, preferring to wipe my
feet on the sod before my door. It is best to avoid the beginnings of
evil.

Not long since I was present at the auction of a deacon’s effects, for
his life had not been ineffectual:—

     “The evil that men do lives after them.”

As usual, a great proportion was trumpery which had begun to accumulate
in his father’s day. Among the rest was a dried tapeworm. And now,
after lying half a century in his garret and other dust holes, these
things were not burned; instead of a _bonfire_, or purifying
destruction of them, there was an _auction_, or increasing of them. The
neighbors eagerly collected to view them, bought them all, and
carefully transported them to their garrets and dust holes, to lie
there till their estates are settled, when they will start again. When
a man dies he kicks the dust.

The customs of some savage nations might, perchance, be profitably
imitated by us, for they at least go through the semblance of casting
their slough annually; they have the idea of the thing, whether they
have the reality or not. Would it not be well if we were to celebrate
such a “busk,” or “feast of first fruits,” as Bartram describes to have
been the custom of the Mucclasse Indians? “When a town celebrates the
busk,” says he, “having previously provided themselves with new
clothes, new pots, pans, and other household utensils and furniture,
they collect all their worn out clothes and other despicable things,
sweep and cleanse their houses, squares, and the whole town of their
filth, which with all the remaining grain and other old provisions they
cast together into one common heap, and consume it with fire. After
having taken medicine, and fasted for three days, all the fire in the
town is extinguished. During this fast they abstain from the
gratification of every appetite and passion whatever. A general amnesty
is proclaimed; all malefactors may return to their town.—”

“On the fourth morning, the high priest, by rubbing dry wood together,
produces new fire in the public square, from whence every habitation in
the town is supplied with the new and pure flame.”

They then feast on the new corn and fruits, and dance and sing for
three days, “and the four following days they receive visits and
rejoice with their friends from neighboring towns who have in like
manner purified and prepared themselves.”

The Mexicans also practised a similar purification at the end of every
fifty-two years, in the belief that it was time for the world to come
to an end.

I have scarcely heard of a truer sacrament, that is, as the dictionary
defines it, “outward and visible sign of an inward and spiritual
grace,” than this, and I have no doubt that they were originally
inspired directly from Heaven to do thus, though they have no biblical
record of the revelation.



For more than five years I maintained myself thus solely by the labor
of my hands, and I found, that by working about six weeks in a year, I
could meet all the expenses of living. The whole of my winters, as well
as most of my summers, I had free and clear for study. I have
thoroughly tried school-keeping, and found that my expenses were in
proportion, or rather out of proportion, to my income, for I was
obliged to dress and train, not to say think and believe, accordingly,
and I lost my time into the bargain. As I did not teach for the good of
my fellow-men, but simply for a livelihood, this was a failure. I have
tried trade; but I found that it would take ten years to get under way
in that, and that then I should probably be on my way to the devil. I
was actually afraid that I might by that time be doing what is called a
good business. When formerly I was looking about to see what I could do
for a living, some sad experience in conforming to the wishes of
friends being fresh in my mind to tax my ingenuity, I thought often and
seriously of picking huckleberries; that surely I could do, and its
small profits might suffice,—for my greatest skill has been to want but
little,—so little capital it required, so little distraction from my
wonted moods, I foolishly thought. While my acquaintances went
unhesitatingly into trade or the professions, I contemplated this
occupation as most like theirs; ranging the hills all summer to pick
the berries which came in my way, and thereafter carelessly dispose of
them; so, to keep the flocks of Admetus. I also dreamed that I might
gather the wild herbs, or carry evergreens to such villagers as loved
to be reminded of the woods, even to the city, by hay-cart loads. But I
have since learned that trade curses everything it handles; and though
you trade in messages from heaven, the whole curse of trade attaches to
the business.

As I preferred some things to others, and especially valued my freedom,
as I could fare hard and yet succeed well, I did not wish to spend my
time in earning rich carpets or other fine furniture, or delicate
cookery, or a house in the Grecian or the Gothic style just yet. If
there are any to whom it is no interruption to acquire these things,
and who know how to use them when acquired, I relinquish to them the
pursuit. Some are “industrious,” and appear to love labor for its own
sake, or perhaps because it keeps them out of worse mischief; to such I
have at present nothing to say. Those who would not know what to do
with more leisure than they now enjoy, I might advise to work twice as
hard as they do,—work till they pay for themselves, and get their free
papers. For myself I found that the occupation of a day-laborer was the
most independent of any, especially as it required only thirty or forty
days in a year to support one. The laborer’s day ends with the going
down of the sun, and he is then free to devote himself to his chosen
pursuit, independent of his labor; but his employer, who speculates
from month to month, has no respite from one end of the year to the
other.

In short, I am convinced, both by faith and experience, that to
maintain one’s self on this earth is not a hardship but a pastime, if
we will live simply and wisely; as the pursuits of the simpler nations
are still the sports of the more artificial. It is not necessary that a
man should earn his living by the sweat of his brow, unless he sweats
easier than I do.

One young man of my acquaintance, who has inherited some acres, told me
that he thought he should live as I did, _if he had the means_. I would
not have any one adopt _my_ mode of living on any account; for, beside
that before he has fairly learned it I may have found out another for
myself, I desire that there may be as many different persons in the
world as possible; but I would have each one be very careful to find
out and pursue _his own_ way, and not his father’s or his mother’s or
his neighbor’s instead. The youth may build or plant or sail, only let
him not be hindered from doing that which he tells me he would like to
do. It is by a mathematical point only that we are wise, as the sailor
or the fugitive slave keeps the polestar in his eye; but that is
sufficient guidance for all our life. We may not arrive at our port
within a calculable period, but we would preserve the true course.

Undoubtedly, in this case, what is true for one is truer still for a
thousand, as a large house is not proportionally more expensive than a
small one, since one roof may cover, one cellar underlie, and one wall
separate several apartments. But for my part, I preferred the solitary
dwelling. Moreover, it will commonly be cheaper to build the whole
yourself than to convince another of the advantage of the common wall;
and when you have done this, the common partition, to be much cheaper,
must be a thin one, and that other may prove a bad neighbor, and also
not keep his side in repair. The only coöperation which is commonly
possible is exceedingly partial and superficial; and what little true
coöperation there is, is as if it were not, being a harmony inaudible
to men. If a man has faith, he will coöperate with equal faith
everywhere; if he has not faith, he will continue to live like the rest
of the world, whatever company he is joined to. To coöperate, in the
highest as well as the lowest sense, means _to get our living
together_. I heard it proposed lately that two young men should travel
together over the world, the one without money, earning his means as he
went, before the mast and behind the plow, the other carrying a bill of
exchange in his pocket. It was easy to see that they could not long be
companions or coöperate, since one would not _operate_ at all. They
would part at the first interesting crisis in their adventures. Above
all, as I have implied, the man who goes alone can start to-day; but he
who travels with another must wait till that other is ready, and it may
be a long time before they get off.



But all this is very selfish, I have heard some of my townsmen say. I
confess that I have hitherto indulged very little in philanthropic
enterprises. I have made some sacrifices to a sense of duty, and among
others have sacrificed this pleasure also. There are those who have
used all their arts to persuade me to undertake the support of some
poor family in the town; and if I had nothing to do,—for the devil
finds employment for the idle,—I might try my hand at some such pastime
as that. However, when I have thought to indulge myself in this
respect, and lay their Heaven under an obligation by maintaining
certain poor persons in all respects as comfortably as I maintain
myself, and have even ventured so far as to make them the offer, they
have one and all unhesitatingly preferred to remain poor. While my
townsmen and women are devoted in so many ways to the good of their
fellows, I trust that one at least may be spared to other and less
humane pursuits. You must have a genius for charity as well as for any
thing else. As for Doing-good, that is one of the professions which are
full. Moreover, I have tried it fairly, and, strange as it may seem, am
satisfied that it does not agree with my constitution. Probably I
should not consciously and deliberately forsake my particular calling
to do the good which society demands of me, to save the universe from
annihilation; and I believe that a like but infinitely greater
steadfastness elsewhere is all that now preserves it. But I would not
stand between any man and his genius; and to him who does this work,
which I decline, with his whole heart and soul and life, I would say,
Persevere, even if the world call it doing evil, as it is most likely
they will.

I am far from supposing that my case is a peculiar one; no doubt many
of my readers would make a similar defence. At doing something,—I will
not engage that my neighbors shall pronounce it good,—I do not hesitate
to say that I should be a capital fellow to hire; but what that is, it
is for my employer to find out. What _good_ I do, in the common sense
of that word, must be aside from my main path, and for the most part
wholly unintended. Men say, practically, Begin where you are and such
as you are, without aiming mainly to become of more worth, and with
kindness aforethought go about doing good. If I were to preach at all
in this strain, I should say rather, Set about being good. As if the
sun should stop when he had kindled his fires up to the splendor of a
moon or a star of the sixth magnitude, and go about like a Robin
Goodfellow, peeping in at every cottage window, inspiring lunatics, and
tainting meats, and making darkness visible, instead of steadily
increasing his genial heat and beneficence till he is of such
brightness that no mortal can look him in the face, and then, and in
the mean while too, going about the world in his own orbit, doing it
good, or rather, as a truer philosophy has discovered, the world going
about him getting good. When Phaeton, wishing to prove his heavenly
birth by his beneficence, had the sun’s chariot but one day, and drove
out of the beaten track, he burned several blocks of houses in the
lower streets of heaven, and scorched the surface of the earth, and
dried up every spring, and made the great desert of Sahara, till at
length Jupiter hurled him headlong to the earth with a thunderbolt, and
the sun, through grief at his death, did not shine for a year.

There is no odor so bad as that which arises from goodness tainted. It
is human, it is divine, carrion. If I knew for a certainty that a man
was coming to my house with the conscious design of doing me good, I
should run for my life, as from that dry and parching wind of the
African deserts called the simoom, which fills the mouth and nose and
ears and eyes with dust till you are suffocated, for fear that I should
get some of his good done to me,—some of its virus mingled with my
blood. No,—in this case I would rather suffer evil the natural way. A
man is not a good _man_ to me because he will feed me if I should be
starving, or warm me if I should be freezing, or pull me out of a ditch
if I should ever fall into one. I can find you a Newfoundland dog that
will do as much. Philanthropy is not love for one’s fellow-man in the
broadest sense. Howard was no doubt an exceedingly kind and worthy man
in his way, and has his reward; but, comparatively speaking, what are a
hundred Howards to _us_, if their philanthropy do not help _us_ in our
best estate, when we are most worthy to be helped? I never heard of a
philanthropic meeting in which it was sincerely proposed to do any good
to me, or the like of me.

The Jesuits were quite balked by those Indians who, being burned at the
stake, suggested new modes of torture to their tormentors. Being
superior to physical suffering, it sometimes chanced that they were
superior to any consolation which the missionaries could offer; and the
law to do as you would be done by fell with less persuasiveness on the
ears of those who, for their part, did not care how they were done by,
who loved their enemies after a new fashion, and came very near freely
forgiving them all they did.

Be sure that you give the poor the aid they most need, though it be
your example which leaves them far behind. If you give money, spend
yourself with it, and do not merely abandon it to them. We make curious
mistakes sometimes. Often the poor man is not so cold and hungry as he
is dirty and ragged and gross. It is partly his taste, and not merely
his misfortune. If you give him money, he will perhaps buy more rags
with it. I was wont to pity the clumsy Irish laborers who cut ice on
the pond, in such mean and ragged clothes, while I shivered in my more
tidy and somewhat more fashionable garments, till, one bitter cold day,
one who had slipped into the water came to my house to warm him, and I
saw him strip off three pairs of pants and two pairs of stockings ere
he got down to the skin, though they were dirty and ragged enough, it
is true, and that he could afford to refuse the _extra_ garments which
I offered him, he had so many _intra_ ones. This ducking was the very
thing he needed. Then I began to pity myself, and I saw that it would
be a greater charity to bestow on me a flannel shirt than a whole
slop-shop on him. There are a thousand hacking at the branches of evil
to one who is striking at the root, and it may be that he who bestows
the largest amount of time and money on the needy is doing the most by
his mode of life to produce that misery which he strives in vain to
relieve. It is the pious slave-breeder devoting the proceeds of every
tenth slave to buy a Sunday’s liberty for the rest. Some show their
kindness to the poor by employing them in their kitchens. Would they
not be kinder if they employed themselves there? You boast of spending
a tenth part of your income in charity; maybe you should spend the nine
tenths so, and done with it. Society recovers only a tenth part of the
property then. Is this owing to the generosity of him in whose
possession it is found, or to the remissness of the officers of
justice?

Philanthropy is almost the only virtue which is sufficiently
appreciated by mankind. Nay, it is greatly overrated; and it is our
selfishness which overrates it. A robust poor man, one sunny day here
in Concord, praised a fellow-townsman to me, because, as he said, he
was kind to the poor; meaning himself. The kind uncles and aunts of the
race are more esteemed than its true spiritual fathers and mothers. I
once heard a reverend lecturer on England, a man of learning and
intelligence, after enumerating her scientific, literary, and political
worthies, Shakespeare, Bacon, Cromwell, Milton, Newton, and others,
speak next of her Christian heroes, whom, as if his profession required
it of him, he elevated to a place far above all the rest, as the
greatest of the great. They were Penn, Howard, and Mrs. Fry. Every one
must feel the falsehood and cant of this. The last were not England’s
best men and women; only, perhaps, her best philanthropists.

I would not subtract any thing from the praise that is due to
philanthropy, but merely demand justice for all who by their lives and
works are a blessing to mankind. I do not value chiefly a man’s
uprightness and benevolence, which are, as it were, his stem and
leaves. Those plants of whose greenness withered we make herb tea for
the sick, serve but a humble use, and are most employed by quacks. I
want the flower and fruit of a man; that some fragrance be wafted over
from him to me, and some ripeness flavor our intercourse. His goodness
must not be a partial and transitory act, but a constant superfluity,
which costs him nothing and of which he is unconscious. This is a
charity that hides a multitude of sins. The philanthropist too often
surrounds mankind with the remembrance of his own cast-off griefs as an
atmosphere, and calls it sympathy. We should impart our courage, and
not our despair, our health and ease, and not our disease, and take
care that this does not spread by contagion. From what southern plains
comes up the voice of wailing? Under what latitudes reside the heathen
to whom we would send light? Who is that intemperate and brutal man
whom we would redeem? If any thing ail a man, so that he does not
perform his functions, if he have a pain in his bowels even,—for that
is the seat of sympathy,—he forthwith sets about reforming—the world.
Being a microcosm himself, he discovers, and it is a true discovery,
and he is the man to make it,—that the world has been eating green
apples; to his eyes, in fact, the globe itself is a great green apple,
which there is danger awful to think of that the children of men will
nibble before it is ripe; and straightway his drastic philanthropy
seeks out the Esquimaux and the Patagonian, and embraces the populous
Indian and Chinese villages; and thus, by a few years of philanthropic
activity, the powers in the mean while using him for their own ends, no
doubt, he cures himself of his dyspepsia, the globe acquires a faint
blush on one or both of its cheeks, as if it were beginning to be ripe,
and life loses its crudity and is once more sweet and wholesome to
live. I never dreamed of any enormity greater than I have committed. I
never knew, and never shall know, a worse man than myself.

I believe that what so saddens the reformer is not his sympathy with
his fellows in distress, but, though he be the holiest son of God, is
his private ail. Let this be righted, let the spring come to him, the
morning rise over his couch, and he will forsake his generous
companions without apology. My excuse for not lecturing against the use
of tobacco is, that I never chewed it; that is a penalty which reformed
tobacco-chewers have to pay; though there are things enough I have
chewed, which I could lecture against. If you should ever be betrayed
into any of these philanthropies, do not let your left hand know what
your right hand does, for it is not worth knowing. Rescue the drowning
and tie your shoe-strings. Take your time, and set about some free
labor.

Our manners have been corrupted by communication with the saints. Our
hymn-books resound with a melodious cursing of God and enduring him
forever. One would say that even the prophets and redeemers had rather
consoled the fears than confirmed the hopes of man. There is nowhere
recorded a simple and irrepressible satisfaction with the gift of life,
any memorable praise of God. All health and success does me good,
however far off and withdrawn it may appear; all disease and failure
helps to make me sad and does me evil, however much sympathy it may
have with me or I with it. If, then, we would indeed restore mankind by
truly Indian, botanic, magnetic, or natural means, let us first be as
simple and well as Nature ourselves, dispel the clouds which hang over
our own brows, and take up a little life into our pores. Do not stay to
be an overseer of the poor, but endeavor to become one of the worthies
of the world.

I read in the Gulistan, or Flower Garden, of Sheik Sadi of Shiraz, that
“They asked a wise man, saying; Of the many celebrated trees which the
Most High God has created lofty and umbrageous, they call none azad, or
free, excepting the cypress, which bears no fruit; what mystery is
there in this? He replied; Each has its appropriate produce, and
appointed season, during the continuance of which it is fresh and
blooming, and during their absence dry and withered; to neither of
which states is the cypress exposed, being always flourishing; and of
this nature are the azads, or religious independents.—Fix not thy heart
on that which is transitory; for the Dijlah, or Tigris, will continue
to flow through Bagdad after the race of caliphs is extinct: if thy
hand has plenty, be liberal as the date tree; but if it affords nothing
to give away, be an azad, or free man, like the cypress.”

     COMPLEMENTAL VERSES

     The Pretensions of Poverty

     “Thou dost presume too much, poor needy wretch,
     To claim a station in the firmament
     Because thy humble cottage, or thy tub,
     Nurses some lazy or pedantic virtue
     In the cheap sunshine or by shady springs,
     With roots and pot-herbs; where thy right hand,
     Tearing those humane passions from the mind,
     Upon whose stocks fair blooming virtues flourish,
     Degradeth nature, and benumbeth sense,
     And, Gorgon-like, turns active men to stone.
     We not require the dull society
     Of your necessitated temperance,
     Or that unnatural stupidity
     That knows nor joy nor sorrow; nor your forc’d
     Falsely exalted passive fortitude
     Above the active. This low abject brood,
     That fix their seats in mediocrity,
     Become your servile minds; but we advance
     Such virtues only as admit excess,
     Brave, bounteous acts, regal magnificence,
     All-seeing prudence, magnanimity
     That knows no bound, and that heroic virtue
     For which antiquity hath left no name,
     But patterns only, such as Hercules,
     Achilles, Theseus. Back to thy loath’d cell;
     And when thou seest the new enlightened sphere,
     Study to know but what those worthies were.”
                                    T. CAREW


Where I Lived, and What I Lived For

At a certain season of our life we are accustomed to consider every
spot as the possible site of a house. I have thus surveyed the country
on every side within a dozen miles of where I live. In imagination I
have bought all the farms in succession, for all were to be bought, and
I knew their price. I walked over each farmer’s premises, tasted his
wild apples, discoursed on husbandry with him, took his farm at his
price, at any price, mortgaging it to him in my mind; even put a higher
price on it,—took everything but a deed of it,—took his word for his
deed, for I dearly love to talk,—cultivated it, and him too to some
extent, I trust, and withdrew when I had enjoyed it long enough,
leaving him to carry it on. This experience entitled me to be regarded
as a sort of real-estate broker by my friends. Wherever I sat, there I
might live, and the landscape radiated from me accordingly. What is a
house but a _sedes_, a seat?—better if a country seat. I discovered
many a site for a house not likely to be soon improved, which some
might have thought too far from the village, but to my eyes the village
was too far from it. Well, there I might live, I said; and there I did
live, for an hour, a summer and a winter life; saw how I could let the
years run off, buffet the winter through, and see the spring come in.
The future inhabitants of this region, wherever they may place their
houses, may be sure that they have been anticipated. An afternoon
sufficed to lay out the land into orchard, woodlot, and pasture, and to
decide what fine oaks or pines should be left to stand before the door,
and whence each blasted tree could be seen to the best advantage; and
then I let it lie, fallow perchance, for a man is rich in proportion to
the number of things which he can afford to let alone.

My imagination carried me so far that I even had the refusal of several
farms,—the refusal was all I wanted,—but I never got my fingers burned
by actual possession. The nearest that I came to actual possession was
when I bought the Hollowell place, and had begun to sort my seeds, and
collected materials with which to make a wheelbarrow to carry it on or
off with; but before the owner gave me a deed of it, his wife—every man
has such a wife—changed her mind and wished to keep it, and he offered
me ten dollars to release him. Now, to speak the truth, I had but ten
cents in the world, and it surpassed my arithmetic to tell, if I was
that man who had ten cents, or who had a farm, or ten dollars, or all
together. However, I let him keep the ten dollars and the farm too, for
I had carried it far enough; or rather, to be generous, I sold him the
farm for just what I gave for it, and, as he was not a rich man, made
him a present of ten dollars, and still had my ten cents, and seeds,
and materials for a wheelbarrow left. I found thus that I had been a
rich man without any damage to my poverty. But I retained the
landscape, and I have since annually carried off what it yielded
without a wheelbarrow. With respect to landscapes,—

     “I am monarch of all I _survey_,
     My right there is none to dispute.”

I have frequently seen a poet withdraw, having enjoyed the most
valuable part of a farm, while the crusty farmer supposed that he had
got a few wild apples only. Why, the owner does not know it for many
years when a poet has put his farm in rhyme, the most admirable kind of
invisible fence, has fairly impounded it, milked it, skimmed it, and
got all the cream, and left the farmer only the skimmed milk.

The real attractions of the Hollowell farm, to me, were; its complete
retirement, being, about two miles from the village, half a mile from
the nearest neighbor, and separated from the highway by a broad field;
its bounding on the river, which the owner said protected it by its
fogs from frosts in the spring, though that was nothing to me; the gray
color and ruinous state of the house and barn, and the dilapidated
fences, which put such an interval between me and the last occupant;
the hollow and lichen-covered apple trees, gnawed by rabbits, showing
what kind of neighbors I should have; but above all, the recollection I
had of it from my earliest voyages up the river, when the house was
concealed behind a dense grove of red maples, through which I heard the
house-dog bark. I was in haste to buy it, before the proprietor
finished getting out some rocks, cutting down the hollow apple trees,
and grubbing up some young birches which had sprung up in the pasture,
or, in short, had made any more of his improvements. To enjoy these
advantages I was ready to carry it on; like Atlas, to take the world on
my shoulders,—I never heard what compensation he received for that,—and
do all those things which had no other motive or excuse but that I
might pay for it and be unmolested in my possession of it; for I knew
all the while that it would yield the most abundant crop of the kind I
wanted if I could only afford to let it alone. But it turned out as I
have said.

All that I could say, then, with respect to farming on a large scale,
(I have always cultivated a garden,) was, that I had had my seeds
ready. Many think that seeds improve with age. I have no doubt that
time discriminates between the good and the bad; and when at last I
shall plant, I shall be less likely to be disappointed. But I would say
to my fellows, once for all, As long as possible live free and
uncommitted. It makes but little difference whether you are committed
to a farm or the county jail.

Old Cato, whose “De Re Rusticâ” is my “Cultivator,” says, and the only
translation I have seen makes sheer nonsense of the passage, “When you
think of getting a farm, turn it thus in your mind, not to buy
greedily; nor spare your pains to look at it, and do not think it
enough to go round it once. The oftener you go there the more it will
please you, if it is good.” I think I shall not buy greedily, but go
round and round it as long as I live, and be buried in it first, that
it may please me the more at last.



The present was my next experiment of this kind, which I purpose to
describe more at length; for convenience, putting the experience of two
years into one. As I have said, I do not propose to write an ode to
dejection, but to brag as lustily as chanticleer in the morning,
standing on his roost, if only to wake my neighbors up.

When first I took up my abode in the woods, that is, began to spend my
nights as well as days there, which, by accident, was on Independence
Day, or the Fourth of July, 1845, my house was not finished for winter,
but was merely a defence against the rain, without plastering or
chimney, the walls being of rough, weather-stained boards, with wide
chinks, which made it cool at night. The upright white hewn studs and
freshly planed door and window casings gave it a clean and airy look,
especially in the morning, when its timbers were saturated with dew, so
that I fancied that by noon some sweet gum would exude from them. To my
imagination it retained throughout the day more or less of this auroral
character, reminding me of a certain house on a mountain which I had
visited the year before. This was an airy and unplastered cabin, fit to
entertain a travelling god, and where a goddess might trail her
garments. The winds which passed over my dwelling were such as sweep
over the ridges of mountains, bearing the broken strains, or celestial
parts only, of terrestrial music. The morning wind forever blows, the
poem of creation is uninterrupted; but few are the ears that hear it.
Olympus is but the outside of the earth every where.

The only house I had been the owner of before, if I except a boat, was
a tent, which I used occasionally when making excursions in the summer,
and this is still rolled up in my garret; but the boat, after passing
from hand to hand, has gone down the stream of time. With this more
substantial shelter about me, I had made some progress toward settling
in the world. This frame, so slightly clad, was a sort of
crystallization around me, and reacted on the builder. It was
suggestive somewhat as a picture in outlines. I did not need to go
outdoors to take the air, for the atmosphere within had lost none of
its freshness. It was not so much within doors as behind a door where I
sat, even in the rainiest weather. The Harivansa says, “An abode
without birds is like a meat without seasoning.” Such was not my abode,
for I found myself suddenly neighbor to the birds; not by having
imprisoned one, but having caged myself near them. I was not only
nearer to some of those which commonly frequent the garden and the
orchard, but to those wilder and more thrilling songsters of the forest
which never, or rarely, serenade a villager,—the wood-thrush, the
veery, the scarlet tanager, the field-sparrow, the whippoorwill, and
many others.

I was seated by the shore of a small pond, about a mile and a half
south of the village of Concord and somewhat higher than it, in the
midst of an extensive wood between that town and Lincoln, and about two
miles south of that our only field known to fame, Concord Battle
Ground; but I was so low in the woods that the opposite shore, half a
mile off, like the rest, covered with wood, was my most distant
horizon. For the first week, whenever I looked out on the pond it
impressed me like a tarn high up on the side of a mountain, its bottom
far above the surface of other lakes, and, as the sun arose, I saw it
throwing off its nightly clothing of mist, and here and there, by
degrees, its soft ripples or its smooth reflecting surface was
revealed, while the mists, like ghosts, were stealthily withdrawing in
every direction into the woods, as at the breaking up of some nocturnal
conventicle. The very dew seemed to hang upon the trees later into the
day than usual, as on the sides of mountains.

This small lake was of most value as a neighbor in the intervals of a
gentle rain storm in August, when, both air and water being perfectly
still, but the sky overcast, mid-afternoon had all the serenity of
evening, and the wood-thrush sang around, and was heard from shore to
shore. A lake like this is never smoother than at such a time; and the
clear portion of the air above it being shallow and darkened by clouds,
the water, full of light and reflections, becomes a lower heaven itself
so much the more important. From a hill top near by, where the wood had
been recently cut off, there was a pleasing vista southward across the
pond, through a wide indentation in the hills which form the shore
there, where their opposite sides sloping toward each other suggested a
stream flowing out in that direction through a wooded valley, but
stream there was none. That way I looked between and over the near
green hills to some distant and higher ones in the horizon, tinged with
blue. Indeed, by standing on tiptoe I could catch a glimpse of some of
the peaks of the still bluer and more distant mountain ranges in the
north-west, those true-blue coins from heaven’s own mint, and also of
some portion of the village. But in other directions, even from this
point, I could not see over or beyond the woods which surrounded me. It
is well to have some water in your neighborhood, to give buoyancy to
and float the earth. One value even of the smallest well is, that when
you look into it you see that earth is not continent but insular. This
is as important as that it keeps butter cool. When I looked across the
pond from this peak toward the Sudbury meadows, which in time of flood
I distinguished elevated perhaps by a mirage in their seething valley,
like a coin in a basin, all the earth beyond the pond appeared like a
thin crust insulated and floated even by this small sheet of
interverting water, and I was reminded that this on which I dwelt was
but _dry land_.

Though the view from my door was still more contracted, I did not feel
crowded or confined in the least. There was pasture enough for my
imagination. The low shrub-oak plateau to which the opposite shore
arose, stretched away toward the prairies of the West and the steppes
of Tartary, affording ample room for all the roving families of men.
“There are none happy in the world but beings who enjoy freely a vast
horizon,”—said Damodara, when his herds required new and larger
pastures.

Both place and time were changed, and I dwelt nearer to those parts of
the universe and to those eras in history which had most attracted me.
Where I lived was as far off as many a region viewed nightly by
astronomers. We are wont to imagine rare and delectable places in some
remote and more celestial corner of the system, behind the
constellation of Cassiopeia’s Chair, far from noise and disturbance. I
discovered that my house actually had its site in such a withdrawn, but
forever new and unprofaned, part of the universe. If it were worth the
while to settle in those parts near to the Pleiades or the Hyades, to
Aldebaran or Altair, then I was really there, or at an equal remoteness
from the life which I had left behind, dwindled and twinkling with as
fine a ray to my nearest neighbor, and to be seen only in moonless
nights by him. Such was that part of creation where I had squatted;—

     “There was a shepherd that did live,
         And held his thoughts as high
     As were the mounts whereon his flocks
         Did hourly feed him by.”

What should we think of the shepherd’s life if his flocks always
wandered to higher pastures than his thoughts?

Every morning was a cheerful invitation to make my life of equal
simplicity, and I may say innocence, with Nature herself. I have been
as sincere a worshipper of Aurora as the Greeks. I got up early and
bathed in the pond; that was a religious exercise, and one of the best
things which I did. They say that characters were engraven on the
bathing tub of king Tching-thang to this effect: “Renew thyself
completely each day; do it again, and again, and forever again.” I can
understand that. Morning brings back the heroic ages. I was as much
affected by the faint hum of a mosquito making its invisible and
unimaginable tour through my apartment at earliest dawn, when I was
sitting with door and windows open, as I could be by any trumpet that
ever sang of fame. It was Homer’s requiem; itself an Iliad and Odyssey
in the air, singing its own wrath and wanderings. There was something
cosmical about it; a standing advertisement, till forbidden, of the
everlasting vigor and fertility of the world. The morning, which is the
most memorable season of the day, is the awakening hour. Then there is
least somnolence in us; and for an hour, at least, some part of us
awakes which slumbers all the rest of the day and night. Little is to
be expected of that day, if it can be called a day, to which we are not
awakened by our Genius, but by the mechanical nudgings of some
servitor, are not awakened by our own newly-acquired force and
aspirations from within, accompanied by the undulations of celestial
music, instead of factory bells, and a fragrance filling the air—to a
higher life than we fell asleep from; and thus the darkness bear its
fruit, and prove itself to be good, no less than the light. That man
who does not believe that each day contains an earlier, more sacred,
and auroral hour than he has yet profaned, has despaired of life, and
is pursuing a descending and darkening way. After a partial cessation
of his sensuous life, the soul of man, or its organs rather, are
reinvigorated each day, and his Genius tries again what noble life it
can make. All memorable events, I should say, transpire in morning time
and in a morning atmosphere. The Vedas say, “All intelligences awake
with the morning.” Poetry and art, and the fairest and most memorable
of the actions of men, date from such an hour. All poets and heroes,
like Memnon, are the children of Aurora, and emit their music at
sunrise. To him whose elastic and vigorous thought keeps pace with the
sun, the day is a perpetual morning. It matters not what the clocks say
or the attitudes and labors of men. Morning is when I am awake and
there is a dawn in me. Moral reform is the effort to throw off sleep.
Why is it that men give so poor an account of their day if they have
not been slumbering? They are not such poor calculators. If they had
not been overcome with drowsiness, they would have performed something.
The millions are awake enough for physical labor; but only one in a
million is awake enough for effective intellectual exertion, only one
in a hundred millions to a poetic or divine life. To be awake is to be
alive. I have never yet met a man who was quite awake. How could I have
looked him in the face?

We must learn to reawaken and keep ourselves awake, not by mechanical
aids, but by an infinite expectation of the dawn, which does not
forsake us in our soundest sleep. I know of no more encouraging fact
than the unquestionable ability of man to elevate his life by a
conscious endeavor. It is something to be able to paint a particular
picture, or to carve a statue, and so to make a few objects beautiful;
but it is far more glorious to carve and paint the very atmosphere and
medium through which we look, which morally we can do. To affect the
quality of the day, that is the highest of arts. Every man is tasked to
make his life, even in its details, worthy of the contemplation of his
most elevated and critical hour. If we refused, or rather used up, such
paltry information as we get, the oracles would distinctly inform us
how this might be done.

I went to the woods because I wished to live deliberately, to front
only the essential facts of life, and see if I could not learn what it
had to teach, and not, when I came to die, discover that I had not
lived. I did not wish to live what was not life, living is so dear; nor
did I wish to practise resignation, unless it was quite necessary. I
wanted to live deep and suck out all the marrow of life, to live so
sturdily and Spartan-like as to put to rout all that was not life, to
cut a broad swath and shave close, to drive life into a corner, and
reduce it to its lowest terms, and, if it proved to be mean, why then
to get the whole and genuine meanness of it, and publish its meanness
to the world; or if it were sublime, to know it by experience, and be
able to give a true account of it in my next excursion. For most men,
it appears to me, are in a strange uncertainty about it, whether it is
of the devil or of God, and have _somewhat hastily_ concluded that it
is the chief end of man here to “glorify God and enjoy him forever.”

Still we live meanly, like ants; though the fable tells us that we were
long ago changed into men; like pygmies we fight with cranes; it is
error upon error, and clout upon clout, and our best virtue has for its
occasion a superfluous and evitable wretchedness. Our life is frittered
away by detail. An honest man has hardly need to count more than his
ten fingers, or in extreme cases he may add his ten toes, and lump the
rest. Simplicity, simplicity, simplicity! I say, let your affairs be as
two or three, and not a hundred or a thousand; instead of a million
count half a dozen, and keep your accounts on your thumb nail. In the
midst of this chopping sea of civilized life, such are the clouds and
storms and quicksands and thousand-and-one items to be allowed for,
that a man has to live, if he would not founder and go to the bottom
and not make his port at all, by dead reckoning, and he must be a great
calculator indeed who succeeds. Simplify, simplify. Instead of three
meals a day, if it be necessary eat but one; instead of a hundred
dishes, five; and reduce other things in proportion. Our life is like a
German Confederacy, made up of petty states, with its boundary forever
fluctuating, so that even a German cannot tell you how it is bounded at
any moment. The nation itself, with all its so called internal
improvements, which, by the way are all external and superficial, is
just such an unwieldy and overgrown establishment, cluttered with
furniture and tripped up by its own traps, ruined by luxury and
heedless expense, by want of calculation and a worthy aim, as the
million households in the land; and the only cure for it as for them is
in a rigid economy, a stern and more than Spartan simplicity of life
and elevation of purpose. It lives too fast. Men think that it is
essential that the _Nation_ have commerce, and export ice, and talk
through a telegraph, and ride thirty miles an hour, without a doubt,
whether _they_ do or not; but whether we should live like baboons or
like men, is a little uncertain. If we do not get out sleepers, and
forge rails, and devote days and nights to the work, but go to
tinkering upon our _lives_ to improve _them_, who will build railroads?
And if railroads are not built, how shall we get to heaven in season?
But if we stay at home and mind our business, who will want railroads?
We do not ride on the railroad; it rides upon us. Did you ever think
what those sleepers are that underlie the railroad? Each one is a man,
an Irish-man, or a Yankee man. The rails are laid on them, and they are
covered with sand, and the cars run smoothly over them. They are sound
sleepers, I assure you. And every few years a new lot is laid down and
run over; so that, if some have the pleasure of riding on a rail,
others have the misfortune to be ridden upon. And when they run over a
man that is walking in his sleep, a supernumerary sleeper in the wrong
position, and wake him up, they suddenly stop the cars, and make a hue
and cry about it, as if this were an exception. I am glad to know that
it takes a gang of men for every five miles to keep the sleepers down
and level in their beds as it is, for this is a sign that they may
sometime get up again.

Why should we live with such hurry and waste of life? We are determined
to be starved before we are hungry. Men say that a stitch in time saves
nine, and so they take a thousand stitches to-day to save nine
to-morrow. As for _work_, we haven’t any of any consequence. We have
the Saint Vitus’ dance, and cannot possibly keep our heads still. If I
should only give a few pulls at the parish bell-rope, as for a fire,
that is, without setting the bell, there is hardly a man on his farm in
the outskirts of Concord, notwithstanding that press of engagements
which was his excuse so many times this morning, nor a boy, nor a
woman, I might almost say, but would forsake all and follow that sound,
not mainly to save property from the flames, but, if we will confess
the truth, much more to see it burn, since burn it must, and we, be it
known, did not set it on fire,—or to see it put out, and have a hand in
it, if that is done as handsomely; yes, even if it were the parish
church itself. Hardly a man takes a half hour’s nap after dinner, but
when he wakes he holds up his head and asks, “What’s the news?” as if
the rest of mankind had stood his sentinels. Some give directions to be
waked every half hour, doubtless for no other purpose; and then, to pay
for it, they tell what they have dreamed. After a night’s sleep the
news is as indispensable as the breakfast. “Pray tell me any thing new
that has happened to a man any where on this globe,”—and he reads it
over his coffee and rolls, that a man has had his eyes gouged out this
morning on the Wachito River; never dreaming the while that he lives in
the dark unfathomed mammoth cave of this world, and has but the
rudiment of an eye himself.

For my part, I could easily do without the post-office. I think that
there are very few important communications made through it. To speak
critically, I never received more than one or two letters in my life—I
wrote this some years ago—that were worth the postage. The penny-post
is, commonly, an institution through which you seriously offer a man
that penny for his thoughts which is so often safely offered in jest.
And I am sure that I never read any memorable news in a newspaper. If
we read of one man robbed, or murdered, or killed by accident, or one
house burned, or one vessel wrecked, or one steamboat blown up, or one
cow run over on the Western Railroad, or one mad dog killed, or one lot
of grasshoppers in the winter,—we never need read of another. One is
enough. If you are acquainted with the principle, what do you care for
a myriad instances and applications? To a philosopher all _news_, as it
is called, is gossip, and they who edit and read it are old women over
their tea. Yet not a few are greedy after this gossip. There was such a
rush, as I hear, the other day at one of the offices to learn the
foreign news by the last arrival, that several large squares of plate
glass belonging to the establishment were broken by the pressure,—news
which I seriously think a ready wit might write a twelve-month, or
twelve years, beforehand with sufficient accuracy. As for Spain, for
instance, if you know how to throw in Don Carlos and the Infanta, and
Don Pedro and Seville and Granada, from time to time in the right
proportions,—they may have changed the names a little since I saw the
papers,—and serve up a bull-fight when other entertainments fail, it
will be true to the letter, and give us as good an idea of the exact
state or ruin of things in Spain as the most succinct and lucid reports
under this head in the newspapers: and as for England, almost the last
significant scrap of news from that quarter was the revolution of 1649;
and if you have learned the history of her crops for an average year,
you never need attend to that thing again, unless your speculations are
of a merely pecuniary character. If one may judge who rarely looks into
the newspapers, nothing new does ever happen in foreign parts, a French
revolution not excepted.

What news! how much more important to know what that is which was never
old! “Kieou-he-yu (great dignitary of the state of Wei) sent a man to
Khoung-tseu to know his news. Khoung-tseu caused the messenger to be
seated near him, and questioned him in these terms: What is your master
doing? The messenger answered with respect: My master desires to
diminish the number of his faults, but he cannot come to the end of
them. The messenger being gone, the philosopher remarked: What a worthy
messenger! What a worthy messenger!” The preacher, instead of vexing
the ears of drowsy farmers on their day of rest at the end of the
week,—for Sunday is the fit conclusion of an ill-spent week, and not
the fresh and brave beginning of a new one,—with this one other
draggle-tail of a sermon, should shout with thundering voice, “Pause!
Avast! Why so seeming fast, but deadly slow?”

Shams and delusions are esteemed for soundest truths, while reality is
fabulous. If men would steadily observe realities only, and not allow
themselves to be deluded, life, to compare it with such things as we
know, would be like a fairy tale and the Arabian Nights’
Entertainments. If we respected only what is inevitable and has a right
to be, music and poetry would resound along the streets. When we are
unhurried and wise, we perceive that only great and worthy things have
any permanent and absolute existence,—that petty fears and petty
pleasures are but the shadow of the reality. This is always
exhilarating and sublime. By closing the eyes and slumbering, and
consenting to be deceived by shows, men establish and confirm their
daily life of routine and habit everywhere, which still is built on
purely illusory foundations. Children, who play life, discern its true
law and relations more clearly than men, who fail to live it worthily,
but who think that they are wiser by experience, that is, by failure. I
have read in a Hindoo book, that “there was a king’s son, who, being
expelled in infancy from his native city, was brought up by a forester,
and, growing up to maturity in that state, imagined himself to belong
to the barbarous race with which he lived. One of his father’s
ministers having discovered him, revealed to him what he was, and the
misconception of his character was removed, and he knew himself to be a
prince. So soul,” continues the Hindoo philosopher, “from the
circumstances in which it is placed, mistakes its own character, until
the truth is revealed to it by some holy teacher, and then it knows
itself to be _Brahme_.” I perceive that we inhabitants of New England
live this mean life that we do because our vision does not penetrate
the surface of things. We think that that _is_ which _appears_ to be.
If a man should walk through this town and see only the reality, where,
think you, would the “Mill-dam” go to? If he should give us an account
of the realities he beheld there, we should not recognize the place in
his description. Look at a meeting-house, or a court-house, or a jail,
or a shop, or a dwelling-house, and say what that thing really is
before a true gaze, and they would all go to pieces in your account of
them. Men esteem truth remote, in the outskirts of the system, behind
the farthest star, before Adam and after the last man. In eternity
there is indeed something true and sublime. But all these times and
places and occasions are now and here. God himself culminates in the
present moment, and will never be more divine in the lapse of all the
ages. And we are enabled to apprehend at all what is sublime and noble
only by the perpetual instilling and drenching of the reality that
surrounds us. The universe constantly and obediently answers to our
conceptions; whether we travel fast or slow, the track is laid for us.
Let us spend our lives in conceiving then. The poet or the artist never
yet had so fair and noble a design but some of his posterity at least
could accomplish it.

Let us spend one day as deliberately as Nature, and not be thrown off
the track by every nutshell and mosquito’s wing that falls on the
rails. Let us rise early and fast, or break fast, gently and without
perturbation; let company come and let company go, let the bells ring
and the children cry,—determined to make a day of it. Why should we
knock under and go with the stream? Let us not be upset and overwhelmed
in that terrible rapid and whirlpool called a dinner, situated in the
meridian shallows. Weather this danger and you are safe, for the rest
of the way is down hill. With unrelaxed nerves, with morning vigor,
sail by it, looking another way, tied to the mast like Ulysses. If the
engine whistles, let it whistle till it is hoarse for its pains. If the
bell rings, why should we run? We will consider what kind of music they
are like. Let us settle ourselves, and work and wedge our feet downward
through the mud and slush of opinion, and prejudice, and tradition, and
delusion, and appearance, that alluvion which covers the globe, through
Paris and London, through New York and Boston and Concord, through
church and state, through poetry and philosophy and religion, till we
come to a hard bottom and rocks in place, which we can call _reality_,
and say, This is, and no mistake; and then begin, having a _point
d’appui_, below freshet and frost and fire, a place where you might
found a wall or a state, or set a lamp-post safely, or perhaps a gauge,
not a Nilometer, but a Realometer, that future ages might know how deep
a freshet of shams and appearances had gathered from time to time. If
you stand right fronting and face to face to a fact, you will see the
sun glimmer on both its surfaces, as if it were a cimeter, and feel its
sweet edge dividing you through the heart and marrow, and so you will
happily conclude your mortal career. Be it life or death, we crave only
reality. If we are really dying, let us hear the rattle in our throats
and feel cold in the extremities; if we are alive, let us go about our
business.

Time is but the stream I go a-fishing in. I drink at it; but while I
drink I see the sandy bottom and detect how shallow it is. Its thin
current slides away, but eternity remains. I would drink deeper; fish
in the sky, whose bottom is pebbly with stars. I cannot count one. I
know not the first letter of the alphabet. I have always been
regretting that I was not as wise as the day I was born. The intellect
is a cleaver; it discerns and rifts its way into the secret of things.
I do not wish to be any more busy with my hands than is necessary. My
head is hands and feet. I feel all my best faculties concentrated in
it. My instinct tells me that my head is an organ for burrowing, as
some creatures use their snout and fore-paws, and with it I would mine
and burrow my way through these hills. I think that the richest vein is
somewhere hereabouts; so by the divining-rod and thin rising vapors I
judge; and here I will begin to mine.


Reading

With a little more deliberation in the choice of their pursuits, all
men would perhaps become essentially students and observers, for
certainly their nature and destiny are interesting to all alike. In
accumulating property for ourselves or our posterity, in founding a
family or a state, or acquiring fame even, we are mortal; but in
dealing with truth we are immortal, and need fear no change nor
accident. The oldest Egyptian or Hindoo philosopher raised a corner of
the veil from the statue of the divinity; and still the trembling robe
remains raised, and I gaze upon as fresh a glory as he did, since it
was I in him that was then so bold, and it is he in me that now reviews
the vision. No dust has settled on that robe; no time has elapsed since
that divinity was revealed. That time which we really improve, or which
is improvable, is neither past, present, nor future.

My residence was more favorable, not only to thought, but to serious
reading, than a university; and though I was beyond the range of the
ordinary circulating library, I had more than ever come within the
influence of those books which circulate round the world, whose
sentences were first written on bark, and are now merely copied from
time to time on to linen paper. Says the poet Mîr Camar Uddîn Mast,
“Being seated to run through the region of the spiritual world; I have
had this advantage in books. To be intoxicated by a single glass of
wine; I have experienced this pleasure when I have drunk the liquor of
the esoteric doctrines.” I kept Homer’s Iliad on my table through the
summer, though I looked at his page only now and then. Incessant labor
with my hands, at first, for I had my house to finish and my beans to
hoe at the same time, made more study impossible. Yet I sustained
myself by the prospect of such reading in future. I read one or two
shallow books of travel in the intervals of my work, till that
employment made me ashamed of myself, and I asked where it was then
that _I_ lived.

The student may read Homer or Æschylus in the Greek without danger of
dissipation or luxuriousness, for it implies that he in some measure
emulate their heroes, and consecrate morning hours to their pages. The
heroic books, even if printed in the character of our mother tongue,
will always be in a language dead to degenerate times; and we must
laboriously seek the meaning of each word and line, conjecturing a
larger sense than common use permits out of what wisdom and valor and
generosity we have. The modern cheap and fertile press, with all its
translations, has done little to bring us nearer to the heroic writers
of antiquity. They seem as solitary, and the letter in which they are
printed as rare and curious, as ever. It is worth the expense of
youthful days and costly hours, if you learn only some words of an
ancient language, which are raised out of the trivialness of the
street, to be perpetual suggestions and provocations. It is not in vain
that the farmer remembers and repeats the few Latin words which he has
heard. Men sometimes speak as if the study of the classics would at
length make way for more modern and practical studies; but the
adventurous student will always study classics, in whatever language
they may be written and however ancient they may be. For what are the
classics but the noblest recorded thoughts of man? They are the only
oracles which are not decayed, and there are such answers to the most
modern inquiry in them as Delphi and Dodona never gave. We might as
well omit to study Nature because she is old. To read well, that is, to
read true books in a true spirit, is a noble exercise, and one that
will task the reader more than any exercise which the customs of the
day esteem. It requires a training such as the athletes underwent, the
steady intention almost of the whole life to this object. Books must be
read as deliberately and reservedly as they were written. It is not
enough even to be able to speak the language of that nation by which
they are written, for there is a memorable interval between the spoken
and the written language, the language heard and the language read. The
one is commonly transitory, a sound, a tongue, a dialect merely, almost
brutish, and we learn it unconsciously, like the brutes, of our
mothers. The other is the maturity and experience of that; if that is
our mother tongue, this is our father tongue, a reserved and select
expression, too significant to be heard by the ear, which we must be
born again in order to speak. The crowds of men who merely _spoke_ the
Greek and Latin tongues in the middle ages were not entitled by the
accident of birth to _read_ the works of genius written in those
languages; for these were not written in that Greek or Latin which they
knew, but in the select language of literature. They had not learned
the nobler dialects of Greece and Rome, but the very materials on which
they were written were waste paper to them, and they prized instead a
cheap contemporary literature. But when the several nations of Europe
had acquired distinct though rude written languages of their own,
sufficient for the purposes of their rising literatures, then first
learning revived, and scholars were enabled to discern from that
remoteness the treasures of antiquity. What the Roman and Grecian
multitude could not _hear_, after the lapse of ages a few scholars
_read_, and a few scholars only are still reading it.

However much we may admire the orator’s occasional bursts of eloquence,
the noblest written words are commonly as far behind or above the
fleeting spoken language as the firmament with its stars is behind the
clouds. _There_ are the stars, and they who can may read them. The
astronomers forever comment on and observe them. They are not
exhalations like our daily colloquies and vaporous breath. What is
called eloquence in the forum is commonly found to be rhetoric in the
study. The orator yields to the inspiration of a transient occasion,
and speaks to the mob before him, to those who can _hear_ him; but the
writer, whose more equable life is his occasion, and who would be
distracted by the event and the crowd which inspire the orator, speaks
to the intellect and health of mankind, to all in any age who can
_understand_ him.

No wonder that Alexander carried the Iliad with him on his expeditions
in a precious casket. A written word is the choicest of relics. It is
something at once more intimate with us and more universal than any
other work of art. It is the work of art nearest to life itself. It may
be translated into every language, and not only be read but actually
breathed from all human lips;—not be represented on canvas or in marble
only, but be carved out of the breath of life itself. The symbol of an
ancient man’s thought becomes a modern man’s speech. Two thousand
summers have imparted to the monuments of Grecian literature, as to her
marbles, only a maturer golden and autumnal tint, for they have carried
their own serene and celestial atmosphere into all lands to protect
them against the corrosion of time. Books are the treasured wealth of
the world and the fit inheritance of generations and nations. Books,
the oldest and the best, stand naturally and rightfully on the shelves
of every cottage. They have no cause of their own to plead, but while
they enlighten and sustain the reader his common sense will not refuse
them. Their authors are a natural and irresistible aristocracy in every
society, and, more than kings or emperors, exert an influence on
mankind. When the illiterate and perhaps scornful trader has earned by
enterprise and industry his coveted leisure and independence, and is
admitted to the circles of wealth and fashion, he turns inevitably at
last to those still higher but yet inaccessible circles of intellect
and genius, and is sensible only of the imperfection of his culture and
the vanity and insufficiency of all his riches, and further proves his
good sense by the pains which he takes to secure for his children that
intellectual culture whose want he so keenly feels; and thus it is that
he becomes the founder of a family.

Those who have not learned to read the ancient classics in the language
in which they were written must have a very imperfect knowledge of the
history of the human race; for it is remarkable that no transcript of
them has ever been made into any modern tongue, unless our civilization
itself may be regarded as such a transcript. Homer has never yet been
printed in English, nor Æschylus, nor Virgil even—works as refined, as
solidly done, and as beautiful almost as the morning itself; for later
writers, say what we will of their genius, have rarely, if ever,
equalled the elaborate beauty and finish and the lifelong and heroic
literary labors of the ancients. They only talk of forgetting them who
never knew them. It will be soon enough to forget them when we have the
learning and the genius which will enable us to attend to and
appreciate them. That age will be rich indeed when those relics which
we call Classics, and the still older and more than classic but even
less known Scriptures of the nations, shall have still further
accumulated, when the Vaticans shall be filled with Vedas and
Zendavestas and Bibles, with Homers and Dantes and Shakespeares, and
all the centuries to come shall have successively deposited their
trophies in the forum of the world. By such a pile we may hope to scale
heaven at last.

The works of the great poets have never yet been read by mankind, for
only great poets can read them. They have only been read as the
multitude read the stars, at most astrologically, not astronomically.
Most men have learned to read to serve a paltry convenience, as they
have learned to cipher in order to keep accounts and not be cheated in
trade; but of reading as a noble intellectual exercise they know little
or nothing; yet this only is reading, in a high sense, not that which
lulls us as a luxury and suffers the nobler faculties to sleep the
while, but what we have to stand on tip-toe to read and devote our most
alert and wakeful hours to.

I think that having learned our letters we should read the best that is
in literature, and not be forever repeating our a b abs, and words of
one syllable, in the fourth or fifth classes, sitting on the lowest and
foremost form all our lives. Most men are satisfied if they read or
hear read, and perchance have been convicted by the wisdom of one good
book, the Bible, and for the rest of their lives vegetate and dissipate
their faculties in what is called easy reading. There is a work in
several volumes in our Circulating Library entitled Little Reading,
which I thought referred to a town of that name which I had not been
to. There are those who, like cormorants and ostriches, can digest all
sorts of this, even after the fullest dinner of meats and vegetables,
for they suffer nothing to be wasted. If others are the machines to
provide this provender, they are the machines to read it. They read the
nine thousandth tale about Zebulon and Sephronia, and how they loved as
none had ever loved before, and neither did the course of their true
love run smooth,—at any rate, how it did run and stumble, and get up
again and go on! how some poor unfortunate got up on to a steeple, who
had better never have gone up as far as the belfry; and then, having
needlessly got him up there, the happy novelist rings the bell for all
the world to come together and hear, O dear! how he did get down again!
For my part, I think that they had better metamorphose all such
aspiring heroes of universal noveldom into man weathercocks, as they
used to put heroes among the constellations, and let them swing round
there till they are rusty, and not come down at all to bother honest
men with their pranks. The next time the novelist rings the bell I will
not stir though the meeting-house burn down. “The Skip of the
Tip-Toe-Hop, a Romance of the Middle Ages, by the celebrated author of
‘Tittle-Tol-Tan,’ to appear in monthly parts; a great rush; don’t all
come together.” All this they read with saucer eyes, and erect and
primitive curiosity, and with unwearied gizzard, whose corrugations
even yet need no sharpening, just as some little four-year-old bencher
his two-cent gilt-covered edition of Cinderella,—without any
improvement, that I can see, in the pronunciation, or accent, or
emphasis, or any more skill in extracting or inserting the moral. The
result is dulness of sight, a stagnation of the vital circulations, and
a general deliquium and sloughing off of all the intellectual
faculties. This sort of gingerbread is baked daily and more sedulously
than pure wheat or rye-and-Indian in almost every oven, and finds a
surer market.

The best books are not read even by those who are called good readers.
What does our Concord culture amount to? There is in this town, with a
very few exceptions, no taste for the best or for very good books even
in English literature, whose words all can read and spell. Even the
college-bred and so called liberally educated men here and elsewhere
have really little or no acquaintance with the English classics; and as
for the recorded wisdom of mankind, the ancient classics and Bibles,
which are accessible to all who will know of them, there are the
feeblest efforts any where made to become acquainted with them. I know
a woodchopper, of middle age, who takes a French paper, not for news as
he says, for he is above that, but to “keep himself in practice,” he
being a Canadian by birth; and when I ask him what he considers the
best thing he can do in this world, he says, beside this, to keep up
and add to his English. This is about as much as the college bred
generally do or aspire to do, and they take an English paper for the
purpose. One who has just come from reading perhaps one of the best
English books will find how many with whom he can converse about it? Or
suppose he comes from reading a Greek or Latin classic in the original,
whose praises are familiar even to the so called illiterate; he will
find nobody at all to speak to, but must keep silence about it. Indeed,
there is hardly the professor in our colleges, who, if he has mastered
the difficulties of the language, has proportionally mastered the
difficulties of the wit and poetry of a Greek poet, and has any
sympathy to impart to the alert and heroic reader; and as for the
sacred Scriptures, or Bibles of mankind, who in this town can tell me
even their titles? Most men do not know that any nation but the Hebrews
have had a scripture. A man, any man, will go considerably out of his
way to pick up a silver dollar; but here are golden words, which the
wisest men of antiquity have uttered, and whose worth the wise of every
succeeding age have assured us of;—and yet we learn to read only as far
as Easy Reading, the primers and class-books, and when we leave school,
the “Little Reading,” and story books, which are for boys and
beginners; and our reading, our conversation and thinking, are all on a
very low level, worthy only of pygmies and manikins.

I aspire to be acquainted with wiser men than this our Concord soil has
produced, whose names are hardly known here. Or shall I hear the name
of Plato and never read his book? As if Plato were my townsman and I
never saw him,—my next neighbor and I never heard him speak or attended
to the wisdom of his words. But how actually is it? His Dialogues,
which contain what was immortal in him, lie on the next shelf, and yet
I never read them. We are underbred and low-lived and illiterate; and
in this respect I confess I do not make any very broad distinction
between the illiterateness of my townsman who cannot read at all, and
the illiterateness of him who has learned to read only what is for
children and feeble intellects. We should be as good as the worthies of
antiquity, but partly by first knowing how good they were. We are a
race of tit-men, and soar but little higher in our intellectual flights
than the columns of the daily paper.

It is not all books that are as dull as their readers. There are
probably words addressed to our condition exactly, which, if we could
really hear and understand, would be more salutary than the morning or
the spring to our lives, and possibly put a new aspect on the face of
things for us. How many a man has dated a new era in his life from the
reading of a book. The book exists for us perchance which will explain
our miracles and reveal new ones. The at present unutterable things we
may find somewhere uttered. These same questions that disturb and
puzzle and confound us have in their turn occurred to all the wise men;
not one has been omitted; and each has answered them, according to his
ability, by his words and his life. Moreover, with wisdom we shall
learn liberality. The solitary hired man on a farm in the outskirts of
Concord, who has had his second birth and peculiar religious
experience, and is driven as he believes into the silent gravity and
exclusiveness by his faith, may think it is not true; but Zoroaster,
thousands of years ago, travelled the same road and had the same
experience; but he, being wise, knew it to be universal, and treated
his neighbors accordingly, and is even said to have invented and
established worship among men. Let him humbly commune with Zoroaster
then, and through the liberalizing influence of all the worthies, with
Jesus Christ himself, and let “our church” go by the board.

We boast that we belong to the nineteenth century and are making the
most rapid strides of any nation. But consider how little this village
does for its own culture. I do not wish to flatter my townsmen, nor to
be flattered by them, for that will not advance either of us. We need
to be provoked,—goaded like oxen, as we are, into a trot. We have a
comparatively decent system of common schools, schools for infants
only; but excepting the half-starved Lyceum in the winter, and latterly
the puny beginning of a library suggested by the state, no school for
ourselves. We spend more on almost any article of bodily aliment or
ailment than on our mental aliment. It is time that we had uncommon
schools, that we did not leave off our education when we begin to be
men and women. It is time that villages were universities, and their
elder inhabitants the fellows of universities, with leisure—if they are
indeed so well off—to pursue liberal studies the rest of their lives.
Shall the world be confined to one Paris or one Oxford forever? Cannot
students be boarded here and get a liberal education under the skies of
Concord? Can we not hire some Abelard to lecture to us? Alas! what with
foddering the cattle and tending the store, we are kept from school too
long, and our education is sadly neglected. In this country, the
village should in some respects take the place of the nobleman of
Europe. It should be the patron of the fine arts. It is rich enough. It
wants only the magnanimity and refinement. It can spend money enough on
such things as farmers and traders value, but it is thought Utopian to
propose spending money for things which more intelligent men know to be
of far more worth. This town has spent seventeen thousand dollars on a
town-house, thank fortune or politics, but probably it will not spend
so much on living wit, the true meat to put into that shell, in a
hundred years. The one hundred and twenty-five dollars annually
subscribed for a Lyceum in the winter is better spent than any other
equal sum raised in the town. If we live in the nineteenth century, why
should we not enjoy the advantages which the nineteenth century offers?
Why should our life be in any respect provincial? If we will read
newspapers, why not skip the gossip of Boston and take the best
newspaper in the world at once?—not be sucking the pap of “neutral
family” papers, or browsing “Olive-Branches” here in New England. Let
the reports of all the learned societies come to us, and we will see if
they know any thing. Why should we leave it to Harper & Brothers and
Redding & Co. to select our reading? As the nobleman of cultivated
taste surrounds himself with whatever conduces to his
culture,—genius—learning—wit—books—paintings—statuary—music—
philosophical instruments, and the like; so let the village do,—not
stop short at a pedagogue, a parson, a sexton, a parish library, and
three selectmen, because our pilgrim forefathers got through a cold
winter once on a bleak rock with these. To act collectively is
according to the spirit of our institutions; and I am confident that,
as our circumstances are more flourishing, our means are greater than
the nobleman’s. New England can hire all the wise men in the world to
come and teach her, and board them round the while, and not be
provincial at all. That is the _uncommon_ school we want. Instead of
noblemen, let us have noble villages of men. If it is necessary, omit
one bridge over the river, go round a little there, and throw one arch
at least over the darker gulf of ignorance which surrounds us.


Sounds

But while we are confined to books, though the most select and classic,
and read only particular written languages, which are themselves but
dialects and provincial, we are in danger of forgetting the language
which all things and events speak without metaphor, which alone is
copious and standard. Much is published, but little printed. The rays
which stream through the shutter will be no longer remembered when the
shutter is wholly removed. No method nor discipline can supersede the
necessity of being forever on the alert. What is a course of history,
or philosophy, or poetry, no matter how well selected, or the best
society, or the most admirable routine of life, compared with the
discipline of looking always at what is to be seen? Will you be a
reader, a student merely, or a seer? Read your fate, see what is before
you, and walk on into futurity.

I did not read books the first summer; I hoed beans. Nay, I often did
better than this. There were times when I could not afford to sacrifice
the bloom of the present moment to any work, whether of the head or
hands. I love a broad margin to my life. Sometimes, in a summer
morning, having taken my accustomed bath, I sat in my sunny doorway
from sunrise till noon, rapt in a revery, amidst the pines and
hickories and sumachs, in undisturbed solitude and stillness, while the
birds sing around or flitted noiseless through the house, until by the
sun falling in at my west window, or the noise of some traveller’s
wagon on the distant highway, I was reminded of the lapse of time. I
grew in those seasons like corn in the night, and they were far better
than any work of the hands would have been. They were not time
subtracted from my life, but so much over and above my usual allowance.
I realized what the Orientals mean by contemplation and the forsaking
of works. For the most part, I minded not how the hours went. The day
advanced as if to light some work of mine; it was morning, and lo, now
it is evening, and nothing memorable is accomplished. Instead of
singing like the birds, I silently smiled at my incessant good fortune.
As the sparrow had its trill, sitting on the hickory before my door, so
had I my chuckle or suppressed warble which he might hear out of my
nest. My days were not days of the week, bearing the stamp of any
heathen deity, nor were they minced into hours and fretted by the
ticking of a clock; for I lived like the Puri Indians, of whom it is
said that “for yesterday, to-day, and to-morrow they have only one
word, and they express the variety of meaning by pointing backward for
yesterday, forward for to-morrow, and overhead for the passing day.”
This was sheer idleness to my fellow-townsmen, no doubt; but if the
birds and flowers had tried me by their standard, I should not have
been found wanting. A man must find his occasions in himself, it is
true. The natural day is very calm, and will hardly reprove his
indolence.

I had this advantage, at least, in my mode of life, over those who were
obliged to look abroad for amusement, to society and the theatre, that
my life itself was become my amusement and never ceased to be novel. It
was a drama of many scenes and without an end. If we were always indeed
getting our living, and regulating our lives according to the last and
best mode we had learned, we should never be troubled with ennui.
Follow your genius closely enough, and it will not fail to show you a
fresh prospect every hour. Housework was a pleasant pastime. When my
floor was dirty, I rose early, and, setting all my furniture out of
doors on the grass, bed and bedstead making but one budget, dashed
water on the floor, and sprinkled white sand from the pond on it, and
then with a broom scrubbed it clean and white; and by the time the
villagers had broken their fast the morning sun had dried my house
sufficiently to allow me to move in again, and my meditations were
almost uninterupted. It was pleasant to see my whole household effects
out on the grass, making a little pile like a gypsy’s pack, and my
three-legged table, from which I did not remove the books and pen and
ink, standing amid the pines and hickories. They seemed glad to get out
themselves, and as if unwilling to be brought in. I was sometimes
tempted to stretch an awning over them and take my seat there. It was
worth the while to see the sun shine on these things, and hear the free
wind blow on them; so much more interesting most familiar objects look
out of doors than in the house. A bird sits on the next bough,
life-everlasting grows under the table, and blackberry vines run round
its legs; pine cones, chestnut burs, and strawberry leaves are strewn
about. It looked as if this was the way these forms came to be
transferred to our furniture, to tables, chairs, and bedsteads,—because
they once stood in their midst.

My house was on the side of a hill, immediately on the edge of the
larger wood, in the midst of a young forest of pitch pines and
hickories, and half a dozen rods from the pond, to which a narrow
footpath led down the hill. In my front yard grew the strawberry,
blackberry, and life-everlasting, johnswort and goldenrod, shrub-oaks
and sand-cherry, blueberry and groundnut. Near the end of May, the
sand-cherry (_Cerasus pumila_,) adorned the sides of the path with its
delicate flowers arranged in umbels cylindrically about its short
stems, which last, in the fall, weighed down with good sized and
handsome cherries, fell over in wreaths like rays on every side. I
tasted them out of compliment to Nature, though they were scarcely
palatable. The sumach (_Rhus glabra_,) grew luxuriantly about the
house, pushing up through the embankment which I had made, and growing
five or six feet the first season. Its broad pinnate tropical leaf was
pleasant though strange to look on. The large buds, suddenly pushing
out late in the spring from dry sticks which had seemed to be dead,
developed themselves as by magic into graceful green and tender boughs,
an inch in diameter; and sometimes, as I sat at my window, so
heedlessly did they grow and tax their weak joints, I heard a fresh and
tender bough suddenly fall like a fan to the ground, when there was not
a breath of air stirring, broken off by its own weight. In August, the
large masses of berries, which, when in flower, had attracted many wild
bees, gradually assumed their bright velvety crimson hue, and by their
weight again bent down and broke the tender limbs.



As I sit at my window this summer afternoon, hawks are circling about
my clearing; the tantivy of wild pigeons, flying by twos and threes
athwart my view, or perching restless on the white-pine boughs behind
my house, gives a voice to the air; a fishhawk dimples the glassy
surface of the pond and brings up a fish; a mink steals out of the
marsh before my door and seizes a frog by the shore; the sedge is
bending under the weight of the reed-birds flitting hither and thither;
and for the last half hour I have heard the rattle of railroad cars,
now dying away and then reviving like the beat of a partridge,
conveying travellers from Boston to the country. For I did not live so
out of the world as that boy who, as I hear, was put out to a farmer in
the east part of the town, but ere long ran away and came home again,
quite down at the heel and homesick. He had never seen such a dull and
out-of-the-way place; the folks were all gone off; why, you couldn’t
even hear the whistle! I doubt if there is such a place in
Massachusetts now:—

     “In truth, our village has become a butt
     For one of those fleet railroad shafts, and o’er
     Our peaceful plain its soothing sound is—Concord.”

The Fitchburg Railroad touches the pond about a hundred rods south of
where I dwell. I usually go to the village along its causeway, and am,
as it were, related to society by this link. The men on the freight
trains, who go over the whole length of the road, bow to me as to an
old acquaintance, they pass me so often, and apparently they take me
for an employee; and so I am. I too would fain be a track-repairer
somewhere in the orbit of the earth.

The whistle of the locomotive penetrates my woods summer and winter,
sounding like the scream of a hawk sailing over some farmer’s yard,
informing me that many restless city merchants are arriving within the
circle of the town, or adventurous country traders from the other side.
As they come under one horizon, they shout their warning to get off the
track to the other, heard sometimes through the circles of two towns.
Here come your groceries, country; your rations, countrymen! Nor is
there any man so independent on his farm that he can say them nay. And
here’s your pay for them! screams the countryman’s whistle; timber like
long battering rams going twenty miles an hour against the city’s
walls, and chairs enough to seat all the weary and heavy laden that
dwell within them. With such huge and lumbering civility the country
hands a chair to the city. All the Indian huckleberry hills are
stripped, all the cranberry meadows are raked into the city. Up comes
the cotton, down goes the woven cloth; up comes the silk, down goes the
woollen; up come the books, but down goes the wit that writes them.

When I meet the engine with its train of cars moving off with planetary
motion,—or, rather, like a comet, for the beholder knows not if with
that velocity and with that direction it will ever revisit this system,
since its orbit does not look like a returning curve,—with its steam
cloud like a banner streaming behind in golden and silver wreaths, like
many a downy cloud which I have seen, high in the heavens, unfolding
its masses to the light,—as if this travelling demigod, this
cloud-compeller, would ere long take the sunset sky for the livery of
his train; when I hear the iron horse make the hills echo with his
snort like thunder, shaking the earth with his feet, and breathing fire
and smoke from his nostrils, (what kind of winged horse or fiery dragon
they will put into the new Mythology I don’t know), it seems as if the
earth had got a race now worthy to inhabit it. If all were as it seems,
and men made the elements their servants for noble ends! If the cloud
that hangs over the engine were the perspiration of heroic deeds, or as
beneficent as that which floats over the farmer’s fields, then the
elements and Nature herself would cheerfully accompany men on their
errands and be their escort.

I watch the passage of the morning cars with the same feeling that I do
the rising of the sun, which is hardly more regular. Their train of
clouds stretching far behind and rising higher and higher, going to
heaven while the cars are going to Boston, conceals the sun for a
minute and casts my distant field into the shade, a celestial train
beside which the petty train of cars which hugs the earth is but the
barb of the spear. The stabler of the iron horse was up early this
winter morning by the light of the stars amid the mountains, to fodder
and harness his steed. Fire, too, was awakened thus early to put the
vital heat in him and get him off. If the enterprise were as innocent
as it is early! If the snow lies deep, they strap on his snow-shoes,
and with the giant plow, plow a furrow from the mountains to the
seaboard, in which the cars, like a following drill-barrow, sprinkle
all the restless men and floating merchandise in the country for seed.
All day the fire-steed flies over the country, stopping only that his
master may rest, and I am awakened by his tramp and defiant snort at
midnight, when in some remote glen in the woods he fronts the elements
incased in ice and snow; and he will reach his stall only with the
morning star, to start once more on his travels without rest or
slumber. Or perchance, at evening, I hear him in his stable blowing off
the superfluous energy of the day, that he may calm his nerves and cool
his liver and brain for a few hours of iron slumber. If the enterprise
were as heroic and commanding as it is protracted and unwearied!

Far through unfrequented woods on the confines of towns, where once
only the hunter penetrated by day, in the darkest night dart these
bright saloons without the knowledge of their inhabitants; this moment
stopping at some brilliant station-house in town or city, where a
social crowd is gathered, the next in the Dismal Swamp, scaring the owl
and fox. The startings and arrivals of the cars are now the epochs in
the village day. They go and come with such regularity and precision,
and their whistle can be heard so far, that the farmers set their
clocks by them, and thus one well conducted institution regulates a
whole country. Have not men improved somewhat in punctuality since the
railroad was invented? Do they not talk and think faster in the depot
than they did in the stage-office? There is something electrifying in
the atmosphere of the former place. I have been astonished at the
miracles it has wrought; that some of my neighbors, who, I should have
prophesied, once for all, would never get to Boston by so prompt a
conveyance, are on hand when the bell rings. To do things “railroad
fashion” is now the by-word; and it is worth the while to be warned so
often and so sincerely by any power to get off its track. There is no
stopping to read the riot act, no firing over the heads of the mob, in
this case. We have constructed a fate, an _Atropos_, that never turns
aside. (Let that be the name of your engine.) Men are advertised that
at a certain hour and minute these bolts will be shot toward particular
points of the compass; yet it interferes with no man’s business, and
the children go to school on the other track. We live the steadier for
it. We are all educated thus to be sons of Tell. The air is full of
invisible bolts. Every path but your own is the path of fate. Keep on
your own track, then.

What recommends commerce to me is its enterprise and bravery. It does
not clasp its hands and pray to Jupiter. I see these men every day go
about their business with more or less courage and content, doing more
even than they suspect, and perchance better employed than they could
have consciously devised. I am less affected by their heroism who stood
up for half an hour in the front line at Buena Vista, than by the
steady and cheerful valor of the men who inhabit the snow-plough for
their winter quarters; who have not merely the three-o’-clock in the
morning courage, which Bonaparte thought was the rarest, but whose
courage does not go to rest so early, who go to sleep only when the
storm sleeps or the sinews of their iron steed are frozen. On this
morning of the Great Snow, perchance, which is still raging and
chilling men’s blood, I hear the muffled tone of their engine bell from
out the fog bank of their chilled breath, which announces that the cars
_are coming_, without long delay, notwithstanding the veto of a New
England north-east snow storm, and I behold the ploughmen covered with
snow and rime, their heads peering, above the mould-board which is
turning down other than daisies and the nests of field-mice, like
bowlders of the Sierra Nevada, that occupy an outside place in the
universe.

Commerce is unexpectedly confident and serene, alert, adventurous, and
unwearied. It is very natural in its methods withal, far more so than
many fantastic enterprises and sentimental experiments, and hence its
singular success. I am refreshed and expanded when the freight train
rattles past me, and I smell the stores which go dispensing their odors
all the way from Long Wharf to Lake Champlain, reminding me of foreign
parts, of coral reefs, and Indian oceans, and tropical climes, and the
extent of the globe. I feel more like a citizen of the world at the
sight of the palm-leaf which will cover so many flaxen New England
heads the next summer, the Manilla hemp and cocoa-nut husks, the old
junk, gunny bags, scrap iron, and rusty nails. This car-load of torn
sails is more legible and interesting now than if they should be
wrought into paper and printed books. Who can write so graphically the
history of the storms they have weathered as these rents have done?
They are proof-sheets which need no correction. Here goes lumber from
the Maine woods, which did not go out to sea in the last freshet, risen
four dollars on the thousand because of what did go out or was split
up; pine, spruce, cedar,—first, second, third, and fourth qualities, so
lately all of one quality, to wave over the bear, and moose, and
caribou. Next rolls Thomaston lime, a prime lot, which will get far
among the hills before it gets slacked. These rags in bales, of all
hues and qualities, the lowest condition to which cotton and linen
descend, the final result of dress,—of patterns which are now no longer
cried up, unless it be in Milwaukie, as those splendid articles,
English, French, or American prints, ginghams, muslins, &c., gathered
from all quarters both of fashion and poverty, going to become paper of
one color or a few shades only, on which forsooth will be written tales
of real life, high and low, and founded on fact! This closed car smells
of salt fish, the strong New England and commercial scent, reminding me
of the Grand Banks and the fisheries. Who has not seen a salt fish,
thoroughly cured for this world, so that nothing can spoil it, and
putting the perseverance of the saints to the blush? with which you may
sweep or pave the streets, and split your kindlings, and the teamster
shelter himself and his lading against sun wind and rain behind it,—and
the trader, as a Concord trader once did, hang it up by his door for a
sign when he commences business, until at last his oldest customer
cannot tell surely whether it be animal, vegetable, or mineral, and yet
it shall be as pure as a snowflake, and if it be put into a pot and
boiled, will come out an excellent dun fish for a Saturday’s dinner.
Next Spanish hides, with the tails still preserving their twist and the
angle of elevation they had when the oxen that wore them were careering
over the pampas of the Spanish main,—a type of all obstinacy, and
evincing how almost hopeless and incurable are all constitutional
vices. I confess, that practically speaking, when I have learned a
man’s real disposition, I have no hopes of changing it for the better
or worse in this state of existence. As the Orientals say, “A cur’s
tail may be warmed, and pressed, and bound round with ligatures, and
after a twelve years’ labor bestowed upon it, still it will retain its
natural form.” The only effectual cure for such inveteracies as these
tails exhibit is to make glue of them, which I believe is what is
usually done with them, and then they will stay put and stick. Here is
a hogshead of molasses or of brandy directed to John Smith,
Cuttingsville, Vermont, some trader among the Green Mountains, who
imports for the farmers near his clearing, and now perchance stands
over his bulk-head and thinks of the last arrivals on the coast, how
they may affect the price for him, telling his customers this moment,
as he has told them twenty times before this morning, that he expects
some by the next train of prime quality. It is advertised in the
Cuttingsville Times.

While these things go up other things come down. Warned by the whizzing
sound, I look up from my book and see some tall pine, hewn on far
northern hills, which has winged its way over the Green Mountains and
the Connecticut, shot like an arrow through the township within ten
minutes, and scarce another eye beholds it; going

     “to be the mast
     Of some great ammiral.”

And hark! here comes the cattle-train bearing the cattle of a thousand
hills, sheepcots, stables, and cow-yards in the air, drovers with their
sticks, and shepherd boys in the midst of their flocks, all but the
mountain pastures, whirled along like leaves blown from the mountains
by the September gales. The air is filled with the bleating of calves
and sheep, and the hustling of oxen, as if a pastoral valley were going
by. When the old bell-wether at the head rattles his bell, the
mountains do indeed skip like rams and the little hills like lambs. A
car-load of drovers, too, in the midst, on a level with their droves
now, their vocation gone, but still clinging to their useless sticks as
their badge of office. But their dogs, where are they? It is a stampede
to them; they are quite thrown out; they have lost the scent. Methinks
I hear them barking behind the Peterboro’ Hills, or panting up the
western slope of the Green Mountains. They will not be in at the death.
Their vocation, too, is gone. Their fidelity and sagacity are below par
now. They will slink back to their kennels in disgrace, or perchance
run wild and strike a league with the wolf and the fox. So is your
pastoral life whirled past and away. But the bell rings, and I must get
off the track and let the cars go by;—

     What’s the railroad to me?
     I never go to see
     Where it ends.
     It fills a few hollows,
     And makes banks for the swallows,
     It sets the sand a-blowing,
     And the blackberries a-growing,

but I cross it like a cart-path in the woods. I will not have my eyes
put out and my ears spoiled by its smoke and steam and hissing.



Now that the cars are gone by and all the restless world with them, and
the fishes in the pond no longer feel their rumbling, I am more alone
than ever. For the rest of the long afternoon, perhaps, my meditations
are interrupted only by the faint rattle of a carriage or team along
the distant highway.

Sometimes, on Sundays, I heard the bells, the Lincoln, Acton, Bedford,
or Concord bell, when the wind was favorable, a faint, sweet, and, as
it were, natural melody, worth importing into the wilderness. At a
sufficient distance over the woods this sound acquires a certain
vibratory hum, as if the pine needles in the horizon were the strings
of a harp which it swept. All sound heard at the greatest possible
distance produces one and the same effect, a vibration of the universal
lyre, just as the intervening atmosphere makes a distant ridge of earth
interesting to our eyes by the azure tint it imparts to it. There came
to me in this case a melody which the air had strained, and which had
conversed with every leaf and needle of the wood, that portion of the
sound which the elements had taken up and modulated and echoed from
vale to vale. The echo is, to some extent, an original sound, and
therein is the magic and charm of it. It is not merely a repetition of
what was worth repeating in the bell, but partly the voice of the wood;
the same trivial words and notes sung by a wood-nymph.

At evening, the distant lowing of some cow in the horizon beyond the
woods sounded sweet and melodious, and at first I would mistake it for
the voices of certain minstrels by whom I was sometimes serenaded, who
might be straying over hill and dale; but soon I was not unpleasantly
disappointed when it was prolonged into the cheap and natural music of
the cow. I do not mean to be satirical, but to express my appreciation
of those youths’ singing, when I state that I perceived clearly that it
was akin to the music of the cow, and they were at length one
articulation of Nature.

Regularly at half past seven, in one part of the summer, after the
evening train had gone by, the whippoorwills chanted their vespers for
half an hour, sitting on a stump by my door, or upon the ridge pole of
the house. They would begin to sing almost with as much precision as a
clock, within five minutes of a particular time, referred to the
setting of the sun, every evening. I had a rare opportunity to become
acquainted with their habits. Sometimes I heard four or five at once in
different parts of the wood, by accident one a bar behind another, and
so near me that I distinguished not only the cluck after each note, but
often that singular buzzing sound like a fly in a spider’s web, only
proportionally louder. Sometimes one would circle round and round me in
the woods a few feet distant as if tethered by a string, when probably
I was near its eggs. They sang at intervals throughout the night, and
were again as musical as ever just before and about dawn.

When other birds are still the screech owls take up the strain, like
mourning women their ancient u-lu-lu. Their dismal scream is truly Ben
Jonsonian. Wise midnight hags! It is no honest and blunt tu-whit tu-who
of the poets, but, without jesting, a most solemn graveyard ditty, the
mutual consolations of suicide lovers remembering the pangs and the
delights of supernal love in the infernal groves. Yet I love to hear
their wailing, their doleful responses, trilled along the wood-side;
reminding me sometimes of music and singing birds; as if it were the
dark and tearful side of music, the regrets and sighs that would fain
be sung. They are the spirits, the low spirits and melancholy
forebodings, of fallen souls that once in human shape night-walked the
earth and did the deeds of darkness, now expiating their sins with
their wailing hymns or threnodies in the scenery of their
transgressions. They give me a new sense of the variety and capacity of
that nature which is our common dwelling. _Oh-o-o-o-o that I never had
been bor-r-r-r-n!_ sighs one on this side of the pond, and circles with
the restlessness of despair to some new perch on the gray oaks.
Then—_that I never had been bor-r-r-r-n!_ echoes another on the farther
side with tremulous sincerity, and—_bor-r-r-r-n!_ comes faintly from
far in the Lincoln woods.

I was also serenaded by a hooting owl. Near at hand you could fancy it
the most melancholy sound in Nature, as if she meant by this to
stereotype and make permanent in her choir the dying moans of a human
being,—some poor weak relic of mortality who has left hope behind, and
howls like an animal, yet with human sobs, on entering the dark valley,
made more awful by a certain gurgling melodiousness,—I find myself
beginning with the letters gl when I try to imitate it,—expressive of a
mind which has reached the gelatinous mildewy stage in the
mortification of all healthy and courageous thought. It reminded me of
ghouls and idiots and insane howlings. But now one answers from far
woods in a strain made really melodious by distance,—_Hoo hoo hoo,
hoorer hoo_; and indeed for the most part it suggested only pleasing
associations, whether heard by day or night, summer or winter.

I rejoice that there are owls. Let them do the idiotic and maniacal
hooting for men. It is a sound admirably suited to swamps and twilight
woods which no day illustrates, suggesting a vast and undeveloped
nature which men have not recognized. They represent the stark twilight
and unsatisfied thoughts which all have. All day the sun has shone on
the surface of some savage swamp, where the single spruce stands hung
with usnea lichens, and small hawks circulate above, and the chickadee
lisps amid the evergreens, and the partridge and rabbit skulk beneath;
but now a more dismal and fitting day dawns, and a different race of
creatures awakes to express the meaning of Nature there.

Late in the evening I heard the distant rumbling of wagons over
bridges,—a sound heard farther than almost any other at night,—the
baying of dogs, and sometimes again the lowing of some disconsolate cow
in a distant barn-yard. In the mean while all the shore rang with the
trump of bullfrogs, the sturdy spirits of ancient wine-bibbers and
wassailers, still unrepentant, trying to sing a catch in their Stygian
lake,—if the Walden nymphs will pardon the comparison, for though there
are almost no weeds, there are frogs there,—who would fain keep up the
hilarious rules of their old festal tables, though their voices have
waxed hoarse and solemnly grave, mocking at mirth, and the wine has
lost its flavor, and become only liquor to distend their paunches, and
sweet intoxication never comes to drown the memory of the past, but
mere saturation and waterloggedness and distention. The most
aldermanic, with his chin upon a heart-leaf, which serves for a napkin
to his drooling chaps, under this northern shore quaffs a deep draught
of the once scorned water, and passes round the cup with the
ejaculation _tr-r-r-oonk, tr-r-r-oonk, tr-r-r-oonk!_ and straightway
comes over the water from some distant cove the same password repeated,
where the next in seniority and girth has gulped down to his mark; and
when this observance has made the circuit of the shores, then
ejaculates the master of ceremonies, with satisfaction, _tr-r-r-oonk!_
and each in his turn repeats the same down to the least distended,
leakiest, and flabbiest paunched, that there be no mistake; and then
the bowl goes round again and again, until the sun disperses the
morning mist, and only the patriarch is not under the pond, but vainly
bellowing _troonk_ from time to time, and pausing for a reply.

I am not sure that I ever heard the sound of cock-crowing from my
clearing, and I thought that it might be worth the while to keep a
cockerel for his music merely, as a singing bird. The note of this once
wild Indian pheasant is certainly the most remarkable of any bird’s,
and if they could be naturalized without being domesticated, it would
soon become the most famous sound in our woods, surpassing the clangor
of the goose and the hooting of the owl; and then imagine the cackling
of the hens to fill the pauses when their lords’ clarions rested! No
wonder that man added this bird to his tame stock,—to say nothing of
the eggs and drumsticks. To walk in a winter morning in a wood where
these birds abounded, their native woods, and hear the wild cockerels
crow on the trees, clear and shrill for miles over the resounding
earth, drowning the feebler notes of other birds,—think of it! It would
put nations on the alert. Who would not be early to rise, and rise
earlier and earlier every successive day of his life, till he became
unspeakably healthy, wealthy, and wise? This foreign bird’s note is
celebrated by the poets of all countries along with the notes of their
native songsters. All climates agree with brave Chanticleer. He is more
indigenous even than the natives. His health is ever good, his lungs
are sound, his spirits never flag. Even the sailor on the Atlantic and
Pacific is awakened by his voice; but its shrill sound never roused me
from my slumbers. I kept neither dog, cat, cow, pig, nor hens, so that
you would have said there was a deficiency of domestic sounds; neither
the churn, nor the spinning wheel, nor even the singing of the kettle,
nor the hissing of the urn, nor children crying, to comfort one. An
old-fashioned man would have lost his senses or died of ennui before
this. Not even rats in the wall, for they were starved out, or rather
were never baited in,—only squirrels on the roof and under the floor, a
whippoorwill on the ridge pole, a blue-jay screaming beneath the
window, a hare or woodchuck under the house, a screech-owl or a cat-owl
behind it, a flock of wild geese or a laughing loon on the pond, and a
fox to bark in the night. Not even a lark or an oriole, those mild
plantation birds, ever visited my clearing. No cockerels to crow nor
hens to cackle in the yard. No yard! but unfenced Nature reaching up to
your very sills. A young forest growing up under your meadows, and wild
sumachs and blackberry vines breaking through into your cellar; sturdy
pitch pines rubbing and creaking against the shingles for want of room,
their roots reaching quite under the house. Instead of a scuttle or a
blind blown off in the gale,—a pine tree snapped off or torn up by the
roots behind your house for fuel. Instead of no path to the front-yard
gate in the Great Snow,—no gate,—no front-yard,—and no path to the
civilized world!


Solitude

This is a delicious evening, when the whole body is one sense, and
imbibes delight through every pore. I go and come with a strange
liberty in Nature, a part of herself. As I walk along the stony shore
of the pond in my shirt sleeves, though it is cool as well as cloudy
and windy, and I see nothing special to attract me, all the elements
are unusually congenial to me. The bullfrogs trump to usher in the
night, and the note of the whippoorwill is borne on the rippling wind
from over the water. Sympathy with the fluttering alder and poplar
leaves almost takes away my breath; yet, like the lake, my serenity is
rippled but not ruffled. These small waves raised by the evening wind
are as remote from storm as the smooth reflecting surface. Though it is
now dark, the wind still blows and roars in the wood, the waves still
dash, and some creatures lull the rest with their notes. The repose is
never complete. The wildest animals do not repose, but seek their prey
now; the fox, and skunk, and rabbit, now roam the fields and woods
without fear. They are Nature’s watchmen,—links which connect the days
of animated life.

When I return to my house I find that visitors have been there and left
their cards, either a bunch of flowers, or a wreath of evergreen, or a
name in pencil on a yellow walnut leaf or a chip. They who come rarely
to the woods take some little piece of the forest into their hands to
play with by the way, which they leave, either intentionally or
accidentally. One has peeled a willow wand, woven it into a ring, and
dropped it on my table. I could always tell if visitors had called in
my absence, either by the bended twigs or grass, or the print of their
shoes, and generally of what sex or age or quality they were by some
slight trace left, as a flower dropped, or a bunch of grass plucked and
thrown away, even as far off as the railroad, half a mile distant, or
by the lingering odor of a cigar or pipe. Nay, I was frequently
notified of the passage of a traveller along the highway sixty rods off
by the scent of his pipe.

There is commonly sufficient space about us. Our horizon is never quite
at our elbows. The thick wood is not just at our door, nor the pond,
but somewhat is always clearing, familiar and worn by us, appropriated
and fenced in some way, and reclaimed from Nature. For what reason have
I this vast range and circuit, some square miles of unfrequented
forest, for my privacy, abandoned to me by men? My nearest neighbor is
a mile distant, and no house is visible from any place but the
hill-tops within half a mile of my own. I have my horizon bounded by
woods all to myself; a distant view of the railroad where it touches
the pond on the one hand, and of the fence which skirts the woodland
road on the other. But for the most part it is as solitary where I live
as on the prairies. It is as much Asia or Africa as New England. I
have, as it were, my own sun and moon and stars, and a little world all
to myself. At night there was never a traveller passed my house, or
knocked at my door, more than if I were the first or last man; unless
it were in the spring, when at long intervals some came from the
village to fish for pouts,—they plainly fished much more in the Walden
Pond of their own natures, and baited their hooks with darkness,—but
they soon retreated, usually with light baskets, and left “the world to
darkness and to me,” and the black kernel of the night was never
profaned by any human neighborhood. I believe that men are generally
still a little afraid of the dark, though the witches are all hung, and
Christianity and candles have been introduced.

Yet I experienced sometimes that the most sweet and tender, the most
innocent and encouraging society may be found in any natural object,
even for the poor misanthrope and most melancholy man. There can be no
very black melancholy to him who lives in the midst of Nature and has
his senses still. There was never yet such a storm but it was Æolian
music to a healthy and innocent ear. Nothing can rightly compel a
simple and brave man to a vulgar sadness. While I enjoy the friendship
of the seasons I trust that nothing can make life a burden to me. The
gentle rain which waters my beans and keeps me in the house to-day is
not drear and melancholy, but good for me too. Though it prevents my
hoeing them, it is of far more worth than my hoeing. If it should
continue so long as to cause the seeds to rot in the ground and destroy
the potatoes in the low lands, it would still be good for the grass on
the uplands, and, being good for the grass, it would be good for me.
Sometimes, when I compare myself with other men, it seems as if I were
more favored by the gods than they, beyond any deserts that I am
conscious of; as if I had a warrant and surety at their hands which my
fellows have not, and were especially guided and guarded. I do not
flatter myself, but if it be possible they flatter me. I have never
felt lonesome, or in the least oppressed by a sense of solitude, but
once, and that was a few weeks after I came to the woods, when, for an
hour, I doubted if the near neighborhood of man was not essential to a
serene and healthy life. To be alone was something unpleasant. But I
was at the same time conscious of a slight insanity in my mood, and
seemed to foresee my recovery. In the midst of a gentle rain while
these thoughts prevailed, I was suddenly sensible of such sweet and
beneficent society in Nature, in the very pattering of the drops, and
in every sound and sight around my house, an infinite and unaccountable
friendliness all at once like an atmosphere sustaining me, as made the
fancied advantages of human neighborhood insignificant, and I have
never thought of them since. Every little pine needle expanded and
swelled with sympathy and befriended me. I was so distinctly made aware
of the presence of something kindred to me, even in scenes which we are
accustomed to call wild and dreary, and also that the nearest of blood
to me and humanest was not a person nor a villager, that I thought no
place could ever be strange to me again.—

     “Mourning untimely consumes the sad;
     Few are their days in the land of the living,
     Beautiful daughter of Toscar.”

Some of my pleasantest hours were during the long rain storms in the
spring or fall, which confined me to the house for the afternoon as
well as the forenoon, soothed by their ceaseless roar and pelting; when
an early twilight ushered in a long evening in which many thoughts had
time to take root and unfold themselves. In those driving north-east
rains which tried the village houses so, when the maids stood ready
with mop and pail in front entries to keep the deluge out, I sat behind
my door in my little house, which was all entry, and thoroughly enjoyed
its protection. In one heavy thunder shower the lightning struck a
large pitch-pine across the pond, making a very conspicuous and
perfectly regular spiral groove from top to bottom, an inch or more
deep, and four or five inches wide, as you would groove a
walking-stick. I passed it again the other day, and was struck with awe
on looking up and beholding that mark, now more distinct than ever,
where a terrific and resistless bolt came down out of the harmless sky
eight years ago. Men frequently say to me, “I should think you would
feel lonesome down there, and want to be nearer to folks, rainy and
snowy days and nights especially.” I am tempted to reply to such,—This
whole earth which we inhabit is but a point in space. How far apart,
think you, dwell the two most distant inhabitants of yonder star, the
breadth of whose disk cannot be appreciated by our instruments? Why
should I feel lonely? is not our planet in the Milky Way? This which
you put seems to me not to be the most important question. What sort of
space is that which separates a man from his fellows and makes him
solitary? I have found that no exertion of the legs can bring two minds
much nearer to one another. What do we want most to dwell near to? Not
to many men surely, the depot, the post-office, the bar-room, the
meeting-house, the school-house, the grocery, Beacon Hill, or the Five
Points, where men most congregate, but to the perennial source of our
life, whence in all our experience we have found that to issue, as the
willow stands near the water and sends out its roots in that direction.
This will vary with different natures, but this is the place where a
wise man will dig his cellar.... I one evening overtook one of my
townsmen, who has accumulated what is called “a handsome
property,”—though I never got a _fair_ view of it,—on the Walden road,
driving a pair of cattle to market, who inquired of me how I could
bring my mind to give up so many of the comforts of life. I answered
that I was very sure I liked it passably well; I was not joking. And so
I went home to my bed, and left him to pick his way through the
darkness and the mud to Brighton,—or Bright-town,—which place he would
reach some time in the morning.

Any prospect of awakening or coming to life to a dead man makes
indifferent all times and places. The place where that may occur is
always the same, and indescribably pleasant to all our senses. For the
most part we allow only outlying and transient circumstances to make
our occasions. They are, in fact, the cause of our distraction. Nearest
to all things is that power which fashions their being. _Next_ to us
the grandest laws are continually being executed. _Next_ to us is not
the workman whom we have hired, with whom we love so well to talk, but
the workman whose work we are.

“How vast and profound is the influence of the subtile powers of Heaven
and of Earth!”

“We seek to perceive them, and we do not see them; we seek to hear
them, and we do not hear them; identified with the substance of things,
they cannot be separated from them.”

“They cause that in all the universe men purify and sanctify their
hearts, and clothe themselves in their holiday garments to offer
sacrifices and oblations to their ancestors. It is an ocean of subtile
intelligences. They are every where, above us, on our left, on our
right; they environ us on all sides.”

We are the subjects of an experiment which is not a little interesting
to me. Can we not do without the society of our gossips a little while
under these circumstances,—have our own thoughts to cheer us? Confucius
says truly, “Virtue does not remain as an abandoned orphan; it must of
necessity have neighbors.”

With thinking we may be beside ourselves in a sane sense. By a
conscious effort of the mind we can stand aloof from actions and their
consequences; and all things, good and bad, go by us like a torrent. We
are not wholly involved in Nature. I may be either the drift-wood in
the stream, or Indra in the sky looking down on it. I _may_ be affected
by a theatrical exhibition; on the other hand, I _may not_ be affected
by an actual event which appears to concern me much more. I only know
myself as a human entity; the scene, so to speak, of thoughts and
affections; and am sensible of a certain doubleness by which I can
stand as remote from myself as from another. However intense my
experience, I am conscious of the presence and criticism of a part of
me, which, as it were, is not a part of me, but spectator, sharing no
experience, but taking note of it; and that is no more I than it is
you. When the play, it may be the tragedy, of life is over, the
spectator goes his way. It was a kind of fiction, a work of the
imagination only, so far as he was concerned. This doubleness may
easily make us poor neighbors and friends sometimes.

I find it wholesome to be alone the greater part of the time. To be in
company, even with the best, is soon wearisome and dissipating. I love
to be alone. I never found the companion that was so companionable as
solitude. We are for the most part more lonely when we go abroad among
men than when we stay in our chambers. A man thinking or working is
always alone, let him be where he will. Solitude is not measured by the
miles of space that intervene between a man and his fellows. The really
diligent student in one of the crowded hives of Cambridge College is as
solitary as a dervish in the desert. The farmer can work alone in the
field or the woods all day, hoeing or chopping, and not feel lonesome,
because he is employed; but when he comes home at night he cannot sit
down in a room alone, at the mercy of his thoughts, but must be where
he can “see the folks,” and recreate, and as he thinks remunerate
himself for his day’s solitude; and hence he wonders how the student
can sit alone in the house all night and most of the day without ennui
and “the blues;” but he does not realize that the student, though in
the house, is still at work in _his_ field, and chopping in _his_
woods, as the farmer in his, and in turn seeks the same recreation and
society that the latter does, though it may be a more condensed form of
it.

Society is commonly too cheap. We meet at very short intervals, not
having had time to acquire any new value for each other. We meet at
meals three times a day, and give each other a new taste of that old
musty cheese that we are. We have had to agree on a certain set of
rules, called etiquette and politeness, to make this frequent meeting
tolerable and that we need not come to open war. We meet at the
post-office, and at the sociable, and about the fireside every night;
we live thick and are in each other’s way, and stumble over one
another, and I think that we thus lose some respect for one another.
Certainly less frequency would suffice for all important and hearty
communications. Consider the girls in a factory,—never alone, hardly in
their dreams. It would be better if there were but one inhabitant to a
square mile, as where I live. The value of a man is not in his skin,
that we should touch him.

I have heard of a man lost in the woods and dying of famine and
exhaustion at the foot of a tree, whose loneliness was relieved by the
grotesque visions with which, owing to bodily weakness, his diseased
imagination surrounded him, and which he believed to be real. So also,
owing to bodily and mental health and strength, we may be continually
cheered by a like but more normal and natural society, and come to know
that we are never alone.

I have a great deal of company in my house; especially in the morning,
when nobody calls. Let me suggest a few comparisons, that some one may
convey an idea of my situation. I am no more lonely than the loon in
the pond that laughs so loud, or than Walden Pond itself. What company
has that lonely lake, I pray? And yet it has not the blue devils, but
the blue angels in it, in the azure tint of its waters. The sun is
alone, except in thick weather, when there sometimes appear to be two,
but one is a mock sun. God is alone,—but the devil, he is far from
being alone; he sees a great deal of company; he is legion. I am no
more lonely than a single mullein or dandelion in a pasture, or a bean
leaf, or sorrel, or a horse-fly, or a bumble-bee. I am no more lonely
than the Mill Brook, or a weathercock, or the north star, or the south
wind, or an April shower, or a January thaw, or the first spider in a
new house.

I have occasional visits in the long winter evenings, when the snow
falls fast and the wind howls in the wood, from an old settler and
original proprietor, who is reported to have dug Walden Pond, and
stoned it, and fringed it with pine woods; who tells me stories of old
time and of new eternity; and between us we manage to pass a cheerful
evening with social mirth and pleasant views of things, even without
apples or cider,—a most wise and humorous friend, whom I love much, who
keeps himself more secret than ever did Goffe or Whalley; and though he
is thought to be dead, none can show where he is buried. An elderly
dame, too, dwells in my neighborhood, invisible to most persons, in
whose odorous herb garden I love to stroll sometimes, gathering simples
and listening to her fables; for she has a genius of unequalled
fertility, and her memory runs back farther than mythology, and she can
tell me the original of every fable, and on what fact every one is
founded, for the incidents occurred when she was young. A ruddy and
lusty old dame, who delights in all weathers and seasons, and is likely
to outlive all her children yet.

The indescribable innocence and beneficence of Nature,—of sun and wind
and rain, of summer and winter,—such health, such cheer, they afford
forever! and such sympathy have they ever with our race, that all
Nature would be affected, and the sun’s brightness fade, and the winds
would sigh humanely, and the clouds rain tears, and the woods shed
their leaves and put on mourning in midsummer, if any man should ever
for a just cause grieve. Shall I not have intelligence with the earth?
Am I not partly leaves and vegetable mould myself?

What is the pill which will keep us well, serene, contented? Not my or
thy great-grandfather’s, but our great-grandmother Nature’s universal,
vegetable, botanic medicines, by which she has kept herself young
always, outlived so many old Parrs in her day, and fed her health with
their decaying fatness. For my panacea, instead of one of those quack
vials of a mixture dipped from Acheron and the Dead Sea, which come out
of those long shallow black-schooner looking wagons which we sometimes
see made to carry bottles, let me have a draught of undiluted morning
air. Morning air! If men will not drink of this at the fountain-head of
the day, why, then, we must even bottle up some and sell it in the
shops, for the benefit of those who have lost their subscription ticket
to morning time in this world. But remember, it will not keep quite
till noon-day even in the coolest cellar, but drive out the stopples
long ere that and follow westward the steps of Aurora. I am no
worshipper of Hygeia, who was the daughter of that old herb-doctor
Æsculapius, and who is represented on monuments holding a serpent in
one hand, and in the other a cup out of which the serpent sometimes
drinks; but rather of Hebe, cupbearer to Jupiter, who was the daughter
of Juno and wild lettuce, and who had the power of restoring gods and
men to the vigor of youth. She was probably the only thoroughly
sound-conditioned, healthy, and robust young lady that ever walked the
globe, and wherever she came it was spring.


Visitors

I think that I love society as much as most, and am ready enough to
fasten myself like a bloodsucker for the time to any full-blooded man
that comes in my way. I am naturally no hermit, but might possibly sit
out the sturdiest frequenter of the bar-room, if my business called me
thither.

I had three chairs in my house; one for solitude, two for friendship,
three for society. When visitors came in larger and unexpected numbers
there was but the third chair for them all, but they generally
economized the room by standing up. It is surprising how many great men
and women a small house will contain. I have had twenty-five or thirty
souls, with their bodies, at once under my roof, and yet we often
parted without being aware that we had come very near to one another.
Many of our houses, both public and private, with their almost
innumerable apartments, their huge halls and their cellars for the
storage of wines and other munitions of peace, appear to me
extravagantly large for their inhabitants. They are so vast and
magnificent that the latter seem to be only vermin which infest them. I
am surprised when the herald blows his summons before some Tremont or
Astor or Middlesex House, to see come creeping out over the piazza for
all inhabitants a ridiculous mouse, which soon again slinks into some
hole in the pavement.

One inconvenience I sometimes experienced in so small a house, the
difficulty of getting to a sufficient distance from my guest when we
began to utter the big thoughts in big words. You want room for your
thoughts to get into sailing trim and run a course or two before they
make their port. The bullet of your thought must have overcome its
lateral and ricochet motion and fallen into its last and steady course
before it reaches the ear of the hearer, else it may plough out again
through the side of his head. Also, our sentences wanted room to unfold
and form their columns in the interval. Individuals, like nations, must
have suitable broad and natural boundaries, even a considerable neutral
ground, between them. I have found it a singular luxury to talk across
the pond to a companion on the opposite side. In my house we were so
near that we could not begin to hear,—we could not speak low enough to
be heard; as when you throw two stones into calm water so near that
they break each other’s undulations. If we are merely loquacious and
loud talkers, then we can afford to stand very near together, cheek by
jowl, and feel each other’s breath; but if we speak reservedly and
thoughtfully, we want to be farther apart, that all animal heat and
moisture may have a chance to evaporate. If we would enjoy the most
intimate society with that in each of us which is without, or above,
being spoken to, we must not only be silent, but commonly so far apart
bodily that we cannot possibly hear each other’s voice in any case.
Referred to this standard, speech is for the convenience of those who
are hard of hearing; but there are many fine things which we cannot say
if we have to shout. As the conversation began to assume a loftier and
grander tone, we gradually shoved our chairs farther apart till they
touched the wall in opposite corners, and then commonly there was not
room enough.

My “best” room, however, my withdrawing room, always ready for company,
on whose carpet the sun rarely fell, was the pine wood behind my house.
Thither in summer days, when distinguished guests came, I took them,
and a priceless domestic swept the floor and dusted the furniture and
kept the things in order.

If one guest came he sometimes partook of my frugal meal, and it was no
interruption to conversation to be stirring a hasty-pudding, or
watching the rising and maturing of a loaf of bread in the ashes, in
the mean while. But if twenty came and sat in my house there was
nothing said about dinner, though there might be bread enough for two,
more than if eating were a forsaken habit; but we naturally practised
abstinence; and this was never felt to be an offence against
hospitality, but the most proper and considerate course. The waste and
decay of physical life, which so often needs repair, seemed
miraculously retarded in such a case, and the vital vigor stood its
ground. I could entertain thus a thousand as well as twenty; and if any
ever went away disappointed or hungry from my house when they found me
at home, they may depend upon it that I sympathized with them at least.
So easy is it, though many housekeepers doubt it, to establish new and
better customs in the place of the old. You need not rest your
reputation on the dinners you give. For my own part, I was never so
effectually deterred from frequenting a man’s house, by any kind of
Cerberus whatever, as by the parade one made about dining me, which I
took to be a very polite and roundabout hint never to trouble him so
again. I think I shall never revisit those scenes. I should be proud to
have for the motto of my cabin those lines of Spenser which one of my
visitors inscribed on a yellow walnut leaf for a card:—

     “Arrivéd there, the little house they fill,
         Ne looke for entertainment where none was;
     Rest is their feast, and all things at their will:
         The noblest mind the best contentment has.”

When Winslow, afterward governor of the Plymouth Colony, went with a
companion on a visit of ceremony to Massasoit on foot through the
woods, and arrived tired and hungry at his lodge, they were well
received by the king, but nothing was said about eating that day. When
the night arrived, to quote their own words,—“He laid us on the bed
with himself and his wife, they at the one end and we at the other, it
being only planks laid a foot from the ground, and a thin mat upon
them. Two more of his chief men, for want of room, pressed by and upon
us; so that we were worse weary of our lodging than of our journey.” At
one o’clock the next day Massasoit “brought two fishes that he had
shot,” about thrice as big as a bream; “these being boiled, there were
at least forty looked for a share in them. The most ate of them. This
meal only we had in two nights and a day; and had not one of us bought
a partridge, we had taken our journey fasting.” Fearing that they would
be light-headed for want of food and also sleep, owing to “the savages’
barbarous singing, (for they used to sing themselves asleep,)” and that
they might get home while they had strength to travel, they departed.
As for lodging, it is true they were but poorly entertained, though
what they found an inconvenience was no doubt intended for an honor;
but as far as eating was concerned, I do not see how the Indians could
have done better. They had nothing to eat themselves, and they were
wiser than to think that apologies could supply the place of food to
their guests; so they drew their belts tighter and said nothing about
it. Another time when Winslow visited them, it being a season of plenty
with them, there was no deficiency in this respect.

As for men, they will hardly fail one any where. I had more visitors
while I lived in the woods than at any other period in my life; I mean
that I had some. I met several there under more favorable circumstances
than I could any where else. But fewer came to see me on trivial
business. In this respect, my company was winnowed by my mere distance
from town. I had withdrawn so far within the great ocean of solitude,
into which the rivers of society empty, that for the most part, so far
as my needs were concerned, only the finest sediment was deposited
around me. Beside, there were wafted to me evidences of unexplored and
uncultivated continents on the other side.

Who should come to my lodge this morning but a true Homeric or
Paphlagonian man,—he had so suitable and poetic a name that I am sorry
I cannot print it here,—a Canadian, a woodchopper and post-maker, who
can hole fifty posts in a day, who made his last supper on a woodchuck
which his dog caught. He, too, has heard of Homer, and, “if it were not
for books,” would “not know what to do rainy days,” though perhaps he
has not read one wholly through for many rainy seasons. Some priest who
could pronounce the Greek itself taught him to read his verse in the
testament in his native parish far away; and now I must translate to
him, while he holds the book, Achilles’ reproof to Patroclus for his
sad countenance.—“Why are you in tears, Patroclus, like a young girl?”—

     “Or have you alone heard some news from Phthia?
     They say that Menœtius lives yet, son of Actor,
     And Peleus lives, son of Æacus, among the Myrmidons,
     Either of whom having died, we should greatly grieve.”

He says, “That’s good.” He has a great bundle of white-oak bark under
his arm for a sick man, gathered this Sunday morning. “I suppose
there’s no harm in going after such a thing to-day,” says he. To him
Homer was a great writer, though what his writing was about he did not
know. A more simple and natural man it would be hard to find. Vice and
disease, which cast such a sombre moral hue over the world, seemed to
have hardly any existence for him. He was about twenty-eight years old,
and had left Canada and his father’s house a dozen years before to work
in the States, and earn money to buy a farm with at last, perhaps in
his native country. He was cast in the coarsest mould; a stout but
sluggish body, yet gracefully carried, with a thick sunburnt neck, dark
bushy hair, and dull sleepy blue eyes, which were occasionally lit up
with expression. He wore a flat gray cloth cap, a dingy wool-colored
greatcoat, and cowhide boots. He was a great consumer of meat, usually
carrying his dinner to his work a couple of miles past my house,—for he
chopped all summer,—in a tin pail; cold meats, often cold woodchucks,
and coffee in a stone bottle which dangled by a string from his belt;
and sometimes he offered me a drink. He came along early, crossing my
bean-field, though without anxiety or haste to get to his work, such as
Yankees exhibit. He wasn’t a-going to hurt himself. He didn’t care if
he only earned his board. Frequently he would leave his dinner in the
bushes, when his dog had caught a woodchuck by the way, and go back a
mile and a half to dress it and leave it in the cellar of the house
where he boarded, after deliberating first for half an hour whether he
could not sink it in the pond safely till nightfall,—loving to dwell
long upon these themes. He would say, as he went by in the morning,
“How thick the pigeons are! If working every day were not my trade, I
could get all the meat I should want by hunting,—pigeons, woodchucks,
rabbits, partridges,—by gosh! I could get all I should want for a week
in one day.”

He was a skilful chopper, and indulged in some flourishes and ornaments
in his art. He cut his trees level and close to the ground, that the
sprouts which came up afterward might be more vigorous and a sled might
slide over the stumps; and instead of leaving a whole tree to support
his corded wood, he would pare it away to a slender stake or splinter
which you could break off with your hand at last.

He interested me because he was so quiet and solitary and so happy
withal; a well of good humor and contentment which overflowed at his
eyes. His mirth was without alloy. Sometimes I saw him at his work in
the woods, felling trees, and he would greet me with a laugh of
inexpressible satisfaction, and a salutation in Canadian French, though
he spoke English as well. When I approached him he would suspend his
work, and with half-suppressed mirth lie along the trunk of a pine
which he had felled, and, peeling off the inner bark, roll it up into a
ball and chew it while he laughed and talked. Such an exuberance of
animal spirits had he that he sometimes tumbled down and rolled on the
ground with laughter at any thing which made him think and tickled him.
Looking round upon the trees he would exclaim,—“By George! I can enjoy
myself well enough here chopping; I want no better sport.” Sometimes,
when at leisure, he amused himself all day in the woods with a pocket
pistol, firing salutes to himself at regular intervals as he walked. In
the winter he had a fire by which at noon he warmed his coffee in a
kettle; and as he sat on a log to eat his dinner the chickadees would
sometimes come round and alight on his arm and peck at the potato in
his fingers; and he said that he “liked to have the little _fellers_
about him.”

In him the animal man chiefly was developed. In physical endurance and
contentment he was cousin to the pine and the rock. I asked him once if
he was not sometimes tired at night, after working all day; and he
answered, with a sincere and serious look, “Gorrappit, I never was
tired in my life.” But the intellectual and what is called spiritual
man in him were slumbering as in an infant. He had been instructed only
in that innocent and ineffectual way in which the Catholic priests
teach the aborigines, by which the pupil is never educated to the
degree of consciousness, but only to the degree of trust and reverence,
and a child is not made a man, but kept a child. When Nature made him,
she gave him a strong body and contentment for his portion, and propped
him on every side with reverence and reliance, that he might live out
his threescore years and ten a child. He was so genuine and
unsophisticated that no introduction would serve to introduce him, more
than if you introduced a woodchuck to your neighbor. He had got to find
him out as you did. He would not play any part. Men paid him wages for
work, and so helped to feed and clothe him; but he never exchanged
opinions with them. He was so simply and naturally humble—if he can be
called humble who never aspires—that humility was no distinct quality
in him, nor could he conceive of it. Wiser men were demigods to him. If
you told him that such a one was coming, he did as if he thought that
any thing so grand would expect nothing of himself, but take all the
responsibility on itself, and let him be forgotten still. He never
heard the sound of praise. He particularly reverenced the writer and
the preacher. Their performances were miracles. When I told him that I
wrote considerably, he thought for a long time that it was merely the
handwriting which I meant, for he could write a remarkably good hand
himself. I sometimes found the name of his native parish handsomely
written in the snow by the highway, with the proper French accent, and
knew that he had passed. I asked him if he ever wished to write his
thoughts. He said that he had read and written letters for those who
could not, but he never tried to write thoughts,—no, he could not, he
could not tell what to put first, it would kill him, and then there was
spelling to be attended to at the same time!

I heard that a distinguished wise man and reformer asked him if he did
not want the world to be changed; but he answered with a chuckle of
surprise in his Canadian accent, not knowing that the question had ever
been entertained before, “No, I like it well enough.” It would have
suggested many things to a philosopher to have dealings with him. To a
stranger he appeared to know nothing of things in general; yet I
sometimes saw in him a man whom I had not seen before, and I did not
know whether he was as wise as Shakespeare or as simply ignorant as a
child, whether to suspect him of a fine poetic consciousness or of
stupidity. A townsman told me that when he met him sauntering through
the village in his small close-fitting cap, and whistling to himself,
he reminded him of a prince in disguise.

His only books were an almanac and an arithmetic, in which last he was
considerably expert. The former was a sort of cyclopædia to him, which
he supposed to contain an abstract of human knowledge, as indeed it
does to a considerable extent. I loved to sound him on the various
reforms of the day, and he never failed to look at them in the most
simple and practical light. He had never heard of such things before.
Could he do without factories? I asked. He had worn the home-made
Vermont gray, he said, and that was good. Could he dispense with tea
and coffee? Did this country afford any beverage beside water? He had
soaked hemlock leaves in water and drank it, and thought that was
better than water in warm weather. When I asked him if he could do
without money, he showed the convenience of money in such a way as to
suggest and coincide with the most philosophical accounts of the origin
of this institution, and the very derivation of the word _pecunia_. If
an ox were his property, and he wished to get needles and thread at the
store, he thought it would be inconvenient and impossible soon to go on
mortgaging some portion of the creature each time to that amount. He
could defend many institutions better than any philosopher, because, in
describing them as they concerned him, he gave the true reason for
their prevalence, and speculation had not suggested to him any other.
At another time, hearing Plato’s definition of a man,—a biped without
feathers,—and that one exhibited a cock plucked and called it Plato’s
man, he thought it an important difference that the _knees_ bent the
wrong way. He would sometimes exclaim, “How I love to talk! By George,
I could talk all day!” I asked him once, when I had not seen him for
many months, if he had got a new idea this summer. “Good Lord,” said
he, “a man that has to work as I do, if he does not forget the ideas he
has had, he will do well. May be the man you hoe with is inclined to
race; then, by gorry, your mind must be there; you think of weeds.” He
would sometimes ask me first on such occasions, if I had made any
improvement. One winter day I asked him if he was always satisfied with
himself, wishing to suggest a substitute within him for the priest
without, and some higher motive for living. “Satisfied!” said he; “some
men are satisfied with one thing, and some with another. One man,
perhaps, if he has got enough, will be satisfied to sit all day with
his back to the fire and his belly to the table, by George!” Yet I
never, by any manœuvring, could get him to take the spiritual view of
things; the highest that he appeared to conceive of was a simple
expediency, such as you might expect an animal to appreciate; and this,
practically, is true of most men. If I suggested any improvement in his
mode of life, he merely answered, without expressing any regret, that
it was too late. Yet he thoroughly believed in honesty and the like
virtues.

There was a certain positive originality, however slight, to be
detected in him, and I occasionally observed that he was thinking for
himself and expressing his own opinion, a phenomenon so rare that I
would any day walk ten miles to observe it, and it amounted to the
re-origination of many of the institutions of society. Though he
hesitated, and perhaps failed to express himself distinctly, he always
had a presentable thought behind. Yet his thinking was so primitive and
immersed in his animal life, that, though more promising than a merely
learned man’s, it rarely ripened to any thing which can be reported. He
suggested that there might be men of genius in the lowest grades of
life, however permanently humble and illiterate, who take their own
view always, or do not pretend to see at all; who are as bottomless
even as Walden Pond was thought to be, though they may be dark and
muddy.



Many a traveller came out of his way to see me and the inside of my
house, and, as an excuse for calling, asked for a glass of water. I
told them that I drank at the pond, and pointed thither, offering to
lend them a dipper. Far off as I lived, I was not exempted from the
annual visitation which occurs, methinks, about the first of April,
when every body is on the move; and I had my share of good luck, though
there were some curious specimens among my visitors. Half-witted men
from the almshouse and elsewhere came to see me; but I endeavored to
make them exercise all the wit they had, and make their confessions to
me; in such cases making wit the theme of our conversation; and so was
compensated. Indeed, I found some of them to be wiser than the so
called _overseers_ of the poor and selectmen of the town, and thought
it was time that the tables were turned. With respect to wit, I learned
that there was not much difference between the half and the whole. One
day, in particular, an inoffensive, simple-minded pauper, whom with
others I had often seen used as fencing stuff, standing or sitting on a
bushel in the fields to keep cattle and himself from straying, visited
me, and expressed a wish to live as I did. He told me, with the utmost
simplicity and truth, quite superior, or rather _inferior_, to any
thing that is called humility, that he was “deficient in intellect.”
These were his words. The Lord had made him so, yet he supposed the
Lord cared as much for him as for another. “I have always been so,”
said he, “from my childhood; I never had much mind; I was not like
other children; I am weak in the head. It was the Lord’s will, I
suppose.” And there he was to prove the truth of his words. He was a
metaphysical puzzle to me. I have rarely met a fellow-man on such
promising ground,—it was so simple and sincere and so true all that he
said. And, true enough, in proportion as he appeared to humble himself
was he exalted. I did not know at first but it was the result of a wise
policy. It seemed that from such a basis of truth and frankness as the
poor weak-headed pauper had laid, our intercourse might go forward to
something better than the intercourse of sages.

I had some guests from those not reckoned commonly among the town’s
poor, but who should be; who are among the world’s poor, at any rate;
guests who appeal, not to your hospitality, but to your
_hospitalality_; who earnestly wish to be helped, and preface their
appeal with the information that they are resolved, for one thing,
never to help themselves. I require of a visitor that he be not
actually starving, though he may have the very best appetite in the
world, however he got it. Objects of charity are not guests. Men who
did not know when their visit had terminated, though I went about my
business again, answering them from greater and greater remoteness. Men
of almost every degree of wit called on me in the migrating season.
Some who had more wits than they knew what to do with; runaway slaves
with plantation manners, who listened from time to time, like the fox
in the fable, as if they heard the hounds a-baying on their track, and
looked at me beseechingly, as much as to say,—

     “O Christian, will you send me back?”

One real runaway slave, among the rest, whom I helped to forward toward
the northstar. Men of one idea, like a hen with one chicken, and that a
duckling; men of a thousand ideas, and unkempt heads, like those hens
which are made to take charge of a hundred chickens, all in pursuit of
one bug, a score of them lost in every morning’s dew,—and become
frizzled and mangy in consequence; men of ideas instead of legs, a sort
of intellectual centipede that made you crawl all over. One man
proposed a book in which visitors should write their names, as at the
White Mountains; but, alas! I have too good a memory to make that
necessary.

I could not but notice some of the peculiarities of my visitors. Girls
and boys and young women generally seemed glad to be in the woods. They
looked in the pond and at the flowers, and improved their time. Men of
business, even farmers, thought only of solitude and employment, and of
the great distance at which I dwelt from something or other; and though
they said that they loved a ramble in the woods occasionally, it was
obvious that they did not. Restless committed men, whose time was all
taken up in getting a living or keeping it; ministers who spoke of God
as if they enjoyed a monopoly of the subject, who could not bear all
kinds of opinions; doctors, lawyers, uneasy housekeepers who pried into
my cupboard and bed when I was out,—how came Mrs. —— to know that my
sheets were not as clean as hers?—young men who had ceased to be young,
and had concluded that it was safest to follow the beaten track of the
professions,—all these generally said that it was not possible to do so
much good in my position. Ay! there was the rub. The old and infirm and
the timid, of whatever age or sex, thought most of sickness, and sudden
accident and death; to them life seemed full of danger,—what danger is
there if you don’t think of any?—and they thought that a prudent man
would carefully select the safest position, where Dr. B. might be on
hand at a moment’s warning. To them the village was literally a
_com-munity_, a league for mutual defence, and you would suppose that
they would not go a-huckleberrying without a medicine chest. The amount
of it is, if a man is alive, there is always _danger_ that he may die,
though the danger must be allowed to be less in proportion as he is
dead-and-alive to begin with. A man sits as many risks as he runs.
Finally, there were the self-styled reformers, the greatest bores of
all, who thought that I was forever singing,—

     This is the house that I built;
     This is the man that lives in the house that I built;

but they did not know that the third line was,—

     These are the folks that worry the man
     That lives in the house that I built.

I did not fear the hen-harriers, for I kept no chickens; but I feared
the men-harriers rather.

I had more cheering visitors than the last. Children come a-berrying,
railroad men taking a Sunday morning walk in clean shirts, fishermen
and hunters, poets and philosophers; in short, all honest pilgrims, who
came out to the woods for freedom’s sake, and really left the village
behind, I was ready to greet with,—“Welcome, Englishmen! welcome,
Englishmen!” for I had had communication with that race.


The Bean-Field

Meanwhile my beans, the length of whose rows, added together, was seven
miles already planted, were impatient to be hoed, for the earliest had
grown considerably before the latest were in the ground; indeed they
were not easily to be put off. What was the meaning of this so steady
and self-respecting, this small Herculean labor, I knew not. I came to
love my rows, my beans, though so many more than I wanted. They
attached me to the earth, and so I got strength like Antæus. But why
should I raise them? Only Heaven knows. This was my curious labor all
summer,—to make this portion of the earth’s surface, which had yielded
only cinquefoil, blackberries, johnswort, and the like, before, sweet
wild fruits and pleasant flowers, produce instead this pulse. What
shall I learn of beans or beans of me? I cherish them, I hoe them,
early and late I have an eye to them; and this is my day’s work. It is
a fine broad leaf to look on. My auxiliaries are the dews and rains
which water this dry soil, and what fertility is in the soil itself,
which for the most part is lean and effete. My enemies are worms, cool
days, and most of all woodchucks. The last have nibbled for me a
quarter of an acre clean. But what right had I to oust johnswort and
the rest, and break up their ancient herb garden? Soon, however, the
remaining beans will be too tough for them, and go forward to meet new
foes.

When I was four years old, as I well remember, I was brought from
Boston to this my native town, through these very woods and this field,
to the pond. It is one of the oldest scenes stamped on my memory. And
now to-night my flute has waked the echoes over that very water. The
pines still stand here older than I; or, if some have fallen, I have
cooked my supper with their stumps, and a new growth is rising all
around, preparing another aspect for new infant eyes. Almost the same
johnswort springs from the same perennial root in this pasture, and
even I have at length helped to clothe that fabulous landscape of my
infant dreams, and one of the results of my presence and influence is
seen in these bean leaves, corn blades, and potato vines.

I planted about two acres and a half of upland; and as it was only
about fifteen years since the land was cleared, and I myself had got
out two or three cords of stumps, I did not give it any manure; but in
the course of the summer it appeared by the arrowheads which I turned
up in hoeing, that an extinct nation had anciently dwelt here and
planted corn and beans ere white men came to clear the land, and so, to
some extent, had exhausted the soil for this very crop.

Before yet any woodchuck or squirrel had run across the road, or the
sun had got above the shrub oaks, while all the dew was on, though the
farmers warned me against it,—I would advise you to do all your work if
possible while the dew is on,—I began to level the ranks of haughty
weeds in my bean-field and throw dust upon their heads. Early in the
morning I worked barefooted, dabbling like a plastic artist in the dewy
and crumbling sand, but later in the day the sun blistered my feet.
There the sun lighted me to hoe beans, pacing slowly backward and
forward over that yellow gravelly upland, between the long green rows,
fifteen rods, the one end terminating in a shrub oak copse where I
could rest in the shade, the other in a blackberry field where the
green berries deepened their tints by the time I had made another bout.
Removing the weeds, putting fresh soil about the bean stems, and
encouraging this weed which I had sown, making the yellow soil express
its summer thought in bean leaves and blossoms rather than in wormwood
and piper and millet grass, making the earth say beans instead of
grass,—this was my daily work. As I had little aid from horses or
cattle, or hired men or boys, or improved implements of husbandry, I
was much slower, and became much more intimate with my beans than
usual. But labor of the hands, even when pursued to the verge of
drudgery, is perhaps never the worst form of idleness. It has a
constant and imperishable moral, and to the scholar it yields a classic
result. A very _agricola laboriosus_ was I to travellers bound westward
through Lincoln and Wayland to nobody knows where; they sitting at
their ease in gigs, with elbows on knees, and reins loosely hanging in
festoons; I the home-staying, laborious native of the soil. But soon my
homestead was out of their sight and thought. It was the only open and
cultivated field for a great distance on either side of the road; so
they made the most of it; and sometimes the man in the field heard more
of travellers’ gossip and comment than was meant for his ear: “Beans so
late! peas so late!”—for I continued to plant when others had begun to
hoe,—the ministerial husbandman had not suspected it. “Corn, my boy,
for fodder; corn for fodder.” “Does he _live_ there?” asks the black
bonnet of the gray coat; and the hard-featured farmer reins up his
grateful dobbin to inquire what you are doing where he sees no manure
in the furrow, and recommends a little chip dirt, or any little waste
stuff, or it may be ashes or plaster. But here were two acres and a
half of furrows, and only a hoe for cart and two hands to draw
it,—there being an aversion to other carts and horses,—and chip dirt
far away. Fellow-travellers as they rattled by compared it aloud with
the fields which they had passed, so that I came to know how I stood in
the agricultural world. This was one field not in Mr. Coleman’s report.
And, by the way, who estimates the value of the crop which nature
yields in the still wilder fields unimproved by man? The crop of
_English_ hay is carefully weighed, the moisture calculated, the
silicates and the potash; but in all dells and pond holes in the woods
and pastures and swamps grows a rich and various crop only unreaped by
man. Mine was, as it were, the connecting link between wild and
cultivated fields; as some states are civilized, and others
half-civilized, and others savage or barbarous, so my field was, though
not in a bad sense, a half-cultivated field. They were beans cheerfully
returning to their wild and primitive state that I cultivated, and my
hoe played the _Ranz des Vaches_ for them.

Near at hand, upon the topmost spray of a birch, sings the
brown-thrasher—or red mavis, as some love to call him—all the morning,
glad of your society, that would find out another farmer’s field if
yours were not here. While you are planting the seed, he cries,—“Drop
it, drop it,—cover it up, cover it up,—pull it up, pull it up, pull it
up.” But this was not corn, and so it was safe from such enemies as he.
You may wonder what his rigmarole, his amateur Paganini performances on
one string or on twenty, have to do with your planting, and yet prefer
it to leached ashes or plaster. It was a cheap sort of top dressing in
which I had entire faith.

As I drew a still fresher soil about the rows with my hoe, I disturbed
the ashes of unchronicled nations who in primeval years lived under
these heavens, and their small implements of war and hunting were
brought to the light of this modern day. They lay mingled with other
natural stones, some of which bore the marks of having been burned by
Indian fires, and some by the sun, and also bits of pottery and glass
brought hither by the recent cultivators of the soil. When my hoe
tinkled against the stones, that music echoed to the woods and the sky,
and was an accompaniment to my labor which yielded an instant and
immeasurable crop. It was no longer beans that I hoed, nor I that hoed
beans; and I remembered with as much pity as pride, if I remembered at
all, my acquaintances who had gone to the city to attend the oratorios.
The night-hawk circled overhead in the sunny afternoons—for I sometimes
made a day of it—like a mote in the eye, or in heaven’s eye, falling
from time to time with a swoop and a sound as if the heavens were rent,
torn at last to very rags and tatters, and yet a seamless cope
remained; small imps that fill the air and lay their eggs on the ground
on bare sand or rocks on the tops of hills, where few have found them;
graceful and slender like ripples caught up from the pond, as leaves
are raised by the wind to float in the heavens; such kindredship is in
Nature. The hawk is aerial brother of the wave which he sails over and
surveys, those his perfect air-inflated wings answering to the
elemental unfledged pinions of the sea. Or sometimes I watched a pair
of hen-hawks circling high in the sky, alternately soaring and
descending, approaching, and leaving one another, as if they were the
embodiment of my own thoughts. Or I was attracted by the passage of
wild pigeons from this wood to that, with a slight quivering winnowing
sound and carrier haste; or from under a rotten stump my hoe turned up
a sluggish portentous and outlandish spotted salamander, a trace of
Egypt and the Nile, yet our contemporary. When I paused to lean on my
hoe, these sounds and sights I heard and saw anywhere in the row, a
part of the inexhaustible entertainment which the country offers.

On gala days the town fires its great guns, which echo like popguns to
these woods, and some waifs of martial music occasionally penetrate
thus far. To me, away there in my bean-field at the other end of the
town, the big guns sounded as if a puffball had burst; and when there
was a military turnout of which I was ignorant, I have sometimes had a
vague sense all the day of some sort of itching and disease in the
horizon, as if some eruption would break out there soon, either
scarlatina or canker-rash, until at length some more favorable puff of
wind, making haste over the fields and up the Wayland road, brought me
information of the “trainers.” It seemed by the distant hum as if
somebody’s bees had swarmed, and that the neighbors, according to
Virgil’s advice, by a faint _tintinnabulum_ upon the most sonorous of
their domestic utensils, were endeavoring to call them down into the
hive again. And when the sound died quite away, and the hum had ceased,
and the most favorable breezes told no tale, I knew that they had got
the last drone of them all safely into the Middlesex hive, and that now
their minds were bent on the honey with which it was smeared.

I felt proud to know that the liberties of Massachusetts and of our
fatherland were in such safe keeping; and as I turned to my hoeing
again I was filled with an inexpressible confidence, and pursued my
labor cheerfully with a calm trust in the future.

When there were several bands of musicians, it sounded as if all the
village was a vast bellows, and all the buildings expanded and
collapsed alternately with a din. But sometimes it was a really noble
and inspiring strain that reached these woods, and the trumpet that
sings of fame, and I felt as if I could spit a Mexican with a good
relish,—for why should we always stand for trifles?—and looked round
for a woodchuck or a skunk to exercise my chivalry upon. These martial
strains seemed as far away as Palestine, and reminded me of a march of
crusaders in the horizon, with a slight tantivy and tremulous motion of
the elm-tree tops which overhang the village. This was one of the
_great_ days; though the sky had from my clearing only the same
everlastingly great look that it wears daily, and I saw no difference
in it.

It was a singular experience that long acquaintance which I cultivated
with beans, what with planting, and hoeing, and harvesting, and
threshing, and picking over and selling them,—the last was the hardest
of all,—I might add eating, for I did taste. I was determined to know
beans. When they were growing, I used to hoe from five o’clock in the
morning till noon, and commonly spent the rest of the day about other
affairs. Consider the intimate and curious acquaintance one makes with
various kinds of weeds,—it will bear some iteration in the account, for
there was no little iteration in the labor,—disturbing their delicate
organizations so ruthlessly, and making such invidious distinctions
with his hoe, levelling whole ranks of one species, and sedulously
cultivating another. That’s Roman wormwood,—that’s pigweed,—that’s
sorrel,—that’s piper-grass,—have at him, chop him up, turn his roots
upward to the sun, don’t let him have a fibre in the shade, if you do
he’ll turn himself t’other side up and be as green as a leek in two
days. A long war, not with cranes, but with weeds, those Trojans who
had sun and rain and dews on their side. Daily the beans saw me come to
their rescue armed with a hoe, and thin the ranks of their enemies,
filling up the trenches with weedy dead. Many a lusty crest-waving
Hector, that towered a whole foot above his crowding comrades, fell
before my weapon and rolled in the dust.

Those summer days which some of my contemporaries devoted to the fine
arts in Boston or Rome, and others to contemplation in India, and
others to trade in London or New York, I thus, with the other farmers
of New England, devoted to husbandry. Not that I wanted beans to eat,
for I am by nature a Pythagorean, so far as beans are concerned,
whether they mean porridge or voting, and exchanged them for rice; but,
perchance, as some must work in fields if only for the sake of tropes
and expression, to serve a parable-maker one day. It was on the whole a
rare amusement, which, continued too long, might have become a
dissipation. Though I gave them no manure, and did not hoe them all
once, I hoed them unusually well as far as I went, and was paid for it
in the end, “there being in truth,” as Evelyn says, “no compost or
lætation whatsoever comparable to this continual motion, repastination,
and turning of the mould with the spade.” “The earth,” he adds
elsewhere, “especially if fresh, has a certain magnetism in it, by
which it attracts the salt, power, or virtue (call it either) which
gives it life, and is the logic of all the labor and stir we keep about
it, to sustain us; all dungings and other sordid temperings being but
the vicars succedaneous to this improvement.” Moreover, this being one
of those “worn-out and exhausted lay fields which enjoy their sabbath,”
had perchance, as Sir Kenelm Digby thinks likely, attracted “vital
spirits” from the air. I harvested twelve bushels of beans.

But to be more particular, for it is complained that Mr. Coleman has
reported chiefly the expensive experiments of gentlemen farmers, my
outgoes were,—


    For a hoe,.................................. $ 0.54
    Ploughing, harrowing, and furrowing,.........  7.50  Too much.
    Beans for seed,..............................  3.12½
    Potatoes for seed,...........................  1.33
    Peas for seed,...............................  0.40
    Turnip seed,.................................  0.06
    White line for crow fence,...................  0.02
    Horse cultivator and boy three hours,........  1.00
    Horse and cart to get crop,..................  0.75
                                                ————
        In all,................................. $14.72½

My income was (patrem familias vendacem, non emacem esse oportet),
from

    Nine bushels and twelve quarts of beans sold,. $16.94
    Five    "    large potatoes,.................... 2.50
    Nine    "    small,............................. 2.25
    Grass,.......................................... 1.00
    Stalks,......................................... 0.75
                                                  ————
        In all,................................... $23.44
    Leaving a pecuniary profit,
        as I have elsewhere said, of..............  $8.71½.

This is the result of my experience in raising beans. Plant the common
small white bush bean about the first of June, in rows three feet by
eighteen inches apart, being careful to select fresh round and unmixed
seed. First look out for worms, and supply vacancies by planting anew.
Then look out for woodchucks, if it is an exposed place, for they will
nibble off the earliest tender leaves almost clean as they go; and
again, when the young tendrils make their appearance, they have notice
of it, and will shear them off with both buds and young pods, sitting
erect like a squirrel. But above all harvest as early as possible, if
you would escape frosts and have a fair and salable crop; you may save
much loss by this means.

This further experience also I gained. I said to myself, I will not
plant beans and corn with so much industry another summer, but such
seeds, if the seed is not lost, as sincerity, truth, simplicity, faith,
innocence, and the like, and see if they will not grow in this soil,
even with less toil and manurance, and sustain me, for surely it has
not been exhausted for these crops. Alas! I said this to myself; but
now another summer is gone, and another, and another, and I am obliged
to say to you, Reader, that the seeds which I planted, if indeed they
_were_ the seeds of those virtues, were wormeaten or had lost their
vitality, and so did not come up. Commonly men will only be brave as
their fathers were brave, or timid. This generation is very sure to
plant corn and beans each new year precisely as the Indians did
centuries ago and taught the first settlers to do, as if there were a
fate in it. I saw an old man the other day, to my astonishment, making
the holes with a hoe for the seventieth time at least, and not for
himself to lie down in! But why should not the New Englander try new
adventures, and not lay so much stress on his grain, his potato and
grass crop, and his orchards,—raise other crops than these? Why concern
ourselves so much about our beans for seed, and not be concerned at all
about a new generation of men? We should really be fed and cheered if
when we met a man we were sure to see that some of the qualities which
I have named, which we all prize more than those other productions, but
which are for the most part broadcast and floating in the air, had
taken root and grown in him. Here comes such a subtile and ineffable
quality, for instance, as truth or justice, though the slightest amount
or new variety of it, along the road. Our ambassadors should be
instructed to send home such seeds as these, and Congress help to
distribute them over all the land. We should never stand upon ceremony
with sincerity. We should never cheat and insult and banish one another
by our meanness, if there were present the kernel of worth and
friendliness. We should not meet thus in haste. Most men I do not meet
at all, for they seem not to have time; they are busy about their
beans. We would not deal with a man thus plodding ever, leaning on a
hoe or a spade as a staff between his work, not as a mushroom, but
partially risen out of the earth, something more than erect, like
swallows alighted and walking on the ground:—

     “And as he spake, his wings would now and then
     Spread, as he meant to fly, then close again,”

so that we should suspect that we might be conversing with an angel.
Bread may not always nourish us; but it always does us good, it even
takes stiffness out of our joints, and makes us supple and buoyant,
when we knew not what ailed us, to recognize any generosity in man or
Nature, to share any unmixed and heroic joy.

Ancient poetry and mythology suggest, at least, that husbandry was once
a sacred art; but it is pursued with irreverent haste and heedlessness
by us, our object being to have large farms and large crops merely. We
have no festival, nor procession, nor ceremony, not excepting our
Cattle-shows and so called Thanksgivings, by which the farmer expresses
a sense of the sacredness of his calling, or is reminded of its sacred
origin. It is the premium and the feast which tempt him. He sacrifices
not to Ceres and the Terrestrial Jove, but to the infernal Plutus
rather. By avarice and selfishness, and a grovelling habit, from which
none of us is free, of regarding the soil as property, or the means of
acquiring property chiefly, the landscape is deformed, husbandry is
degraded with us, and the farmer leads the meanest of lives. He knows
Nature but as a robber. Cato says that the profits of agriculture are
particularly pious or just, (_maximeque pius quæstus_), and according
to Varro the old Romans “called the same earth Mother and Ceres, and
thought that they who cultivated it led a pious and useful life, and
that they alone were left of the race of King Saturn.”

We are wont to forget that the sun looks on our cultivated fields and
on the prairies and forests without distinction. They all reflect and
absorb his rays alike, and the former make but a small part of the
glorious picture which he beholds in his daily course. In his view the
earth is all equally cultivated like a garden. Therefore we should
receive the benefit of his light and heat with a corresponding trust
and magnanimity. What though I value the seed of these beans, and
harvest that in the fall of the year? This broad field which I have
looked at so long looks not to me as the principal cultivator, but away
from me to influences more genial to it, which water and make it green.
These beans have results which are not harvested by me. Do they not
grow for woodchucks partly? The ear of wheat (in Latin _spica_,
obsoletely _speca_, from _spe_, hope) should not be the only hope of
the husbandman; its kernel or grain (_granum_ from _gerendo_, bearing)
is not all that it bears. How, then, can our harvest fail? Shall I not
rejoice also at the abundance of the weeds whose seeds are the granary
of the birds? It matters little comparatively whether the fields fill
the farmer’s barns. The true husbandman will cease from anxiety, as the
squirrels manifest no concern whether the woods will bear chestnuts
this year or not, and finish his labor with every day, relinquishing
all claim to the produce of his fields, and sacrificing in his mind not
only his first but his last fruits also.


The Village

After hoeing, or perhaps reading and writing, in the forenoon, I
usually bathed again in the pond, swimming across one of its coves for
a stint, and washed the dust of labor from my person, or smoothed out
the last wrinkle which study had made, and for the afternoon was
absolutely free. Every day or two I strolled to the village to hear
some of the gossip which is incessantly going on there, circulating
either from mouth to mouth, or from newspaper to newspaper, and which,
taken in homœopathic doses, was really as refreshing in its way as the
rustle of leaves and the peeping of frogs. As I walked in the woods to
see the birds and squirrels, so I walked in the village to see the men
and boys; instead of the wind among the pines I heard the carts rattle.
In one direction from my house there was a colony of muskrats in the
river meadows; under the grove of elms and buttonwoods in the other
horizon was a village of busy men, as curious to me as if they had been
prairie dogs, each sitting at the mouth of its burrow, or running over
to a neighbor’s to gossip. I went there frequently to observe their
habits. The village appeared to me a great news room; and on one side,
to support it, as once at Redding & Company’s on State Street, they
kept nuts and raisins, or salt and meal and other groceries. Some have
such a vast appetite for the former commodity, that is, the news, and
such sound digestive organs, that they can sit forever in public
avenues without stirring, and let it simmer and whisper through them
like the Etesian winds, or as if inhaling ether, it only producing
numbness and insensibility to pain,—otherwise it would often be painful
to hear,—without affecting the consciousness. I hardly ever failed,
when I rambled through the village, to see a row of such worthies,
either sitting on a ladder sunning themselves, with their bodies
inclined forward and their eyes glancing along the line this way and
that, from time to time, with a voluptuous expression, or else leaning
against a barn with their hands in their pockets, like caryatides, as
if to prop it up. They, being commonly out of doors, heard whatever was
in the wind. These are the coarsest mills, in which all gossip is first
rudely digested or cracked up before it is emptied into finer and more
delicate hoppers within doors. I observed that the vitals of the
village were the grocery, the bar-room, the post-office, and the bank;
and, as a necessary part of the machinery, they kept a bell, a big gun,
and a fire-engine, at convenient places; and the houses were so
arranged as to make the most of mankind, in lanes and fronting one
another, so that every traveller had to run the gantlet, and every man,
woman, and child might get a lick at him. Of course, those who were
stationed nearest to the head of the line, where they could most see
and be seen, and have the first blow at him, paid the highest prices
for their places; and the few straggling inhabitants in the outskirts,
where long gaps in the line began to occur, and the traveller could get
over walls or turn aside into cow paths, and so escape, paid a very
slight ground or window tax. Signs were hung out on all sides to allure
him; some to catch him by the appetite, as the tavern and victualling
cellar; some by the fancy, as the dry goods store and the jeweller’s;
and others by the hair or the feet or the skirts, as the barber, the
shoemaker, or the tailor. Besides, there was a still more terrible
standing invitation to call at every one of these houses, and company
expected about these times. For the most part I escaped wonderfully
from these dangers, either by proceeding at once boldly and without
deliberation to the goal, as is recommended to those who run the
gantlet, or by keeping my thoughts on high things, like Orpheus, who,
“loudly singing the praises of the gods to his lyre, drowned the voices
of the Sirens, and kept out of danger.” Sometimes I bolted suddenly,
and nobody could tell my whereabouts, for I did not stand much about
gracefulness, and never hesitated at a gap in a fence. I was even
accustomed to make an irruption into some houses, where I was well
entertained, and after learning the kernels and very last sieve-ful of
news, what had subsided, the prospects of war and peace, and whether
the world was likely to hold together much longer, I was let out
through the rear avenues, and so escaped to the woods again.

It was very pleasant, when I stayed late in town, to launch myself into
the night, especially if it was dark and tempestuous, and set sail from
some bright village parlor or lecture room, with a bag of rye or Indian
meal upon my shoulder, for my snug harbor in the woods, having made all
tight without and withdrawn under hatches with a merry crew of
thoughts, leaving only my outer man at the helm, or even tying up the
helm when it was plain sailing. I had many a genial thought by the
cabin fire “as I sailed.” I was never cast away nor distressed in any
weather, though I encountered some severe storms. It is darker in the
woods, even in common nights, than most suppose. I frequently had to
look up at the opening between the trees above the path in order to
learn my route, and, where there was no cart-path, to feel with my feet
the faint track which I had worn, or steer by the known relation of
particular trees which I felt with my hands, passing between two pines
for instance, not more than eighteen inches apart, in the midst of the
woods, invariably, in the darkest night. Sometimes, after coming home
thus late in a dark and muggy night, when my feet felt the path which
my eyes could not see, dreaming and absent-minded all the way, until I
was aroused by having to raise my hand to lift the latch, I have not
been able to recall a single step of my walk, and I have thought that
perhaps my body would find its way home if its master should forsake
it, as the hand finds its way to the mouth without assistance. Several
times, when a visitor chanced to stay into evening, and it proved a
dark night, I was obliged to conduct him to the cart-path in the rear
of the house, and then point out to him the direction he was to pursue,
and in keeping which he was to be guided rather by his feet than his
eyes. One very dark night I directed thus on their way two young men
who had been fishing in the pond. They lived about a mile off through
the woods, and were quite used to the route. A day or two after one of
them told me that they wandered about the greater part of the night,
close by their own premises, and did not get home till toward morning,
by which time, as there had been several heavy showers in the mean
while, and the leaves were very wet, they were drenched to their skins.
I have heard of many going astray even in the village streets, when the
darkness was so thick that you could cut it with a knife, as the saying
is. Some who live in the outskirts, having come to town a-shopping in
their wagons, have been obliged to put up for the night; and gentlemen
and ladies making a call have gone half a mile out of their way,
feeling the sidewalk only with their feet, and not knowing when they
turned. It is a surprising and memorable, as well as valuable
experience, to be lost in the woods any time. Often in a snow storm,
even by day, one will come out upon a well-known road and yet find it
impossible to tell which way leads to the village. Though he knows that
he has travelled it a thousand times, he cannot recognize a feature in
it, but it is as strange to him as if it were a road in Siberia. By
night, of course, the perplexity is infinitely greater. In our most
trivial walks, we are constantly, though unconsciously, steering like
pilots by certain well-known beacons and headlands, and if we go beyond
our usual course we still carry in our minds the bearing of some
neighboring cape; and not till we are completely lost, or turned
round,—for a man needs only to be turned round once with his eyes shut
in this world to be lost,—do we appreciate the vastness and strangeness
of Nature. Every man has to learn the points of compass again as often
as he awakes, whether from sleep or any abstraction. Not till we are
lost, in other words not till we have lost the world, do we begin to
find ourselves, and realize where we are and the infinite extent of our
relations.

One afternoon, near the end of the first summer, when I went to the
village to get a shoe from the cobbler’s, I was seized and put into
jail, because, as I have elsewhere related, I did not pay a tax to, or
recognize the authority of, the state which buys and sells men, women,
and children, like cattle at the door of its senate-house. I had gone
down to the woods for other purposes. But, wherever a man goes, men
will pursue and paw him with their dirty institutions, and, if they
can, constrain him to belong to their desperate odd-fellow society. It
is true, I might have resisted forcibly with more or less effect, might
have run “amok” against society; but I preferred that society should
run “amok” against me, it being the desperate party. However, I was
released the next day, obtained my mended shoe, and returned to the
woods in season to get my dinner of huckleberries on Fair-Haven Hill. I
was never molested by any person but those who represented the state. I
had no lock nor bolt but for the desk which held my papers, not even a
nail to put over my latch or windows. I never fastened my door night or
day, though I was to be absent several days; not even when the next
fall I spent a fortnight in the woods of Maine. And yet my house was
more respected than if it had been surrounded by a file of soldiers.
The tired rambler could rest and warm himself by my fire, the literary
amuse himself with the few books on my table, or the curious, by
opening my closet door, see what was left of my dinner, and what
prospect I had of a supper. Yet, though many people of every class came
this way to the pond, I suffered no serious inconvenience from these
sources, and I never missed anything but one small book, a volume of
Homer, which perhaps was improperly gilded, and this I trust a soldier
of our camp has found by this time. I am convinced, that if all men
were to live as simply as I then did, thieving and robbery would be
unknown. These take place only in communities where some have got more
than is sufficient while others have not enough. The Pope’s Homers
would soon get properly distributed.—

                     “Nec bella fuerunt,
     Faginus astabat dum scyphus ante dapes.”

                     “Nor wars did men molest,
     When only beechen bowls were in request.”

“You who govern public affairs, what need have you to employ
punishments? Love virtue, and the people will be virtuous. The virtues
of a superior man are like the wind; the virtues of a common man are
like the grass; the grass, when the wind passes over it, bends.”


The Ponds

Sometimes, having had a surfeit of human society and gossip, and worn
out all my village friends, I rambled still farther westward than I
habitually dwell, into yet more unfrequented parts of the town, “to
fresh woods and pastures new,” or, while the sun was setting, made my
supper of huckleberries and blueberries on Fair Haven Hill, and laid up
a store for several days. The fruits do not yield their true flavor to
the purchaser of them, nor to him who raises them for the market. There
is but one way to obtain it, yet few take that way. If you would know
the flavor of huckleberries, ask the cow-boy or the partridge. It is a
vulgar error to suppose that you have tasted huckleberries who never
plucked them. A huckleberry never reaches Boston; they have not been
known there since they grew on her three hills. The ambrosial and
essential part of the fruit is lost with the bloom which is rubbed off
in the market cart, and they become mere provender. As long as Eternal
Justice reigns, not one innocent huckleberry can be transported thither
from the country’s hills.

Occasionally, after my hoeing was done for the day, I joined some
impatient companion who had been fishing on the pond since morning, as
silent and motionless as a duck or a floating leaf, and, after
practising various kinds of philosophy, had concluded commonly, by the
time I arrived, that he belonged to the ancient sect of Cœnobites.
There was one older man, an excellent fisher and skilled in all kinds
of woodcraft, who was pleased to look upon my house as a building
erected for the convenience of fishermen; and I was equally pleased
when he sat in my doorway to arrange his lines. Once in a while we sat
together on the pond, he at one end of the boat, and I at the other;
but not many words passed between us, for he had grown deaf in his
later years, but he occasionally hummed a psalm, which harmonized well
enough with my philosophy. Our intercourse was thus altogether one of
unbroken harmony, far more pleasing to remember than if it had been
carried on by speech. When, as was commonly the case, I had none to
commune with, I used to raise the echoes by striking with a paddle on
the side of my boat, filling the surrounding woods with circling and
dilating sound, stirring them up as the keeper of a menagerie his wild
beasts, until I elicited a growl from every wooded vale and hill-side.

In warm evenings I frequently sat in the boat playing the flute, and
saw the perch, which I seemed to have charmed, hovering around me, and
the moon travelling over the ribbed bottom, which was strewed with the
wrecks of the forest. Formerly I had come to this pond adventurously,
from time to time, in dark summer nights, with a companion, and making
a fire close to the water’s edge, which we thought attracted the
fishes, we caught pouts with a bunch of worms strung on a thread; and
when we had done, far in the night, threw the burning brands high into
the air like skyrockets, which, coming down into the pond, were
quenched with a loud hissing, and we were suddenly groping in total
darkness. Through this, whistling a tune, we took our way to the haunts
of men again. But now I had made my home by the shore.

Sometimes, after staying in a village parlor till the family had all
retired, I have returned to the woods, and, partly with a view to the
next day’s dinner, spent the hours of midnight fishing from a boat by
moonlight, serenaded by owls and foxes, and hearing, from time to time,
the creaking note of some unknown bird close at hand. These experiences
were very memorable and valuable to me,—anchored in forty feet of
water, and twenty or thirty rods from the shore, surrounded sometimes
by thousands of small perch and shiners, dimpling the surface with
their tails in the moonlight, and communicating by a long flaxen line
with mysterious nocturnal fishes which had their dwelling forty feet
below, or sometimes dragging sixty feet of line about the pond as I
drifted in the gentle night breeze, now and then feeling a slight
vibration along it, indicative of some life prowling about its
extremity, of dull uncertain blundering purpose there, and slow to make
up its mind. At length you slowly raise, pulling hand over hand, some
horned pout squeaking and squirming to the upper air. It was very
queer, especially in dark nights, when your thoughts had wandered to
vast and cosmogonal themes in other spheres, to feel this faint jerk,
which came to interrupt your dreams and link you to Nature again. It
seemed as if I might next cast my line upward into the air, as well as
downward into this element, which was scarcely more dense. Thus I
caught two fishes as it were with one hook.



The scenery of Walden is on a humble scale, and, though very beautiful,
does not approach to grandeur, nor can it much concern one who has not
long frequented it or lived by its shore; yet this pond is so
remarkable for its depth and purity as to merit a particular
description. It is a clear and deep green well, half a mile long and a
mile and three quarters in circumference, and contains about sixty-one
and a half acres; a perennial spring in the midst of pine and oak
woods, without any visible inlet or outlet except by the clouds and
evaporation. The surrounding hills rise abruptly from the water to the
height of forty to eighty feet, though on the south-east and east they
attain to about one hundred and one hundred and fifty feet
respectively, within a quarter and a third of a mile. They are
exclusively woodland. All our Concord waters have two colors at least;
one when viewed at a distance, and another, more proper, close at hand.
The first depends more on the light, and follows the sky. In clear
weather, in summer, they appear blue at a little distance, especially
if agitated, and at a great distance all appear alike. In stormy
weather they are sometimes of a dark slate color. The sea, however, is
said to be blue one day and green another without any perceptible
change in the atmosphere. I have seen our river, when, the landscape
being covered with snow, both water and ice were almost as green as
grass. Some consider blue “to be the color of pure water, whether
liquid or solid.” But, looking directly down into our waters from a
boat, they are seen to be of very different colors. Walden is blue at
one time and green at another, even from the same point of view. Lying
between the earth and the heavens, it partakes of the color of both.
Viewed from a hill-top it reflects the color of the sky; but near at
hand it is of a yellowish tint next the shore where you can see the
sand, then a light green, which gradually deepens to a uniform dark
green in the body of the pond. In some lights, viewed even from a
hill-top, it is of a vivid green next the shore. Some have referred
this to the reflection of the verdure; but it is equally green there
against the railroad sand-bank, and in the spring, before the leaves
are expanded, and it may be simply the result of the prevailing blue
mixed with the yellow of the sand. Such is the color of its iris. This
is that portion, also, where in the spring, the ice being warmed by the
heat of the sun reflected from the bottom, and also transmitted through
the earth, melts first and forms a narrow canal about the still frozen
middle. Like the rest of our waters, when much agitated, in clear
weather, so that the surface of the waves may reflect the sky at the
right angle, or because there is more light mixed with it, it appears
at a little distance of a darker blue than the sky itself; and at such
a time, being on its surface, and looking with divided vision, so as to
see the reflection, I have discerned a matchless and indescribable
light blue, such as watered or changeable silks and sword blades
suggest, more cerulean than the sky itself, alternating with the
original dark green on the opposite sides of the waves, which last
appeared but muddy in comparison. It is a vitreous greenish blue, as I
remember it, like those patches of the winter sky seen through cloud
vistas in the west before sundown. Yet a single glass of its water held
up to the light is as colorless as an equal quantity of air. It is well
known that a large plate of glass will have a green tint, owing, as the
makers say, to its “body,” but a small piece of the same will be
colorless. How large a body of Walden water would be required to
reflect a green tint I have never proved. The water of our river is
black or a very dark brown to one looking directly down on it, and,
like that of most ponds, imparts to the body of one bathing in it a
yellowish tinge; but this water is of such crystalline purity that the
body of the bather appears of an alabaster whiteness, still more
unnatural, which, as the limbs are magnified and distorted withal,
produces a monstrous effect, making fit studies for a Michael Angelo.

The water is so transparent that the bottom can easily be discerned at
the depth of twenty-five or thirty feet. Paddling over it, you may see,
many feet beneath the surface the schools of perch and shiners, perhaps
only an inch long, yet the former easily distinguished by their
transverse bars, and you think that they must be ascetic fish that find
a subsistence there. Once, in the winter, many years ago, when I had
been cutting holes through the ice in order to catch pickerel, as I
stepped ashore I tossed my axe back on to the ice, but, as if some evil
genius had directed it, it slid four or five rods directly into one of
the holes, where the water was twenty-five feet deep. Out of curiosity,
I lay down on the ice and looked through the hole, until I saw the axe
a little on one side, standing on its head, with its helve erect and
gently swaying to and fro with the pulse of the pond; and there it
might have stood erect and swaying till in the course of time the
handle rotted off, if I had not disturbed it. Making another hole
directly over it with an ice chisel which I had, and cutting down the
longest birch which I could find in the neighborhood with my knife, I
made a slip-noose, which I attached to its end, and, letting it down
carefully, passed it over the knob of the handle, and drew it by a line
along the birch, and so pulled the axe out again.

The shore is composed of a belt of smooth rounded white stones like
paving stones, excepting one or two short sand beaches, and is so steep
that in many places a single leap will carry you into water over your
head; and were it not for its remarkable transparency, that would be
the last to be seen of its bottom till it rose on the opposite side.
Some think it is bottomless. It is nowhere muddy, and a casual observer
would say that there were no weeds at all in it; and of noticeable
plants, except in the little meadows recently overflowed, which do not
properly belong to it, a closer scrutiny does not detect a flag nor a
bulrush, nor even a lily, yellow or white, but only a few small
heart-leaves and potamogetons, and perhaps a water-target or two; all
which however a bather might not perceive; and these plants are clean
and bright like the element they grow in. The stones extend a rod or
two into the water, and then the bottom is pure sand, except in the
deepest parts, where there is usually a little sediment, probably from
the decay of the leaves which have been wafted on to it so many
successive falls, and a bright green weed is brought up on anchors even
in midwinter.

We have one other pond just like this, White Pond, in Nine Acre Corner,
about two and a half miles westerly; but, though I am acquainted with
most of the ponds within a dozen miles of this centre I do not know a
third of this pure and well-like character. Successive nations
perchance have drank at, admired, and fathomed it, and passed away, and
still its water is green and pellucid as ever. Not an intermitting
spring! Perhaps on that spring morning when Adam and Eve were driven
out of Eden Walden Pond was already in existence, and even then
breaking up in a gentle spring rain accompanied with mist and a
southerly wind, and covered with myriads of ducks and geese, which had
not heard of the fall, when still such pure lakes sufficed them. Even
then it had commenced to rise and fall, and had clarified its waters
and colored them of the hue they now wear, and obtained a patent of
heaven to be the only Walden Pond in the world and distiller of
celestial dews. Who knows in how many unremembered nations’ literatures
this has been the Castalian Fountain? or what nymphs presided over it
in the Golden Age? It is a gem of the first water which Concord wears
in her coronet.

Yet perchance the first who came to this well have left some trace of
their footsteps. I have been surprised to detect encircling the pond,
even where a thick wood has just been cut down on the shore, a narrow
shelf-like path in the steep hill-side, alternately rising and falling,
approaching and receding from the water’s edge, as old probably as the
race of man here, worn by the feet of aboriginal hunters, and still
from time to time unwittingly trodden by the present occupants of the
land. This is particularly distinct to one standing on the middle of
the pond in winter, just after a light snow has fallen, appearing as a
clear undulating white line, unobscured by weeds and twigs, and very
obvious a quarter of a mile off in many places where in summer it is
hardly distinguishable close at hand. The snow reprints it, as it were,
in clear white type alto-relievo. The ornamented grounds of villas
which will one day be built here may still preserve some trace of this.

The pond rises and falls, but whether regularly or not, and within what
period, nobody knows, though, as usual, many pretend to know. It is
commonly higher in the winter and lower in the summer, though not
corresponding to the general wet and dryness. I can remember when it
was a foot or two lower, and also when it was at least five feet
higher, than when I lived by it. There is a narrow sand-bar running
into it, with very deep water on one side, on which I helped boil a
kettle of chowder, some six rods from the main shore, about the year
1824, which it has not been possible to do for twenty-five years; and
on the other hand, my friends used to listen with incredulity when I
told them, that a few years later I was accustomed to fish from a boat
in a secluded cove in the woods, fifteen rods from the only shore they
knew, which place was long since converted into a meadow. But the pond
has risen steadily for two years, and now, in the summer of ’52, is
just five feet higher than when I lived there, or as high as it was
thirty years ago, and fishing goes on again in the meadow. This makes a
difference of level, at the outside, of six or seven feet; and yet the
water shed by the surrounding hills is insignificant in amount, and
this overflow must be referred to causes which affect the deep springs.
This same summer the pond has begun to fall again. It is remarkable
that this fluctuation, whether periodical or not, appears thus to
require many years for its accomplishment. I have observed one rise and
a part of two falls, and I expect that a dozen or fifteen years hence
the water will again be as low as I have ever known it. Flint’s Pond, a
mile eastward, allowing for the disturbance occasioned by its inlets
and outlets, and the smaller intermediate ponds also, sympathize with
Walden, and recently attained their greatest height at the same time
with the latter. The same is true, as far as my observation goes, of
White Pond.

This rise and fall of Walden at long intervals serves this use at
least; the water standing at this great height for a year or more,
though it makes it difficult to walk round it, kills the shrubs and
trees which have sprung up about its edge since the last rise,
pitch-pines, birches, alders, aspens, and others, and, falling again,
leaves an unobstructed shore; for, unlike many ponds and all waters
which are subject to a daily tide, its shore is cleanest when the water
is lowest. On the side of the pond next my house, a row of pitch pines
fifteen feet high has been killed and tipped over as if by a lever, and
thus a stop put to their encroachments; and their size indicates how
many years have elapsed since the last rise to this height. By this
fluctuation the pond asserts its title to a shore, and thus the _shore_
is _shorn_, and the trees cannot hold it by right of possession. These
are the lips of the lake on which no beard grows. It licks its chaps
from time to time. When the water is at its height, the alders,
willows, and maples send forth a mass of fibrous red roots several feet
long from all sides of their stems in the water, and to the height of
three or four feet from the ground, in the effort to maintain
themselves; and I have known the high-blueberry bushes about the shore,
which commonly produce no fruit, bear an abundant crop under these
circumstances.

Some have been puzzled to tell how the shore became so regularly paved.
My townsmen have all heard the tradition, the oldest people tell me
that they heard it in their youth, that anciently the Indians were
holding a pow-wow upon a hill here, which rose as high into the heavens
as the pond now sinks deep into the earth, and they used much
profanity, as the story goes, though this vice is one of which the
Indians were never guilty, and while they were thus engaged the hill
shook and suddenly sank, and only one old squaw, named Walden, escaped,
and from her the pond was named. It has been conjectured that when the
hill shook these stones rolled down its side and became the present
shore. It is very certain, at any rate, that once there was no pond
here, and now there is one; and this Indian fable does not in any
respect conflict with the account of that ancient settler whom I have
mentioned, who remembers so well when he first came here with his
divining rod, saw a thin vapor rising from the sward, and the hazel
pointed steadily downward, and he concluded to dig a well here. As for
the stones, many still think that they are hardly to be accounted for
by the action of the waves on these hills; but I observe that the
surrounding hills are remarkably full of the same kind of stones, so
that they have been obliged to pile them up in walls on both sides of
the railroad cut nearest the pond; and, moreover, there are most stones
where the shore is most abrupt; so that, unfortunately, it is no longer
a mystery to me. I detect the paver. If the name was not derived from
that of some English locality,—Saffron Walden, for instance,—one might
suppose that it was called originally _Walled-in_ Pond.

The pond was my well ready dug. For four months in the year its water
is as cold as it is pure at all times; and I think that it is then as
good as any, if not the best, in the town. In the winter, all water
which is exposed to the air is colder than springs and wells which are
protected from it. The temperature of the pond water which had stood in
the room where I sat from five o’clock in the afternoon till noon the
next day, the sixth of March, 1846, the thermometer having been up to
65° or 70° some of the time, owing partly to the sun on the roof, was
42°, or one degree colder than the water of one of the coldest wells in
the village just drawn. The temperature of the Boiling Spring the same
day was 45°, or the warmest of any water tried, though it is the
coldest that I know of in summer, when, beside, shallow and stagnant
surface water is not mingled with it. Moreover, in summer, Walden never
becomes so warm as most water which is exposed to the sun, on account
of its depth. In the warmest weather I usually placed a pailful in my
cellar, where it became cool in the night, and remained so during the
day; though I also resorted to a spring in the neighborhood. It was as
good when a week old as the day it was dipped, and had no taste of the
pump. Whoever camps for a week in summer by the shore of a pond, needs
only bury a pail of water a few feet deep in the shade of his camp to
be independent of the luxury of ice.

There have been caught in Walden pickerel, one weighing seven pounds,
to say nothing of another which carried off a reel with great velocity,
which the fisherman safely set down at eight pounds because he did not
see him, perch and pouts, some of each weighing over two pounds,
shiners, chivins or roach (_Leuciscus pulchellus_), a very few breams,
and a couple of eels, one weighing four pounds,—I am thus particular
because the weight of a fish is commonly its only title to fame, and
these are the only eels I have heard of here;—also, I have a faint
recollection of a little fish some five inches long, with silvery sides
and a greenish back, somewhat dace-like in its character, which I
mention here chiefly to link my facts to fable. Nevertheless, this pond
is not very fertile in fish. Its pickerel, though not abundant, are its
chief boast. I have seen at one time lying on the ice pickerel of at
least three different kinds; a long and shallow one, steel-colored,
most like those caught in the river; a bright golden kind, with
greenish reflections and remarkably deep, which is the most common
here; and another, golden-colored, and shaped like the last, but
peppered on the sides with small dark brown or black spots, intermixed
with a few faint blood-red ones, very much like a trout. The specific
name _reticulatus_ would not apply to this; it should be _guttatus_
rather. These are all very firm fish, and weigh more than their size
promises. The shiners, pouts, and perch also, and indeed all the fishes
which inhabit this pond, are much cleaner, handsomer, and firmer
fleshed than those in the river and most other ponds, as the water is
purer, and they can easily be distinguished from them. Probably many
ichthyologists would make new varieties of some of them. There are also
a clean race of frogs and tortoises, and a few muscels in it; muskrats
and minks leave their traces about it, and occasionally a travelling
mud-turtle visits it. Sometimes, when I pushed off my boat in the
morning, I disturbed a great mud-turtle which had secreted himself
under the boat in the night. Ducks and geese frequent it in the spring
and fall, the white-bellied swallows (_Hirundo bicolor_) skim over it,
and the peetweets (_Totanus macularius_) “teter” along its stony shores
all summer. I have sometimes disturbed a fishhawk sitting on a
white-pine over the water; but I doubt if it is ever profaned by the
wing of a gull, like Fair Haven. At most, it tolerates one annual loon.
These are all the animals of consequence which frequent it now.

You may see from a boat, in calm weather, near the sandy eastern shore,
where the water is eight or ten feet deep, and also in some other parts
of the pond, some circular heaps half a dozen feet in diameter by a
foot in height, consisting of small stones less than a hen’s egg in
size, where all around is bare sand. At first you wonder if the Indians
could have formed them on the ice for any purpose, and so, when the ice
melted, they sank to the bottom; but they are too regular and some of
them plainly too fresh for that. They are similar to those found in
rivers; but as there are no suckers nor lampreys here, I know not by
what fish they could be made. Perhaps they are the nests of the chivin.
These lend a pleasing mystery to the bottom.

The shore is irregular enough not to be monotonous. I have in my mind’s
eye the western indented with deep bays, the bolder northern, and the
beautifully scalloped southern shore, where successive capes overlap
each other and suggest unexplored coves between. The forest has never
so good a setting, nor is so distinctly beautiful, as when seen from
the middle of a small lake amid hills which rise from the water’s edge;
for the water in which it is reflected not only makes the best
foreground in such a case, but, with its winding shore, the most
natural and agreeable boundary to it. There is no rawness nor
imperfection in its edge there, as where the axe has cleared a part, or
a cultivated field abuts on it. The trees have ample room to expand on
the water side, and each sends forth its most vigorous branch in that
direction. There Nature has woven a natural selvage, and the eye rises
by just gradations from the low shrubs of the shore to the highest
trees. There are few traces of man’s hand to be seen. The water laves
the shore as it did a thousand years ago.

A lake is the landscape’s most beautiful and expressive feature. It is
earth’s eye; looking into which the beholder measures the depth of his
own nature. The fluviatile trees next the shore are the slender
eyelashes which fringe it, and the wooded hills and cliffs around are
its overhanging brows.

Standing on the smooth sandy beach at the east end of the pond, in a
calm September afternoon, when a slight haze makes the opposite shore
line indistinct, I have seen whence came the expression, “the glassy
surface of a lake.” When you invert your head, it looks like a thread
of finest gossamer stretched across the valley, and gleaming against
the distant pine woods, separating one stratum of the atmosphere from
another. You would think that you could walk dry under it to the
opposite hills, and that the swallows which skim over might perch on
it. Indeed, they sometimes dive below this line, as it were by mistake,
and are undeceived. As you look over the pond westward you are obliged
to employ both your hands to defend your eyes against the reflected as
well as the true sun, for they are equally bright; and if, between the
two, you survey its surface critically, it is literally as smooth as
glass, except where the skater insects, at equal intervals scattered
over its whole extent, by their motions in the sun produce the finest
imaginable sparkle on it, or, perchance, a duck plumes itself, or, as I
have said, a swallow skims so low as to touch it. It may be that in the
distance a fish describes an arc of three or four feet in the air, and
there is one bright flash where it emerges, and another where it
strikes the water; sometimes the whole silvery arc is revealed; or here
and there, perhaps, is a thistle-down floating on its surface, which
the fishes dart at and so dimple it again. It is like molten glass
cooled but not congealed, and the few motes in it are pure and
beautiful like the imperfections in glass. You may often detect a yet
smoother and darker water, separated from the rest as if by an
invisible cobweb, boom of the water nymphs, resting on it. From a
hill-top you can see a fish leap in almost any part; for not a pickerel
or shiner picks an insect from this smooth surface but it manifestly
disturbs the equilibrium of the whole lake. It is wonderful with what
elaborateness this simple fact is advertised,—this piscine murder will
out,—and from my distant perch I distinguish the circling undulations
when they are half a dozen rods in diameter. You can even detect a
water-bug (_Gyrinus_) ceaselessly progressing over the smooth surface a
quarter of a mile off; for they furrow the water slightly, making a
conspicuous ripple bounded by two diverging lines, but the skaters
glide over it without rippling it perceptibly. When the surface is
considerably agitated there are no skaters nor water-bugs on it, but
apparently, in calm days, they leave their havens and adventurously
glide forth from the shore by short impulses till they completely cover
it. It is a soothing employment, on one of those fine days in the fall
when all the warmth of the sun is fully appreciated, to sit on a stump
on such a height as this, overlooking the pond, and study the dimpling
circles which are incessantly inscribed on its otherwise invisible
surface amid the reflected skies and trees. Over this great expanse
there is no disturbance but it is thus at once gently smoothed away and
assuaged, as, when a vase of water is jarred, the trembling circles
seek the shore and all is smooth again. Not a fish can leap or an
insect fall on the pond but it is thus reported in circling dimples, in
lines of beauty, as it were the constant welling up of its fountain,
the gentle pulsing of its life, the heaving of its breast. The thrills
of joy and thrills of pain are undistinguishable. How peaceful the
phenomena of the lake! Again the works of man shine as in the spring.
Ay, every leaf and twig and stone and cobweb sparkles now at
mid-afternoon as when covered with dew in a spring morning. Every
motion of an oar or an insect produces a flash of light; and if an oar
falls, how sweet the echo!

In such a day, in September or October, Walden is a perfect forest
mirror, set round with stones as precious to my eye as if fewer or
rarer. Nothing so fair, so pure, and at the same time so large, as a
lake, perchance, lies on the surface of the earth. Sky water. It needs
no fence. Nations come and go without defiling it. It is a mirror which
no stone can crack, whose quicksilver will never wear off, whose
gilding Nature continually repairs; no storms, no dust, can dim its
surface ever fresh;—a mirror in which all impurity presented to it
sinks, swept and dusted by the sun’s hazy brush,—this the light
dust-cloth,—which retains no breath that is breathed on it, but sends
its own to float as clouds high above its surface, and be reflected in
its bosom still.

A field of water betrays the spirit that is in the air. It is
continually receiving new life and motion from above. It is
intermediate in its nature between land and sky. On land only the grass
and trees wave, but the water itself is rippled by the wind. I see
where the breeze dashes across it by the streaks or flakes of light. It
is remarkable that we can look down on its surface. We shall, perhaps,
look down thus on the surface of air at length, and mark where a still
subtler spirit sweeps over it.

The skaters and water-bugs finally disappear in the latter part of
October, when the severe frosts have come; and then and in November,
usually, in a calm day, there is absolutely nothing to ripple the
surface. One November afternoon, in the calm at the end of a rain storm
of several days’ duration, when the sky was still completely overcast
and the air was full of mist, I observed that the pond was remarkably
smooth, so that it was difficult to distinguish its surface; though it
no longer reflected the bright tints of October, but the sombre
November colors of the surrounding hills. Though I passed over it as
gently as possible, the slight undulations produced by my boat extended
almost as far as I could see, and gave a ribbed appearance to the
reflections. But, as I was looking over the surface, I saw here and
there at a distance a faint glimmer, as if some skater insects which
had escaped the frosts might be collected there, or, perchance, the
surface, being so smooth, betrayed where a spring welled up from the
bottom. Paddling gently to one of these places, I was surprised to find
myself surrounded by myriads of small perch, about five inches long, of
a rich bronze color in the green water, sporting there, and constantly
rising to the surface and dimpling it, sometimes leaving bubbles on it.
In such transparent and seemingly bottomless water, reflecting the
clouds, I seemed to be floating through the air as in a balloon, and
their swimming impressed me as a kind of flight or hovering, as if they
were a compact flock of birds passing just beneath my level on the
right or left, their fins, like sails, set all around them. There were
many such schools in the pond, apparently improving the short season
before winter would draw an icy shutter over their broad skylight,
sometimes giving to the surface an appearance as if a slight breeze
struck it, or a few rain-drops fell there. When I approached carelessly
and alarmed them, they made a sudden splash and rippling with their
tails, as if one had struck the water with a brushy bough, and
instantly took refuge in the depths. At length the wind rose, the mist
increased, and the waves began to run, and the perch leaped much higher
than before, half out of water, a hundred black points, three inches
long, at once above the surface. Even as late as the fifth of December,
one year, I saw some dimples on the surface, and thinking it was going
to rain hard immediately, the air being full of mist, I made haste to
take my place at the oars and row homeward; already the rain seemed
rapidly increasing, though I felt none on my cheek, and I anticipated a
thorough soaking. But suddenly the dimples ceased, for they were
produced by the perch, which the noise of my oars had seared into the
depths, and I saw their schools dimly disappearing; so I spent a dry
afternoon after all.

An old man who used to frequent this pond nearly sixty years ago, when
it was dark with surrounding forests, tells me that in those days he
sometimes saw it all alive with ducks and other water fowl, and that
there were many eagles about it. He came here a-fishing, and used an
old log canoe which he found on the shore. It was made of two
white-pine logs dug out and pinned together, and was cut off square at
the ends. It was very clumsy, but lasted a great many years before it
became water-logged and perhaps sank to the bottom. He did not know
whose it was; it belonged to the pond. He used to make a cable for his
anchor of strips of hickory bark tied together. An old man, a potter,
who lived by the pond before the Revolution, told him once that there
was an iron chest at the bottom, and that he had seen it. Sometimes it
would come floating up to the shore; but when you went toward it, it
would go back into deep water and disappear. I was pleased to hear of
the old log canoe, which took the place of an Indian one of the same
material but more graceful construction, which perchance had first been
a tree on the bank, and then, as it were, fell into the water, to float
there for a generation, the most proper vessel for the lake. I remember
that when I first looked into these depths there were many large trunks
to be seen indistinctly lying on the bottom, which had either been
blown over formerly, or left on the ice at the last cutting, when wood
was cheaper; but now they have mostly disappeared.

When I first paddled a boat on Walden, it was completely surrounded by
thick and lofty pine and oak woods, and in some of its coves grape
vines had run over the trees next the water and formed bowers under
which a boat could pass. The hills which form its shores are so steep,
and the woods on them were then so high, that, as you looked down from
the west end, it had the appearance of an amphitheatre for some kind of
sylvan spectacle. I have spent many an hour, when I was younger,
floating over its surface as the zephyr willed, having paddled my boat
to the middle, and lying on my back across the seats, in a summer
forenoon, dreaming awake, until I was aroused by the boat touching the
sand, and I arose to see what shore my fates had impelled me to; days
when idleness was the most attractive and productive industry. Many a
forenoon have I stolen away, preferring to spend thus the most valued
part of the day; for I was rich, if not in money, in sunny hours and
summer days, and spent them lavishly; nor do I regret that I did not
waste more of them in the workshop or the teacher’s desk. But since I
left those shores the woodchoppers have still further laid them waste,
and now for many a year there will be no more rambling through the
aisles of the wood, with occasional vistas through which you see the
water. My Muse may be excused if she is silent henceforth. How can you
expect the birds to sing when their groves are cut down?

Now the trunks of trees on the bottom, and the old log canoe, and the
dark surrounding woods, are gone, and the villagers, who scarcely know
where it lies, instead of going to the pond to bathe or drink, are
thinking to bring its water, which should be as sacred as the Ganges at
least, to the village in a pipe, to wash their dishes with!—to earn
their Walden by the turning of a cock or drawing of a plug! That
devilish Iron Horse, whose ear-rending neigh is heard throughout the
town, has muddied the Boiling Spring with his foot, and he it is that
has browsed off all the woods on Walden shore, that Trojan horse, with
a thousand men in his belly, introduced by mercenary Greeks! Where is
the country’s champion, the Moore of Moore Hill, to meet him at the
Deep Cut and thrust an avenging lance between the ribs of the bloated
pest?

Nevertheless, of all the characters I have known, perhaps Walden wears
best, and best preserves its purity. Many men have been likened to it,
but few deserve that honor. Though the woodchoppers have laid bare
first this shore and then that, and the Irish have built their sties by
it, and the railroad has infringed on its border, and the ice-men have
skimmed it once, it is itself unchanged, the same water which my
youthful eyes fell on; all the change is in me. It has not acquired one
permanent wrinkle after all its ripples. It is perennially young, and I
may stand and see a swallow dip apparently to pick an insect from its
surface as of yore. It struck me again tonight, as if I had not seen it
almost daily for more than twenty years,—Why, here is Walden, the same
woodland lake that I discovered so many years ago; where a forest was
cut down last winter another is springing up by its shore as lustily as
ever; the same thought is welling up to its surface that was then; it
is the same liquid joy and happiness to itself and its Maker, ay, and
it _may_ be to me. It is the work of a brave man surely, in whom there
was no guile! He rounded this water with his hand, deepened and
clarified it in his thought, and in his will bequeathed it to Concord.
I see by its face that it is visited by the same reflection; and I can
almost say, Walden, is it you?

     It is no dream of mine,
     To ornament a line;
     I cannot come nearer to God and Heaven
     Than I live to Walden even.
     I am its stony shore,
     And the breeze that passes o’er;
     In the hollow of my hand
     Are its water and its sand,
     And its deepest resort
     Lies high in my thought.

The cars never pause to look at it; yet I fancy that the engineers and
firemen and brakemen, and those passengers who have a season ticket and
see it often, are better men for the sight. The engineer does not
forget at night, or his nature does not, that he has beheld this vision
of serenity and purity once at least during the day. Though seen but
once, it helps to wash out State-street and the engine’s soot. One
proposes that it be called “God’s Drop.”

I have said that Walden has no visible inlet nor outlet, but it is on
the one hand distantly and indirectly related to Flint’s Pond, which is
more elevated, by a chain of small ponds coming from that quarter, and
on the other directly and manifestly to Concord River, which is lower,
by a similar chain of ponds through which in some other geological
period it may have flowed, and by a little digging, which God forbid,
it can be made to flow thither again. If by living thus reserved and
austere, like a hermit in the woods, so long, it has acquired such
wonderful purity, who would not regret that the comparatively impure
waters of Flint’s Pond should be mingled with it, or itself should ever
go to waste its sweetness in the ocean wave?



Flint’s, or Sandy Pond, in Lincoln, our greatest lake and inland sea,
lies about a mile east of Walden. It is much larger, being said to
contain one hundred and ninety-seven acres, and is more fertile in
fish; but it is comparatively shallow, and not remarkably pure. A walk
through the woods thither was often my recreation. It was worth the
while, if only to feel the wind blow on your cheek freely, and see the
waves run, and remember the life of mariners. I went a-chestnutting
there in the fall, on windy days, when the nuts were dropping into the
water and were washed to my feet; and one day, as I crept along its
sedgy shore, the fresh spray blowing in my face, I came upon the
mouldering wreck of a boat, the sides gone, and hardly more than the
impression of its flat bottom left amid the rushes; yet its model was
sharply defined, as if it were a large decayed pad, with its veins. It
was as impressive a wreck as one could imagine on the sea-shore, and
had as good a moral. It is by this time mere vegetable mould and
undistinguishable pond shore, through which rushes and flags have
pushed up. I used to admire the ripple marks on the sandy bottom, at
the north end of this pond, made firm and hard to the feet of the wader
by the pressure of the water, and the rushes which grew in Indian file,
in waving lines, corresponding to these marks, rank behind rank, as if
the waves had planted them. There also I have found, in considerable
quantities, curious balls, composed apparently of fine grass or roots,
of pipewort perhaps, from half an inch to four inches in diameter, and
perfectly spherical. These wash back and forth in shallow water on a
sandy bottom, and are sometimes cast on the shore. They are either
solid grass, or have a little sand in the middle. At first you would
say that they were formed by the action of the waves, like a pebble;
yet the smallest are made of equally coarse materials, half an inch
long, and they are produced only at one season of the year. Moreover,
the waves, I suspect, do not so much construct as wear down a material
which has already acquired consistency. They preserve their form when
dry for an indefinite period.

_Flint’s Pond!_ Such is the poverty of our nomenclature. What right had
the unclean and stupid farmer, whose farm abutted on this sky water,
whose shores he has ruthlessly laid bare, to give his name to it? Some
skin-flint, who loved better the reflecting surface of a dollar, or a
bright cent, in which he could see his own brazen face; who regarded
even the wild ducks which settled in it as trespassers; his fingers
grown into crooked and horny talons from the long habit of grasping
harpy-like;—so it is not named for me. I go not there to see him nor to
hear of him; who never _saw_ it, who never bathed in it, who never
loved it, who never protected it, who never spoke a good word for it,
nor thanked God that he had made it. Rather let it be named from the
fishes that swim in it, the wild fowl or quadrupeds which frequent it,
the wild flowers which grow by its shores, or some wild man or child
the thread of whose history is interwoven with its own; not from him
who could show no title to it but the deed which a like-minded neighbor
or legislature gave him,—him who thought only of its money value; whose
presence perchance cursed all the shore; who exhausted the land around
it, and would fain have exhausted the waters within it; who regretted
only that it was not English hay or cranberry meadow,—there was nothing
to redeem it, forsooth, in his eyes,—and would have drained and sold it
for the mud at its bottom. It did not turn his mill, and it was no
_privilege_ to him to behold it. I respect not his labors, his farm
where every thing has its price; who would carry the landscape, who
would carry his God, to market, if he could get any thing for him; who
goes to market _for_ his god as it is; on whose farm nothing grows
free, whose fields bear no crops, whose meadows no flowers, whose trees
no fruits, but dollars; who loves not the beauty of his fruits, whose
fruits are not ripe for him till they are turned to dollars. Give me
the poverty that enjoys true wealth. Farmers are respectable and
interesting to me in proportion as they are poor,—poor farmers. A model
farm! where the house stands like a fungus in a muck-heap, chambers for
men, horses, oxen, and swine, cleansed and uncleansed, all contiguous
to one another! Stocked with men! A great grease-spot, redolent of
manures and buttermilk! Under a high state of cultivation, being
manured with the hearts and brains of men! As if you were to raise your
potatoes in the church-yard! Such is a model farm.

No, no; if the fairest features of the landscape are to be named after
men, let them be the noblest and worthiest men alone. Let our lakes
receive as true names at least as the Icarian Sea, where “still the
shore” a “brave attempt resounds.”



Goose Pond, of small extent, is on my way to Flint’s; Fair-Haven, an
expansion of Concord River, said to contain some seventy acres, is a
mile south-west; and White Pond, of about forty acres, is a mile and a
half beyond Fair-Haven. This is my lake country. These, with Concord
River, are my water privileges; and night and day, year in year out,
they grind such grist as I carry to them.

Since the woodcutters, and the railroad, and I myself have profaned
Walden, perhaps the most attractive, if not the most beautiful, of all
our lakes, the gem of the woods, is White Pond;—a poor name from its
commonness, whether derived from the remarkable purity of its waters or
the color of its sands. In these as in other respects, however, it is a
lesser twin of Walden. They are so much alike that you would say they
must be connected under ground. It has the same stony shore, and its
waters are of the same hue. As at Walden, in sultry dog-day weather,
looking down through the woods on some of its bays which are not so
deep but that the reflection from the bottom tinges them, its waters
are of a misty bluish-green or glaucous color. Many years since I used
to go there to collect the sand by cart-loads, to make sand-paper with,
and I have continued to visit it ever since. One who frequents it
proposes to call it Virid Lake. Perhaps it might be called Yellow-Pine
Lake, from the following circumstance. About fifteen years ago you
could see the top of a pitch-pine, of the kind called yellow-pine
hereabouts, though it is not a distinct species, projecting above the
surface in deep water, many rods from the shore. It was even supposed
by some that the pond had sunk, and this was one of the primitive
forest that formerly stood there. I find that even so long ago as 1792,
in a “Topographical Description of the Town of Concord,” by one of its
citizens, in the Collections of the Massachusetts Historical Society,
the author, after speaking of Walden and White Ponds, adds: “In the
middle of the latter may be seen, when the water is very low, a tree
which appears as if it grew in the place where it now stands, although
the roots are fifty feet below the surface of the water; the top of
this tree is broken off, and at that place measures fourteen inches in
diameter.” In the spring of ’49 I talked with the man who lives nearest
the pond in Sudbury, who told me that it was he who got out this tree
ten or fifteen years before. As near as he could remember, it stood
twelve or fifteen rods from the shore, where the water was thirty or
forty feet deep. It was in the winter, and he had been getting out ice
in the forenoon, and had resolved that in the afternoon, with the aid
of his neighbors, he would take out the old yellow-pine. He sawed a
channel in the ice toward the shore, and hauled it over and along and
out on to the ice with oxen; but, before he had gone far in his work,
he was surprised to find that it was wrong end upward, with the stumps
of the branches pointing down, and the small end firmly fastened in the
sandy bottom. It was about a foot in diameter at the big end, and he
had expected to get a good saw-log, but it was so rotten as to be fit
only for fuel, if for that. He had some of it in his shed then. There
were marks of an axe and of woodpeckers on the butt. He thought that it
might have been a dead tree on the shore, but was finally blown over
into the pond, and after the top had become waterlogged, while the
butt-end was still dry and light, had drifted out and sunk wrong end
up. His father, eighty years old, could not remember when it was not
there. Several pretty large logs may still be seen lying on the bottom,
where, owing to the undulation of the surface, they look like huge
water snakes in motion.

This pond has rarely been profaned by a boat, for there is little in it
to tempt a fisherman. Instead of the white lily, which requires mud, or
the common sweet flag, the blue flag (_Iris versicolor_) grows thinly
in the pure water, rising from the stony bottom all around the shore,
where it is visited by humming birds in June; and the color both of its
bluish blades and its flowers, and especially their reflections, are in
singular harmony with the glaucous water.

White Pond and Walden are great crystals on the surface of the earth,
Lakes of Light. If they were permanently congealed, and small enough to
be clutched, they would, perchance, be carried off by slaves, like
precious stones, to adorn the heads of emperors; but being liquid, and
ample, and secured to us and our successors forever, we disregard them,
and run after the diamond of Kohinoor. They are too pure to have a
market value; they contain no muck. How much more beautiful than our
lives, how much more transparent than our characters, are they! We
never learned meanness of them. How much fairer than the pool before
the farmer’s door, in which his ducks swim! Hither the clean wild ducks
come. Nature has no human inhabitant who appreciates her. The birds
with their plumage and their notes are in harmony with the flowers, but
what youth or maiden conspires with the wild luxuriant beauty of
Nature? She flourishes most alone, far from the towns where they
reside. Talk of heaven! ye disgrace earth.


Baker Farm

Sometimes I rambled to pine groves, standing like temples, or like
fleets at sea, full-rigged, with wavy boughs, and rippling with light,
so soft and green and shady that the Druids would have forsaken their
oaks to worship in them; or to the cedar wood beyond Flint’s Pond,
where the trees, covered with hoary blue berries, spiring higher and
higher, are fit to stand before Valhalla, and the creeping juniper
covers the ground with wreaths full of fruit; or to swamps where the
usnea lichen hangs in festoons from the white-spruce trees, and
toad-stools, round tables of the swamp gods, cover the ground, and more
beautiful fungi adorn the stumps, like butterflies or shells, vegetable
winkles; where the swamp-pink and dogwood grow, the red alder-berry
glows like eyes of imps, the waxwork grooves and crushes the hardest
woods in its folds, and the wild-holly berries make the beholder forget
his home with their beauty, and he is dazzled and tempted by nameless
other wild forbidden fruits, too fair for mortal taste. Instead of
calling on some scholar, I paid many a visit to particular trees, of
kinds which are rare in this neighborhood, standing far away in the
middle of some pasture, or in the depths of a wood or swamp, or on a
hill-top; such as the black-birch, of which we have some handsome
specimens two feet in diameter; its cousin, the yellow birch, with its
loose golden vest, perfumed like the first; the beech, which has so
neat a bole and beautifully lichen-painted, perfect in all its details,
of which, excepting scattered specimens, I know but one small grove of
sizable trees left in the township, supposed by some to have been
planted by the pigeons that were once baited with beech nuts near by;
it is worth the while to see the silver grain sparkle when you split
this wood; the bass; the hornbeam; the _Celtis occidentalis_, or false
elm, of which we have but one well-grown; some taller mast of a pine, a
shingle tree, or a more perfect hemlock than usual, standing like a
pagoda in the midst of the woods; and many others I could mention.
These were the shrines I visited both summer and winter.

Once it chanced that I stood in the very abutment of a rainbow’s arch,
which filled the lower stratum of the atmosphere, tinging the grass and
leaves around, and dazzling me as if I looked through colored crystal.
It was a lake of rainbow light, in which, for a short while, I lived
like a dolphin. If it had lasted longer it might have tinged my
employments and life. As I walked on the railroad causeway, I used to
wonder at the halo of light around my shadow, and would fain fancy
myself one of the elect. One who visited me declared that the shadows
of some Irishmen before him had no halo about them, that it was only
natives that were so distinguished. Benvenuto Cellini tells us in his
memoirs, that, after a certain terrible dream or vision which he had
during his confinement in the castle of St. Angelo, a resplendent light
appeared over the shadow of his head at morning and evening, whether he
was in Italy or France, and it was particularly conspicuous when the
grass was moist with dew. This was probably the same phenomenon to
which I have referred, which is especially observed in the morning, but
also at other times, and even by moonlight. Though a constant one, it
is not commonly noticed, and, in the case of an excitable imagination
like Cellini’s, it would be basis enough for superstition. Beside, he
tells us that he showed it to very few. But are they not indeed
distinguished who are conscious that they are regarded at all?



I set out one afternoon to go a-fishing to Fair-Haven, through the
woods, to eke out my scanty fare of vegetables. My way led through
Pleasant Meadow, an adjunct of the Baker Farm, that retreat of which a
poet has since sung, beginning,—

     “Thy entry is a pleasant field,
     Which some mossy fruit trees yield
     Partly to a ruddy brook,
     By gliding musquash undertook,
     And mercurial trout,
     Darting about.”

I thought of living there before I went to Walden. I “hooked” the
apples, leaped the brook, and scared the musquash and the trout. It was
one of those afternoons which seem indefinitely long before one, in
which many events may happen, a large portion of our natural life,
though it was already half spent when I started. By the way there came
up a shower, which compelled me to stand half an hour under a pine,
piling boughs over my head, and wearing my handkerchief for a shed; and
when at length I had made one cast over the pickerel-weed, standing up
to my middle in water, I found myself suddenly in the shadow of a
cloud, and the thunder began to rumble with such emphasis that I could
do no more than listen to it. The gods must be proud, thought I, with
such forked flashes to rout a poor unarmed fisherman. So I made haste
for shelter to the nearest hut, which stood half a mile from any road,
but so much the nearer to the pond, and had long been uninhabited:—

     “And here a poet builded,
         In the completed years,
     For behold a trivial cabin
         That to destruction steers.”

So the Muse fables. But therein, as I found, dwelt now John Field, an
Irishman, and his wife, and several children, from the broad-faced boy
who assisted his father at his work, and now came running by his side
from the bog to escape the rain, to the wrinkled, sibyl-like,
cone-headed infant that sat upon its father’s knee as in the palaces of
nobles, and looked out from its home in the midst of wet and hunger
inquisitively upon the stranger, with the privilege of infancy, not
knowing but it was the last of a noble line, and the hope and cynosure
of the world, instead of John Field’s poor starveling brat. There we
sat together under that part of the roof which leaked the least, while
it showered and thundered without. I had sat there many times of old
before the ship was built that floated his family to America. An
honest, hard-working, but shiftless man plainly was John Field; and his
wife, she too was brave to cook so many successive dinners in the
recesses of that lofty stove; with round greasy face and bare breast,
still thinking to improve her condition one day; with the never absent
mop in one hand, and yet no effects of it visible anywhere. The
chickens, which had also taken shelter here from the rain, stalked
about the room like members of the family, too humanized methought to
roast well. They stood and looked in my eye or pecked at my shoe
significantly. Meanwhile my host told me his story, how hard he worked
“bogging” for a neighboring farmer, turning up a meadow with a spade or
bog hoe at the rate of ten dollars an acre and the use of the land with
manure for one year, and his little broad-faced son worked cheerfully
at his father’s side the while, not knowing how poor a bargain the
latter had made. I tried to help him with my experience, telling him
that he was one of my nearest neighbors, and that I too, who came
a-fishing here, and looked like a loafer, was getting my living like
himself; that I lived in a tight, light, and clean house, which hardly
cost more than the annual rent of such a ruin as his commonly amounts
to; and how, if he chose, he might in a month or two build himself a
palace of his own; that I did not use tea, nor coffee, nor butter, nor
milk, nor fresh meat, and so did not have to work to get them; again,
as I did not work hard, I did not have to eat hard, and it cost me but
a trifle for my food; but as he began with tea, and coffee, and butter,
and milk, and beef, he had to work hard to pay for them, and when he
had worked hard he had to eat hard again to repair the waste of his
system,—and so it was as broad as it was long, indeed it was broader
than it was long, for he was discontented and wasted his life into the
bargain; and yet he had rated it as a gain in coming to America, that
here you could get tea, and coffee, and meat every day. But the only
true America is that country where you are at liberty to pursue such a
mode of life as may enable you to do without these, and where the state
does not endeavor to compel you to sustain the slavery and war and
other superfluous expenses which directly or indirectly result from the
use of such things. For I purposely talked to him as if he were a
philosopher, or desired to be one. I should be glad if all the meadows
on the earth were left in a wild state, if that were the consequence of
men’s beginning to redeem themselves. A man will not need to study
history to find out what is best for his own culture. But alas! the
culture of an Irishman is an enterprise to be undertaken with a sort of
moral bog hoe. I told him, that as he worked so hard at bogging, he
required thick boots and stout clothing, which yet were soon soiled and
worn out, but I wore light shoes and thin clothing, which cost not half
so much, though he might think that I was dressed like a gentleman,
(which, however, was not the case,) and in an hour or two, without
labor, but as a recreation, I could, if I wished, catch as many fish as
I should want for two days, or earn enough money to support me a week.
If he and his family would live simply, they might all go
a-huckleberrying in the summer for their amusement. John heaved a sigh
at this, and his wife stared with arms a-kimbo, and both appeared to be
wondering if they had capital enough to begin such a course with, or
arithmetic enough to carry it through. It was sailing by dead reckoning
to them, and they saw not clearly how to make their port so; therefore
I suppose they still take life bravely, after their fashion, face to
face, giving it tooth and nail, not having skill to split its massive
columns with any fine entering wedge, and rout it in detail;—thinking
to deal with it roughly, as one should handle a thistle. But they fight
at an overwhelming disadvantage,—living, John Field, alas! without
arithmetic, and failing so.

“Do you ever fish?” I asked. “Oh yes, I catch a mess now and then when
I am lying by; good perch I catch.” “What’s your bait?” “I catch
shiners with fish-worms, and bait the perch with them.” “You’d better
go now, John,” said his wife, with glistening and hopeful face; but
John demurred.

The shower was now over, and a rainbow above the eastern woods promised
a fair evening; so I took my departure. When I had got without I asked
for a drink, hoping to get a sight of the well bottom, to complete my
survey of the premises; but there, alas! are shallows and quicksands,
and rope broken withal, and bucket irrecoverable. Meanwhile the right
culinary vessel was selected, water was seemingly distilled, and after
consultation and long delay passed out to the thirsty one,—not yet
suffered to cool, not yet to settle. Such gruel sustains life here, I
thought; so, shutting my eyes, and excluding the motes by a skilfully
directed under-current, I drank to genuine hospitality the heartiest
draught I could. I am not squeamish in such cases when manners are
concerned.

As I was leaving the Irishman’s roof after the rain, bending my steps
again to the pond, my haste to catch pickerel, wading in retired
meadows, in sloughs and bog-holes, in forlorn and savage places,
appeared for an instant trivial to me who had been sent to school and
college; but as I ran down the hill toward the reddening west, with the
rainbow over my shoulder, and some faint tinkling sounds borne to my
ear through the cleansed air, from I know not what quarter, my Good
Genius seemed to say,—Go fish and hunt far and wide day by day,—farther
and wider,—and rest thee by many brooks and hearth-sides without
misgiving. Remember thy Creator in the days of thy youth. Rise free
from care before the dawn, and seek adventures. Let the noon find thee
by other lakes, and the night overtake thee everywhere at home. There
are no larger fields than these, no worthier games than may here be
played. Grow wild according to thy nature, like these sedges and
brakes, which will never become English hay. Let the thunder rumble;
what if it threaten ruin to farmers’ crops? that is not its errand to
thee. Take shelter under the cloud, while they flee to carts and sheds.
Let not to get a living be thy trade, but thy sport. Enjoy the land,
but own it not. Through want of enterprise and faith men are where they
are, buying and selling, and spending their lives like serfs.

O Baker Farm!

     “Landscape where the richest element
     Is a little sunshine innocent.” * *

     “No one runs to revel
     On thy rail-fenced lea.” * *

     “Debate with no man hast thou,
         With questions art never perplexed,
     As tame at the first sight as now,
         In thy plain russet gabardine dressed.” * *

     “Come ye who love,
         And ye who hate,
     Children of the Holy Dove,
         And Guy Faux of the state,
     And hang conspiracies
     From the tough rafters of the trees!”

Men come tamely home at night only from the next field or street, where
their household echoes haunt, and their life pines because it breathes
its own breath over again; their shadows morning and evening reach
farther than their daily steps. We should come home from far, from
adventures, and perils, and discoveries every day, with new experience
and character.

Before I had reached the pond some fresh impulse had brought out John
Field, with altered mind, letting go “bogging” ere this sunset. But he,
poor man, disturbed only a couple of fins while I was catching a fair
string, and he said it was his luck; but when we changed seats in the
boat luck changed seats too. Poor John Field!—I trust he does not read
this, unless he will improve by it,—thinking to live by some derivative
old country mode in this primitive new country,—to catch perch with
shiners. It is good bait sometimes, I allow. With his horizon all his
own, yet he a poor man, born to be poor, with his inherited Irish
poverty or poor life, his Adam’s grandmother and boggy ways, not to
rise in this world, he nor his posterity, till their wading webbed
bog-trotting feet get _talaria_ to their heels.


Higher Laws

As I came home through the woods with my string of fish, trailing my
pole, it being now quite dark, I caught a glimpse of a woodchuck
stealing across my path, and felt a strange thrill of savage delight,
and was strongly tempted to seize and devour him raw; not that I was
hungry then, except for that wildness which he represented. Once or
twice, however, while I lived at the pond, I found myself ranging the
woods, like a half-starved hound, with a strange abandonment, seeking
some kind of venison which I might devour, and no morsel could have
been too savage for me. The wildest scenes had become unaccountably
familiar. I found in myself, and still find, an instinct toward a
higher, or, as it is named, spiritual life, as do most men, and another
toward a primitive rank and savage one, and I reverence them both. I
love the wild not less than the good. The wildness and adventure that
are in fishing still recommended it to me. I like sometimes to take
rank hold on life and spend my day more as the animals do. Perhaps I
have owed to this employment and to hunting, when quite young, my
closest acquaintance with Nature. They early introduce us to and detain
us in scenery with which otherwise, at that age, we should have little
acquaintance. Fishermen, hunters, woodchoppers, and others, spending
their lives in the fields and woods, in a peculiar sense a part of
Nature themselves, are often in a more favorable mood for observing
her, in the intervals of their pursuits, than philosophers or poets
even, who approach her with expectation. She is not afraid to exhibit
herself to them. The traveller on the prairie is naturally a hunter, on
the head waters of the Missouri and Columbia a trapper, and at the
Falls of St. Mary a fisherman. He who is only a traveller learns things
at second-hand and by the halves, and is poor authority. We are most
interested when science reports what those men already know practically
or instinctively, for that alone is a true _humanity_, or account of
human experience.

They mistake who assert that the Yankee has few amusements, because he
has not so many public holidays, and men and boys do not play so many
games as they do in England, for here the more primitive but solitary
amusements of hunting fishing and the like have not yet given place to
the former. Almost every New England boy among my contemporaries
shouldered a fowling piece between the ages of ten and fourteen; and
his hunting and fishing grounds were not limited, like the preserves of
an English nobleman, but were more boundless even than those of a
savage. No wonder, then, that he did not oftener stay to play on the
common. But already a change is taking place, owing, not to an
increased humanity, but to an increased scarcity of game, for perhaps
the hunter is the greatest friend of the animals hunted, not excepting
the Humane Society.

Moreover, when at the pond, I wished sometimes to add fish to my fare
for variety. I have actually fished from the same kind of necessity
that the first fishers did. Whatever humanity I might conjure up
against it was all factitious, and concerned my philosophy more than my
feelings. I speak of fishing only now, for I had long felt differently
about fowling, and sold my gun before I went to the woods. Not that I
am less humane than others, but I did not perceive that my feelings
were much affected. I did not pity the fishes nor the worms. This was
habit. As for fowling, during the last years that I carried a gun my
excuse was that I was studying ornithology, and sought only new or rare
birds. But I confess that I am now inclined to think that there is a
finer way of studying ornithology than this. It requires so much closer
attention to the habits of the birds, that, if for that reason only, I
have been willing to omit the gun. Yet notwithstanding the objection on
the score of humanity, I am compelled to doubt if equally valuable
sports are ever substituted for these; and when some of my friends have
asked me anxiously about their boys, whether they should let them hunt,
I have answered, yes,—remembering that it was one of the best parts of
my education,—_make_ them hunters, though sportsmen only at first, if
possible, mighty hunters at last, so that they shall not find game
large enough for them in this or any vegetable wilderness,—hunters as
well as fishers of men. Thus far I am of the opinion of Chaucer’s nun,
who

     “yave not of the text a pulled hen
     That saith that hunters ben not holy men.”

There is a period in the history of the individual, as of the race,
when the hunters are the “best men,” as the Algonquins called them. We
cannot but pity the boy who has never fired a gun; he is no more
humane, while his education has been sadly neglected. This was my
answer with respect to those youths who were bent on this pursuit,
trusting that they would soon outgrow it. No humane being, past the
thoughtless age of boyhood, will wantonly murder any creature which
holds its life by the same tenure that he does. The hare in its
extremity cries like a child. I warn you, mothers, that my sympathies
do not always make the usual phil-_anthropic_ distinctions.

Such is oftenest the young man’s introduction to the forest, and the
most original part of himself. He goes thither at first as a hunter and
fisher, until at last, if he has the seeds of a better life in him, he
distinguishes his proper objects, as a poet or naturalist it may be,
and leaves the gun and fish-pole behind. The mass of men are still and
always young in this respect. In some countries a hunting parson is no
uncommon sight. Such a one might make a good shepherd’s dog, but is far
from being the Good Shepherd. I have been surprised to consider that
the only obvious employment, except wood-chopping, ice-cutting, or the
like business, which ever to my knowledge detained at Walden Pond for a
whole half day any of my fellow-citizens, whether fathers or children
of the town, with just one exception, was fishing. Commonly they did
not think that they were lucky, or well paid for their time, unless
they got a long string of fish, though they had the opportunity of
seeing the pond all the while. They might go there a thousand times
before the sediment of fishing would sink to the bottom and leave their
purpose pure; but no doubt such a clarifying process would be going on
all the while. The governor and his council faintly remember the pond,
for they went a-fishing there when they were boys; but now they are too
old and dignified to go a-fishing, and so they know it no more forever.
Yet even they expect to go to heaven at last. If the legislature
regards it, it is chiefly to regulate the number of hooks to be used
there; but they know nothing about the hook of hooks with which to
angle for the pond itself, impaling the legislature for a bait. Thus,
even in civilized communities, the embryo man passes through the hunter
stage of development.

I have found repeatedly, of late years, that I cannot fish without
falling a little in self-respect. I have tried it again and again. I
have skill at it, and, like many of my fellows, a certain instinct for
it, which revives from time to time, but always when I have done I feel
that it would have been better if I had not fished. I think that I do
not mistake. It is a faint intimation, yet so are the first streaks of
morning. There is unquestionably this instinct in me which belongs to
the lower orders of creation; yet with every year I am less a
fisherman, though without more humanity or even wisdom; at present I am
no fisherman at all. But I see that if I were to live in a wilderness I
should again be tempted to become a fisher and hunter in earnest.
Beside, there is something essentially unclean about this diet and all
flesh, and I began to see where housework commences, and whence the
endeavor, which costs so much, to wear a tidy and respectable
appearance each day, to keep the house sweet and free from all ill
odors and sights. Having been my own butcher and scullion and cook, as
well as the gentleman for whom the dishes were served up, I can speak
from an unusually complete experience. The practical objection to
animal food in my case was its uncleanness; and, besides, when I had
caught and cleaned and cooked and eaten my fish, they seemed not to
have fed me essentially. It was insignificant and unnecessary, and cost
more than it came to. A little bread or a few potatoes would have done
as well, with less trouble and filth. Like many of my contemporaries, I
had rarely for many years used animal food, or tea, or coffee, &c.; not
so much because of any ill effects which I had traced to them, as
because they were not agreeable to my imagination. The repugnance to
animal food is not the effect of experience, but is an instinct. It
appeared more beautiful to live low and fare hard in many respects; and
though I never did so, I went far enough to please my imagination. I
believe that every man who has ever been earnest to preserve his higher
or poetic faculties in the best condition has been particularly
inclined to abstain from animal food, and from much food of any kind.
It is a significant fact, stated by entomologists, I find it in Kirby
and Spence, that “some insects in their perfect state, though furnished
with organs of feeding, make no use of them;” and they lay it down as
“a general rule, that almost all insects in this state eat much less
than in that of larvæ. The voracious caterpillar when transformed into
a butterfly,” . . “and the gluttonous maggot when become a fly,”
content themselves with a drop or two of honey or some other sweet
liquid. The abdomen under the wings of the butterfly still represents
the larva. This is the tid-bit which tempts his insectivorous fate. The
gross feeder is a man in the larva state; and there are whole nations
in that condition, nations without fancy or imagination, whose vast
abdomens betray them.

It is hard to provide and cook so simple and clean a diet as will not
offend the imagination; but this, I think, is to be fed when we feed
the body; they should both sit down at the same table. Yet perhaps this
may be done. The fruits eaten temperately need not make us ashamed of
our appetites, nor interrupt the worthiest pursuits. But put an extra
condiment into your dish, and it will poison you. It is not worth the
while to live by rich cookery. Most men would feel shame if caught
preparing with their own hands precisely such a dinner, whether of
animal or vegetable food, as is every day prepared for them by others.
Yet till this is otherwise we are not civilized, and, if gentlemen and
ladies, are not true men and women. This certainly suggests what change
is to be made. It may be vain to ask why the imagination will not be
reconciled to flesh and fat. I am satisfied that it is not. Is it not a
reproach that man is a carnivorous animal? True, he can and does live,
in a great measure, by preying on other animals; but this is a
miserable way,—as any one who will go to snaring rabbits, or
slaughtering lambs, may learn,—and he will be regarded as a benefactor
of his race who shall teach man to confine himself to a more innocent
and wholesome diet. Whatever my own practice may be, I have no doubt
that it is a part of the destiny of the human race, in its gradual
improvement, to leave off eating animals, as surely as the savage
tribes have left off eating each other when they came in contact with
the more civilized.

If one listens to the faintest but constant suggestions of his genius,
which are certainly true, he sees not to what extremes, or even
insanity, it may lead him; and yet that way, as he grows more resolute
and faithful, his road lies. The faintest assured objection which one
healthy man feels will at length prevail over the arguments and customs
of mankind. No man ever followed his genius till it misled him. Though
the result were bodily weakness, yet perhaps no one can say that the
consequences were to be regretted, for these were a life in conformity
to higher principles. If the day and the night are such that you greet
them with joy, and life emits a fragrance like flowers and
sweet-scented herbs, is more elastic, more starry, more immortal,—that
is your success. All nature is your congratulation, and you have cause
momentarily to bless yourself. The greatest gains and values are
farthest from being appreciated. We easily come to doubt if they exist.
We soon forget them. They are the highest reality. Perhaps the facts
most astounding and most real are never communicated by man to man. The
true harvest of my daily life is somewhat as intangible and
indescribable as the tints of morning or evening. It is a little
star-dust caught, a segment of the rainbow which I have clutched.

Yet, for my part, I was never unusually squeamish; I could sometimes
eat a fried rat with a good relish, if it were necessary. I am glad to
have drunk water so long, for the same reason that I prefer the natural
sky to an opium-eater’s heaven. I would fain keep sober always; and
there are infinite degrees of drunkenness. I believe that water is the
only drink for a wise man; wine is not so noble a liquor; and think of
dashing the hopes of a morning with a cup of warm coffee, or of an
evening with a dish of tea! Ah, how low I fall when I am tempted by
them! Even music may be intoxicating. Such apparently slight causes
destroyed Greece and Rome, and will destroy England and America. Of all
ebriosity, who does not prefer to be intoxicated by the air he
breathes? I have found it to be the most serious objection to coarse
labors long continued, that they compelled me to eat and drink coarsely
also. But to tell the truth, I find myself at present somewhat less
particular in these respects. I carry less religion to the table, ask
no blessing; not because I am wiser than I was, but, I am obliged to
confess, because, however much it is to be regretted, with years I have
grown more coarse and indifferent. Perhaps these questions are
entertained only in youth, as most believe of poetry. My practice is
“nowhere,” my opinion is here. Nevertheless I am far from regarding
myself as one of those privileged ones to whom the Ved refers when it
says, that “he who has true faith in the Omnipresent Supreme Being may
eat all that exists,” that is, is not bound to inquire what is his
food, or who prepares it; and even in their case it is to be observed,
as a Hindoo commentator has remarked, that the Vedant limits this
privilege to “the time of distress.”

Who has not sometimes derived an inexpressible satisfaction from his
food in which appetite had no share? I have been thrilled to think that
I owed a mental perception to the commonly gross sense of taste, that I
have been inspired through the palate, that some berries which I had
eaten on a hill-side had fed my genius. “The soul not being mistress of
herself,” says Thseng-tseu, “one looks, and one does not see; one
listens, and one does not hear; one eats, and one does not know the
savor of food.” He who distinguishes the true savor of his food can
never be a glutton; he who does not cannot be otherwise. A puritan may
go to his brown-bread crust with as gross an appetite as ever an
alderman to his turtle. Not that food which entereth into the mouth
defileth a man, but the appetite with which it is eaten. It is neither
the quality nor the quantity, but the devotion to sensual savors; when
that which is eaten is not a viand to sustain our animal, or inspire
our spiritual life, but food for the worms that possess us. If the
hunter has a taste for mud-turtles, muskrats, and other such savage
tid-bits, the fine lady indulges a taste for jelly made of a calf’s
foot, or for sardines from over the sea, and they are even. He goes to
the mill-pond, she to her preserve-pot. The wonder is how they, how you
and I, can live this slimy, beastly life, eating and drinking.

Our whole life is startlingly moral. There is never an instant’s truce
between virtue and vice. Goodness is the only investment that never
fails. In the music of the harp which trembles round the world it is
the insisting on this which thrills us. The harp is the travelling
patterer for the Universe’s Insurance Company, recommending its laws,
and our little goodness is all the assessment that we pay. Though the
youth at last grows indifferent, the laws of the universe are not
indifferent, but are forever on the side of the most sensitive. Listen
to every zephyr for some reproof, for it is surely there, and he is
unfortunate who does not hear it. We cannot touch a string or move a
stop but the charming moral transfixes us. Many an irksome noise, go a
long way off, is heard as music, a proud sweet satire on the meanness
of our lives.

We are conscious of an animal in us, which awakens in proportion as our
higher nature slumbers. It is reptile and sensual, and perhaps cannot
be wholly expelled; like the worms which, even in life and health,
occupy our bodies. Possibly we may withdraw from it, but never change
its nature. I fear that it may enjoy a certain health of its own; that
we may be well, yet not pure. The other day I picked up the lower jaw
of a hog, with white and sound teeth and tusks, which suggested that
there was an animal health and vigor distinct from the spiritual. This
creature succeeded by other means than temperance and purity. “That in
which men differ from brute beasts,” says Mencius, “is a thing very
inconsiderable; the common herd lose it very soon; superior men
preserve it carefully.” Who knows what sort of life would result if we
had attained to purity? If I knew so wise a man as could teach me
purity I would go to seek him forthwith. “A command over our passions,
and over the external senses of the body, and good acts, are declared
by the Ved to be indispensable in the mind’s approximation to God.” Yet
the spirit can for the time pervade and control every member and
function of the body, and transmute what in form is the grossest
sensuality into purity and devotion. The generative energy, which, when
we are loose, dissipates and makes us unclean, when we are continent
invigorates and inspires us. Chastity is the flowering of man; and what
are called Genius, Heroism, Holiness, and the like, are but various
fruits which succeed it. Man flows at once to God when the channel of
purity is open. By turns our purity inspires and our impurity casts us
down. He is blessed who is assured that the animal is dying out in him
day by day, and the divine being established. Perhaps there is none but
has cause for shame on account of the inferior and brutish nature to
which he is allied. I fear that we are such gods or demigods only as
fauns and satyrs, the divine allied to beasts, the creatures of
appetite, and that, to some extent, our very life is our disgrace.—

     “How happy’s he who hath due place assigned
     To his beasts and disafforested his mind!
                *    *    *    *    *
     Can use this horse, goat, wolf, and ev’ry beast,
     And is not ass himself to all the rest!
     Else man not only is the herd of swine,
     But he’s those devils too which did incline
     Them to a headlong rage, and made them worse.”

All sensuality is one, though it takes many forms; all purity is one.
It is the same whether a man eat, or drink, or cohabit, or sleep
sensually. They are but one appetite, and we only need to see a person
do any one of these things to know how great a sensualist he is. The
impure can neither stand nor sit with purity. When the reptile is
attacked at one mouth of his burrow, he shows himself at another. If
you would be chaste, you must be temperate. What is chastity? How shall
a man know if he is chaste? He shall not know it. We have heard of this
virtue, but we know not what it is. We speak conformably to the rumor
which we have heard. From exertion come wisdom and purity; from sloth
ignorance and sensuality. In the student sensuality is a sluggish habit
of mind. An unclean person is universally a slothful one, one who sits
by a stove, whom the sun shines on prostrate, who reposes without being
fatigued. If you would avoid uncleanness, and all the sins, work
earnestly, though it be at cleaning a stable. Nature is hard to be
overcome, but she must be overcome. What avails it that you are
Christian, if you are not purer than the heathen, if you deny yourself
no more, if you are not more religious? I know of many systems of
religion esteemed heathenish whose precepts fill the reader with shame,
and provoke him to new endeavors, though it be to the performance of
rites merely.

I hesitate to say these things, but it is not because of the subject,—I
care not how obscene my _words_ are,—but because I cannot speak of them
without betraying my impurity. We discourse freely without shame of one
form of sensuality, and are silent about another. We are so degraded
that we cannot speak simply of the necessary functions of human nature.
In earlier ages, in some countries, every function was reverently
spoken of and regulated by law. Nothing was too trivial for the Hindoo
lawgiver, however offensive it may be to modern taste. He teaches how
to eat, drink, cohabit, void excrement and urine, and the like,
elevating what is mean, and does not falsely excuse himself by calling
these things trifles.

Every man is the builder of a temple, called his body, to the god he
worships, after a style purely his own, nor can he get off by hammering
marble instead. We are all sculptors and painters, and our material is
our own flesh and blood and bones. Any nobleness begins at once to
refine a man’s features, any meanness or sensuality to imbrute them.

John Farmer sat at his door one September evening, after a hard day’s
work, his mind still running on his labor more or less. Having bathed,
he sat down to re-create his intellectual man. It was a rather cool
evening, and some of his neighbors were apprehending a frost. He had
not attended to the train of his thoughts long when he heard some one
playing on a flute, and that sound harmonized with his mood. Still he
thought of his work; but the burden of his thought was, that though
this kept running in his head, and he found himself planning and
contriving it against his will, yet it concerned him very little. It
was no more than the scurf of his skin, which was constantly shuffled
off. But the notes of the flute came home to his ears out of a
different sphere from that he worked in, and suggested work for certain
faculties which slumbered in him. They gently did away with the street,
and the village, and the state in which he lived. A voice said to
him,—Why do you stay here and live this mean moiling life, when a
glorious existence is possible for you? Those same stars twinkle over
other fields than these.—But how to come out of this condition and
actually migrate thither? All that he could think of was to practise
some new austerity, to let his mind descend into his body and redeem
it, and treat himself with ever increasing respect.


Brute Neighbors

Sometimes I had a companion in my fishing, who came through the village
to my house from the other side of the town, and the catching of the
dinner was as much a social exercise as the eating of it.

_Hermit._ I wonder what the world is doing now. I have not heard so
much as a locust over the sweet-fern these three hours. The pigeons are
all asleep upon their roosts,—no flutter from them. Was that a farmer’s
noon horn which sounded from beyond the woods just now? The hands are
coming in to boiled salt beef and cider and Indian bread. Why will men
worry themselves so? He that does not eat need not work. I wonder how
much they have reaped. Who would live there where a body can never
think for the barking of Bose? And O, the housekeeping! to keep bright
the devil’s door-knobs, and scour his tubs this bright day! Better not
keep a house. Say, some hollow tree; and then for morning calls and
dinner-parties! Only a woodpecker tapping. O, they swarm; the sun is
too warm there; they are born too far into life for me. I have water
from the spring, and a loaf of brown bread on the shelf.—Hark! I hear a
rustling of the leaves. Is it some ill-fed village hound yielding to
the instinct of the chase? or the lost pig which is said to be in these
woods, whose tracks I saw after the rain? It comes on apace; my sumachs
and sweet-briers tremble.—Eh, Mr. Poet, is it you? How do you like the
world to-day?

_Poet._ See those clouds; how they hang! That’s the greatest thing I
have seen to-day. There’s nothing like it in old paintings, nothing
like it in foreign lands,—unless when we were off the coast of Spain.
That’s a true Mediterranean sky. I thought, as I have my living to get,
and have not eaten to-day, that I might go a-fishing. That’s the true
industry for poets. It is the only trade I have learned. Come, let’s
along.

_Hermit._ I cannot resist. My brown bread will soon be gone. I will go
with you gladly soon, but I am just concluding a serious meditation. I
think that I am near the end of it. Leave me alone, then, for a while.
But that we may not be delayed, you shall be digging the bait
meanwhile. Angle-worms are rarely to be met with in these parts, where
the soil was never fattened with manure; the race is nearly extinct.
The sport of digging the bait is nearly equal to that of catching the
fish, when one’s appetite is not too keen; and this you may have all to
yourself to-day. I would advise you to set in the spade down yonder
among the ground-nuts, where you see the johnswort waving. I think that
I may warrant you one worm to every three sods you turn up, if you look
well in among the roots of the grass, as if you were weeding. Or, if
you choose to go farther, it will not be unwise, for I have found the
increase of fair bait to be very nearly as the squares of the
distances.

_Hermit alone._ Let me see; where was I? Methinks I was nearly in this
frame of mind; the world lay about at this angle. Shall I go to heaven
or a-fishing? If I should soon bring this meditation to an end, would
another so sweet occasion be likely to offer? I was as near being
resolved into the essence of things as ever I was in my life. I fear my
thoughts will not come back to me. If it would do any good, I would
whistle for them. When they make us an offer, is it wise to say, We
will think of it? My thoughts have left no track, and I cannot find the
path again. What was it that I was thinking of? It was a very hazy day.
I will just try these three sentences of Con-fut-see; they may fetch
that state about again. I know not whether it was the dumps or a
budding ecstasy. Mem. There never is but one opportunity of a kind.

_Poet._ How now, Hermit, is it too soon? I have got just thirteen whole
ones, beside several which are imperfect or undersized; but they will
do for the smaller fry; they do not cover up the hook so much. Those
village worms are quite too large; a shiner may make a meal off one
without finding the skewer.

_Hermit._ Well, then, let’s be off. Shall we to the Concord? There’s
good sport there if the water be not too high.



Why do precisely these objects which we behold make a world? Why has
man just these species of animals for his neighbors; as if nothing but
a mouse could have filled this crevice? I suspect that Pilpay & Co.
have put animals to their best use, for they are all beasts of burden,
in a sense, made to carry some portion of our thoughts.

The mice which haunted my house were not the common ones, which are
said to have been introduced into the country, but a wild native kind
not found in the village. I sent one to a distinguished naturalist, and
it interested him much. When I was building, one of these had its nest
underneath the house, and before I had laid the second floor, and swept
out the shavings, would come out regularly at lunch time and pick up
the crumbs at my feet. It probably had never seen a man before; and it
soon became quite familiar, and would run over my shoes and up my
clothes. It could readily ascend the sides of the room by short
impulses, like a squirrel, which it resembled in its motions. At
length, as I leaned with my elbow on the bench one day, it ran up my
clothes, and along my sleeve, and round and round the paper which held
my dinner, while I kept the latter close, and dodged and played at
bopeep with it; and when at last I held still a piece of cheese between
my thumb and finger, it came and nibbled it, sitting in my hand, and
afterward cleaned its face and paws, like a fly, and walked away.

A phœbe soon built in my shed, and a robin for protection in a pine
which grew against the house. In June the partridge (_Tetrao
umbellus_,) which is so shy a bird, led her brood past my windows, from
the woods in the rear to the front of my house, clucking and calling to
them like a hen, and in all her behavior proving herself the hen of the
woods. The young suddenly disperse on your approach, at a signal from
the mother, as if a whirlwind had swept them away, and they so exactly
resemble the dried leaves and twigs that many a traveler has placed his
foot in the midst of a brood, and heard the whir of the old bird as she
flew off, and her anxious calls and mewing, or seen her trail her wings
to attract his attention, without suspecting their neighborhood. The
parent will sometimes roll and spin round before you in such a
dishabille, that you cannot, for a few moments, detect what kind of
creature it is. The young squat still and flat, often running their
heads under a leaf, and mind only their mother’s directions given from
a distance, nor will your approach make them run again and betray
themselves. You may even tread on them, or have your eyes on them for a
minute, without discovering them. I have held them in my open hand at
such a time, and still their only care, obedient to their mother and
their instinct, was to squat there without fear or trembling. So
perfect is this instinct, that once, when I had laid them on the leaves
again, and one accidentally fell on its side, it was found with the
rest in exactly the same position ten minutes afterward. They are not
callow like the young of most birds, but more perfectly developed and
precocious even than chickens. The remarkably adult yet innocent
expression of their open and serene eyes is very memorable. All
intelligence seems reflected in them. They suggest not merely the
purity of infancy, but a wisdom clarified by experience. Such an eye
was not born when the bird was, but is coeval with the sky it reflects.
The woods do not yield another such a gem. The traveller does not often
look into such a limpid well. The ignorant or reckless sportsman often
shoots the parent at such a time, and leaves these innocents to fall a
prey to some prowling beast or bird, or gradually mingle with the
decaying leaves which they so much resemble. It is said that when
hatched by a hen they will directly disperse on some alarm, and so are
lost, for they never hear the mother’s call which gathers them again.
These were my hens and chickens.

It is remarkable how many creatures live wild and free though secret in
the woods, and still sustain themselves in the neighborhood of towns,
suspected by hunters only. How retired the otter manages to live here!
He grows to be four feet long, as big as a small boy, perhaps without
any human being getting a glimpse of him. I formerly saw the raccoon in
the woods behind where my house is built, and probably still heard
their whinnering at night. Commonly I rested an hour or two in the
shade at noon, after planting, and ate my lunch, and read a little by a
spring which was the source of a swamp and of a brook, oozing from
under Brister’s Hill, half a mile from my field. The approach to this
was through a succession of descending grassy hollows, full of young
pitch-pines, into a larger wood about the swamp. There, in a very
secluded and shaded spot, under a spreading white-pine, there was yet a
clean, firm sward to sit on. I had dug out the spring and made a well
of clear gray water, where I could dip up a pailful without roiling it,
and thither I went for this purpose almost every day in midsummer, when
the pond was warmest. Thither, too, the wood-cock led her brood, to
probe the mud for worms, flying but a foot above them down the bank,
while they ran in a troop beneath; but at last, spying me, she would
leave her young and circle round and round me, nearer and nearer till
within four or five feet, pretending broken wings and legs, to attract
my attention, and get off her young, who would already have taken up
their march, with faint wiry peep, single file through the swamp, as
she directed. Or I heard the peep of the young when I could not see the
parent bird. There too the turtle-doves sat over the spring, or
fluttered from bough to bough of the soft white-pines over my head; or
the red squirrel, coursing down the nearest bough, was particularly
familiar and inquisitive. You only need sit still long enough in some
attractive spot in the woods that all its inhabitants may exhibit
themselves to you by turns.

I was witness to events of a less peaceful character. One day when I
went out to my wood-pile, or rather my pile of stumps, I observed two
large ants, the one red, the other much larger, nearly half an inch
long, and black, fiercely contending with one another. Having once got
hold they never let go, but struggled and wrestled and rolled on the
chips incessantly. Looking farther, I was surprised to find that the
chips were covered with such combatants, that it was not a _duellum_,
but a _bellum_, a war between two races of ants, the red always pitted
against the black, and frequently two red ones to one black. The
legions of these Myrmidons covered all the hills and vales in my
wood-yard, and the ground was already strewn with the dead and dying,
both red and black. It was the only battle which I have ever witnessed,
the only battle-field I ever trod while the battle was raging;
internecine war; the red republicans on the one hand, and the black
imperialists on the other. On every side they were engaged in deadly
combat, yet without any noise that I could hear, and human soldiers
never fought so resolutely. I watched a couple that were fast locked in
each other’s embraces, in a little sunny valley amid the chips, now at
noon-day prepared to fight till the sun went down, or life went out.
The smaller red champion had fastened himself like a vice to his
adversary’s front, and through all the tumblings on that field never
for an instant ceased to gnaw at one of his feelers near the root,
having already caused the other to go by the board; while the stronger
black one dashed him from side to side, and, as I saw on looking
nearer, had already divested him of several of his members. They fought
with more pertinacity than bull-dogs. Neither manifested the least
disposition to retreat. It was evident that their battle-cry was
Conquer or die. In the mean while there came along a single red ant on
the hill-side of this valley, evidently full of excitement, who either
had despatched his foe, or had not yet taken part in the battle;
probably the latter, for he had lost none of his limbs; whose mother
had charged him to return with his shield or upon it. Or perchance he
was some Achilles, who had nourished his wrath apart, and had now come
to avenge or rescue his Patroclus. He saw this unequal combat from
afar,—for the blacks were nearly twice the size of the red,—he drew
near with rapid pace till he stood on his guard within half an inch of
the combatants; then, watching his opportunity, he sprang upon the
black warrior, and commenced his operations near the root of his right
fore-leg, leaving the foe to select among his own members; and so there
were three united for life, as if a new kind of attraction had been
invented which put all other locks and cements to shame. I should not
have wondered by this time to find that they had their respective
musical bands stationed on some eminent chip, and playing their
national airs the while, to excite the slow and cheer the dying
combatants. I was myself excited somewhat even as if they had been men.
The more you think of it, the less the difference. And certainly there
is not the fight recorded in Concord history, at least, if in the
history of America, that will bear a moment’s comparison with this,
whether for the numbers engaged in it, or for the patriotism and
heroism displayed. For numbers and for carnage it was an Austerlitz or
Dresden. Concord Fight! Two killed on the patriots’ side, and Luther
Blanchard wounded! Why here every ant was a Buttrick,—“Fire! for God’s
sake fire!”—and thousands shared the fate of Davis and Hosmer. There
was not one hireling there. I have no doubt that it was a principle
they fought for, as much as our ancestors, and not to avoid a
three-penny tax on their tea; and the results of this battle will be as
important and memorable to those whom it concerns as those of the
battle of Bunker Hill, at least.

I took up the chip on which the three I have particularly described
were struggling, carried it into my house, and placed it under a
tumbler on my window-sill, in order to see the issue. Holding a
microscope to the first-mentioned red ant, I saw that, though he was
assiduously gnawing at the near fore-leg of his enemy, having severed
his remaining feeler, his own breast was all torn away, exposing what
vitals he had there to the jaws of the black warrior, whose breastplate
was apparently too thick for him to pierce; and the dark carbuncles of
the sufferer’s eyes shone with ferocity such as war only could excite.
They struggled half an hour longer under the tumbler, and when I looked
again the black soldier had severed the heads of his foes from their
bodies, and the still living heads were hanging on either side of him
like ghastly trophies at his saddle-bow, still apparently as firmly
fastened as ever, and he was endeavoring with feeble struggles, being
without feelers and with only the remnant of a leg, and I know not how
many other wounds, to divest himself of them; which at length, after
half an hour more, he accomplished. I raised the glass, and he went off
over the window-sill in that crippled state. Whether he finally
survived that combat, and spent the remainder of his days in some Hotel
des Invalides, I do not know; but I thought that his industry would not
be worth much thereafter. I never learned which party was victorious,
nor the cause of the war; but I felt for the rest of that day as if I
had had my feelings excited and harrowed by witnessing the struggle,
the ferocity and carnage, of a human battle before my door.

Kirby and Spence tell us that the battles of ants have long been
celebrated and the date of them recorded, though they say that Huber is
the only modern author who appears to have witnessed them. “Æneas
Sylvius,” say they, “after giving a very circumstantial account of one
contested with great obstinacy by a great and small species on the
trunk of a pear tree,” adds that “‘This action was fought in the
pontificate of Eugenius the Fourth, in the presence of Nicholas
Pistoriensis, an eminent lawyer, who related the whole history of the
battle with the greatest fidelity.’ A similar engagement between great
and small ants is recorded by Olaus Magnus, in which the small ones,
being victorious, are said to have buried the bodies of their own
soldiers, but left those of their giant enemies a prey to the birds.
This event happened previous to the expulsion of the tyrant Christiern
the Second from Sweden.” The battle which I witnessed took place in the
Presidency of Polk, five years before the passage of Webster’s
Fugitive-Slave Bill.

Many a village Bose, fit only to course a mud-turtle in a victualling
cellar, sported his heavy quarters in the woods, without the knowledge
of his master, and ineffectually smelled at old fox burrows and
woodchucks’ holes; led perchance by some slight cur which nimbly
threaded the wood, and might still inspire a natural terror in its
denizens;—now far behind his guide, barking like a canine bull toward
some small squirrel which had treed itself for scrutiny, then,
cantering off, bending the bushes with his weight, imagining that he is
on the track of some stray member of the jerbilla family. Once I was
surprised to see a cat walking along the stony shore of the pond, for
they rarely wander so far from home. The surprise was mutual.
Nevertheless the most domestic cat, which has lain on a rug all her
days, appears quite at home in the woods, and, by her sly and stealthy
behavior, proves herself more native there than the regular
inhabitants. Once, when berrying, I met with a cat with young kittens
in the woods, quite wild, and they all, like their mother, had their
backs up and were fiercely spitting at me. A few years before I lived
in the woods there was what was called a “winged cat” in one of the
farm-houses in Lincoln nearest the pond, Mr. Gilian Baker’s. When I
called to see her in June, 1842, she was gone a-hunting in the woods,
as was her wont, (I am not sure whether it was a male or female, and so
use the more common pronoun,) but her mistress told me that she came
into the neighborhood a little more than a year before, in April, and
was finally taken into their house; that she was of a dark
brownish-gray color, with a white spot on her throat, and white feet,
and had a large bushy tail like a fox; that in the winter the fur grew
thick and flatted out along her sides, forming stripes ten or twelve
inches long by two and a half wide, and under her chin like a muff, the
upper side loose, the under matted like felt, and in the spring these
appendages dropped off. They gave me a pair of her “wings,” which I
keep still. There is no appearance of a membrane about them. Some
thought it was part flying-squirrel or some other wild animal, which is
not impossible, for, according to naturalists, prolific hybrids have
been produced by the union of the marten and domestic cat. This would
have been the right kind of cat for me to keep, if I had kept any; for
why should not a poet’s cat be winged as well as his horse?

In the fall the loon (_Colymbus glacialis_) came, as usual, to moult
and bathe in the pond, making the woods ring with his wild laughter
before I had risen. At rumor of his arrival all the Mill-dam sportsmen
are on the alert, in gigs and on foot, two by two and three by three,
with patent rifles and conical balls and spy-glasses. They come
rustling through the woods like autumn leaves, at least ten men to one
loon. Some station themselves on this side of the pond, some on that,
for the poor bird cannot be omnipresent; if he dive here he must come
up there. But now the kind October wind rises, rustling the leaves and
rippling the surface of the water, so that no loon can be heard or
seen, though his foes sweep the pond with spy-glasses, and make the
woods resound with their discharges. The waves generously rise and dash
angrily, taking sides with all water-fowl, and our sportsmen must beat
a retreat to town and shop and unfinished jobs. But they were too often
successful. When I went to get a pail of water early in the morning I
frequently saw this stately bird sailing out of my cove within a few
rods. If I endeavored to overtake him in a boat, in order to see how he
would manœuvre, he would dive and be completely lost, so that I did not
discover him again, sometimes, till the latter part of the day. But I
was more than a match for him on the surface. He commonly went off in a
rain.

As I was paddling along the north shore one very calm October
afternoon, for such days especially they settle on to the lakes, like
the milkweed down, having looked in vain over the pond for a loon,
suddenly one, sailing out from the shore toward the middle a few rods
in front of me, set up his wild laugh and betrayed himself. I pursued
with a paddle and he dived, but when he came up I was nearer than
before. He dived again, but I miscalculated the direction he would
take, and we were fifty rods apart when he came to the surface this
time, for I had helped to widen the interval; and again he laughed long
and loud, and with more reason than before. He manœuvred so cunningly
that I could not get within half a dozen rods of him. Each time, when
he came to the surface, turning his head this way and that, he cooly
surveyed the water and the land, and apparently chose his course so
that he might come up where there was the widest expanse of water and
at the greatest distance from the boat. It was surprising how quickly
he made up his mind and put his resolve into execution. He led me at
once to the widest part of the pond, and could not be driven from it.
While he was thinking one thing in his brain, I was endeavoring to
divine his thought in mine. It was a pretty game, played on the smooth
surface of the pond, a man against a loon. Suddenly your adversary’s
checker disappears beneath the board, and the problem is to place yours
nearest to where his will appear again. Sometimes he would come up
unexpectedly on the opposite side of me, having apparently passed
directly under the boat. So long-winded was he and so unweariable, that
when he had swum farthest he would immediately plunge again,
nevertheless; and then no wit could divine where in the deep pond,
beneath the smooth surface, he might be speeding his way like a fish,
for he had time and ability to visit the bottom of the pond in its
deepest part. It is said that loons have been caught in the New York
lakes eighty feet beneath the surface, with hooks set for trout,—though
Walden is deeper than that. How surprised must the fishes be to see
this ungainly visitor from another sphere speeding his way amid their
schools! Yet he appeared to know his course as surely under water as on
the surface, and swam much faster there. Once or twice I saw a ripple
where he approached the surface, just put his head out to reconnoitre,
and instantly dived again. I found that it was as well for me to rest
on my oars and wait his reappearing as to endeavor to calculate where
he would rise; for again and again, when I was straining my eyes over
the surface one way, I would suddenly be startled by his unearthly
laugh behind me. But why, after displaying so much cunning, did he
invariably betray himself the moment he came up by that loud laugh? Did
not his white breast enough betray him? He was indeed a silly loon, I
thought. I could commonly hear the splash of the water when he came up,
and so also detected him. But after an hour he seemed as fresh as ever,
dived as willingly and swam yet farther than at first. It was
surprising to see how serenely he sailed off with unruffled breast when
he came to the surface, doing all the work with his webbed feet
beneath. His usual note was this demoniac laughter, yet somewhat like
that of a water-fowl; but occasionally, when he had balked me most
successfully and come up a long way off, he uttered a long-drawn
unearthly howl, probably more like that of a wolf than any bird; as
when a beast puts his muzzle to the ground and deliberately howls. This
was his looning,—perhaps the wildest sound that is ever heard here,
making the woods ring far and wide. I concluded that he laughed in
derision of my efforts, confident of his own resources. Though the sky
was by this time overcast, the pond was so smooth that I could see
where he broke the surface when I did not hear him. His white breast,
the stillness of the air, and the smoothness of the water were all
against him. At length, having come up fifty rods off, he uttered one
of those prolonged howls, as if calling on the god of loons to aid him,
and immediately there came a wind from the east and rippled the
surface, and filled the whole air with misty rain, and I was impressed
as if it were the prayer of the loon answered, and his god was angry
with me; and so I left him disappearing far away on the tumultuous
surface.

For hours, in fall days, I watched the ducks cunningly tack and veer
and hold the middle of the pond, far from the sportsman; tricks which
they will have less need to practise in Louisiana bayous. When
compelled to rise they would sometimes circle round and round and over
the pond at a considerable height, from which they could easily see to
other ponds and the river, like black motes in the sky; and, when I
thought they had gone off thither long since, they would settle down by
a slanting flight of a quarter of a mile on to a distant part which was
left free; but what beside safety they got by sailing in the middle of
Walden I do not know, unless they love its water for the same reason
that I do.


House-Warming

In October I went a-graping to the river meadows, and loaded myself
with clusters more precious for their beauty and fragrance than for
food. There too I admired, though I did not gather, the cranberries,
small waxen gems, pendants of the meadow grass, pearly and red, which
the farmer plucks with an ugly rake, leaving the smooth meadow in a
snarl, heedlessly measuring them by the bushel and the dollar only, and
sells the spoils of the meads to Boston and New York; destined to be
_jammed_, to satisfy the tastes of lovers of Nature there. So butchers
rake the tongues of bison out of the prairie grass, regardless of the
torn and drooping plant. The barberry’s brilliant fruit was likewise
food for my eyes merely; but I collected a small store of wild apples
for coddling, which the proprietor and travellers had overlooked. When
chestnuts were ripe I laid up half a bushel for winter. It was very
exciting at that season to roam the then boundless chestnut woods of
Lincoln,—they now sleep their long sleep under the railroad,—with a bag
on my shoulder, and a stick to open burrs with in my hand, for I did
not always wait for the frost, amid the rustling of leaves and the loud
reproofs of the red-squirrels and the jays, whose half-consumed nuts I
sometimes stole, for the burrs which they had selected were sure to
contain sound ones. Occasionally I climbed and shook the trees. They
grew also behind my house, and one large tree, which almost
overshadowed it, was, when in flower, a bouquet which scented the whole
neighborhood, but the squirrels and the jays got most of its fruit; the
last coming in flocks early in the morning and picking the nuts out of
the burrs before they fell. I relinquished these trees to them and
visited the more distant woods composed wholly of chestnut. These nuts,
as far as they went, were a good substitute for bread. Many other
substitutes might, perhaps, be found. Digging one day for fish-worms, I
discovered the ground-nut (_Apios tuberosa_) on its string, the potato
of the aborigines, a sort of fabulous fruit, which I had begun to doubt
if I had ever dug and eaten in childhood, as I had told, and had not
dreamed it. I had often since seen its crimpled red velvety blossom
supported by the stems of other plants without knowing it to be the
same. Cultivation has well nigh exterminated it. It has a sweetish
taste, much like that of a frostbitten potato, and I found it better
boiled than roasted. This tuber seemed like a faint promise of Nature
to rear her own children and feed them simply here at some future
period. In these days of fatted cattle and waving grain-fields this
humble root, which was once the _totem_ of an Indian tribe, is quite
forgotten, or known only by its flowering vine; but let wild Nature
reign here once more, and the tender and luxurious English grains will
probably disappear before a myriad of foes, and without the care of man
the crow may carry back even the last seed of corn to the great
cornfield of the Indian’s God in the south-west, whence he is said to
have brought it; but the now almost exterminated ground-nut will
perhaps revive and flourish in spite of frosts and wildness, prove
itself indigenous, and resume its ancient importance and dignity as the
diet of the hunter tribe. Some Indian Ceres or Minerva must have been
the inventor and bestower of it; and when the reign of poetry commences
here, its leaves and string of nuts may be represented on our works of
art.

Already, by the first of September, I had seen two or three small
maples turned scarlet across the pond, beneath where the white stems of
three aspens diverged, at the point of a promontory, next the water.
Ah, many a tale their color told! And gradually from week to week the
character of each tree came out, and it admired itself reflected in the
smooth mirror of the lake. Each morning the manager of this gallery
substituted some new picture, distinguished by more brilliant or
harmonious coloring, for the old upon the walls.

The wasps came by thousands to my lodge in October, as to winter
quarters, and settled on my windows within and on the walls over-head,
sometimes deterring visitors from entering. Each morning, when they
were numbed with cold, I swept some of them out, but I did not trouble
myself much to get rid of them; I even felt complimented by their
regarding my house as a desirable shelter. They never molested me
seriously, though they bedded with me; and they gradually disappeared,
into what crevices I do not know, avoiding winter and unspeakable cold.

Like the wasps, before I finally went into winter quarters in November,
I used to resort to the north-east side of Walden, which the sun,
reflected from the pitch-pine woods and the stony shore, made the
fire-side of the pond; it is so much pleasanter and wholesomer to be
warmed by the sun while you can be, than by an artificial fire. I thus
warmed myself by the still glowing embers which the summer, like a
departed hunter, had left.



When I came to build my chimney I studied masonry. My bricks being
second-hand ones required to be cleaned with a trowel, so that I
learned more than usual of the qualities of bricks and trowels. The
mortar on them was fifty years old, and was said to be still growing
harder; but this is one of those sayings which men love to repeat
whether they are true or not. Such sayings themselves grow harder and
adhere more firmly with age, and it would take many blows with a trowel
to clean an old wiseacre of them. Many of the villages of Mesopotamia
are built of second-hand bricks of a very good quality, obtained from
the ruins of Babylon, and the cement on them is older and probably
harder still. However that may be, I was struck by the peculiar
toughness of the steel which bore so many violent blows without being
worn out. As my bricks had been in a chimney before, though I did not
read the name of Nebuchadnezzar on them, I picked out as many
fire-place bricks as I could find, to save work and waste, and I filled
the spaces between the bricks about the fire-place with stones from the
pond shore, and also made my mortar with the white sand from the same
place. I lingered most about the fireplace, as the most vital part of
the house. Indeed, I worked so deliberately, that though I commenced at
the ground in the morning, a course of bricks raised a few inches above
the floor served for my pillow at night; yet I did not get a stiff neck
for it that I remember; my stiff neck is of older date. I took a poet
to board for a fortnight about those times, which caused me to be put
to it for room. He brought his own knife, though I had two, and we used
to scour them by thrusting them into the earth. He shared with me the
labors of cooking. I was pleased to see my work rising so square and
solid by degrees, and reflected, that, if it proceeded slowly, it was
calculated to endure a long time. The chimney is to some extent an
independent structure, standing on the ground and rising through the
house to the heavens; even after the house is burned it still stands
sometimes, and its importance and independence are apparent. This was
toward the end of summer. It was now November.



The north wind had already begun to cool the pond, though it took many
weeks of steady blowing to accomplish it, it is so deep. When I began
to have a fire at evening, before I plastered my house, the chimney
carried smoke particularly well, because of the numerous chinks between
the boards. Yet I passed some cheerful evenings in that cool and airy
apartment, surrounded by the rough brown boards full of knots, and
rafters with the bark on high overhead. My house never pleased my eye
so much after it was plastered, though I was obliged to confess that it
was more comfortable. Should not every apartment in which man dwells be
lofty enough to create some obscurity over-head, where flickering
shadows may play at evening about the rafters? These forms are more
agreeable to the fancy and imagination than fresco paintings or other
the most expensive furniture. I now first began to inhabit my house, I
may say, when I began to use it for warmth as well as shelter. I had
got a couple of old fire-dogs to keep the wood from the hearth, and it
did me good to see the soot form on the back of the chimney which I had
built, and I poked the fire with more right and more satisfaction than
usual. My dwelling was small, and I could hardly entertain an echo in
it; but it seemed larger for being a single apartment and remote from
neighbors. All the attractions of a house were concentrated in one
room; it was kitchen, chamber, parlor, and keeping-room; and whatever
satisfaction parent or child, master or servant, derive from living in
a house, I enjoyed it all. Cato says, the master of a family
(_patremfamilias_) must have in his rustic villa “cellam oleariam,
vinariam, dolia multa, uti lubeat caritatem expectare, et rei, et
virtuti, et gloriæ erit,” that is, “an oil and wine cellar, many casks,
so that it may be pleasant to expect hard times; it will be for his
advantage, and virtue, and glory.” I had in my cellar a firkin of
potatoes, about two quarts of peas with the weevil in them, and on my
shelf a little rice, a jug of molasses, and of rye and Indian meal a
peck each.

I sometimes dream of a larger and more populous house, standing in a
golden age, of enduring materials, and without ginger-bread work, which
shall still consist of only one room, a vast, rude, substantial,
primitive hall, without ceiling or plastering, with bare rafters and
purlins supporting a sort of lower heaven over one’s head,—useful to
keep off rain and snow; where the king and queen posts stand out to
receive your homage, when you have done reverence to the prostrate
Saturn of an older dynasty on stepping over the sill; a cavernous
house, wherein you must reach up a torch upon a pole to see the roof;
where some may live in the fire-place, some in the recess of a window,
and some on settles, some at one end of the hall, some at another, and
some aloft on rafters with the spiders, if they choose; a house which
you have got into when you have opened the outside door, and the
ceremony is over; where the weary traveller may wash, and eat, and
converse, and sleep, without further journey; such a shelter as you
would be glad to reach in a tempestuous night, containing all the
essentials of a house, and nothing for house-keeping; where you can see
all the treasures of the house at one view, and every thing hangs upon
its peg, that a man should use; at once kitchen, pantry, parlor,
chamber, store-house, and garret; where you can see so necessary a
thing as a barrel or a ladder, so convenient a thing as a cupboard, and
hear the pot boil, and pay your respects to the fire that cooks your
dinner and the oven that bakes your bread, and the necessary furniture
and utensils are the chief ornaments; where the washing is not put out,
nor the fire, nor the mistress, and perhaps you are sometimes requested
to move from off the trap-door, when the cook would descend into the
cellar, and so learn whether the ground is solid or hollow beneath you
without stamping. A house whose inside is as open and manifest as a
bird’s nest, and you cannot go in at the front door and out at the back
without seeing some of its inhabitants; where to be a guest is to be
presented with the freedom of the house, and not to be carefully
excluded from seven eighths of it, shut up in a particular cell, and
told to make yourself at home there,—in solitary confinement. Nowadays
the host does not admit you to _his_ hearth, but has got the mason to
build one for yourself somewhere in his alley, and hospitality is the
art of _keeping_ you at the greatest distance. There is as much secrecy
about the cooking as if he had a design to poison you. I am aware that
I have been on many a man’s premises, and might have been legally
ordered off, but I am not aware that I have been in many men’s houses.
I might visit in my old clothes a king and queen who lived simply in
such a house as I have described, if I were going their way; but
backing out of a modern palace will be all that I shall desire to
learn, if ever I am caught in one.

It would seem as if the very language of our parlors would lose all its
nerve and degenerate into _palaver_ wholly, our lives pass at such
remoteness from its symbols, and its metaphors and tropes are
necessarily so far fetched, through slides and dumb-waiters, as it
were; in other words, the parlor is so far from the kitchen and
workshop. The dinner even is only the parable of a dinner, commonly. As
if only the savage dwelt near enough to Nature and Truth to borrow a
trope from them. How can the scholar, who dwells away in the North West
Territory or the Isle of Man, tell what is parliamentary in the
kitchen?

However, only one or two of my guests were ever bold enough to stay and
eat a hasty-pudding with me; but when they saw that crisis approaching
they beat a hasty retreat rather, as if it would shake the house to its
foundations. Nevertheless, it stood through a great many
hasty-puddings.

I did not plaster till it was freezing weather. I brought over some
whiter and cleaner sand for this purpose from the opposite shore of the
pond in a boat, a sort of conveyance which would have tempted me to go
much farther if necessary. My house had in the mean while been shingled
down to the ground on every side. In lathing I was pleased to be able
to send home each nail with a single blow of the hammer, and it was my
ambition to transfer the plaster from the board to the wall neatly and
rapidly. I remembered the story of a conceited fellow, who, in fine
clothes, was wont to lounge about the village once, giving advice to
workmen. Venturing one day to substitute deeds for words, he turned up
his cuffs, seized a plasterer’s board, and having loaded his trowel
without mishap, with a complacent look toward the lathing overhead,
made a bold gesture thitherward; and straightway, to his complete
discomfiture, received the whole contents in his ruffled bosom. I
admired anew the economy and convenience of plastering, which so
effectually shuts out the cold and takes a handsome finish, and I
learned the various casualties to which the plasterer is liable. I was
surprised to see how thirsty the bricks were which drank up all the
moisture in my plaster before I had smoothed it, and how many pailfuls
of water it takes to christen a new hearth. I had the previous winter
made a small quantity of lime by burning the shells of the _Unio
fluviatilis_, which our river affords, for the sake of the experiment;
so that I knew where my materials came from. I might have got good
limestone within a mile or two and burned it myself, if I had cared to
do so.



The pond had in the mean while skimmed over in the shadiest and
shallowest coves, some days or even weeks before the general freezing.
The first ice is especially interesting and perfect, being hard, dark,
and transparent, and affords the best opportunity that ever offers for
examining the bottom where it is shallow; for you can lie at your
length on ice only an inch thick, like a skater insect on the surface
of the water, and study the bottom at your leisure, only two or three
inches distant, like a picture behind a glass, and the water is
necessarily always smooth then. There are many furrows in the sand
where some creature has travelled about and doubled on its tracks; and,
for wrecks, it is strewn with the cases of cadis worms made of minute
grains of white quartz. Perhaps these have creased it, for you find
some of their cases in the furrows, though they are deep and broad for
them to make. But the ice itself is the object of most interest, though
you must improve the earliest opportunity to study it. If you examine
it closely the morning after it freezes, you find that the greater part
of the bubbles, which at first appeared to be within it, are against
its under surface, and that more are continually rising from the
bottom; while the ice is as yet comparatively solid and dark, that is,
you see the water through it. These bubbles are from an eightieth to an
eighth of an inch in diameter, very clear and beautiful, and you see
your face reflected in them through the ice. There may be thirty or
forty of them to a square inch. There are also already within the ice
narrow oblong perpendicular bubbles about half an inch long, sharp
cones with the apex upward; or oftener, if the ice is quite fresh,
minute spherical bubbles one directly above another, like a string of
beads. But these within the ice are not so numerous nor obvious as
those beneath. I sometimes used to cast on stones to try the strength
of the ice, and those which broke through carried in air with them,
which formed very large and conspicuous white bubbles beneath. One day
when I came to the same place forty-eight hours afterward, I found that
those large bubbles were still perfect, though an inch more of ice had
formed, as I could see distinctly by the seam in the edge of a cake.
But as the last two days had been very warm, like an Indian summer, the
ice was not now transparent, showing the dark green color of the water,
and the bottom, but opaque and whitish or gray, and though twice as
thick was hardly stronger than before, for the air bubbles had greatly
expanded under this heat and run together, and lost their regularity;
they were no longer one directly over another, but often like silvery
coins poured from a bag, one overlapping another, or in thin flakes, as
if occupying slight cleavages. The beauty of the ice was gone, and it
was too late to study the bottom. Being curious to know what position
my great bubbles occupied with regard to the new ice, I broke out a
cake containing a middling sized one, and turned it bottom upward. The
new ice had formed around and under the bubble, so that it was included
between the two ices. It was wholly in the lower ice, but close against
the upper, and was flattish, or perhaps slightly lenticular, with a
rounded edge, a quarter of an inch deep by four inches in diameter; and
I was surprised to find that directly under the bubble the ice was
melted with great regularity in the form of a saucer reversed, to the
height of five eighths of an inch in the middle, leaving a thin
partition there between the water and the bubble, hardly an eighth of
an inch thick; and in many places the small bubbles in this partition
had burst out downward, and probably there was no ice at all under the
largest bubbles, which were a foot in diameter. I inferred that the
infinite number of minute bubbles which I had first seen against the
under surface of the ice were now frozen in likewise, and that each, in
its degree, had operated like a burning glass on the ice beneath to
melt and rot it. These are the little air-guns which contribute to make
the ice crack and whoop.



At length the winter set in in good earnest, just as I had finished
plastering, and the wind began to howl around the house as if it had
not had permission to do so till then. Night after night the geese came
lumbering in in the dark with a clangor and a whistling of wings, even
after the ground was covered with snow, some to alight in Walden, and
some flying low over the woods toward Fair Haven, bound for Mexico.
Several times, when returning from the village at ten or eleven o’clock
at night, I heard the tread of a flock of geese, or else ducks, on the
dry leaves in the woods by a pond-hole behind my dwelling, where they
had come up to feed, and the faint honk or quack of their leader as
they hurried off. In 1845 Walden froze entirely over for the first time
on the night of the 22d of December, Flint’s and other shallower ponds
and the river having been frozen ten days or more; in ’46, the 16th; in
’49, about the 31st; and in ’50, about the 27th of December; in ’52,
the 5th of January; in ’53, the 31st of December. The snow had already
covered the ground since the 25th of November, and surrounded me
suddenly with the scenery of winter. I withdrew yet farther into my
shell, and endeavored to keep a bright fire both within my house and
within my breast. My employment out of doors now was to collect the
dead wood in the forest, bringing it in my hands or on my shoulders, or
sometimes trailing a dead pine tree under each arm to my shed. An old
forest fence which had seen its best days was a great haul for me. I
sacrificed it to Vulcan, for it was past serving the god Terminus. How
much more interesting an event is that man’s supper who has just been
forth in the snow to hunt, nay, you might say, steal, the fuel to cook
it with! His bread and meat are sweet. There are enough fagots and
waste wood of all kinds in the forests of most of our towns to support
many fires, but which at present warm none, and, some think, hinder the
growth of the young wood. There was also the drift-wood of the pond. In
the course of the summer I had discovered a raft of pitch-pine logs
with the bark on, pinned together by the Irish when the railroad was
built. This I hauled up partly on the shore. After soaking two years
and then lying high six months it was perfectly sound, though
waterlogged past drying. I amused myself one winter day with sliding
this piecemeal across the pond, nearly half a mile, skating behind with
one end of a log fifteen feet long on my shoulder, and the other on the
ice; or I tied several logs together with a birch withe, and then, with
a longer birch or alder which had a hook at the end, dragged them
across. Though completely waterlogged and almost as heavy as lead, they
not only burned long, but made a very hot fire; nay, I thought that
they burned better for the soaking, as if the pitch, being confined by
the water, burned longer, as in a lamp.

Gilpin, in his account of the forest borderers of England, says that
“the encroachments of trespassers, and the houses and fences thus
raised on the borders of the forest,” were “considered as great
nuisances by the old forest law, and were severely punished under the
name of _purprestures_, as tending _ad terrorem ferarum—ad nocumentum
forestæ_, &c.,” to the frightening of the game and the detriment of the
forest. But I was interested in the preservation of the venison and the
vert more than the hunters or woodchoppers, and as much as though I had
been the Lord Warden himself; and if any part was burned, though I
burned it myself by accident, I grieved with a grief that lasted longer
and was more inconsolable than that of the proprietors; nay, I grieved
when it was cut down by the proprietors themselves. I would that our
farmers when they cut down a forest felt some of that awe which the old
Romans did when they came to thin, or let in the light to, a
consecrated grove (_lucum conlucare_), that is, would believe that it
is sacred to some god. The Roman made an expiatory offering, and
prayed, Whatever god or goddess thou art to whom this grove is sacred,
be propitious to me, my family, and children, &c.

It is remarkable what a value is still put upon wood even in this age
and in this new country, a value more permanent and universal than that
of gold. After all our discoveries and inventions no man will go by a
pile of wood. It is as precious to us as it was to our Saxon and Norman
ancestors. If they made their bows of it, we make our gun-stocks of it.
Michaux, more than thirty years ago, says that the price of wood for
fuel in New York and Philadelphia “nearly equals, and sometimes
exceeds, that of the best wood in Paris, though this immense capital
annually requires more than three hundred thousand cords, and is
surrounded to the distance of three hundred miles by cultivated
plains.” In this town the price of wood rises almost steadily, and the
only question is, how much higher it is to be this year than it was the
last. Mechanics and tradesmen who come in person to the forest on no
other errand, are sure to attend the wood auction, and even pay a high
price for the privilege of gleaning after the woodchopper. It is now
many years that men have resorted to the forest for fuel and the
materials of the arts; the New Englander and the New Hollander, the
Parisian and the Celt, the farmer and Robinhood, Goody Blake and Harry
Gill, in most parts of the world the prince and the peasant, the
scholar and the savage, equally require still a few sticks from the
forest to warm them and cook their food. Neither could I do without
them.

Every man looks at his wood-pile with a kind of affection. I love to
have mine before my window, and the more chips the better to remind me
of my pleasing work. I had an old axe which nobody claimed, with which
by spells in winter days, on the sunny side of the house, I played
about the stumps which I had got out of my bean-field. As my driver
prophesied when I was ploughing, they warmed me twice, once while I was
splitting them, and again when they were on the fire, so that no fuel
could give out more heat. As for the axe, I was advised to get the
village blacksmith to “jump” it; but I jumped him, and, putting a
hickory helve from the woods into it, made it do. If it was dull, it
was at least hung true.

A few pieces of fat pine were a great treasure. It is interesting to
remember how much of this food for fire is still concealed in the
bowels of the earth. In previous years I had often gone “prospecting”
over some bare hill-side, where a pitch-pine wood had formerly stood,
and got out the fat pine roots. They are almost indestructible. Stumps
thirty or forty years old, at least, will still be sound at the core,
though the sapwood has all become vegetable mould, as appears by the
scales of the thick bark forming a ring level with the earth four or
five inches distant from the heart. With axe and shovel you explore
this mine, and follow the marrowy store, yellow as beef tallow, or as
if you had struck on a vein of gold, deep into the earth. But commonly
I kindled my fire with the dry leaves of the forest, which I had stored
up in my shed before the snow came. Green hickory finely split makes
the woodchopper’s kindlings, when he has a camp in the woods. Once in a
while I got a little of this. When the villagers were lighting their
fires beyond the horizon, I too gave notice to the various wild
inhabitants of Walden vale, by a smoky streamer from my chimney, that I
was awake.—

     Light-winged Smoke, Icarian bird,
     Melting thy pinions in thy upward flight,
     Lark without song, and messenger of dawn,
     Circling above the hamlets as thy nest;
     Or else, departing dream, and shadowy form
     Of midnight vision, gathering up thy skirts;
     By night star-veiling, and by day
     Darkening the light and blotting out the sun;
     Go thou my incense upward from this hearth,
     And ask the gods to pardon this clear flame.

Hard green wood just cut, though I used but little of that, answered my
purpose better than any other. I sometimes left a good fire when I went
to take a walk in a winter afternoon; and when I returned, three or
four hours afterward, it would be still alive and glowing. My house was
not empty though I was gone. It was as if I had left a cheerful
housekeeper behind. It was I and Fire that lived there; and commonly my
housekeeper proved trustworthy. One day, however, as I was splitting
wood, I thought that I would just look in at the window and see if the
house was not on fire; it was the only time I remember to have been
particularly anxious on this score; so I looked and saw that a spark
had caught my bed, and I went in and extinguished it when it had burned
a place as big as my hand. But my house occupied so sunny and sheltered
a position, and its roof was so low, that I could afford to let the
fire go out in the middle of almost any winter day.

The moles nested in my cellar, nibbling every third potato, and making
a snug bed even there of some hair left after plastering and of brown
paper; for even the wildest animals love comfort and warmth as well as
man, and they survive the winter only because they are so careful to
secure them. Some of my friends spoke as if I was coming to the woods
on purpose to freeze myself. The animal merely makes a bed, which he
warms with his body, in a sheltered place; but man, having discovered
fire, boxes up some air in a spacious apartment, and warms that,
instead of robbing himself, makes that his bed, in which he can move
about divested of more cumbrous clothing, maintain a kind of summer in
the midst of winter, and by means of windows even admit the light, and
with a lamp lengthen out the day. Thus he goes a step or two beyond
instinct, and saves a little time for the fine arts. Though, when I had
been exposed to the rudest blasts a long time, my whole body began to
grow torpid, when I reached the genial atmosphere of my house I soon
recovered my faculties and prolonged my life. But the most luxuriously
housed has little to boast of in this respect, nor need we trouble
ourselves to speculate how the human race may be at last destroyed. It
would be easy to cut their threads any time with a little sharper blast
from the north. We go on dating from Cold Fridays and Great Snows; but
a little colder Friday, or greater snow, would put a period to man’s
existence on the globe.

The next winter I used a small cooking-stove for economy, since I did
not own the forest; but it did not keep fire so well as the open
fire-place. Cooking was then, for the most part, no longer a poetic,
but merely a chemic process. It will soon be forgotten, in these days
of stoves, that we used to roast potatoes in the ashes, after the
Indian fashion. The stove not only took up room and scented the house,
but it concealed the fire, and I felt as if I had lost a companion. You
can always see a face in the fire. The laborer, looking into it at
evening, purifies his thoughts of the dross and earthiness which they
have accumulated during the day. But I could no longer sit and look
into the fire, and the pertinent words of a poet recurred to me with
new force.—

     “Never, bright flame, may be denied to me
     Thy dear, life imaging, close sympathy.
     What but my hopes shot upward e’er so bright?
     What but my fortunes sunk so low in night?

     Why art thou banished from our hearth and hall,
     Thou who art welcomed and beloved by all?
     Was thy existence then too fanciful
     For our life’s common light, who are so dull?
     Did thy bright gleam mysterious converse hold
     With our congenial souls? secrets too bold?
     Well, we are safe and strong, for now we sit
     Beside a hearth where no dim shadows flit,
     Where nothing cheers nor saddens, but a fire
     Warms feet and hands—nor does to more aspire;
     By whose compact utilitarian heap
     The present may sit down and go to sleep,
     Nor fear the ghosts who from the dim past walked,
     And with us by the unequal light of the old wood fire talked.”



Former Inhabitants and Winter Visitors

I weathered some merry snow storms, and spent some cheerful winter
evenings by my fire-side, while the snow whirled wildly without, and
even the hooting of the owl was hushed. For many weeks I met no one in
my walks but those who came occasionally to cut wood and sled it to the
village. The elements, however, abetted me in making a path through the
deepest snow in the woods, for when I had once gone through the wind
blew the oak leaves into my tracks, where they lodged, and by absorbing
the rays of the sun melted the snow, and so not only made a dry bed for
my feet, but in the night their dark line was my guide. For human
society I was obliged to conjure up the former occupants of these
woods. Within the memory of many of my townsmen the road near which my
house stands resounded with the laugh and gossip of inhabitants, and
the woods which border it were notched and dotted here and there with
their little gardens and dwellings, though it was then much more shut
in by the forest than now. In some places, within my own remembrance,
the pines would scrape both sides of a chaise at once, and women and
children who were compelled to go this way to Lincoln alone and on foot
did it with fear, and often ran a good part of the distance. Though
mainly but a humble route to neighboring villages, or for the woodman’s
team, it once amused the traveller more than now by its variety, and
lingered longer in his memory. Where now firm open fields stretch from
the village to the woods, it then ran through a maple swamp on a
foundation of logs, the remnants of which, doubtless, still underlie
the present dusty highway, from the Stratton, now the Alms House, Farm,
to Brister’s Hill.

East of my bean-field, across the road, lived Cato Ingraham, slave of
Duncan Ingraham, Esquire, gentleman, of Concord village, who built his
slave a house, and gave him permission to live in Walden Woods;—Cato,
not Uticensis, but Concordiensis. Some say that he was a Guinea Negro.
There are a few who remember his little patch among the walnuts, which
he let grow up till he should be old and need them; but a younger and
whiter speculator got them at last. He too, however, occupies an
equally narrow house at present. Cato’s half-obliterated cellar hole
still remains, though known to few, being concealed from the traveller
by a fringe of pines. It is now filled with the smooth sumach (_Rhus
glabra_,) and one of the earliest species of golden-rod (_Solidago
stricta_) grows there luxuriantly.

Here, by the very corner of my field, still nearer to town, Zilpha, a
colored woman, had her little house, where she spun linen for the
townsfolk, making the Walden Woods ring with her shrill singing, for
she had a loud and notable voice. At length, in the war of 1812, her
dwelling was set on fire by English soldiers, prisoners on parole, when
she was away, and her cat and dog and hens were all burned up together.
She led a hard life, and somewhat inhumane. One old frequenter of these
woods remembers, that as he passed her house one noon he heard her
muttering to herself over her gurgling pot,—“Ye are all bones, bones!”
I have seen bricks amid the oak copse there.

Down the road, on the right hand, on Brister’s Hill, lived Brister
Freeman, “a handy Negro,” slave of Squire Cummings once,—there where
grow still the apple-trees which Brister planted and tended; large old
trees now, but their fruit still wild and ciderish to my taste. Not
long since I read his epitaph in the old Lincoln burying-ground, a
little on one side, near the unmarked graves of some British grenadiers
who fell in the retreat from Concord,—where he is styled “Sippio
Brister,”—Scipio Africanus he had some title to be called,—“a man of
color,” as if he were discolored. It also told me, with staring
emphasis, when he died; which was but an indirect way of informing me
that he ever lived. With him dwelt Fenda, his hospitable wife, who told
fortunes, yet pleasantly,—large, round, and black, blacker than any of
the children of night, such a dusky orb as never rose on Concord before
or since.

Farther down the hill, on the left, on the old road in the woods, are
marks of some homestead of the Stratton family; whose orchard once
covered all the slope of Brister’s Hill, but was long since killed out
by pitch pines, excepting a few stumps, whose old roots furnish still
the wild stocks of many a thrifty village tree.

Nearer yet to town, you come to Breed’s location, on the other side of
the way, just on the edge of the wood; ground famous for the pranks of
a demon not distinctly named in old mythology, who has acted a
prominent and astounding part in our New England life, and deserves, as
much as any mythological character, to have his biography written one
day; who first comes in the guise of a friend or hired man, and then
robs and murders the whole family,—New-England Rum. But history must
not yet tell the tragedies enacted here; let time intervene in some
measure to assuage and lend an azure tint to them. Here the most
indistinct and dubious tradition says that once a tavern stood; the
well the same, which tempered the traveller’s beverage and refreshed
his steed. Here then men saluted one another, and heard and told the
news, and went their ways again.

Breed’s hut was standing only a dozen years ago, though it had long
been unoccupied. It was about the size of mine. It was set on fire by
mischievous boys, one Election night, if I do not mistake. I lived on
the edge of the village then, and had just lost myself over Davenant’s
Gondibert, that winter that I labored with a lethargy,—which, by the
way, I never knew whether to regard as a family complaint, having an
uncle who goes to sleep shaving himself, and is obliged to sprout
potatoes in a cellar Sundays, in order to keep awake and keep the
Sabbath, or as the consequence of my attempt to read Chalmers’
collection of English poetry without skipping. It fairly overcame my
Nervii. I had just sunk my head on this when the bells rung fire, and
in hot haste the engines rolled that way, led by a straggling troop of
men and boys, and I among the foremost, for I had leaped the brook. We
thought it was far south over the woods,—we who had run to fires
before,—barn, shop, or dwelling-house, or all together. “It’s Baker’s
barn,” cried one. “It is the Codman place,” affirmed another. And then
fresh sparks went up above the wood, as if the roof fell in, and we all
shouted “Concord to the rescue!” Wagons shot past with furious speed
and crushing loads, bearing, perchance, among the rest, the agent of
the Insurance Company, who was bound to go however far; and ever and
anon the engine bell tinkled behind, more slow and sure; and rearmost
of all, as it was afterward whispered, came they who set the fire and
gave the alarm. Thus we kept on like true idealists, rejecting the
evidence of our senses, until at a turn in the road we heard the
crackling and actually felt the heat of the fire from over the wall,
and realized, alas! that we were there. The very nearness of the fire
but cooled our ardor. At first we thought to throw a frog-pond on to
it; but concluded to let it burn, it was so far gone and so worthless.
So we stood round our engine, jostled one another, expressed our
sentiments through speaking-trumpets, or in lower tone referred to the
great conflagrations which the world has witnessed, including Bascom’s
shop, and, between ourselves, we thought that, were we there in season
with our “tub,” and a full frog-pond by, we could turn that threatened
last and universal one into another flood. We finally retreated without
doing any mischief,—returned to sleep and Gondibert. But as for
Gondibert, I would except that passage in the preface about wit being
the soul’s powder,—“but most of mankind are strangers to wit, as
Indians are to powder.”

It chanced that I walked that way across the fields the following
night, about the same hour, and hearing a low moaning at this spot, I
drew near in the dark, and discovered the only survivor of the family
that I know, the heir of both its virtues and its vices, who alone was
interested in this burning, lying on his stomach and looking over the
cellar wall at the still smouldering cinders beneath, muttering to
himself, as is his wont. He had been working far off in the river
meadows all day, and had improved the first moments that he could call
his own to visit the home of his fathers and his youth. He gazed into
the cellar from all sides and points of view by turns, always lying
down to it, as if there was some treasure, which he remembered,
concealed between the stones, where there was absolutely nothing but a
heap of bricks and ashes. The house being gone, he looked at what there
was left. He was soothed by the sympathy which my mere presence
implied, and showed me, as well as the darkness permitted, where the
well was covered up; which, thank Heaven, could never be burned; and he
groped long about the wall to find the well-sweep which his father had
cut and mounted, feeling for the iron hook or staple by which a burden
had been fastened to the heavy end,—all that he could now cling to,—to
convince me that it was no common “rider.” I felt it, and still remark
it almost daily in my walks, for by it hangs the history of a family.

Once more, on the left, where are seen the well and lilac bushes by the
wall, in the now open field, lived Nutting and Le Grosse. But to return
toward Lincoln.

Farther in the woods than any of these, where the road approaches
nearest to the pond, Wyman the potter squatted, and furnished his
townsmen with earthen ware, and left descendants to succeed him.
Neither were they rich in worldly goods, holding the land by sufferance
while they lived; and there often the sheriff came in vain to collect
the taxes, and “attached a chip,” for form’s sake, as I have read in
his accounts, there being nothing else that he could lay his hands on.
One day in midsummer, when I was hoeing, a man who was carrying a load
of pottery to market stopped his horse against my field and inquired
concerning Wyman the younger. He had long ago bought a potter’s wheel
of him, and wished to know what had become of him. I had read of the
potter’s clay and wheel in Scripture, but it had never occurred to me
that the pots we use were not such as had come down unbroken from those
days, or grown on trees like gourds somewhere, and I was pleased to
hear that so fictile an art was ever practiced in my neighborhood.

The last inhabitant of these woods before me was an Irishman, Hugh
Quoil (if I have spelt his name with coil enough,) who occupied Wyman’s
tenement,—Col. Quoil, he was called. Rumor said that he had been a
soldier at Waterloo. If he had lived I should have made him fight his
battles over again. His trade here was that of a ditcher. Napoleon went
to St. Helena; Quoil came to Walden Woods. All I know of him is tragic.
He was a man of manners, like one who had seen the world, and was
capable of more civil speech than you could well attend to. He wore a
great coat in mid-summer, being affected with the trembling delirium,
and his face was the color of carmine. He died in the road at the foot
of Brister’s Hill shortly after I came to the woods, so that I have not
remembered him as a neighbor. Before his house was pulled down, when
his comrades avoided it as “an unlucky castle,” I visited it. There lay
his old clothes curled up by use, as if they were himself, upon his
raised plank bed. His pipe lay broken on the hearth, instead of a bowl
broken at the fountain. The last could never have been the symbol of
his death, for he confessed to me that, though he had heard of
Brister’s Spring, he had never seen it; and soiled cards, kings of
diamonds spades and hearts, were scattered over the floor. One black
chicken which the administrator could not catch, black as night and as
silent, not even croaking, awaiting Reynard, still went to roost in the
next apartment. In the rear there was the dim outline of a garden,
which had been planted but had never received its first hoeing, owing
to those terrible shaking fits, though it was now harvest time. It was
over-run with Roman wormwood and beggar-ticks, which last stuck to my
clothes for all fruit. The skin of a woodchuck was freshly stretched
upon the back of the house, a trophy of his last Waterloo; but no warm
cap or mittens would he want more.

Now only a dent in the earth marks the site of these dwellings, with
buried cellar stones, and strawberries, raspberries, thimble-berries,
hazel-bushes, and sumachs growing in the sunny sward there; some
pitch-pine or gnarled oak occupies what was the chimney nook, and a
sweet-scented black-birch, perhaps, waves where the door-stone was.
Sometimes the well dent is visible, where once a spring oozed; now dry
and tearless grass; or it was covered deep,—not to be discovered till
some late day,—with a flat stone under the sod, when the last of the
race departed. What a sorrowful act must that be,—the covering up of
wells! coincident with the opening of wells of tears. These cellar
dents, like deserted fox burrows, old holes, are all that is left where
once were the stir and bustle of human life, and “fate, free-will,
foreknowledge absolute,” in some form and dialect or other were by
turns discussed. But all I can learn of their conclusions amounts to
just this, that “Cato and Brister pulled wool;” which is about as
edifying as the history of more famous schools of philosophy.

Still grows the vivacious lilac a generation after the door and lintel
and the sill are gone, unfolding its sweet-scented flowers each spring,
to be plucked by the musing traveller; planted and tended once by
children’s hands, in front-yard plots,—now standing by wall-sides in
retired pastures, and giving place to new-rising forests;—the last of
that stirp, sole survivor of that family. Little did the dusky children
think that the puny slip with its two eyes only, which they stuck in
the ground in the shadow of the house and daily watered, would root
itself so, and outlive them, and house itself in the rear that shaded
it, and grown man’s garden and orchard, and tell their story faintly to
the lone wanderer a half century after they had grown up and
died,—blossoming as fair, and smelling as sweet, as in that first
spring. I mark its still tender, civil, cheerful, lilac colors.

But this small village, germ of something more, why did it fail while
Concord keeps its ground? Were there no natural advantages,—no water
privileges, forsooth? Ay, the deep Walden Pond and cool Brister’s
Spring,—privilege to drink long and healthy draughts at these, all
unimproved by these men but to dilute their glass. They were
universally a thirsty race. Might not the basket, stable-broom,
mat-making, corn-parching, linen-spinning, and pottery business have
thrived here, making the wilderness to blossom like the rose, and a
numerous posterity have inherited the land of their fathers? The
sterile soil would at least have been proof against a low-land
degeneracy. Alas! how little does the memory of these human inhabitants
enhance the beauty of the landscape! Again, perhaps, Nature will try,
with me for a first settler, and my house raised last spring to be the
oldest in the hamlet.

I am not aware that any man has ever built on the spot which I occupy.
Deliver me from a city built on the site of a more ancient city, whose
materials are ruins, whose gardens cemeteries. The soil is blanched and
accursed there, and before that becomes necessary the earth itself will
be destroyed. With such reminiscences I repeopled the woods and lulled
myself asleep.



At this season I seldom had a visitor. When the snow lay deepest no
wanderer ventured near my house for a week or fortnight at a time, but
there I lived as snug as a meadow mouse, or as cattle and poultry which
are said to have survived for a long time buried in drifts, even
without food; or like that early settler’s family in the town of
Sutton, in this state, whose cottage was completely covered by the
great snow of 1717 when he was absent, and an Indian found it only by
the hole which the chimney’s breath made in the drift, and so relieved
the family. But no friendly Indian concerned himself about me; nor
needed he, for the master of the house was at home. The Great Snow! How
cheerful it is to hear of! When the farmers could not get to the woods
and swamps with their teams, and were obliged to cut down the shade
trees before their houses, and when the crust was harder, cut off the
trees in the swamps, ten feet from the ground, as it appeared the next
spring.

In the deepest snows, the path which I used from the highway to my
house, about half a mile long, might have been represented by a
meandering dotted line, with wide intervals between the dots. For a
week of even weather I took exactly the same number of steps, and of
the same length, coming and going, stepping deliberately and with the
precision of a pair of dividers in my own deep tracks,—to such routine
the winter reduces us,—yet often they were filled with heaven’s own
blue. But no weather interfered fatally with my walks, or rather my
going abroad, for I frequently tramped eight or ten miles through the
deepest snow to keep an appointment with a beech-tree, or a
yellow-birch, or an old acquaintance among the pines; when the ice and
snow causing their limbs to droop, and so sharpening their tops, had
changed the pines into fir-trees; wading to the tops of the highest
hills when the snow was nearly two feet deep on a level, and shaking
down another snow-storm on my head at every step; or sometimes creeping
and floundering thither on my hands and knees, when the hunters had
gone into winter quarters. One afternoon I amused myself by watching a
barred owl (_Strix nebulosa_) sitting on one of the lower dead limbs of
a white-pine, close to the trunk, in broad daylight, I standing within
a rod of him. He could hear me when I moved and cronched the snow with
my feet, but could not plainly see me. When I made most noise he would
stretch out his neck, and erect his neck feathers, and open his eyes
wide; but their lids soon fell again, and he began to nod. I too felt a
slumberous influence after watching him half an hour, as he sat thus
with his eyes half open, like a cat, winged brother of the cat. There
was only a narrow slit left between their lids, by which he preserved a
peninsular relation to me; thus, with half-shut eyes, looking out from
the land of dreams, and endeavoring to realize me, vague object or mote
that interrupted his visions. At length, on some louder noise or my
nearer approach, he would grow uneasy and sluggishly turn about on his
perch, as if impatient at having his dreams disturbed; and when he
launched himself off and flapped through the pines, spreading his wings
to unexpected breadth, I could not hear the slightest sound from them.
Thus, guided amid the pine boughs rather by a delicate sense of their
neighborhood than by sight, feeling his twilight way as it were with
his sensitive pinions, he found a new perch, where he might in peace
await the dawning of his day.

As I walked over the long causeway made for the railroad through the
meadows, I encountered many a blustering and nipping wind, for nowhere
has it freer play; and when the frost had smitten me on one cheek,
heathen as I was, I turned to it the other also. Nor was it much better
by the carriage road from Brister’s Hill. For I came to town still,
like a friendly Indian, when the contents of the broad open fields were
all piled up between the walls of the Walden road, and half an hour
sufficed to obliterate the tracks of the last traveller. And when I
returned new drifts would have formed, through which I floundered,
where the busy north-west wind had been depositing the powdery snow
round a sharp angle in the road, and not a rabbit’s track, nor even the
fine print, the small type, of a meadow mouse was to be seen. Yet I
rarely failed to find, even in mid-winter, some warm and springly swamp
where the grass and the skunk-cabbage still put forth with perennial
verdure, and some hardier bird occasionally awaited the return of
spring.

Sometimes, notwithstanding the snow, when I returned from my walk at
evening I crossed the deep tracks of a woodchopper leading from my
door, and found his pile of whittlings on the hearth, and my house
filled with the odor of his pipe. Or on a Sunday afternoon, if I
chanced to be at home, I heard the cronching of the snow made by the
step of a long-headed farmer, who from far through the woods sought my
house, to have a social “crack;” one of the few of his vocation who are
“men on their farms;” who donned a frock instead of a professor’s gown,
and is as ready to extract the moral out of church or state as to haul
a load of manure from his barn-yard. We talked of rude and simple
times, when men sat about large fires in cold bracing weather, with
clear heads; and when other dessert failed, we tried our teeth on many
a nut which wise squirrels have long since abandoned, for those which
have the thickest shells are commonly empty.

The one who came from farthest to my lodge, through deepest snows and
most dismal tempests, was a poet. A farmer, a hunter, a soldier, a
reporter, even a philosopher, may be daunted; but nothing can deter a
poet, for he is actuated by pure love. Who can predict his comings and
goings? His business calls him out at all hours, even when doctors
sleep. We made that small house ring with boisterous mirth and resound
with the murmur of much sober talk, making amends then to Walden vale
for the long silences. Broadway was still and deserted in comparison.
At suitable intervals there were regular salutes of laughter, which
might have been referred indifferently to the last uttered or the
forth-coming jest. We made many a “bran new” theory of life over a thin
dish of gruel, which combined the advantages of conviviality with the
clear-headedness which philosophy requires.

I should not forget that during my last winter at the pond there was
another welcome visitor, who at one time came through the village,
through snow and rain and darkness, till he saw my lamp through the
trees, and shared with me some long winter evenings. One of the last of
the philosophers,—Connecticut gave him to the world,—he peddled first
her wares, afterwards, as he declares, his brains. These he peddles
still, prompting God and disgracing man, bearing for fruit his brain
only, like the nut its kernel. I think that he must be the man of the
most faith of any alive. His words and attitude always suppose a better
state of things than other men are acquainted with, and he will be the
last man to be disappointed as the ages revolve. He has no venture in
the present. But though comparatively disregarded now, when his day
comes, laws unsuspected by most will take effect, and masters of
families and rulers will come to him for advice.—

     “How blind that cannot see serenity!”


A true friend of man; almost the only friend of human progress. An Old
Mortality, say rather an Immortality, with unwearied patience and faith
making plain the image engraven in men’s bodies, the God of whom they
are but defaced and leaning monuments. With his hospitable intellect he
embraces children, beggars, insane, and scholars, and entertains the
thought of all, adding to it commonly some breadth and elegance. I
think that he should keep a caravansary on the world’s highway, where
philosophers of all nations might put up, and on his sign should be
printed, “Entertainment for man, but not for his beast. Enter ye that
have leisure and a quiet mind, who earnestly seek the right road.” He
is perhaps the sanest man and has the fewest crotchets of any I chance
to know; the same yesterday and tomorrow. Of yore we had sauntered and
talked, and effectually put the world behind us; for he was pledged to
no institution in it, freeborn, _ingenuus_. Whichever way we turned, it
seemed that the heavens and the earth had met together, since he
enhanced the beauty of the landscape. A blue-robed man, whose fittest
roof is the overarching sky which reflects his serenity. I do not see
how he can ever die; Nature cannot spare him.

Having each some shingles of thought well dried, we sat and whittled
them, trying our knives, and admiring the clear yellowish grain of the
pumpkin pine. We waded so gently and reverently, or we pulled together
so smoothly, that the fishes of thought were not scared from the
stream, nor feared any angler on the bank, but came and went grandly,
like the clouds which float through the western sky, and the
mother-o’-pearl flocks which sometimes form and dissolve there. There
we worked, revising mythology, rounding a fable here and there, and
building castles in the air for which earth offered no worthy
foundation. Great Looker! Great Expecter! to converse with whom was a
New England Night’s Entertainment. Ah! such discourse we had, hermit
and philosopher, and the old settler I have spoken of,—we three,—it
expanded and racked my little house; I should not dare to say how many
pounds’ weight there was above the atmospheric pressure on every
circular inch; it opened its seams so that they had to be calked with
much dulness thereafter to stop the consequent leak;—but I had enough
of that kind of oakum already picked.

There was one other with whom I had “solid seasons,” long to be
remembered, at his house in the village, and who looked in upon me from
time to time; but I had no more for society there.

There too, as every where, I sometimes expected the Visitor who never
comes. The Vishnu Purana says, “The house-holder is to remain at
eventide in his court-yard as long as it takes to milk a cow, or longer
if he pleases, to await the arrival of a guest.” I often performed this
duty of hospitality, waited long enough to milk a whole herd of cows,
but did not see the man approaching from the town.


Winter Animals

When the ponds were firmly frozen, they afforded not only new and
shorter routes to many points, but new views from their surfaces of the
familiar landscape around them. When I crossed Flint’s Pond, after it
was covered with snow, though I had often paddled about and skated over
it, it was so unexpectedly wide and so strange that I could think of
nothing but Baffin’s Bay. The Lincoln hills rose up around me at the
extremity of a snowy plain, in which I did not remember to have stood
before; and the fishermen, at an indeterminable distance over the ice,
moving slowly about with their wolfish dogs, passed for sealers or
Esquimaux, or in misty weather loomed like fabulous creatures, and I
did not know whether they were giants or pygmies. I took this course
when I went to lecture in Lincoln in the evening, travelling in no road
and passing no house between my own hut and the lecture room. In Goose
Pond, which lay in my way, a colony of muskrats dwelt, and raised their
cabins high above the ice, though none could be seen abroad when I
crossed it. Walden, being like the rest usually bare of snow, or with
only shallow and interrupted drifts on it, was my yard, where I could
walk freely when the snow was nearly two feet deep on a level elsewhere
and the villagers were confined to their streets. There, far from the
village street, and except at very long intervals, from the jingle of
sleigh-bells, I slid and skated, as in a vast moose-yard well trodden,
overhung by oak woods and solemn pines bent down with snow or bristling
with icicles.

For sounds in winter nights, and often in winter days, I heard the
forlorn but melodious note of a hooting owl indefinitely far; such a
sound as the frozen earth would yield if struck with a suitable
plectrum, the very _lingua vernacula_ of Walden Wood, and quite
familiar to me at last, though I never saw the bird while it was making
it. I seldom opened my door in a winter evening without hearing it;
_Hoo hoo hoo, hoorer, hoo,_ sounded sonorously, and the first three
syllables accented somewhat like _how der do_; or sometimes _hoo hoo_
only. One night in the beginning of winter, before the pond froze over,
about nine o’clock, I was startled by the loud honking of a goose, and,
stepping to the door, heard the sound of their wings like a tempest in
the woods as they flew low over my house. They passed over the pond
toward Fair Haven, seemingly deterred from settling by my light, their
commodore honking all the while with a regular beat. Suddenly an
unmistakable cat-owl from very near me, with the most harsh and
tremendous voice I ever heard from any inhabitant of the woods,
responded at regular intervals to the goose, as if determined to expose
and disgrace this intruder from Hudson’s Bay by exhibiting a greater
compass and volume of voice in a native, and _boo-hoo_ him out of
Concord horizon. What do you mean by alarming the citadel at this time
of night consecrated to me? Do you think I am ever caught napping at
such an hour, and that I have not got lungs and a larynx as well as
yourself? _Boo-hoo, boo-hoo, boo-hoo!_ It was one of the most thrilling
discords I ever heard. And yet, if you had a discriminating ear, there
were in it the elements of a concord such as these plains never saw nor
heard.

I also heard the whooping of the ice in the pond, my great bed-fellow
in that part of Concord, as if it were restless in its bed and would
fain turn over, were troubled with flatulency and had dreams; or I was
waked by the cracking of the ground by the frost, as if some one had
driven a team against my door, and in the morning would find a crack in
the earth a quarter of a mile long and a third of an inch wide.

Sometimes I heard the foxes as they ranged over the snow crust, in
moonlight nights, in search of a partridge or other game, barking
raggedly and demoniacally like forest dogs, as if laboring with some
anxiety, or seeking expression, struggling for light and to be dogs
outright and run freely in the streets; for if we take the ages into
our account, may there not be a civilization going on among brutes as
well as men? They seemed to me to be rudimental, burrowing men, still
standing on their defence, awaiting their transformation. Sometimes one
came near to my window, attracted by my light, barked a vulpine curse
at me, and then retreated.

Usually the red squirrel (_Sciurus Hudsonius_) waked me in the dawn,
coursing over the roof and up and down the sides of the house, as if
sent out of the woods for this purpose. In the course of the winter I
threw out half a bushel of ears of sweet-corn, which had not got ripe,
on to the snow crust by my door, and was amused by watching the motions
of the various animals which were baited by it. In the twilight and the
night the rabbits came regularly and made a hearty meal. All day long
the red squirrels came and went, and afforded me much entertainment by
their manœuvres. One would approach at first warily through the
shrub-oaks, running over the snow crust by fits and starts like a leaf
blown by the wind, now a few paces this way, with wonderful speed and
waste of energy, making inconceivable haste with his “trotters,” as if
it were for a wager, and now as many paces that way, but never getting
on more than half a rod at a time; and then suddenly pausing with a
ludicrous expression and a gratuitous somerset, as if all the eyes in
the universe were fixed on him,—for all the motions of a squirrel, even
in the most solitary recesses of the forest, imply spectators as much
as those of a dancing girl,—wasting more time in delay and
circumspection than would have sufficed to walk the whole distance,—I
never saw one walk,—and then suddenly, before you could say Jack
Robinson, he would be in the top of a young pitch-pine, winding up his
clock and chiding all imaginary spectators, soliloquizing and talking
to all the universe at the same time,—for no reason that I could ever
detect, or he himself was aware of, I suspect. At length he would reach
the corn, and selecting a suitable ear, frisk about in the same
uncertain trigonometrical way to the top-most stick of my wood-pile,
before my window, where he looked me in the face, and there sit for
hours, supplying himself with a new ear from time to time, nibbling at
first voraciously and throwing the half-naked cobs about; till at
length he grew more dainty still and played with his food, tasting only
the inside of the kernel, and the ear, which was held balanced over the
stick by one paw, slipped from his careless grasp and fell to the
ground, when he would look over at it with a ludicrous expression of
uncertainty, as if suspecting that it had life, with a mind not made up
whether to get it again, or a new one, or be off; now thinking of corn,
then listening to hear what was in the wind. So the little impudent
fellow would waste many an ear in a forenoon; till at last, seizing
some longer and plumper one, considerably bigger than himself, and
skilfully balancing it, he would set out with it to the woods, like a
tiger with a buffalo, by the same zig-zag course and frequent pauses,
scratching along with it as if it were too heavy for him and falling
all the while, making its fall a diagonal between a perpendicular and
horizontal, being determined to put it through at any rate;—a
singularly frivolous and whimsical fellow;—and so he would get off with
it to where he lived, perhaps carry it to the top of a pine tree forty
or fifty rods distant, and I would afterwards find the cobs strewn
about the woods in various directions.

At length the jays arrive, whose discordant screams were heard long
before, as they were warily making their approach an eighth of a mile
off, and in a stealthy and sneaking manner they flit from tree to tree,
nearer and nearer, and pick up the kernels which the squirrels have
dropped. Then, sitting on a pitch-pine bough, they attempt to swallow
in their haste a kernel which is too big for their throats and chokes
them; and after great labor they disgorge it, and spend an hour in the
endeavor to crack it by repeated blows with their bills. They were
manifestly thieves, and I had not much respect for them; but the
squirrels, though at first shy, went to work as if they were taking
what was their own.

Meanwhile also came the chickadees in flocks, which, picking up the
crumbs the squirrels had dropped, flew to the nearest twig, and,
placing them under their claws, hammered away at them with their little
bills, as if it were an insect in the bark, till they were sufficiently
reduced for their slender throats. A little flock of these tit-mice
came daily to pick a dinner out of my wood-pile, or the crumbs at my
door, with faint flitting lisping notes, like the tinkling of icicles
in the grass, or else with sprightly _day day day_, or more rarely, in
spring-like days, a wiry summery _phe-be_ from the wood-side. They were
so familiar that at length one alighted on an armful of wood which I
was carrying in, and pecked at the sticks without fear. I once had a
sparrow alight upon my shoulder for a moment while I was hoeing in a
village garden, and I felt that I was more distinguished by that
circumstance than I should have been by any epaulet I could have worn.
The squirrels also grew at last to be quite familiar, and occasionally
stepped upon my shoe, when that was the nearest way.

When the ground was not yet quite covered, and again near the end of
winter, when the snow was melted on my south hill-side and about my
wood-pile, the partridges came out of the woods morning and evening to
feed there. Whichever side you walk in the woods the partridge bursts
away on whirring wings, jarring the snow from the dry leaves and twigs
on high, which comes sifting down in the sun-beams like golden dust;
for this brave bird is not to be scared by winter. It is frequently
covered up by drifts, and, it is said, “sometimes plunges from on wing
into the soft snow, where it remains concealed for a day or two.” I
used to start them in the open land also, where they had come out of
the woods at sunset to “bud” the wild apple-trees. They will come
regularly every evening to particular trees, where the cunning
sportsman lies in wait for them, and the distant orchards next the
woods suffer thus not a little. I am glad that the partridge gets fed,
at any rate. It is Nature’s own bird which lives on buds and
diet-drink.

In dark winter mornings, or in short winter afternoons, I sometimes
heard a pack of hounds threading all the woods with hounding cry and
yelp, unable to resist the instinct of the chase, and the note of the
hunting horn at intervals, proving that man was in the rear. The woods
ring again, and yet no fox bursts forth on to the open level of the
pond, nor following pack pursuing their Actæon. And perhaps at evening
I see the hunters returning with a single brush trailing from their
sleigh for a trophy, seeking their inn. They tell me that if the fox
would remain in the bosom of the frozen earth he would be safe, or if
he would run in a straight line away no fox-hound could overtake him;
but, having left his pursuers far behind, he stops to rest and listen
till they come up, and when he runs he circles round to his old haunts,
where the hunters await him. Sometimes, however, he will run upon a
wall many rods, and then leap off far to one side, and he appears to
know that water will not retain his scent. A hunter told me that he
once saw a fox pursued by hounds burst out on to Walden when the ice
was covered with shallow puddles, run part way across, and then return
to the same shore. Ere long the hounds arrived, but here they lost the
scent. Sometimes a pack hunting by themselves would pass my door, and
circle round my house, and yelp and hound without regarding me, as if
afflicted by a species of madness, so that nothing could divert them
from the pursuit. Thus they circle until they fall upon the recent
trail of a fox, for a wise hound will forsake every thing else for
this. One day a man came to my hut from Lexington to inquire after his
hound that made a large track, and had been hunting for a week by
himself. But I fear that he was not the wiser for all I told him, for
every time I attempted to answer his questions he interrupted me by
asking, “What do you do here?” He had lost a dog, but found a man.

One old hunter who has a dry tongue, who used to come to bathe in
Walden once every year when the water was warmest, and at such times
looked in upon me, told me that many years ago he took his gun one
afternoon and went out for a cruise in Walden Wood; and as he walked
the Wayland road he heard the cry of hounds approaching, and ere long a
fox leaped the wall into the road, and as quick as thought leaped the
other wall out of the road, and his swift bullet had not touched him.
Some way behind came an old hound and her three pups in full pursuit,
hunting on their own account, and disappeared again in the woods. Late
in the afternoon, as he was resting in the thick woods south of Walden,
he heard the voice of the hounds far over toward Fair Haven still
pursuing the fox; and on they came, their hounding cry which made all
the woods ring sounding nearer and nearer, now from Well-Meadow, now
from the Baker Farm. For a long time he stood still and listened to
their music, so sweet to a hunter’s ear, when suddenly the fox
appeared, threading the solemn aisles with an easy coursing pace, whose
sound was concealed by a sympathetic rustle of the leaves, swift and
still, keeping the ground, leaving his pursuers far behind; and,
leaping upon a rock amid the woods, he sat erect and listening, with
his back to the hunter. For a moment compassion restrained the latter’s
arm; but that was a short-lived mood, and as quick as thought can
follow thought his piece was levelled, and _whang!_—the fox rolling
over the rock lay dead on the ground. The hunter still kept his place
and listened to the hounds. Still on they came, and now the near woods
resounded through all their aisles with their demoniac cry. At length
the old hound burst into view with muzzle to the ground, and snapping
the air as if possessed, and ran directly to the rock; but spying the
dead fox she suddenly ceased her hounding as if struck dumb with
amazement, and walked round and round him in silence; and one by one
her pups arrived, and, like their mother, were sobered into silence by
the mystery. Then the hunter came forward and stood in their midst, and
the mystery was solved. They waited in silence while he skinned the
fox, then followed the brush a while, and at length turned off into the
woods again. That evening a Weston Squire came to the Concord hunter’s
cottage to inquire for his hounds, and told how for a week they had
been hunting on their own account from Weston woods. The Concord hunter
told him what he knew and offered him the skin; but the other declined
it and departed. He did not find his hounds that night, but the next
day learned that they had crossed the river and put up at a farm-house
for the night, whence, having been well fed, they took their departure
early in the morning.

The hunter who told me this could remember one Sam Nutting, who used to
hunt bears on Fair Haven Ledges, and exchange their skins for rum in
Concord village; who told him, even, that he had seen a moose there.
Nutting had a famous fox-hound named Burgoyne,—he pronounced it
Bugine,—which my informant used to borrow. In the “Wast Book” of an old
trader of this town, who was also a captain, town-clerk, and
representative, I find the following entry. Jan. 18th, 1742–3, “John
Melven Cr. by 1 Grey Fox 0—2—3;” they are not now found here; and in
his ledger, Feb. 7th, 1743, Hezekiah Stratton has credit “by ½ a Catt
skin 0—1—4½;” of course, a wild-cat, for Stratton was a sergeant in the
old French war, and would not have got credit for hunting less noble
game. Credit is given for deer skins also, and they were daily sold.
One man still preserves the horns of the last deer that was killed in
this vicinity, and another has told me the particulars of the hunt in
which his uncle was engaged. The hunters were formerly a numerous and
merry crew here. I remember well one gaunt Nimrod who would catch up a
leaf by the road-side and play a strain on it wilder and more
melodious, if my memory serves me, than any hunting-horn.

At midnight, when there was a moon, I sometimes met with hounds in my
path prowling about the woods, which would skulk out of my way, as if
afraid, and stand silent amid the bushes till I had passed.

Squirrels and wild mice disputed for my store of nuts. There were
scores of pitch-pines around my house, from one to four inches in
diameter, which had been gnawed by mice the previous winter,—a
Norwegian winter for them, for the snow lay long and deep, and they
were obliged to mix a large proportion of pine bark with their other
diet. These trees were alive and apparently flourishing at mid-summer,
and many of them had grown a foot, though completely girdled; but after
another winter such were without exception dead. It is remarkable that
a single mouse should thus be allowed a whole pine tree for its dinner,
gnawing round instead of up and down it; but perhaps it is necessary in
order to thin these trees, which are wont to grow up densely.

The hares (_Lepus Americanus_) were very familiar. One had her form
under my house all winter, separated from me only by the flooring, and
she startled me each morning by her hasty departure when I began to
stir,—thump, thump, thump, striking her head against the floor timbers
in her hurry. They used to come round my door at dusk to nibble the
potato parings which I had thrown out, and were so nearly the color of
the ground that they could hardly be distinguished when still.
Sometimes in the twilight I alternately lost and recovered sight of one
sitting motionless under my window. When I opened my door in the
evening, off they would go with a squeak and a bounce. Near at hand
they only excited my pity. One evening one sat by my door two paces
from me, at first trembling with fear, yet unwilling to move; a poor
wee thing, lean and bony, with ragged ears and sharp nose, scant tail
and slender paws. It looked as if Nature no longer contained the breed
of nobler bloods, but stood on her last toes. Its large eyes appeared
young and unhealthy, almost dropsical. I took a step, and lo, away it
scud with an elastic spring over the snow crust, straightening its body
and its limbs into graceful length, and soon put the forest between me
and itself,—the wild free venison, asserting its vigor and the dignity
of Nature. Not without reason was its slenderness. Such then was its
nature. (_Lepus_, _levipes_, light-foot, some think.)

What is a country without rabbits and partridges? They are among the
most simple and indigenous animal products; ancient and venerable
families known to antiquity as to modern times; of the very hue and
substance of Nature, nearest allied to leaves and to the ground,—and to
one another; it is either winged or it is legged. It is hardly as if
you had seen a wild creature when a rabbit or a partridge bursts away,
only a natural one, as much to be expected as rustling leaves. The
partridge and the rabbit are still sure to thrive, like true natives of
the soil, whatever revolutions occur. If the forest is cut off, the
sprouts and bushes which spring up afford them concealment, and they
become more numerous than ever. That must be a poor country indeed that
does not support a hare. Our woods teem with them both, and around
every swamp may be seen the partridge or rabbit walk, beset with twiggy
fences and horse-hair snares, which some cow-boy tends.


The Pond in Winter

After a still winter night I awoke with the impression that some
question had been put to me, which I had been endeavoring in vain to
answer in my sleep, as what—how—when—where? But there was dawning
Nature, in whom all creatures live, looking in at my broad windows with
serene and satisfied face, and no question on _her_ lips. I awoke to an
answered question, to Nature and daylight. The snow lying deep on the
earth dotted with young pines, and the very slope of the hill on which
my house is placed, seemed to say, Forward! Nature puts no question and
answers none which we mortals ask. She has long ago taken her
resolution. “O Prince, our eyes contemplate with admiration and
transmit to the soul the wonderful and varied spectacle of this
universe. The night veils without doubt a part of this glorious
creation; but day comes to reveal to us this great work, which extends
from earth even into the plains of the ether.”

Then to my morning work. First I take an axe and pail and go in search
of water, if that be not a dream. After a cold and snowy night it
needed a divining rod to find it. Every winter the liquid and trembling
surface of the pond, which was so sensitive to every breath, and
reflected every light and shadow, becomes solid to the depth of a foot
or a foot and a half, so that it will support the heaviest teams, and
perchance the snow covers it to an equal depth, and it is not to be
distinguished from any level field. Like the marmots in the surrounding
hills, it closes its eye-lids and becomes dormant for three months or
more. Standing on the snow-covered plain, as if in a pasture amid the
hills, I cut my way first through a foot of snow, and then a foot of
ice, and open a window under my feet, where, kneeling to drink, I look
down into the quiet parlor of the fishes, pervaded by a softened light
as through a window of ground glass, with its bright sanded floor the
same as in summer; there a perennial waveless serenity reigns as in the
amber twilight sky, corresponding to the cool and even temperament of
the inhabitants. Heaven is under our feet as well as over our heads.

Early in the morning, while all things are crisp with frost, men come
with fishing reels and slender lunch, and let down their fine lines
through the snowy field to take pickerel and perch; wild men, who
instinctively follow other fashions and trust other authorities than
their townsmen, and by their goings and comings stitch towns together
in parts where else they would be ripped. They sit and eat their
luncheon in stout fear-naughts on the dry oak leaves on the shore, as
wise in natural lore as the citizen is in artificial. They never
consulted with books, and know and can tell much less than they have
done. The things which they practise are said not yet to be known. Here
is one fishing for pickerel with grown perch for bait. You look into
his pail with wonder as into a summer pond, as if he kept summer locked
up at home, or knew where she had retreated. How, pray, did he get
these in mid-winter? O, he got worms out of rotten logs since the
ground froze, and so he caught them. His life itself passes deeper in
Nature than the studies of the naturalist penetrate; himself a subject
for the naturalist. The latter raises the moss and bark gently with his
knife in search of insects; the former lays open logs to their core
with his axe, and moss and bark fly far and wide. He gets his living by
barking trees. Such a man has some right to fish, and I love to see
Nature carried out in him. The perch swallows the grub-worm, the
pickerel swallows the perch, and the fisher-man swallows the pickerel;
and so all the chinks in the scale of being are filled.

When I strolled around the pond in misty weather I was sometimes amused
by the primitive mode which some ruder fisherman had adopted. He would
perhaps have placed alder branches over the narrow holes in the ice,
which were four or five rods apart and an equal distance from the
shore, and having fastened the end of the line to a stick to prevent
its being pulled through, have passed the slack line over a twig of the
alder, a foot or more above the ice, and tied a dry oak leaf to it,
which, being pulled down, would show when he had a bite. These alders
loomed through the mist at regular intervals as you walked half way
round the pond.

Ah, the pickerel of Walden! when I see them lying on the ice, or in the
well which the fisherman cuts in the ice, making a little hole to admit
the water, I am always surprised by their rare beauty, as if they were
fabulous fishes, they are so foreign to the streets, even to the woods,
foreign as Arabia to our Concord life. They possess a quite dazzling
and transcendent beauty which separates them by a wide interval from
the cadaverous cod and haddock whose fame is trumpeted in our streets.
They are not green like the pines, nor gray like the stones, nor blue
like the sky; but they have, to my eyes, if possible, yet rarer colors,
like flowers and precious stones, as if they were the pearls, the
animalized _nuclei_ or crystals of the Walden water. They, of course,
are Walden all over and all through; are themselves small Waldens in
the animal kingdom, Waldenses. It is surprising that they are caught
here,—that in this deep and capacious spring, far beneath the rattling
teams and chaises and tinkling sleighs that travel the Walden road,
this great gold and emerald fish swims. I never chanced to see its kind
in any market; it would be the cynosure of all eyes there. Easily, with
a few convulsive quirks, they give up their watery ghosts, like a
mortal translated before his time to the thin air of heaven.



walden_pond_map



As I was desirous to recover the long lost bottom of Walden Pond, I
surveyed it carefully, before the ice broke up, early in ’46, with
compass and chain and sounding line. There have been many stories told
about the bottom, or rather no bottom, of this pond, which certainly
had no foundation for themselves. It is remarkable how long men will
believe in the bottomlessness of a pond without taking the trouble to
sound it. I have visited two such Bottomless Ponds in one walk in this
neighborhood. Many have believed that Walden reached quite through to
the other side of the globe. Some who have lain flat on the ice for a
long time, looking down through the illusive medium, perchance with
watery eyes into the bargain, and driven to hasty conclusions by the
fear of catching cold in their breasts, have seen vast holes “into
which a load of hay might be driven,” if there were any body to drive
it, the undoubted source of the Styx and entrance to the Infernal
Regions from these parts. Others have gone down from the village with a
“fifty-six” and a wagon load of inch rope, but yet have failed to find
any bottom; for while the “fifty-six” was resting by the way, they were
paying out the rope in the vain attempt to fathom their truly
immeasurable capacity for marvellousness. But I can assure my readers
that Walden has a reasonably tight bottom at a not unreasonable, though
at an unusual, depth. I fathomed it easily with a cod-line and a stone
weighing about a pound and a half, and could tell accurately when the
stone left the bottom, by having to pull so much harder before the
water got underneath to help me. The greatest depth was exactly one
hundred and two feet; to which may be added the five feet which it has
risen since, making one hundred and seven. This is a remarkable depth
for so small an area; yet not an inch of it can be spared by the
imagination. What if all ponds were shallow? Would it not react on the
minds of men? I am thankful that this pond was made deep and pure for a
symbol. While men believe in the infinite some ponds will be thought to
be bottomless.

A factory owner, hearing what depth I had found, thought that it could
not be true, for, judging from his acquaintance with dams, sand would
not lie at so steep an angle. But the deepest ponds are not so deep in
proportion to their area as most suppose, and, if drained, would not
leave very remarkable valleys. They are not like cups between the
hills; for this one, which is so unusually deep for its area, appears
in a vertical section through its centre not deeper than a shallow
plate. Most ponds, emptied, would leave a meadow no more hollow than we
frequently see. William Gilpin, who is so admirable in all that relates
to landscapes, and usually so correct, standing at the head of Loch
Fyne, in Scotland, which he describes as “a bay of salt water, sixty or
seventy fathoms deep, four miles in breadth,” and about fifty miles
long, surrounded by mountains, observes, “If we could have seen it
immediately after the diluvian crash, or whatever convulsion of Nature
occasioned it, before the waters gushed in, what a horrid chasm must it
have appeared!

     So high as heaved the tumid hills, so low
     Down sunk a hollow bottom broad and deep,
     Capacious bed of waters—.”

But if, using the shortest diameter of Loch Fyne, we apply these
proportions to Walden, which, as we have seen, appears already in a
vertical section only like a shallow plate, it will appear four times
as shallow. So much for the _increased_ horrors of the chasm of Loch
Fyne when emptied. No doubt many a smiling valley with its stretching
cornfields occupies exactly such a “horrid chasm,” from which the
waters have receded, though it requires the insight and the far sight
of the geologist to convince the unsuspecting inhabitants of this fact.
Often an inquisitive eye may detect the shores of a primitive lake in
the low horizon hills, and no subsequent elevation of the plain has
been necessary to conceal their history. But it is easiest, as they who
work on the highways know, to find the hollows by the puddles after a
shower. The amount of it is, the imagination give it the least license,
dives deeper and soars higher than Nature goes. So, probably, the depth
of the ocean will be found to be very inconsiderable compared with its
breadth.

As I sounded through the ice I could determine the shape of the bottom
with greater accuracy than is possible in surveying harbors which do
not freeze over, and I was surprised at its general regularity. In the
deepest part there are several acres more level than almost any field
which is exposed to the sun wind and plough. In one instance, on a line
arbitrarily chosen, the depth did not vary more than one foot in thirty
rods; and generally, near the middle, I could calculate the variation
for each one hundred feet in any direction beforehand within three or
four inches. Some are accustomed to speak of deep and dangerous holes
even in quiet sandy ponds like this, but the effect of water under
these circumstances is to level all inequalities. The regularity of the
bottom and its conformity to the shores and the range of the
neighboring hills were so perfect that a distant promontory betrayed
itself in the soundings quite across the pond, and its direction could
be determined by observing the opposite shore. Cape becomes bar, and
plain shoal, and valley and gorge deep water and channel.

When I had mapped the pond by the scale of ten rods to an inch, and put
down the soundings, more than a hundred in all, I observed this
remarkable coincidence. Having noticed that the number indicating the
greatest depth was apparently in the centre of the map, I laid a rule
on the map lengthwise, and then breadthwise, and found, to my surprise,
that the line of greatest length intersected the line of greatest
breadth _exactly_ at the point of greatest depth, notwithstanding that
the middle is so nearly level, the outline of the pond far from
regular, and the extreme length and breadth were got by measuring into
the coves; and I said to myself, Who knows but this hint would conduct
to the deepest part of the ocean as well as of a pond or puddle? Is not
this the rule also for the height of mountains, regarded as the
opposite of valleys? We know that a hill is not highest at its
narrowest part.

Of five coves, three, or all which had been sounded, were observed to
have a bar quite across their mouths and deeper water within, so that
the bay tended to be an expansion of water within the land not only
horizontally but vertically, and to form a basin or independent pond,
the direction of the two capes showing the course of the bar. Every
harbor on the sea-coast, also, has its bar at its entrance. In
proportion as the mouth of the cove was wider compared with its length,
the water over the bar was deeper compared with that in the basin.
Given, then, the length and breadth of the cove, and the character of
the surrounding shore, and you have almost elements enough to make out
a formula for all cases.

In order to see how nearly I could guess, with this experience, at the
deepest point in a pond, by observing the outlines of its surface and
the character of its shores alone, I made a plan of White Pond, which
contains about forty-one acres, and, like this, has no island in it,
nor any visible inlet or outlet; and as the line of greatest breadth
fell very near the line of least breadth, where two opposite capes
approached each other and two opposite bays receded, I ventured to mark
a point a short distance from the latter line, but still on the line of
greatest length, as the deepest. The deepest part was found to be
within one hundred feet of this, still farther in the direction to
which I had inclined, and was only one foot deeper, namely, sixty feet.
Of course, a stream running through, or an island in the pond, would
make the problem much more complicated.

If we knew all the laws of Nature, we should need only one fact, or the
description of one actual phenomenon, to infer all the particular
results at that point. Now we know only a few laws, and our result is
vitiated, not, of course, by any confusion or irregularity in Nature,
but by our ignorance of essential elements in the calculation. Our
notions of law and harmony are commonly confined to those instances
which we detect; but the harmony which results from a far greater
number of seemingly conflicting, but really concurring, laws, which we
have not detected, is still more wonderful. The particular laws are as
our points of view, as, to the traveller, a mountain outline varies
with every step, and it has an infinite number of profiles, though
absolutely but one form. Even when cleft or bored through it is not
comprehended in its entireness.

What I have observed of the pond is no less true in ethics. It is the
law of average. Such a rule of the two diameters not only guides us
toward the sun in the system and the heart in man, but draw lines
through the length and breadth of the aggregate of a man’s particular
daily behaviors and waves of life into his coves and inlets, and where
they intersect will be the height or depth of his character. Perhaps we
need only to know how his shores trend and his adjacent country or
circumstances, to infer his depth and concealed bottom. If he is
surrounded by mountainous circumstances, an Achillean shore, whose
peaks overshadow and are reflected in his bosom, they suggest a
corresponding depth in him. But a low and smooth shore proves him
shallow on that side. In our bodies, a bold projecting brow falls off
to and indicates a corresponding depth of thought. Also there is a bar
across the entrance of our every cove, or particular inclination; each
is our harbor for a season, in which we are detained and partially
land-locked. These inclinations are not whimsical usually, but their
form, size, and direction are determined by the promontories of the
shore, the ancient axes of elevation. When this bar is gradually
increased by storms, tides, or currents, or there is a subsidence of
the waters, so that it reaches to the surface, that which was at first
but an inclination in the shore in which a thought was harbored becomes
an individual lake, cut off from the ocean, wherein the thought secures
its own conditions, changes, perhaps, from salt to fresh, becomes a
sweet sea, dead sea, or a marsh. At the advent of each individual into
this life, may we not suppose that such a bar has risen to the surface
somewhere? It is true, we are such poor navigators that our thoughts,
for the most part, stand off and on upon a harborless coast, are
conversant only with the bights of the bays of poesy, or steer for the
public ports of entry, and go into the dry docks of science, where they
merely refit for this world, and no natural currents concur to
individualize them.

As for the inlet or outlet of Walden, I have not discovered any but
rain and snow and evaporation, though perhaps, with a thermometer and a
line, such places may be found, for where the water flows into the pond
it will probably be coldest in summer and warmest in winter. When the
ice-men were at work here in ’46–7, the cakes sent to the shore were
one day rejected by those who were stacking them up there, not being
thick enough to lie side by side with the rest; and the cutters thus
discovered that the ice over a small space was two or three inches
thinner than elsewhere, which made them think that there was an inlet
there. They also showed me in another place what they thought was a
“leach hole,” through which the pond leaked out under a hill into a
neighboring meadow, pushing me out on a cake of ice to see it. It was a
small cavity under ten feet of water; but I think that I can warrant
the pond not to need soldering till they find a worse leak than that.
One has suggested, that if such a “leach hole” should be found, its
connection with the meadow, if any existed, might be proved by
conveying some colored powder or sawdust to the mouth of the hole, and
then putting a strainer over the spring in the meadow, which would
catch some of the particles carried through by the current.

While I was surveying, the ice, which was sixteen inches thick,
undulated under a slight wind like water. It is well known that a level
cannot be used on ice. At one rod from the shore its greatest
fluctuation, when observed by means of a level on land directed toward
a graduated staff on the ice, was three quarters of an inch, though the
ice appeared firmly attached to the shore. It was probably greater in
the middle. Who knows but if our instruments were delicate enough we
might detect an undulation in the crust of the earth? When two legs of
my level were on the shore and the third on the ice, and the sights
were directed over the latter, a rise or fall of the ice of an almost
infinitesimal amount made a difference of several feet on a tree across
the pond. When I began to cut holes for sounding, there were three or
four inches of water on the ice under a deep snow which had sunk it
thus far; but the water began immediately to run into these holes, and
continued to run for two days in deep streams, which wore away the ice
on every side, and contributed essentially, if not mainly, to dry the
surface of the pond; for, as the water ran in, it raised and floated
the ice. This was somewhat like cutting a hole in the bottom of a ship
to let the water out. When such holes freeze, and a rain succeeds, and
finally a new freezing forms a fresh smooth ice over all, it is
beautifully mottled internally by dark figures, shaped somewhat like a
spider’s web, what you may call ice rosettes, produced by the channels
worn by the water flowing from all sides to a centre. Sometimes, also,
when the ice was covered with shallow puddles, I saw a double shadow of
myself, one standing on the head of the other, one on the ice, the
other on the trees or hill-side.



While yet it is cold January, and snow and ice are thick and solid, the
prudent landlord comes from the village to get ice to cool his summer
drink; impressively, even pathetically wise, to foresee the heat and
thirst of July now in January,—wearing a thick coat and mittens! when
so many things are not provided for. It may be that he lays up no
treasures in this world which will cool his summer drink in the next.
He cuts and saws the solid pond, unroofs the house of fishes, and carts
off their very element and air, held fast by chains and stakes like
corded wood, through the favoring winter air, to wintry cellars, to
underlie the summer there. It looks like solidified azure, as, far off,
it is drawn through the streets. These ice-cutters are a merry race,
full of jest and sport, and when I went among them they were wont to
invite me to saw pit-fashion with them, I standing underneath.

In the winter of ’46–7 there came a hundred men of Hyperborean
extraction swoop down on to our pond one morning, with many car-loads
of ungainly-looking farming tools, sleds, ploughs, drill-barrows,
turf-knives, spades, saws, rakes, and each man was armed with a
double-pointed pike-staff, such as is not described in the New-England
Farmer or the Cultivator. I did not know whether they had come to sow a
crop of winter rye, or some other kind of grain recently introduced
from Iceland. As I saw no manure, I judged that they meant to skim the
land, as I had done, thinking the soil was deep and had lain fallow
long enough. They said that a gentleman farmer, who was behind the
scenes, wanted to double his money, which, as I understood, amounted to
half a million already; but in order to cover each one of his dollars
with another, he took off the only coat, ay, the skin itself, of Walden
Pond in the midst of a hard winter. They went to work at once,
ploughing, harrowing, rolling, furrowing, in admirable order, as if
they were bent on making this a model farm; but when I was looking
sharp to see what kind of seed they dropped into the furrow, a gang of
fellows by my side suddenly began to hook up the virgin mould itself,
with a peculiar jerk, clean down to the sand, or rather the water,—for
it was a very springy soil,—indeed all the _terra firma_ there was,—and
haul it away on sleds, and then I guessed that they must be cutting
peat in a bog. So they came and went every day, with a peculiar shriek
from the locomotive, from and to some point of the polar regions, as it
seemed to me, like a flock of arctic snow-birds. But sometimes Squaw
Walden had her revenge, and a hired man, walking behind his team,
slipped through a crack in the ground down toward Tartarus, and he who
was so brave before suddenly became but the ninth part of a man, almost
gave up his animal heat, and was glad to take refuge in my house, and
acknowledged that there was some virtue in a stove; or sometimes the
frozen soil took a piece of steel out of a ploughshare, or a plough got
set in the furrow and had to be cut out.

To speak literally, a hundred Irishmen, with Yankee overseers, came
from Cambridge every day to get out the ice. They divided it into cakes
by methods too well known to require description, and these, being
sledded to the shore, were rapidly hauled off on to an ice platform,
and raised by grappling irons and block and tackle, worked by horses,
on to a stack, as surely as so many barrels of flour, and there placed
evenly side by side, and row upon row, as if they formed the solid base
of an obelisk designed to pierce the clouds. They told me that in a
good day they could get out a thousand tons, which was the yield of
about one acre. Deep ruts and “cradle holes” were worn in the ice, as
on _terra firma_, by the passage of the sleds over the same track, and
the horses invariably ate their oats out of cakes of ice hollowed out
like buckets. They stacked up the cakes thus in the open air in a pile
thirty-five feet high on one side and six or seven rods square, putting
hay between the outside layers to exclude the air; for when the wind,
though never so cold, finds a passage through, it will wear large
cavities, leaving slight supports or studs only here and there, and
finally topple it down. At first it looked like a vast blue fort or
Valhalla; but when they began to tuck the coarse meadow hay into the
crevices, and this became covered with rime and icicles, it looked like
a venerable moss-grown and hoary ruin, built of azure-tinted marble,
the abode of Winter, that old man we see in the almanac,—his shanty, as
if he had a design to estivate with us. They calculated that not
twenty-five per cent of this would reach its destination, and that two
or three per cent would be wasted in the cars. However, a still greater
part of this heap had a different destiny from what was intended; for,
either because the ice was found not to keep so well as was expected,
containing more air than usual, or for some other reason, it never got
to market. This heap, made in the winter of ’46–7 and estimated to
contain ten thousand tons, was finally covered with hay and boards; and
though it was unroofed the following July, and a part of it carried
off, the rest remaining exposed to the sun, it stood over that summer
and the next winter, and was not quite melted till September 1848. Thus
the pond recovered the greater part.

Like the water, the Walden ice, seen near at hand, has a green tint,
but at a distance is beautifully blue, and you can easily tell it from
the white ice of the river, or the merely greenish ice of some ponds, a
quarter of a mile off. Sometimes one of those great cakes slips from
the ice-man’s sled into the village street, and lies there for a week
like a great emerald, an object of interest to all passers. I have
noticed that a portion of Walden which in the state of water was green
will often, when frozen, appear from the same point of view blue. So
the hollows about this pond will, sometimes, in the winter, be filled
with a greenish water somewhat like its own, but the next day will have
frozen blue. Perhaps the blue color of water and ice is due to the
light and air they contain, and the most transparent is the bluest. Ice
is an interesting subject for contemplation. They told me that they had
some in the ice-houses at Fresh Pond five years old which was as good
as ever. Why is it that a bucket of water soon becomes putrid, but
frozen remains sweet forever? It is commonly said that this is the
difference between the affections and the intellect.

Thus for sixteen days I saw from my window a hundred men at work like
busy husbandmen, with teams and horses and apparently all the
implements of farming, such a picture as we see on the first page of
the almanac; and as often as I looked out I was reminded of the fable
of the lark and the reapers, or the parable of the sower, and the like;
and now they are all gone, and in thirty days more, probably, I shall
look from the same window on the pure sea-green Walden water there,
reflecting the clouds and the trees, and sending up its evaporations in
solitude, and no traces will appear that a man has ever stood there.
Perhaps I shall hear a solitary loon laugh as he dives and plumes
himself, or shall see a lonely fisher in his boat, like a floating
leaf, beholding his form reflected in the waves, where lately a hundred
men securely labored.

Thus it appears that the sweltering inhabitants of Charleston and New
Orleans, of Madras and Bombay and Calcutta, drink at my well. In the
morning I bathe my intellect in the stupendous and cosmogonal
philosophy of the Bhagvat Geeta, since whose composition years of the
gods have elapsed, and in comparison with which our modern world and
its literature seem puny and trivial; and I doubt if that philosophy is
not to be referred to a previous state of existence, so remote is its
sublimity from our conceptions. I lay down the book and go to my well
for water, and lo! there I meet the servant of the Bramin, priest of
Brahma and Vishnu and Indra, who still sits in his temple on the Ganges
reading the Vedas, or dwells at the root of a tree with his crust and
water jug. I meet his servant come to draw water for his master, and
our buckets as it were grate together in the same well. The pure Walden
water is mingled with the sacred water of the Ganges. With favoring
winds it is wafted past the site of the fabulous islands of Atlantis
and the Hesperides, makes the periplus of Hanno, and, floating by
Ternate and Tidore and the mouth of the Persian Gulf, melts in the
tropic gales of the Indian seas, and is landed in ports of which
Alexander only heard the names.


Spring

The opening of large tracts by the ice-cutters commonly causes a pond
to break up earlier; for the water, agitated by the wind, even in cold
weather, wears away the surrounding ice. But such was not the effect on
Walden that year, for she had soon got a thick new garment to take the
place of the old. This pond never breaks up so soon as the others in
this neighborhood, on account both of its greater depth and its having
no stream passing through it to melt or wear away the ice. I never knew
it to open in the course of a winter, not excepting that of ’52–3,
which gave the ponds so severe a trial. It commonly opens about the
first of April, a week or ten days later than Flint’s Pond and
Fair-Haven, beginning to melt on the north side and in the shallower
parts where it began to freeze. It indicates better than any water
hereabouts the absolute progress of the season, being least affected by
transient changes of temperature. A severe cold of a few days’ duration
in March may very much retard the opening of the former ponds, while
the temperature of Walden increases almost uninterruptedly. A
thermometer thrust into the middle of Walden on the 6th of March, 1847,
stood at 32°, or freezing point; near the shore at 33°; in the middle
of Flint’s Pond, the same day, at 32½°; at a dozen rods from the shore,
in shallow water, under ice a foot thick, at 36°. This difference of
three and a half degrees between the temperature of the deep water and
the shallow in the latter pond, and the fact that a great proportion of
it is comparatively shallow, show why it should break up so much sooner
than Walden. The ice in the shallowest part was at this time several
inches thinner than in the middle. In mid-winter the middle had been
the warmest and the ice thinnest there. So, also, every one who has
waded about the shores of the pond in summer must have perceived how
much warmer the water is close to the shore, where only three or four
inches deep, than a little distance out, and on the surface where it is
deep, than near the bottom. In spring the sun not only exerts an
influence through the increased temperature of the air and earth, but
its heat passes through ice a foot or more thick, and is reflected from
the bottom in shallow water, and so also warms the water and melts the
under side of the ice, at the same time that it is melting it more
directly above, making it uneven, and causing the air bubbles which it
contains to extend themselves upward and downward until it is
completely honeycombed, and at last disappears suddenly in a single
spring rain. Ice has its grain as well as wood, and when a cake begins
to rot or “comb,” that is, assume the appearance of honey-comb,
whatever may be its position, the air cells are at right angles with
what was the water surface. Where there is a rock or a log rising near
to the surface the ice over it is much thinner, and is frequently quite
dissolved by this reflected heat; and I have been told that in the
experiment at Cambridge to freeze water in a shallow wooden pond,
though the cold air circulated underneath, and so had access to both
sides, the reflection of the sun from the bottom more than
counterbalanced this advantage. When a warm rain in the middle of the
winter melts off the snow-ice from Walden, and leaves a hard dark or
transparent ice on the middle, there will be a strip of rotten though
thicker white ice, a rod or more wide, about the shores, created by
this reflected heat. Also, as I have said, the bubbles themselves
within the ice operate as burning-glasses to melt the ice beneath.

The phenomena of the year take place every day in a pond on a small
scale. Every morning, generally speaking, the shallow water is being
warmed more rapidly than the deep, though it may not be made so warm
after all, and every evening it is being cooled more rapidly until the
morning. The day is an epitome of the year. The night is the winter,
the morning and evening are the spring and fall, and the noon is the
summer. The cracking and booming of the ice indicate a change of
temperature. One pleasant morning after a cold night, February 24th,
1850, having gone to Flint’s Pond to spend the day, I noticed with
surprise, that when I struck the ice with the head of my axe, it
resounded like a gong for many rods around, or as if I had struck on a
tight drum-head. The pond began to boom about an hour after sunrise,
when it felt the influence of the sun’s rays slanted upon it from over
the hills; it stretched itself and yawned like a waking man with a
gradually increasing tumult, which was kept up three or four hours. It
took a short siesta at noon, and boomed once more toward night, as the
sun was withdrawing his influence. In the right stage of the weather a
pond fires its evening gun with great regularity. But in the middle of
the day, being full of cracks, and the air also being less elastic, it
had completely lost its resonance, and probably fishes and muskrats
could not then have been stunned by a blow on it. The fishermen say
that the “thundering of the pond” scares the fishes and prevents their
biting. The pond does not thunder every evening, and I cannot tell
surely when to expect its thundering; but though I may perceive no
difference in the weather, it does. Who would have suspected so large
and cold and thick-skinned a thing to be so sensitive? Yet it has its
law to which it thunders obedience when it should as surely as the buds
expand in the spring. The earth is all alive and covered with papillæ.
The largest pond is as sensitive to atmospheric changes as the globule
of mercury in its tube.



One attraction in coming to the woods to live was that I should have
leisure and opportunity to see the spring come in. The ice in the pond
at length begins to be honey-combed, and I can set my heel in it as I
walk. Fogs and rains and warmer suns are gradually melting the snow;
the days have grown sensibly longer; and I see how I shall get through
the winter without adding to my wood-pile, for large fires are no
longer necessary. I am on the alert for the first signs of spring, to
hear the chance note of some arriving bird, or the striped squirrel’s
chirp, for his stores must be now nearly exhausted, or see the
woodchuck venture out of his winter quarters. On the 13th of March,
after I had heard the bluebird, song-sparrow, and red-wing, the ice was
still nearly a foot thick. As the weather grew warmer it was not
sensibly worn away by the water, nor broken up and floated off as in
rivers, but, though it was completely melted for half a rod in width
about the shore, the middle was merely honey-combed and saturated with
water, so that you could put your foot through it when six inches
thick; but by the next day evening, perhaps, after a warm rain followed
by fog, it would have wholly disappeared, all gone off with the fog,
spirited away. One year I went across the middle only five days before
it disappeared entirely. In 1845 Walden was first completely open on
the 1st of April; in ’46, the 25th of March; in ’47, the 8th of April;
in ’51, the 28th of March; in ’52, the 18th of April; in ’53, the 23d
of March; in ’54, about the 7th of April.

Every incident connected with the breaking up of the rivers and ponds
and the settling of the weather is particularly interesting to us who
live in a climate of so great extremes. When the warmer days come, they
who dwell near the river hear the ice crack at night with a startling
whoop as loud as artillery, as if its icy fetters were rent from end to
end, and within a few days see it rapidly going out. So the alligator
comes out of the mud with quakings of the earth. One old man, who has
been a close observer of Nature, and seems as thoroughly wise in regard
to all her operations as if she had been put upon the stocks when he
was a boy, and he had helped to lay her keel,—who has come to his
growth, and can hardly acquire more of natural lore if he should live
to the age of Methuselah,—told me, and I was surprised to hear him
express wonder at any of Nature’s operations, for I thought that there
were no secrets between them, that one spring day he took his gun and
boat, and thought that he would have a little sport with the ducks.
There was ice still on the meadows, but it was all gone out of the
river, and he dropped down without obstruction from Sudbury, where he
lived, to Fair-Haven Pond, which he found, unexpectedly, covered for
the most part with a firm field of ice. It was a warm day, and he was
surprised to see so great a body of ice remaining. Not seeing any
ducks, he hid his boat on the north or back side of an island in the
pond, and then concealed himself in the bushes on the south side, to
await them. The ice was melted for three or four rods from the shore,
and there was a smooth and warm sheet of water, with a muddy bottom,
such as the ducks love, within, and he thought it likely that some
would be along pretty soon. After he had lain still there about an hour
he heard a low and seemingly very distant sound, but singularly grand
and impressive, unlike any thing he had ever heard, gradually swelling
and increasing as if it would have a universal and memorable ending, a
sullen rush and roar, which seemed to him all at once like the sound of
a vast body of fowl coming in to settle there, and, seizing his gun, he
started up in haste and excited; but he found, to his surprise, that
the whole body of the ice had started while he lay there, and drifted
in to the shore, and the sound he had heard was made by its edge
grating on the shore,—at first gently nibbled and crumbled off, but at
length heaving up and scattering its wrecks along the island to a
considerable height before it came to a stand still.

At length the sun’s rays have attained the right angle, and warm winds
blow up mist and rain and melt the snow banks, and the sun dispersing
the mist smiles on a checkered landscape of russet and white smoking
with incense, through which the traveller picks his way from islet to
islet, cheered by the music of a thousand tinkling rills and rivulets
whose veins are filled with the blood of winter which they are bearing
off.

Few phenomena gave me more delight than to observe the forms which
thawing sand and clay assume in flowing down the sides of a deep cut on
the railroad through which I passed on my way to the village, a
phenomenon not very common on so large a scale, though the number of
freshly exposed banks of the right material must have been greatly
multiplied since railroads were invented. The material was sand of
every degree of fineness and of various rich colors, commonly mixed
with a little clay. When the frost comes out in the spring, and even in
a thawing day in the winter, the sand begins to flow down the slopes
like lava, sometimes bursting out through the snow and overflowing it
where no sand was to be seen before. Innumerable little streams overlap
and interlace one with another, exhibiting a sort of hybrid product,
which obeys half way the law of currents, and half way that of
vegetation. As it flows it takes the forms of sappy leaves or vines,
making heaps of pulpy sprays a foot or more in depth, and resembling,
as you look down on them, the laciniated, lobed, and imbricated
thalluses of some lichens; or you are reminded of coral, of leopard’s
paws or birds’ feet, of brains or lungs or bowels, and excrements of
all kinds. It is a truly _grotesque_ vegetation, whose forms and color
we see imitated in bronze, a sort of architectural foliage more ancient
and typical than acanthus, chiccory, ivy, vine, or any vegetable
leaves; destined perhaps, under some circumstances, to become a puzzle
to future geologists. The whole cut impressed me as if it were a cave
with its stalactites laid open to the light. The various shades of the
sand are singularly rich and agreeable, embracing the different iron
colors, brown, gray, yellowish, and reddish. When the flowing mass
reaches the drain at the foot of the bank it spreads out flatter into
_strands_, the separate streams losing their semi-cylindrical form and
gradually becoming more flat and broad, running together as they are
more moist, till they form an almost flat _sand_, still variously and
beautifully shaded, but in which you can trace the original forms of
vegetation; till at length, in the water itself, they are converted
into _banks_, like those formed off the mouths of rivers, and the forms
of vegetation are lost in the ripple marks on the bottom.

The whole bank, which is from twenty to forty feet high, is sometimes
overlaid with a mass of this kind of foliage, or sandy rupture, for a
quarter of a mile on one or both sides, the produce of one spring day.
What makes this sand foliage remarkable is its springing into existence
thus suddenly. When I see on the one side the inert bank,—for the sun
acts on one side first,—and on the other this luxuriant foliage, the
creation of an hour, I am affected as if in a peculiar sense I stood in
the laboratory of the Artist who made the world and me,—had come to
where he was still at work, sporting on this bank, and with excess of
energy strewing his fresh designs about. I feel as if I were nearer to
the vitals of the globe, for this sandy overflow is something such a
foliaceous mass as the vitals of the animal body. You find thus in the
very sands an anticipation of the vegetable leaf. No wonder that the
earth expresses itself outwardly in leaves, it so labors with the idea
inwardly. The atoms have already learned this law, and are pregnant by
it. The overhanging leaf sees here its prototype. _Internally_, whether
in the globe or animal body, it is a moist thick _lobe_, a word
especially applicable to the liver and lungs and the _leaves_ of fat,
(?e?ß?, _labor_, _lapsus_, to flow or slip downward, a lapsing; ??ß??,
_globus_, lobe, globe; also lap, flap, and many other words,)
_externally_ a dry thin _leaf_, even as the _f_ and _v_ are a pressed
and dried _b_. The radicals of _lobe_ are _lb_, the soft mass of the
_b_ (single lobed, or B, double lobed,) with the liquid _l_ behind it
pressing it forward. In globe, _glb_, the guttural _g_ adds to the
meaning the capacity of the throat. The feathers and wings of birds are
still drier and thinner leaves. Thus, also, you pass from the lumpish
grub in the earth to the airy and fluttering butterfly. The very globe
continually transcends and translates itself, and becomes winged in its
orbit. Even ice begins with delicate crystal leaves, as if it had
flowed into moulds which the fronds of water plants have impressed on
the watery mirror. The whole tree itself is but one leaf, and rivers
are still vaster leaves whose pulp is intervening earth, and towns and
cities are the ova of insects in their axils.

When the sun withdraws the sand ceases to flow, but in the morning the
streams will start once more and branch and branch again into a myriad
of others. You here see perchance how blood vessels are formed. If you
look closely you observe that first there pushes forward from the
thawing mass a stream of softened sand with a drop-like point, like the
ball of the finger, feeling its way slowly and blindly downward, until
at last with more heat and moisture, as the sun gets higher, the most
fluid portion, in its effort to obey the law to which the most inert
also yields, separates from the latter and forms for itself a
meandering channel or artery within that, in which is seen a little
silvery stream glancing like lightning from one stage of pulpy leaves
or branches to another, and ever and anon swallowed up in the sand. It
is wonderful how rapidly yet perfectly the sand organizes itself as it
flows, using the best material its mass affords to form the sharp edges
of its channel. Such are the sources of rivers. In the silicious matter
which the water deposits is perhaps the bony system, and in the still
finer soil and organic matter the fleshy fibre or cellular tissue. What
is man but a mass of thawing clay? The ball of the human finger is but
a drop congealed. The fingers and toes flow to their extent from the
thawing mass of the body. Who knows what the human body would expand
and flow out to under a more genial heaven? Is not the hand a spreading
_palm_ leaf with its lobes and veins? The ear may be regarded,
fancifully, as a lichen, _umbilicaria_, on the side of the head, with
its lobe or drop. The lip—_labium_, from _labor_ (?)—laps or lapses
from the sides of the cavernous mouth. The nose is a manifest congealed
drop or stalactite. The chin is a still larger drop, the confluent
dripping of the face. The cheeks are a slide from the brows into the
valley of the face, opposed and diffused by the cheek bones. Each
rounded lobe of the vegetable leaf, too, is a thick and now loitering
drop, larger or smaller; the lobes are the fingers of the leaf; and as
many lobes as it has, in so many directions it tends to flow, and more
heat or other genial influences would have caused it to flow yet
farther.

Thus it seemed that this one hillside illustrated the principle of all
the operations of Nature. The Maker of this earth but patented a leaf.
What Champollion will decipher this hieroglyphic for us, that we may
turn over a new leaf at last? This phenomenon is more exhilarating to
me than the luxuriance and fertility of vineyards. True, it is somewhat
excrementitious in its character, and there is no end to the heaps of
liver lights and bowels, as if the globe were turned wrong side
outward; but this suggests at least that Nature has some bowels, and
there again is mother of humanity. This is the frost coming out of the
ground; this is Spring. It precedes the green and flowery spring, as
mythology precedes regular poetry. I know of nothing more purgative of
winter fumes and indigestions. It convinces me that Earth is still in
her swaddling clothes, and stretches forth baby fingers on every side.
Fresh curls spring from the baldest brow. There is nothing inorganic.
These foliaceous heaps lie along the bank like the slag of a furnace,
showing that Nature is “in full blast” within. The earth is not a mere
fragment of dead history, stratum upon stratum like the leaves of a
book, to be studied by geologists and antiquaries chiefly, but living
poetry like the leaves of a tree, which precede flowers and fruit,—not
a fossil earth, but a living earth; compared with whose great central
life all animal and vegetable life is merely parasitic. Its throes will
heave our exuviæ from their graves. You may melt your metals and cast
them into the most beautiful moulds you can; they will never excite me
like the forms which this molten earth flows out into. And not only it,
but the institutions upon it, are plastic like clay in the hands of the
potter.



Ere long, not only on these banks, but on every hill and plain and in
every hollow, the frost comes out of the ground like a dormant
quadruped from its burrow, and seeks the sea with music, or migrates to
other climes in clouds. Thaw with his gentle persuasion is more
powerful than Thor with his hammer. The one melts, the other but breaks
in pieces.

When the ground was partially bare of snow, and a few warm days had
dried its surface somewhat, it was pleasant to compare the first tender
signs of the infant year just peeping forth with the stately beauty of
the withered vegetation which had withstood the
winter,—life-everlasting, golden-rods, pinweeds, and graceful wild
grasses, more obvious and interesting frequently than in summer even,
as if their beauty was not ripe till then; even cotton-grass,
cat-tails, mulleins, johnswort, hard-hack, meadow-sweet, and other
strong stemmed plants, those unexhausted granaries which entertain the
earliest birds,—decent weeds, at least, which widowed Nature wears. I
am particularly attracted by the arching and sheaf-like top of the
wool-grass; it brings back the summer to our winter memories, and is
among the forms which art loves to copy, and which, in the vegetable
kingdom, have the same relation to types already in the mind of man
that astronomy has. It is an antique style older than Greek or
Egyptian. Many of the phenomena of Winter are suggestive of an
inexpressible tenderness and fragile delicacy. We are accustomed to
hear this king described as a rude and boisterous tyrant; but with the
gentleness of a lover he adorns the tresses of Summer.

At the approach of spring the red-squirrels got under my house, two at
a time, directly under my feet as I sat reading or writing, and kept up
the queerest chuckling and chirruping and vocal pirouetting and
gurgling sounds that ever were heard; and when I stamped they only
chirruped the louder, as if past all fear and respect in their mad
pranks, defying humanity to stop them. No you
don’t—chickaree—chickaree. They were wholly deaf to my arguments, or
failed to perceive their force, and fell into a strain of invective
that was irresistible.

The first sparrow of spring! The year beginning with younger hope than
ever! The faint silvery warblings heard over the partially bare and
moist fields from the blue-bird, the song-sparrow, and the red-wing, as
if the last flakes of winter tinkled as they fell! What at such a time
are histories, chronologies, traditions, and all written revelations?
The brooks sing carols and glees to the spring. The marsh-hawk sailing
low over the meadow is already seeking the first slimy life that
awakes. The sinking sound of melting snow is heard in all dells, and
the ice dissolves apace in the ponds. The grass flames up on the
hillsides like a spring fire,—“et primitus oritur herba imbribus
primoribus evocata,”—as if the earth sent forth an inward heat to greet
the returning sun; not yellow but green is the color of its flame;—the
symbol of perpetual youth, the grass-blade, like a long green ribbon,
streams from the sod into the summer, checked indeed by the frost, but
anon pushing on again, lifting its spear of last year’s hay with the
fresh life below. It grows as steadily as the rill oozes out of the
ground. It is almost identical with that, for in the growing days of
June, when the rills are dry, the grass blades are their channels, and
from year to year the herds drink at this perennial green stream, and
the mower draws from it betimes their winter supply. So our human life
but dies down to its root, and still puts forth its green blade to
eternity.

Walden is melting apace. There is a canal two rods wide along the
northerly and westerly sides, and wider still at the east end. A great
field of ice has cracked off from the main body. I hear a song-sparrow
singing from the bushes on the shore,—_olit_, _olit_, _olit,_—_chip_,
_chip_, _chip_, _che char_,—_che wiss_, _wiss_, _wiss_. He too is
helping to crack it. How handsome the great sweeping curves in the edge
of the ice, answering somewhat to those of the shore, but more regular!
It is unusually hard, owing to the recent severe but transient cold,
and all watered or waved like a palace floor. But the wind slides
eastward over its opaque surface in vain, till it reaches the living
surface beyond. It is glorious to behold this ribbon of water sparkling
in the sun, the bare face of the pond full of glee and youth, as if it
spoke the joy of the fishes within it, and of the sands on its shore,—a
silvery sheen as from the scales of a _leuciscus_, as it were all one
active fish. Such is the contrast between winter and spring. Walden was
dead and is alive again. But this spring it broke up more steadily, as
I have said.

The change from storm and winter to serene and mild weather, from dark
and sluggish hours to bright and elastic ones, is a memorable crisis
which all things proclaim. It is seemingly instantaneous at last.
Suddenly an influx of light filled my house, though the evening was at
hand, and the clouds of winter still overhung it, and the eaves were
dripping with sleety rain. I looked out the window, and lo! where
yesterday was cold gray ice there lay the transparent pond already calm
and full of hope as in a summer evening, reflecting a summer evening
sky in its bosom, though none was visible overhead, as if it had
intelligence with some remote horizon. I heard a robin in the distance,
the first I had heard for many a thousand years, methought, whose note
I shall not forget for many a thousand more,—the same sweet and
powerful song as of yore. O the evening robin, at the end of a New
England summer day! If I could ever find the twig he sits upon! I mean
_he_; I mean _the twig_. This at least is not the _Turdus migratorius_.
The pitch-pines and shrub-oaks about my house, which had so long
drooped, suddenly resumed their several characters, looked brighter,
greener, and more erect and alive, as if effectually cleansed and
restored by the rain. I knew that it would not rain any more. You may
tell by looking at any twig of the forest, ay, at your very wood-pile,
whether its winter is past or not. As it grew darker, I was startled by
the _honking_ of geese flying low over the woods, like weary travellers
getting in late from southern lakes, and indulging at last in
unrestrained complaint and mutual consolation. Standing at my door, I
could hear the rush of their wings; when, driving toward my house, they
suddenly spied my light, and with hushed clamor wheeled and settled in
the pond. So I came in, and shut the door, and passed my first spring
night in the woods.

In the morning I watched the geese from the door through the mist,
sailing in the middle of the pond, fifty rods off, so large and
tumultuous that Walden appeared like an artificial pond for their
amusement. But when I stood on the shore they at once rose up with a
great flapping of wings at the signal of their commander, and when they
had got into rank circled about over my head, twenty-nine of them, and
then steered straight to Canada, with a regular _honk_ from the leader
at intervals, trusting to break their fast in muddier pools. A “plump”
of ducks rose at the same time and took the route to the north in the
wake of their noisier cousins.

For a week I heard the circling, groping clangor of some solitary goose
in the foggy mornings, seeking its companion, and still peopling the
woods with the sound of a larger life than they could sustain. In April
the pigeons were seen again flying express in small flocks, and in due
time I heard the martins twittering over my clearing, though it had not
seemed that the township contained so many that it could afford me any,
and I fancied that they were peculiarly of the ancient race that dwelt
in hollow trees ere white men came. In almost all climes the tortoise
and the frog are among the precursors and heralds of this season, and
birds fly with song and glancing plumage, and plants spring and bloom,
and winds blow, to correct this slight oscillation of the poles and
preserve the equilibrium of Nature.

As every season seems best to us in its turn, so the coming in of
spring is like the creation of Cosmos out of Chaos and the realization
of the Golden Age.—

     “Eurus ad Auroram Nabathæaque regna recessit,
     Persidaque, et radiis juga subdita matutinis.”

     “The East-Wind withdrew to Aurora and the Nabathæan kingdom,
     And the Persian, and the ridges placed under the morning rays

            *    *    *    *

     Man was born.  Whether that Artificer of things,
     The origin of a better world, made him from the divine seed;
     Or the earth, being recent and lately sundered from the high
     Ether, retained some seeds of cognate heaven.”

A single gentle rain makes the grass many shades greener. So our
prospects brighten on the influx of better thoughts. We should be
blessed if we lived in the present always, and took advantage of every
accident that befell us, like the grass which confesses the influence
of the slightest dew that falls on it; and did not spend our time in
atoning for the neglect of past opportunities, which we call doing our
duty. We loiter in winter while it is already spring. In a pleasant
spring morning all men’s sins are forgiven. Such a day is a truce to
vice. While such a sun holds out to burn, the vilest sinner may return.
Through our own recovered innocence we discern the innocence of our
neighbors. You may have known your neighbor yesterday for a thief, a
drunkard, or a sensualist, and merely pitied or despised him, and
despaired of the world; but the sun shines bright and warm this first
spring morning, re-creating the world, and you meet him at some serene
work, and see how his exhausted and debauched veins expand with still
joy and bless the new day, feel the spring influence with the innocence
of infancy, and all his faults are forgotten. There is not only an
atmosphere of good will about him, but even a savor of holiness groping
for expression, blindly and ineffectually perhaps, like a new-born
instinct, and for a short hour the south hill-side echoes to no vulgar
jest. You see some innocent fair shoots preparing to burst from his
gnarled rind and try another year’s life, tender and fresh as the
youngest plant. Even he has entered into the joy of his Lord. Why the
jailer does not leave open his prison doors,—why the judge does not
dismis his case,—why the preacher does not dismiss his congregation! It
is because they do not obey the hint which God gives them, nor accept
the pardon which he freely offers to all.

“A return to goodness produced each day in the tranquil and beneficent
breath of the morning, causes that in respect to the love of virtue and
the hatred of vice, one approaches a little the primitive nature of
man, as the sprouts of the forest which has been felled. In like manner
the evil which one does in the interval of a day prevents the germs of
virtues which began to spring up again from developing themselves and
destroys them.

“After the germs of virtue have thus been prevented many times from
developing themselves, then the beneficent breath of evening does not
suffice to preserve them. As soon as the breath of evening does not
suffice longer to preserve them, then the nature of man does not differ
much from that of the brute. Men seeing the nature of this man like
that of the brute, think that he has never possessed the innate faculty
of reason. Are those the true and natural sentiments of man?”

     “The Golden Age was first created, which without any avenger
     Spontaneously without law cherished fidelity and rectitude.
     Punishment and fear were not; nor were threatening words read
     On suspended brass; nor did the suppliant crowd fear
     The words of their judge; but were safe without an avenger.
     Not yet the pine felled on its mountains had descended
     To the liquid waves that it might see a foreign world,
     And mortals knew no shores but their own.
            *    *    *    *
     There was eternal spring, and placid zephyrs with warm
     Blasts soothed the flowers born without seed.”

On the 29th of April, as I was fishing from the bank of the river near
the Nine-Acre-Corner bridge, standing on the quaking grass and willow
roots, where the muskrats lurk, I heard a singular rattling sound,
somewhat like that of the sticks which boys play with their fingers,
when, looking up, I observed a very slight and graceful hawk, like a
night-hawk, alternately soaring like a ripple and tumbling a rod or two
over and over, showing the underside of its wings, which gleamed like a
satin ribbon in the sun, or like the pearly inside of a shell. This
sight reminded me of falconry and what nobleness and poetry are
associated with that sport. The Merlin it seemed to me it might be
called: but I care not for its name. It was the most ethereal flight I
had ever witnessed. It did not simply flutter like a butterfly, nor
soar like the larger hawks, but it sported with proud reliance in the
fields of air; mounting again and again with its strange chuckle, it
repeated its free and beautiful fall, turning over and over like a
kite, and then recovering from its lofty tumbling, as if it had never
set its foot on _terra firma_. It appeared to have no companion in the
universe,—sporting there alone,—and to need none but the morning and
the ether with which it played. It was not lonely, but made all the
earth lonely beneath it. Where was the parent which hatched it, its
kindred, and its father in the heavens? The tenant of the air, it
seemed related to the earth but by an egg hatched some time in the
crevice of a crag;—or was its native nest made in the angle of a cloud,
woven of the rainbow’s trimmings and the sunset sky, and lined with
some soft midsummer haze caught up from earth? Its eyry now some cliffy
cloud.

Beside this I got a rare mess of golden and silver and bright cupreous
fishes, which looked like a string of jewels. Ah! I have penetrated to
those meadows on the morning of many a first spring day, jumping from
hummock to hummock, from willow root to willow root, when the wild
river valley and the woods were bathed in so pure and bright a light as
would have waked the dead, if they had been slumbering in their graves,
as some suppose. There needs no stronger proof of immortality. All
things must live in such a light. O Death, where was thy sting? O
Grave, where was thy victory, then?

Our village life would stagnate if it were not for the unexplored
forests and meadows which surround it. We need the tonic of
wildness,—to wade sometimes in marshes where the bittern and the
meadow-hen lurk, and hear the booming of the snipe; to smell the
whispering sedge where only some wilder and more solitary fowl builds
her nest, and the mink crawls with its belly close to the ground. At
the same time that we are earnest to explore and learn all things, we
require that all things be mysterious and unexplorable, that land and
sea be infinitely wild, unsurveyed and unfathomed by us because
unfathomable. We can never have enough of Nature. We must be refreshed
by the sight of inexhaustible vigor, vast and Titanic features, the
sea-coast with its wrecks, the wilderness with its living and its
decaying trees, the thunder cloud, and the rain which lasts three weeks
and produces freshets. We need to witness our own limits transgressed,
and some life pasturing freely where we never wander. We are cheered
when we observe the vulture feeding on the carrion which disgusts and
disheartens us and deriving health and strength from the repast. There
was a dead horse in the hollow by the path to my house, which compelled
me sometimes to go out of my way, especially in the night when the air
was heavy, but the assurance it gave me of the strong appetite and
inviolable health of Nature was my compensation for this. I love to see
that Nature is so rife with life that myriads can be afforded to be
sacrificed and suffered to prey on one another; that tender
organizations can be so serenely squashed out of existence like
pulp,—tadpoles which herons gobble up, and tortoises and toads run over
in the road; and that sometimes it has rained flesh and blood! With the
liability to accident, we must see how little account is to be made of
it. The impression made on a wise man is that of universal innocence.
Poison is not poisonous after all, nor are any wounds fatal. Compassion
is a very untenable ground. It must be expeditious. Its pleadings will
not bear to be stereotyped.

Early in May, the oaks, hickories, maples, and other trees, just
putting out amidst the pine woods around the pond, imparted a
brightness like sunshine to the landscape, especially in cloudy days,
as if the sun were breaking through mists and shining faintly on the
hill-sides here and there. On the third or fourth of May I saw a loon
in the pond, and during the first week of the month I heard the
whippoorwill, the brown-thrasher, the veery, the wood-pewee, the
chewink, and other birds. I had heard the wood-thrush long before. The
phœbe had already come once more and looked in at my door and window,
to see if my house was cavern-like enough for her, sustaining herself
on humming wings with clinched talons, as if she held by the air, while
she surveyed the premises. The sulphur-like pollen of the pitch-pine
soon covered the pond and the stones and rotten wood along the shore,
so that you could have collected a barrel-ful. This is the “sulphur
showers” we hear of. Even in Calidas’ drama of Sacontala, we read of
“rills dyed yellow with the golden dust of the lotus.” And so the
seasons went rolling on into summer, as one rambles into higher and
higher grass.

Thus was my first year’s life in the woods completed; and the second
year was similar to it. I finally left Walden September 6th, 1847.


Conclusion

To the sick the doctors wisely recommend a change of air and scenery.
Thank Heaven, here is not all the world. The buck-eye does not grow in
New England, and the mocking-bird is rarely heard here. The wild-goose
is more of a cosmopolite than we; he breaks his fast in Canada, takes a
luncheon in the Ohio, and plumes himself for the night in a southern
bayou. Even the bison, to some extent, keeps pace with the seasons,
cropping the pastures of the Colorado only till a greener and sweeter
grass awaits him by the Yellowstone. Yet we think that if rail-fences
are pulled down, and stone-walls piled up on our farms, bounds are
henceforth set to our lives and our fates decided. If you are chosen
town-clerk, forsooth, you cannot go to Tierra del Fuego this summer:
but you may go to the land of infernal fire nevertheless. The universe
is wider than our views of it.

Yet we should oftener look over the tafferel of our craft, like curious
passengers, and not make the voyage like stupid sailors picking oakum.
The other side of the globe is but the home of our correspondent. Our
voyaging is only great-circle sailing, and the doctors prescribe for
diseases of the skin merely. One hastens to Southern Africa to chase
the giraffe; but surely that is not the game he would be after. How
long, pray, would a man hunt giraffes if he could? Snipes and woodcocks
also may afford rare sport; but I trust it would be nobler game to
shoot one’s self.—

     “Direct your eye right inward, and you’ll find
     A thousand regions in your mind
     Yet undiscovered. Travel them, and be
     Expert in home-cosmography.”

What does Africa,—what does the West stand for? Is not our own interior
white on the chart? black though it may prove, like the coast, when
discovered. Is it the source of the Nile, or the Niger, or the
Mississippi, or a North-West Passage around this continent, that we
would find? Are these the problems which most concern mankind? Is
Franklin the only man who is lost, that his wife should be so earnest
to find him? Does Mr. Grinnell know where he himself is? Be rather the
Mungo Park, the Lewis and Clarke and Frobisher, of your own streams and
oceans; explore your own higher latitudes,—with shiploads of preserved
meats to support you, if they be necessary; and pile the empty cans
sky-high for a sign. Were preserved meats invented to preserve meat
merely? Nay, be a Columbus to whole new continents and worlds within
you, opening new channels, not of trade, but of thought. Every man is
the lord of a realm beside which the earthly empire of the Czar is but
a petty state, a hummock left by the ice. Yet some can be patriotic who
have no _self_-respect, and sacrifice the greater to the less. They
love the soil which makes their graves, but have no sympathy with the
spirit which may still animate their clay. Patriotism is a maggot in
their heads. What was the meaning of that South-Sea Exploring
Expedition, with all its parade and expense, but an indirect
recognition of the fact, that there are continents and seas in the
moral world to which every man is an isthmus or an inlet, yet
unexplored by him, but that it is easier to sail many thousand miles
through cold and storm and cannibals, in a government ship, with five
hundred men and boys to assist one, than it is to explore the private
sea, the Atlantic and Pacific Ocean of one’s being alone.—

     “Erret, et extremos alter scrutetur Iberos.
     Plus habet hic vitæ, plus habet ille viæ.”

     Let them wander and scrutinize the outlandish Australians.
     I have more of God, they more of the road.

It is not worth the while to go round the world to count the cats in
Zanzibar. Yet do this even till you can do better, and you may perhaps
find some “Symmes’ Hole” by which to get at the inside at last. England
and France, Spain and Portugal, Gold Coast and Slave Coast, all front
on this private sea; but no bark from them has ventured out of sight of
land, though it is without doubt the direct way to India. If you would
learn to speak all tongues and conform to the customs of all nations,
if you would travel farther than all travellers, be naturalized in all
climes, and cause the Sphinx to dash her head against a stone, even
obey the precept of the old philosopher, and Explore thyself. Herein
are demanded the eye and the nerve. Only the defeated and deserters go
to the wars, cowards that run away and enlist. Start now on that
farthest western way, which does not pause at the Mississippi or the
Pacific, nor conduct toward a worn-out China or Japan, but leads on
direct a tangent to this sphere, summer and winter, day and night, sun
down, moon down, and at last earth down too.

It is said that Mirabeau took to highway robbery “to ascertain what
degree of resolution was necessary in order to place one’s self in
formal opposition to the most sacred laws of society.” He declared that
“a soldier who fights in the ranks does not require half so much
courage as a foot-pad,”—“that honor and religion have never stood in
the way of a well-considered and a firm resolve.” This was manly, as
the world goes; and yet it was idle, if not desperate. A saner man
would have found himself often enough “in formal opposition” to what
are deemed “the most sacred laws of society,” through obedience to yet
more sacred laws, and so have tested his resolution without going out
of his way. It is not for a man to put himself in such an attitude to
society, but to maintain himself in whatever attitude he find himself
through obedience to the laws of his being, which will never be one of
opposition to a just government, if he should chance to meet with such.

I left the woods for as good a reason as I went there. Perhaps it
seemed to me that I had several more lives to live, and could not spare
any more time for that one. It is remarkable how easily and insensibly
we fall into a particular route, and make a beaten track for ourselves.
I had not lived there a week before my feet wore a path from my door to
the pond-side; and though it is five or six years since I trod it, it
is still quite distinct. It is true, I fear that others may have fallen
into it, and so helped to keep it open. The surface of the earth is
soft and impressible by the feet of men; and so with the paths which
the mind travels. How worn and dusty, then, must be the highways of the
world, how deep the ruts of tradition and conformity! I did not wish to
take a cabin passage, but rather to go before the mast and on the deck
of the world, for there I could best see the moonlight amid the
mountains. I do not wish to go below now.

I learned this, at least, by my experiment; that if one advances
confidently in the direction of his dreams, and endeavors to live the
life which he has imagined, he will meet with a success unexpected in
common hours. He will put some things behind, will pass an invisible
boundary; new, universal, and more liberal laws will begin to establish
themselves around and within him; or the old laws be expanded, and
interpreted in his favor in a more liberal sense, and he will live with
the license of a higher order of beings. In proportion as he simplifies
his life, the laws of the universe will appear less complex, and
solitude will not be solitude, nor poverty poverty, nor weakness
weakness. If you have built castles in the air, your work need not be
lost; that is where they should be. Now put the foundations under them.

It is a ridiculous demand which England and America make, that you
shall speak so that they can understand you. Neither men nor
toad-stools grow so. As if that were important, and there were not
enough to understand you without them. As if Nature could support but
one order of understandings, could not sustain birds as well as
quadrupeds, flying as well as creeping things, and _hush_ and _who_,
which Bright can understand, were the best English. As if there were
safety in stupidity alone. I fear chiefly lest my expression may not be
_extra-vagant_ enough, may not wander far enough beyond the narrow
limits of my daily experience, so as to be adequate to the truth of
which I have been convinced. _Extra vagance!_ it depends on how you are
yarded. The migrating buffalo, which seeks new pastures in another
latitude, is not extravagant like the cow which kicks over the pail,
leaps the cow-yard fence, and runs after her calf, in milking time. I
desire to speak somewhere _without_ bounds; like a man in a waking
moment, to men in their waking moments; for I am convinced that I
cannot exaggerate enough even to lay the foundation of a true
expression. Who that has heard a strain of music feared then lest he
should speak extravagantly any more forever? In view of the future or
possible, we should live quite laxly and undefined in front, our
outlines dim and misty on that side; as our shadows reveal an
insensible perspiration toward the sun. The volatile truth of our words
should continually betray the inadequacy of the residual statement.
Their truth is instantly _translated_; its literal monument alone
remains. The words which express our faith and piety are not definite;
yet they are significant and fragrant like frankincense to superior
natures.

Why level downward to our dullest perception always, and praise that as
common sense? The commonest sense is the sense of men asleep, which
they express by snoring. Sometimes we are inclined to class those who
are once-and-a-half-witted with the half-witted, because we appreciate
only a third part of their wit. Some would find fault with the
morning-red, if they ever got up early enough. “They pretend,” as I
hear, “that the verses of Kabir have four different senses; illusion,
spirit, intellect, and the exoteric doctrine of the Vedas;” but in this
part of the world it is considered a ground for complaint if a man’s
writings admit of more than one interpretation. While England endeavors
to cure the potato-rot, will not any endeavor to cure the brain-rot,
which prevails so much more widely and fatally?

I do not suppose that I have attained to obscurity, but I should be
proud if no more fatal fault were found with my pages on this score
than was found with the Walden ice. Southern customers objected to its
blue color, which is the evidence of its purity, as if it were muddy,
and preferred the Cambridge ice, which is white, but tastes of weeds.
The purity men love is like the mists which envelop the earth, and not
like the azure ether beyond.

Some are dinning in our ears that we Americans, and moderns generally,
are intellectual dwarfs compared with the ancients, or even the
Elizabethan men. But what is that to the purpose? A living dog is
better than a dead lion. Shall a man go and hang himself because he
belongs to the race of pygmies, and not be the biggest pygmy that he
can? Let every one mind his own business, and endeavor to be what he
was made.

Why should we be in such desperate haste to succeed, and in such
desperate enterprises? If a man does not keep pace with his companions,
perhaps it is because he hears a different drummer. Let him step to the
music which he hears, however measured or far away. It is not important
that he should mature as soon as an apple-tree or an oak. Shall he turn
his spring into summer? If the condition of things which we were made
for is not yet, what were any reality which we can substitute? We will
not be shipwrecked on a vain reality. Shall we with pains erect a
heaven of blue glass over ourselves, though when it is done we shall be
sure to gaze still at the true ethereal heaven far above, as if the
former were not?

There was an artist in the city of Kouroo who was disposed to strive
after perfection. One day it came into his mind to make a staff. Having
considered that in an imperfect work time is an ingredient, but into a
perfect work time does not enter, he said to himself, It shall be
perfect in all respects, though I should do nothing else in my life. He
proceeded instantly to the forest for wood, being resolved that it
should not be made of unsuitable material; and as he searched for and
rejected stick after stick, his friends gradually deserted him, for
they grew old in their works and died, but he grew not older by a
moment. His singleness of purpose and resolution, and his elevated
piety, endowed him, without his knowledge, with perennial youth. As he
made no compromise with Time, Time kept out of his way, and only sighed
at a distance because he could not overcome him. Before he had found a
stock in all respects suitable the city of Kouroo was a hoary ruin, and
he sat on one of its mounds to peel the stick. Before he had given it
the proper shape the dynasty of the Candahars was at an end, and with
the point of the stick he wrote the name of the last of that race in
the sand, and then resumed his work. By the time he had smoothed and
polished the staff Kalpa was no longer the pole-star; and ere he had
put on the ferrule and the head adorned with precious stones, Brahma
had awoke and slumbered many times. But why do I stay to mention these
things? When the finishing stroke was put to his work, it suddenly
expanded before the eyes of the astonished artist into the fairest of
all the creations of Brahma. He had made a new system in making a
staff, a world with full and fair proportions; in which, though the old
cities and dynasties had passed away, fairer and more glorious ones had
taken their places. And now he saw by the heap of shavings still fresh
at his feet, that, for him and his work, the former lapse of time had
been an illusion, and that no more time had elapsed than is required
for a single scintillation from the brain of Brahma to fall on and
inflame the tinder of a mortal brain. The material was pure, and his
art was pure; how could the result be other than wonderful?

No face which we can give to a matter will stead us so well at last as
the truth. This alone wears well. For the most part, we are not where
we are, but in a false position. Through an infinity of our natures, we
suppose a case, and put ourselves into it, and hence are in two cases
at the same time, and it is doubly difficult to get out. In sane
moments we regard only the facts, the case that is. Say what you have
to say, not what you ought. Any truth is better than make-believe. Tom
Hyde, the tinker, standing on the gallows, was asked if he had anything
to say. “Tell the tailors,” said he, “to remember to make a knot in
their thread before they take the first stitch.” His companion’s prayer
is forgotten.

However mean your life is, meet it and live it; do not shun it and call
it hard names. It is not so bad as you are. It looks poorest when you
are richest. The fault-finder will find faults even in paradise. Love
your life, poor as it is. You may perhaps have some pleasant,
thrilling, glorious hours, even in a poor-house. The setting sun is
reflected from the windows of the alms-house as brightly as from the
rich man’s abode; the snow melts before its door as early in the
spring. I do not see but a quiet mind may live as contentedly there,
and have as cheering thoughts, as in a palace. The town’s poor seem to
me often to live the most independent lives of any. May be they are
simply great enough to receive without misgiving. Most think that they
are above being supported by the town; but it oftener happens that they
are not above supporting themselves by dishonest means, which should be
more disreputable. Cultivate poverty like a garden herb, like sage. Do
not trouble yourself much to get new things, whether clothes or
friends. Turn the old; return to them. Things do not change; we change.
Sell your clothes and keep your thoughts. God will see that you do not
want society. If I were confined to a corner of a garret all my days,
like a spider, the world would be just as large to me while I had my
thoughts about me. The philosopher said: “From an army of three
divisions one can take away its general, and put it in disorder; from
the man the most abject and vulgar one cannot take away his thought.”
Do not seek so anxiously to be developed, to subject yourself to many
influences to be played on; it is all dissipation. Humility like
darkness reveals the heavenly lights. The shadows of poverty and
meanness gather around us, “and lo! creation widens to our view.” We
are often reminded that if there were bestowed on us the wealth of
Crœsus, our aims must still be the same, and our means essentially the
same. Moreover, if you are restricted in your range by poverty, if you
cannot buy books and newspapers, for instance, you are but confined to
the most significant and vital experiences; you are compelled to deal
with the material which yields the most sugar and the most starch. It
is life near the bone where it is sweetest. You are defended from being
a trifler. No man loses ever on a lower level by magnanimity on a
higher. Superfluous wealth can buy superfluities only. Money is not
required to buy one necessary of the soul.

I live in the angle of a leaden wall, into whose composition was poured
a little alloy of bell metal. Often, in the repose of my mid-day, there
reaches my ears a confused _tintinnabulum_ from without. It is the
noise of my contemporaries. My neighbors tell me of their adventures
with famous gentlemen and ladies, what notabilities they met at the
dinner-table; but I am no more interested in such things than in the
contents of the Daily Times. The interest and the conversation are
about costume and manners chiefly; but a goose is a goose still, dress
it as you will. They tell me of California and Texas, of England and
the Indies, of the Hon. Mr. —— of Georgia or of Massachusetts, all
transient and fleeting phenomena, till I am ready to leap from their
court-yard like the Mameluke bey. I delight to come to my bearings,—not
walk in procession with pomp and parade, in a conspicuous place, but to
walk even with the Builder of the universe, if I may,—not to live in
this restless, nervous, bustling, trivial Nineteenth Century, but stand
or sit thoughtfully while it goes by. What are men celebrating? They
are all on a committee of arrangements, and hourly expect a speech from
somebody. God is only the president of the day, and Webster is his
orator. I love to weigh, to settle, to gravitate toward that which most
strongly and rightfully attracts me;—not hang by the beam of the scale
and try to weigh less,—not suppose a case, but take the case that is;
to travel the only path I can, and that on which no power can resist
me. It affords me no satisfaction to commence to spring an arch before
I have got a solid foundation. Let us not play at kittly-benders. There
is a solid bottom every where. We read that the traveller asked the boy
if the swamp before him had a hard bottom. The boy replied that it had.
But presently the traveller’s horse sank in up to the girths, and he
observed to the boy, “I thought you said that this bog had a hard
bottom.” “So it has,” answered the latter, “but you have not got half
way to it yet.” So it is with the bogs and quicksands of society; but
he is an old boy that knows it. Only what is thought, said, or done at a
certain rare coincidence is good. I would not be one of those who will
foolishly drive a nail into mere lath and plastering; such a deed would
keep me awake nights. Give me a hammer, and let me feel for the
furring. Do not depend on the putty. Drive a nail home and clinch it so
faithfully that you can wake up in the night and think of your work
with satisfaction,—a work at which you would not be ashamed to invoke
the Muse. So will help you God, and so only. Every nail driven should
be as another rivet in the machine of the universe, you carrying on the
work.

Rather than love, than money, than fame, give me truth. I sat at a
table where were rich food and wine in abundance, and obsequious
attendance, but sincerity and truth were not; and I went away hungry
from the inhospitable board. The hospitality was as cold as the ices. I
thought that there was no need of ice to freeze them. They talked to me
of the age of the wine and the fame of the vintage; but I thought of an
older, a newer, and purer wine, of a more glorious vintage, which they
had not got, and could not buy. The style, the house and grounds and
“entertainment” pass for nothing with me. I called on the king, but he
made me wait in his hall, and conducted like a man incapacitated for
hospitality. There was a man in my neighborhood who lived in a hollow
tree. His manners were truly regal. I should have done better had I
called on him.

How long shall we sit in our porticoes practising idle and musty
virtues, which any work would make impertinent? As if one were to begin
the day with long-suffering, and hire a man to hoe his potatoes; and in
the afternoon go forth to practise Christian meekness and charity with
goodness aforethought! Consider the China pride and stagnant
self-complacency of mankind. This generation inclines a little to
congratulate itself on being the last of an illustrious line; and in
Boston and London and Paris and Rome, thinking of its long descent, it
speaks of its progress in art and science and literature with
satisfaction. There are the Records of the Philosophical Societies, and
the public Eulogies of _Great Men!_ It is the good Adam contemplating
his own virtue. “Yes, we have done great deeds, and sung divine songs,
which shall never die,”—that is, as long as _we_ can remember them. The
learned societies and great men of Assyria,—where are they? What
youthful philosophers and experimentalists we are! There is not one of
my readers who has yet lived a whole human life. These may be but the
spring months in the life of the race. If we have had the seven-years’
itch, we have not seen the seventeen-year locust yet in Concord. We are
acquainted with a mere pellicle of the globe on which we live. Most
have not delved six feet beneath the surface, nor leaped as many above
it. We know not where we are. Beside, we are sound asleep nearly half
our time. Yet we esteem ourselves wise, and have an established order
on the surface. Truly, we are deep thinkers, we are ambitious spirits!
As I stand over the insect crawling amid the pine needles on the forest
floor, and endeavoring to conceal itself from my sight, and ask myself
why it will cherish those humble thoughts, and hide its head from me
who might, perhaps, be its benefactor, and impart to its race some
cheering information, I am reminded of the greater Benefactor and
Intelligence that stands over me the human insect.

There is an incessant influx of novelty into the world, and yet we
tolerate incredible dulness. I need only suggest what kind of sermons
are still listened to in the most enlightened countries. There are such
words as joy and sorrow, but they are only the burden of a psalm, sung
with a nasal twang, while we believe in the ordinary and mean. We think
that we can change our clothes only. It is said that the British Empire
is very large and respectable, and that the United States are a
first-rate power. We do not believe that a tide rises and falls behind
every man which can float the British Empire like a chip, if he should
ever harbor it in his mind. Who knows what sort of seventeen-year
locust will next come out of the ground? The government of the world I
live in was not framed, like that of Britain, in after-dinner
conversations over the wine.

The life in us is like the water in the river. It may rise this year
higher than man has ever known it, and flood the parched uplands; even
this may be the eventful year, which will drown out all our muskrats.
It was not always dry land where we dwell. I see far inland the banks
which the stream anciently washed, before science began to record its
freshets. Every one has heard the story which has gone the rounds of
New England, of a strong and beautiful bug which came out of the dry
leaf of an old table of apple-tree wood, which had stood in a farmer’s
kitchen for sixty years, first in Connecticut, and afterward in
Massachusetts,—from an egg deposited in the living tree many years
earlier still, as appeared by counting the annual layers beyond it;
which was heard gnawing out for several weeks, hatched perchance by the
heat of an urn. Who does not feel his faith in a resurrection and
immortality strengthened by hearing of this? Who knows what beautiful
and winged life, whose egg has been buried for ages under many
concentric layers of woodenness in the dead dry life of society,
deposited at first in the alburnum of the green and living tree, which
has been gradually converted into the semblance of its well-seasoned
tomb,—heard perchance gnawing out now for years by the astonished
family of man, as they sat round the festive board,—may unexpectedly
come forth from amidst society’s most trivial and handselled furniture,
to enjoy its perfect summer life at last!

I do not say that John or Jonathan will realize all this; but such is
the character of that morrow which mere lapse of time can never make to
dawn. The light which puts out our eyes is darkness to us. Only that
day dawns to which we are awake. There is more day to dawn. The sun is
but a morning star.

THE END



ON THE DUTY OF CIVIL DISOBEDIENCE

I heartily accept the motto,—“That government is best which governs
least;” and I should like to see it acted up to more rapidly and
systematically. Carried out, it finally amounts to this, which also I
believe—“That government is best which governs not at all;” and when
men are prepared for it, that will be the kind of government which they
will have. Government is at best but an expedient; but most governments
are usually, and all governments are sometimes, inexpedient. The
objections which have been brought against a standing army, and they
are many and weighty, and deserve to prevail, may also at last be
brought against a standing government. The standing army is only an arm
of the standing government. The government itself, which is only the
mode which the people have chosen to execute their will, is equally
liable to be abused and perverted before the people can act through it.
Witness the present Mexican war, the work of comparatively a few
individuals using the standing government as their tool; for, in the
outset, the people would not have consented to this measure.

This American government,—what is it but a tradition, though a recent
one, endeavoring to transmit itself unimpaired to posterity, but each
instant losing some of its integrity? It has not the vitality and force
of a single living man; for a single man can bend it to his will. It is
a sort of wooden gun to the people themselves; and, if ever they should
use it in earnest as a real one against each other, it will surely
split. But it is not the less necessary for this; for the people must
have some complicated machinery or other, and hear its din, to satisfy
that idea of government which they have. Governments show thus how
successfully men can be imposed on, even impose on themselves, for
their own advantage. It is excellent, we must all allow; yet this
government never of itself furthered any enterprise, but by the
alacrity with which it got out of its way. _It_ does not keep the
country free. _It_ does not settle the West. _It_ does not educate. The
character inherent in the American people has done all that has been
accomplished; and it would have done somewhat more, if the government
had not sometimes got in its way. For government is an expedient, by
which men would fain succeed in letting one another alone; and, as has
been said, when it is most expedient, the governed are most let alone
by it. Trade and commerce, if they were not made of India rubber, would
never manage to bounce over obstacles which legislators are continually
putting in their way; and, if one were to judge these men wholly by the
effects of their actions, and not partly by their intentions, they
would deserve to be classed and punished with those mischievous persons
who put obstructions on the railroads.

But, to speak practically and as a citizen, unlike those who call
themselves no-government men, I ask for, not at once no government, but
_at once_ a better government. Let every man make known what kind of
government would command his respect, and that will be one step toward
obtaining it.

After all, the practical reason why, when the power is once in the
hands of the people, a majority are permitted, and for a long period
continue, to rule, is not because they are most likely to be in the
right, nor because this seems fairest to the minority, but because they
are physically the strongest. But a government in which the majority
rule in all cases can not be based on justice, even as far as men
understand it. Can there not be a government in which the majorities do
not virtually decide right and wrong, but conscience?—in which
majorities decide only those questions to which the rule of expediency
is applicable? Must the citizen ever for a moment, or in the least
degree, resign his conscience to the legislator? Why has every man a
conscience, then? I think that we should be men first, and subjects
afterward. It is not desirable to cultivate a respect for the law, so
much as for the right. The only obligation which I have a right to
assume, is to do at any time what I think right. It is truly enough
said that a corporation has no conscience; but a corporation of
conscientious men is a corporation _with_ a conscience. Law never made
men a whit more just; and, by means of their respect for it, even the
well-disposed are daily made the agents of injustice. A common and
natural result of an undue respect for the law is, that you may see a
file of soldiers, colonel, captain, corporal, privates, powder-monkeys
and all, marching in admirable order over hill and dale to the wars,
against their wills, aye, against their common sense and consciences,
which makes it very steep marching indeed, and produces a palpitation
of the heart. They have no doubt that it is a damnable business in
which they are concerned; they are all peaceably inclined. Now, what
are they? Men at all? or small movable forts and magazines, at the
service of some unscrupulous man in power? Visit the Navy Yard, and
behold a marine, such a man as an American government can make, or such
as it can make a man with its black arts, a mere shadow and
reminiscence of humanity, a man laid out alive and standing, and
already, as one may say, buried under arms with funeral accompaniment,
though it may be

     “Not a drum was heard, not a funeral note,
         As his corpse to the ramparts we hurried;
     Not a soldier discharged his farewell shot
         O’er the grave where our hero we buried.”

The mass of men serve the State thus, not as men mainly, but as
machines, with their bodies. They are the standing army, and the
militia, jailers, constables, _posse comitatus_, &c. In most cases
there is no free exercise whatever of the judgment or of the moral
sense; but they put themselves on a level with wood and earth and
stones; and wooden men can perhaps be manufactured that will serve the
purpose as well. Such command no more respect than men of straw, or a
lump of dirt. They have the same sort of worth only as horses and dogs.
Yet such as these even are commonly esteemed good citizens. Others, as
most legislators, politicians, lawyers, ministers, and office-holders,
serve the state chiefly with their heads; and, as they rarely make any
moral distinctions, they are as likely to serve the devil, without
_intending_ it, as God. A very few, as heroes, patriots, martyrs,
reformers in the great sense, and _men_, serve the State with their
consciences also, and so necessarily resist it for the most part; and
they are commonly treated by it as enemies. A wise man will only be
useful as a man, and will not submit to be “clay,” and “stop a hole to
keep the wind away,” but leave that office to his dust at least:

     “I am too high-born to be propertied,
     To be a secondary at control,
     Or useful serving-man and instrument
     To any sovereign state throughout the world.”

He who gives himself entirely to his fellow-men appears to them useless
and selfish; but he who gives himself partially to them is pronounced a
benefactor and philanthropist.

How does it become a man to behave toward the American government
today? I answer that he cannot without disgrace be associated with it.
I cannot for an instant recognize that political organization as _my_
government which is the _slave’s_ government also.

All men recognize the right of revolution; that is, the right to refuse
allegiance to and to resist the government, when its tyranny or its
inefficiency are great and unendurable. But almost all say that such is
not the case now. But such was the case, they think, in the Revolution
of ’75. If one were to tell me that this was a bad government because
it taxed certain foreign commodities brought to its ports, it is most
probable that I should not make an ado about it, for I can do without
them: all machines have their friction; and possibly this does enough
good to counter-balance the evil. At any rate, it is a great evil to
make a stir about it. But when the friction comes to have its machine,
and oppression and robbery are organized, I say, let us not have such a
machine any longer. In other words, when a sixth of the population of a
nation which has undertaken to be the refuge of liberty are slaves, and
a whole country is unjustly overrun and conquered by a foreign army,
and subjected to military law, I think that it is not too soon for
honest men to rebel and revolutionize. What makes this duty the more
urgent is that fact, that the country so overrun is not our own, but
ours is the invading army.

Paley, a common authority with many on moral questions, in his chapter
on the “Duty of Submission to Civil Government,” resolves all civil
obligation into expediency; and he proceeds to say, “that so long as
the interest of the whole society requires it, that is, so long as the
established government cannot be resisted or changed without public
inconveniency, it is the will of God that the established government be
obeyed, and no longer.”—“This principle being admitted, the justice of
every particular case of resistance is reduced to a computation of the
quantity of the danger and grievance on the one side, and of the
probability and expense of redressing it on the other.” Of this, he
says, every man shall judge for himself. But Paley appears never to
have contemplated those cases to which the rule of expediency does not
apply, in which a people, as well as an individual, must do justice,
cost what it may. If I have unjustly wrested a plank from a drowning
man, I must restore it to him though I drown myself. This, according to
Paley, would be inconvenient. But he that would save his life, in such
a case, shall lose it. This people must cease to hold slaves, and to
make war on Mexico, though it cost them their existence as a people.

In their practice, nations agree with Paley; but does anyone think that
Massachusetts does exactly what is right at the present crisis?

     “A drab of state, a cloth-o’-silver slut,
     To have her train borne up, and her soul trail in the dirt.”

Practically speaking, the opponents to a reform in Massachusetts are
not a hundred thousand politicians at the South, but a hundred thousand
merchants and farmers here, who are more interested in commerce and
agriculture than they are in humanity, and are not prepared to do
justice to the slave and to Mexico, _cost what it may_. I quarrel not
with far-off foes, but with those who, near at home, co-operate with,
and do the bidding of those far away, and without whom the latter would
be harmless. We are accustomed to say, that the mass of men are
unprepared; but improvement is slow, because the few are not materially
wiser or better than the many. It is not so important that many should
be as good as you, as that there be some absolute goodness somewhere;
for that will leaven the whole lump. There are thousands who are _in
opinion_ opposed to slavery and to the war, who yet in effect do
nothing to put an end to them; who, esteeming themselves children of
Washington and Franklin, sit down with their hands in their pockets,
and say that they know not what to do, and do nothing; who even
postpone the question of freedom to the question of free-trade, and
quietly read the prices-current along with the latest advices from
Mexico, after dinner, and, it may be, fall asleep over them both. What
is the price-current of an honest man and patriot today? They hesitate,
and they regret, and sometimes they petition; but they do nothing in
earnest and with effect. They will wait, well disposed, for others to
remedy the evil, that they may no longer have it to regret. At most,
they give only a cheap vote, and a feeble countenance and Godspeed, to
the right, as it goes by them. There are nine hundred and ninety-nine
patrons of virtue to one virtuous man; but it is easier to deal with
the real possessor of a thing than with the temporary guardian of it.

All voting is a sort of gaming, like chequers or backgammon, with a
slight moral tinge to it, a playing with right and wrong, with moral
questions; and betting naturally accompanies it. The character of the
voters is not staked. I cast my vote, perchance, as I think right; but
I am not vitally concerned that that right should prevail. I am willing
to leave it to the majority. Its obligation, therefore, never exceeds
that of expediency. Even voting _for the right_ is _doing_ nothing for
it. It is only expressing to men feebly your desire that it should
prevail. A wise man will not leave the right to the mercy of chance,
nor wish it to prevail through the power of the majority. There is but
little virtue in the action of masses of men. When the majority shall
at length vote for the abolition of slavery, it will be because they
are indifferent to slavery, or because there is but little slavery left
to be abolished by their vote. _They_ will then be the only slaves.
Only _his_ vote can hasten the abolition of slavery who asserts his own
freedom by his vote.

I hear of a convention to be held at Baltimore, or elsewhere, for the
selection of a candidate for the Presidency, made up chiefly of
editors, and men who are politicians by profession; but I think, what
is it to any independent, intelligent, and respectable man what
decision they may come to, shall we not have the advantage of his
wisdom and honesty, nevertheless? Can we not count upon some
independent votes? Are there not many individuals in the country who do
not attend conventions? But no: I find that the respectable man, so
called, has immediately drifted from his position, and despairs of his
country, when his country has more reasons to despair of him. He
forthwith adopts one of the candidates thus selected as the only
_available_ one, thus proving that he is himself _available_ for any
purposes of the demagogue. His vote is of no more worth than that of
any unprincipled foreigner or hireling native, who may have been
bought. Oh for a man who is a _man_, and, as my neighbor says, has a
bone in his back which you cannot pass your hand through! Our
statistics are at fault: the population has been returned too large.
How many _men_ are there to a square thousand miles in the country?
Hardly one. Does not America offer any inducement for men to settle
here? The American has dwindled into an Odd Fellow,—one who may be
known by the development of his organ of gregariousness, and a manifest
lack of intellect and cheerful self-reliance; whose first and chief
concern, on coming into the world, is to see that the alms-houses are
in good repair; and, before yet he has lawfully donned the virile garb,
to collect a fund for the support of the widows and orphans that may
be; who, in short, ventures to live only by the aid of the Mutual
Insurance company, which has promised to bury him decently.

It is not a man’s duty, as a matter of course, to devote himself to the
eradication of any, even the most enormous wrong; he may still properly
have other concerns to engage him; but it is his duty, at least, to
wash his hands of it, and, if he gives it no thought longer, not to
give it practically his support. If I devote myself to other pursuits
and contemplations, I must first see, at least, that I do not pursue
them sitting upon another man’s shoulders. I must get off him first,
that he may pursue his contemplations too. See what gross inconsistency
is tolerated. I have heard some of my townsmen say, “I should like to
have them order me out to help put down an insurrection of the slaves,
or to march to Mexico,—see if I would go;” and yet these very men have
each, directly by their allegiance, and so indirectly, at least, by
their money, furnished a substitute. The soldier is applauded who
refuses to serve in an unjust war by those who do not refuse to sustain
the unjust government which makes the war; is applauded by those whose
own act and authority he disregards and sets at naught; as if the State
were penitent to that degree that it hired one to scourge it while it
sinned, but not to that degree that it left off sinning for a moment.
Thus, under the name of Order and Civil Government, we are all made at
last to pay homage to and support our own meanness. After the first
blush of sin, comes its indifference; and from immoral it becomes, as
it were, _un_moral, and not quite unnecessary to that life which we
have made.

The broadest and most prevalent error requires the most disinterested
virtue to sustain it. The slight reproach to which the virtue of
patriotism is commonly liable, the noble are most likely to incur.
Those who, while they disapprove of the character and measures of a
government, yield to it their allegiance and support, are undoubtedly
its most conscientious supporters, and so frequently the most serious
obstacles to reform. Some are petitioning the State to dissolve the
Union, to disregard the requisitions of the President. Why do they not
dissolve it themselves,—the union between themselves and the State,—and
refuse to pay their quota into its treasury? Do not they stand in same
relation to the State, that the State does to the Union? And have not
the same reasons prevented the State from resisting the Union, which
have prevented them from resisting the State?

How can a man be satisfied to entertain an opinion merely, and enjoy
_it?_ Is there any enjoyment in it, if his opinion is that he is
aggrieved? If you are cheated out of a single dollar by your neighbor,
you do not rest satisfied with knowing you are cheated, or with saying
that you are cheated, or even with petitioning him to pay you your due;
but you take effectual steps at once to obtain the full amount, and see
that you are never cheated again. Action from principle,—the perception
and the performance of right,—changes things and relations; it is
essentially revolutionary, and does not consist wholly with anything
which was. It not only divided states and churches, it divides
families; aye, it divides the _individual_, separating the diabolical
in him from the divine.

Unjust laws exist: shall we be content to obey them, or shall we
endeavor to amend them, and obey them until we have succeeded, or shall
we transgress them at once? Men generally, under such a government as
this, think that they ought to wait until they have persuaded the
majority to alter them. They think that, if they should resist, the
remedy would be worse than the evil. But it is the fault of the
government itself that the remedy _is_ worse than the evil. _It_ makes
it worse. Why is it not more apt to anticipate and provide for reform?
Why does it not cherish its wise minority? Why does it cry and resist
before it is hurt? Why does it not encourage its citizens to be on the
alert to point out its faults, and _do_ better than it would have them?
Why does it always crucify Christ, and excommunicate Copernicus and
Luther, and pronounce Washington and Franklin rebels?

One would think, that a deliberate and practical denial of its
authority was the only offence never contemplated by government; else,
why has it not assigned its definite, its suitable and proportionate
penalty? If a man who has no property refuses but once to earn nine
shillings for the State, he is put in prison for a period unlimited by
any law that I know, and determined only by the discretion of those who
placed him there; but if he should steal ninety times nine shillings
from the State, he is soon permitted to go at large again.

If the injustice is part of the necessary friction of the machine of
government, let it go, let it go: perchance it will wear
smooth,—certainly the machine will wear out. If the injustice has a
spring, or a pulley, or a rope, or a crank, exclusively for itself,
then perhaps you may consider whether the remedy will not be worse than
the evil; but if it is of such a nature that it requires you to be the
agent of injustice to another, then, I say, break the law. Let your
life be a counter friction to stop the machine. What I have to do is to
see, at any rate, that I do not lend myself to the wrong which I
condemn.

As for adopting the ways which the State has provided for remedying the
evil, I know not of such ways. They take too much time, and a man’s
life will be gone. I have other affairs to attend to. I came into this
world, not chiefly to make this a good place to live in, but to live in
it, be it good or bad. A man has not every thing to do, but something;
and because he cannot do _every thing_, it is not necessary that he
should do _something_ wrong. It is not my business to be petitioning
the Governor or the Legislature any more than it is theirs to petition
me; and, if they should not hear my petition, what should I do then?
But in this case the State has provided no way: its very Constitution
is the evil. This may seem to be harsh and stubborn and
unconcilliatory; but it is to treat with the utmost kindness and
consideration the only spirit that can appreciate or deserves it. So is
all change for the better, like birth and death which convulse the
body.

I do not hesitate to say, that those who call themselves abolitionists
should at once effectually withdraw their support, both in person and
property, from the government of Massachusetts, and not wait till they
constitute a majority of one, before they suffer the right to prevail
through them. I think that it is enough if they have God on their side,
without waiting for that other one. Moreover, any man more right than
his neighbors constitutes a majority of one already.

I meet this American government, or its representative, the State
government, directly, and face to face, once a year, no more, in the
person of its tax-gatherer; this is the only mode in which a man
situated as I am necessarily meets it; and it then says distinctly,
Recognize me; and the simplest, the most effectual, and, in the present
posture of affairs, the indispensablest mode of treating with it on
this head, of expressing your little satisfaction with and love for it,
is to deny it then. My civil neighbor, the tax-gatherer, is the very
man I have to deal with,—for it is, after all, with men and not with
parchment that I quarrel,—and he has voluntarily chosen to be an agent
of the government. How shall he ever know well what he is and does as
an officer of the government, or as a man, until he is obliged to
consider whether he shall treat me, his neighbor, for whom he has
respect, as a neighbor and well-disposed man, or as a maniac and
disturber of the peace, and see if he can get over this obstruction to
his neighborliness without a ruder and more impetuous thought or speech
corresponding with his action? I know this well, that if one thousand,
if one hundred, if ten men whom I could name,—if ten _honest_ men
only,—aye, if _one_ HONEST man, in this State of Massachusetts,
_ceasing to hold slaves_, were actually to withdraw from this
copartnership, and be locked up in the county jail therefor, it would
be the abolition of slavery in America. For it matters not how small
the beginning may seem to be: what is once well done is done for ever.
But we love better to talk about it: that we say is our mission. Reform
keeps many scores of newspapers in its service, but not one man. If my
esteemed neighbor, the State’s ambassador, who will devote his days to
the settlement of the question of human rights in the Council Chamber,
instead of being threatened with the prisons of Carolina, were to sit
down the prisoner of Massachusetts, that State which is so anxious to
foist the sin of slavery upon her sister,—though at present she can
discover only an act of inhospitality to be the ground of a quarrel
with her,—the Legislature would not wholly waive the subject of the
following winter.

Under a government which imprisons any unjustly, the true place for a
just man is also a prison. The proper place today, the only place which
Massachusetts has provided for her freer and less desponding spirits,
is in her prisons, to be put out and locked out of the State by her own
act, as they have already put themselves out by their principles. It is
there that the fugitive slave, and the Mexican prisoner on parole, and
the Indian come to plead the wrongs of his race, should find them; on
that separate, but more free and honorable ground, where the State
places those who are not _with_ her but _against_ her,—the only house
in a slave-state in which a free man can abide with honor. If any think
that their influence would be lost there, and their voices no longer
afflict the ear of the State, that they would not be as an enemy within
its walls, they do not know by how much truth is stronger than error,
nor how much more eloquently and effectively he can combat injustice
who has experienced a little in his own person. Cast your whole vote,
not a strip of paper merely, but your whole influence. A minority is
powerless while it conforms to the majority; it is not even a minority
then; but it is irresistible when it clogs by its whole weight. If the
alternative is to keep all just men in prison, or give up war and
slavery, the State will not hesitate which to choose. If a thousand men
were not to pay their tax-bills this year, that would not be a violent
and bloody measure, as it would be to pay them, and enable the State to
commit violence and shed innocent blood. This is, in fact, the
definition of a peaceable revolution, if any such is possible. If the
tax-gatherer, or any other public officer, asks me, as one has done,
“But what shall I do?” my answer is, “If you really wish to do any
thing, resign your office.” When the subject has refused allegiance,
and the officer has resigned his office, then the revolution is
accomplished. But even suppose blood should flow. Is there not a sort
of blood shed when the conscience is wounded? Through this wound a
man’s real manhood and immortality flow out, and he bleeds to an
everlasting death. I see this blood flowing now.

I have contemplated the imprisonment of the offender, rather than the
seizure of his goods,—though both will serve the same purpose,—because
they who assert the purest right, and consequently are most dangerous
to a corrupt State, commonly have not spent much time in accumulating
property. To such the State renders comparatively small service, and a
slight tax is wont to appear exorbitant, particularly if they are
obliged to earn it by special labor with their hands. If there were one
who lived wholly without the use of money, the State itself would
hesitate to demand it of him. But the rich man—not to make any
invidious comparison—is always sold to the institution which makes him
rich. Absolutely speaking, the more money, the less virtue; for money
comes between a man and his objects, and obtains them for him; it was
certainly no great virtue to obtain it. It puts to rest many questions
which he would otherwise be taxed to answer; while the only new
question which it puts is the hard but superfluous one, how to spend
it. Thus his moral ground is taken from under his feet. The
opportunities of living are diminished in proportion as what are called
the “means” are increased. The best thing a man can do for his culture
when he is rich is to endeavor to carry out those schemes which he
entertained when he was poor. Christ answered the Herodians according
to their condition. “Show me the tribute-money,” said he;—and one took
a penny out of his pocket;—if you use money which has the image of
Cæsar on it, and which he has made current and valuable, that is, _if
you are men of the State_, and gladly enjoy the advantages of Cæsar’s
government, then pay him back some of his own when he demands it;
“Render therefore to Cæsar that which is Cæsar’s and to God those
things which are God’s,”—leaving them no wiser than before as to which
was which; for they did not wish to know.

When I converse with the freest of my neighbors, I perceive that,
whatever they may say about the magnitude and seriousness of the
question, and their regard for the public tranquillity, the long and
the short of the matter is, that they cannot spare the protection of
the existing government, and they dread the consequences of
disobedience to it to their property and families. For my own part, I
should not like to think that I ever rely on the protection of the
State. But, if I deny the authority of the State when it presents its
tax-bill, it will soon take and waste all my property, and so harass me
and my children without end. This is hard. This makes it impossible for
a man to live honestly and at the same time comfortably in outward
respects. It will not be worth the while to accumulate property; that
would be sure to go again. You must hire or squat somewhere, and raise
but a small crop, and eat that soon. You must live within yourself, and
depend upon yourself, always tucked up and ready for a start, and not
have many affairs. A man may grow rich in Turkey even, if he will be in
all respects a good subject of the Turkish government. Confucius
said,—“If a State is governed by the principles of reason, poverty and
misery are subjects of shame; if a State is not governed by the
principles of reason, riches and honors are the subjects of shame.” No:
until I want the protection of Massachusetts to be extended to me in
some distant southern port, where my liberty is endangered, or until I
am bent solely on building up an estate at home by peaceful enterprise,
I can afford to refuse allegiance to Massachusetts, and her right to my
property and life. It costs me less in every sense to incur the penalty
of disobedience to the State, than it would to obey. I should feel as
if I were worth less in that case.

Some years ago, the State met me in behalf of the church, and commanded
me to pay a certain sum toward the support of a clergyman whose
preaching my father attended, but never I myself. “Pay it,” it said,
“or be locked up in the jail.” I declined to pay. But, unfortunately,
another man saw fit to pay it. I did not see why the schoolmaster
should be taxed to support the priest, and not the priest the
schoolmaster; for I was not the State’s schoolmaster, but I supported
myself by voluntary subscription. I did not see why the lyceum should
not present its tax-bill, and have the State to back its demand, as
well as the church. However, at the request of the selectmen, I
condescended to make some such statement as this in writing:—“Know all
men by these presents, that I, Henry Thoreau, do not wish to be
regarded as a member of any incorporated society which I have not
joined.” This I gave to the town-clerk; and he has it. The State,
having thus learned that I did not wish to be regarded as a member of
that church, has never made a like demand on me since; though it said
that it must adhere to its original presumption that time. If I had
known how to name them, I should then have signed off in detail from
all the societies which I never signed on to; but I did not know where
to find such a complete list.

I have paid no poll-tax for six years. I was put into a jail once on
this account, for one night; and, as I stood considering the walls of
solid stone, two or three feet thick, the door of wood and iron, a foot
thick, and the iron grating which strained the light, I could not help
being struck with the foolishness of that institution which treated me
as if I were mere flesh and blood and bones, to be locked up. I
wondered that it should have concluded at length that this was the best
use it could put me to, and had never thought to avail itself of my
services in some way. I saw that, if there was a wall of stone between
me and my townsmen, there was a still more difficult one to climb or
break through, before they could get to be as free as I was. I did nor
for a moment feel confined, and the walls seemed a great waste of stone
and mortar. I felt as if I alone of all my townsmen had paid my tax.
They plainly did not know how to treat me, but behaved like persons who
are underbred. In every threat and in every compliment there was a
blunder; for they thought that my chief desire was to stand the other
side of that stone wall. I could not but smile to see how industriously
they locked the door on my meditations, which followed them out again
without let or hindrance, and _they_ were really all that was
dangerous. As they could not reach me, they had resolved to punish my
body; just as boys, if they cannot come at some person against whom
they have a spite, will abuse his dog. I saw that the State was
half-witted, that it was timid as a lone woman with her silver spoons,
and that it did not know its friends from its foes, and I lost all my
remaining respect for it, and pitied it.

Thus the state never intentionally confronts a man’s sense,
intellectual or moral, but only his body, his senses. It is not armed
with superior wit or honesty, but with superior physical strength. I
was not born to be forced. I will breathe after my own fashion. Let us
see who is the strongest. What force has a multitude? They only can
force me who obey a higher law than I. They force me to become like
themselves. I do not hear of _men_ being _forced_ to live this way or
that by masses of men. What sort of life were that to live? When I meet
a government which says to me, “Your money or your life,” why should I
be in haste to give it my money? It may be in a great strait, and not
know what to do: I cannot help that. It must help itself; do as I do.
It is not worth the while to snivel about it. I am not responsible for
the successful working of the machinery of society. I am not the son of
the engineer. I perceive that, when an acorn and a chestnut fall side
by side, the one does not remain inert to make way for the other, but
both obey their own laws, and spring and grow and flourish as best they
can, till one, perchance, overshadows and destroys the other. If a
plant cannot live according to its nature, it dies; and so a man.



The night in prison was novel and interesting enough. The prisoners in
their shirt-sleeves were enjoying a chat and the evening air in the
door-way, when I entered. But the jailer said, “Come, boys, it is time
to lock up;” and so they dispersed, and I heard the sound of their
steps returning into the hollow apartments. My room-mate was introduced
to me by the jailer as “a first-rate fellow and a clever man.” When the
door was locked, he showed me where to hang my hat, and how he managed
matters there. The rooms were whitewashed once a month; and this one,
at least, was the whitest, most simply furnished, and probably the
neatest apartment in town. He naturally wanted to know where I came
from, and what brought me there; and, when I had told him, I asked him
in my turn how he came there, presuming him to be an honest man, of
course; and, as the world goes, I believe he was. “Why,” said he, “they
accuse me of burning a barn; but I never did it.” As near as I could
discover, he had probably gone to bed in a barn when drunk, and smoked
his pipe there; and so a barn was burnt. He had the reputation of being
a clever man, had been there some three months waiting for his trial to
come on, and would have to wait as much longer; but he was quite
domesticated and contented, since he got his board for nothing, and
thought that he was well treated.

He occupied one window, and I the other; and I saw, that, if one stayed
there long, his principal business would be to look out the window. I
had soon read all the tracts that were left there, and examined where
former prisoners had broken out, and where a grate had been sawed off,
and heard the history of the various occupants of that room; for I
found that even here there was a history and a gossip which never
circulated beyond the walls of the jail. Probably this is the only
house in the town where verses are composed, which are afterward
printed in a circular form, but not published. I was shown quite a long
list of verses which were composed by some young men who had been
detected in an attempt to escape, who avenged themselves by singing
them.

I pumped my fellow-prisoner as dry as I could, for fear I should never
see him again; but at length he showed me which was my bed, and left me
to blow out the lamp.

It was like travelling into a far country, such as I had never expected
to behold, to lie there for one night. It seemed to me that I never had
heard the town-clock strike before, nor the evening sounds of the
village; for we slept with the windows open, which were inside the
grating. It was to see my native village in the light of the Middle
Ages, and our Concord was turned into a Rhine stream, and visions of
knights and castles passed before me. They were the voices of old
burghers that I heard in the streets. I was an involuntary spectator
and auditor of whatever was done and said in the kitchen of the
adjacent village-inn—a wholly new and rare experience to me. It was a
closer view of my native town. I was fairly inside of it. I never had
seen its institutions before. This is one of its peculiar institutions;
for it is a shire town. I began to comprehend what its inhabitants were
about.

In the morning, our breakfasts were put through the hole in the door,
in small oblong-square tin pans, made to fit, and holding a pint of
chocolate, with brown bread, and an iron spoon. When they called for
the vessels again, I was green enough to return what bread I had left;
but my comrade seized it, and said that I should lay that up for lunch
or dinner. Soon after, he was let out to work at haying in a
neighboring field, whither he went every day, and would not be back
till noon; so he bade me good-day, saying that he doubted if he should
see me again.

When I came out of prison,—for some one interfered, and paid the tax,—I
did not perceive that great changes had taken place on the common, such
as he observed who went in a youth, and emerged a gray-headed man; and
yet a change had to my eyes come over the scene,—the town, and State,
and country,—greater than any that mere time could effect. I saw yet
more distinctly the State in which I lived. I saw to what extent the
people among whom I lived could be trusted as good neighbors and
friends; that their friendship was for summer weather only; that they
did not greatly purpose to do right; that they were a distinct race
from me by their prejudices and superstitions, as the Chinamen and
Malays are; that, in their sacrifices to humanity they ran no risks,
not even to their property; that, after all, they were not so noble but
they treated the thief as he had treated them, and hoped, by a certain
outward observance and a few prayers, and by walking in a particular
straight though useless path from time to time, to save their souls.
This may be to judge my neighbors harshly; for I believe that most of
them are not aware that they have such an institution as the jail in
their village.

It was formerly the custom in our village, when a poor debtor came out
of jail, for his acquaintances to salute him, looking through their
fingers, which were crossed to represent the grating of a jail window,
“How do ye do?” My neighbors did not thus salute me, but first looked
at me, and then at one another, as if I had returned from a long
journey. I was put into jail as I was going to the shoemaker’s to get a
shoe which was mended. When I was let out the next morning, I proceeded
to finish my errand, and, having put on my mended shoe, joined a
huckleberry party, who were impatient to put themselves under my
conduct; and in half an hour,—for the horse was soon tackled,—was in
the midst of a huckleberry field, on one of our highest hills, two
miles off; and then the State was nowhere to be seen.

This is the whole history of “My Prisons.”



I have never declined paying the highway tax, because I am as desirous
of being a good neighbor as I am of being a bad subject; and, as for
supporting schools, I am doing my part to educate my fellow-countrymen
now. It is for no particular item in the tax-bill that I refuse to pay
it. I simply wish to refuse allegiance to the State, to withdraw and
stand aloof from it effectually. I do not care to trace the course of
my dollar, if I could, till it buys a man, or a musket to shoot one
with,—the dollar is innocent,—but I am concerned to trace the effects
of my allegiance. In fact, I quietly declare war with the State, after
my fashion, though I will still make use and get what advantages of her
I can, as is usual in such cases.

If others pay the tax which is demanded of me, from a sympathy with the
State, they do but what they have already done in their own case, or
rather they abet injustice to a greater extent than the State requires.
If they pay the tax from a mistaken interest in the individual taxed,
to save his property or prevent his going to jail, it is because they
have not considered wisely how far they let their private feelings
interfere with the public good.

This, then, is my position at present. But one cannot be too much on
his guard in such a case, lest his actions be biassed by obstinacy, or
an undue regard for the opinions of men. Let him see that he does only
what belongs to himself and to the hour.

I think sometimes, Why, this people mean well; they are only ignorant;
they would do better if they knew how: why give your neighbors this
pain to treat you as they are not inclined to? But I think, again, this
is no reason why I should do as they do, or permit others to suffer
much greater pain of a different kind. Again, I sometimes say to
myself, When many millions of men, without heat, without ill-will,
without personal feeling of any kind, demand of you a few shillings
only, without the possibility, such is their constitution, of
retracting or altering their present demand, and without the
possibility, on your side, of appeal to any other millions, why expose
yourself to this overwhelming brute force? You do not resist cold and
hunger, the winds and the waves, thus obstinately; you quietly submit
to a thousand similar necessities. You do not put your head into the
fire. But just in proportion as I regard this as not wholly a brute
force, but partly a human force, and consider that I have relations to
those millions as to so many millions of men, and not of mere brute or
inanimate things, I see that appeal is possible, first and
instantaneously, from them to the Maker of them, and, secondly, from
them to themselves. But, if I put my head deliberately into the fire,
there is no appeal to fire or to the Maker of fire, and I have only
myself to blame. If I could convince myself that I have any right to be
satisfied with men as they are, and to treat them accordingly, and not
according, in some respects, to my requisitions and expectations of
what they and I ought to be, then, like a good Mussulman and fatalist,
I should endeavor to be satisfied with things as they are, and say it
is the will of God. And, above all, there is this difference between
resisting this and a purely brute or natural force, that I can resist
this with some effect; but I cannot expect, like Orpheus, to change the
nature of the rocks and trees and beasts.

I do not wish to quarrel with any man or nation. I do not wish to split
hairs, to make fine distinctions, or set myself up as better than my
neighbors. I seek rather, I may say, even an excuse for conforming to
the laws of the land. I am but too ready to conform to them. Indeed I
have reason to suspect myself on this head; and each year, as the
tax-gatherer comes round, I find myself disposed to review the acts and
position of the general and state governments, and the spirit of the
people to discover a pretext for conformity.

     “We must affect our country as our parents,
     And if at any time we alienate
     Out love of industry from doing it honor,
     We must respect effects and teach the soul
     Matter of conscience and religion,
     And not desire of rule or benefit.”

I believe that the State will soon be able to take all my work of this
sort out of my hands, and then I shall be no better patriot than my
fellow-countrymen. Seen from a lower point of view, the Constitution,
with all its faults, is very good; the law and the courts are very
respectable; even this State and this American government are, in many
respects, very admirable, and rare things, to be thankful for, such as
a great many have described them; seen from a higher still, and the
highest, who shall say what they are, or that they are worth looking at
or thinking of at all?

However, the government does not concern me much, and I shall bestow
the fewest possible thoughts on it. It is not many moments that I live
under a government, even in this world. If a man is thought-free,
fancy-free, imagination-free, that which _is not_ never for a long time
appearing _to be_ to him, unwise rulers or reformers cannot fatally
interrupt him.

I know that most men think differently from myself; but those whose
lives are by profession devoted to the study of these or kindred
subjects content me as little as any. Statesmen and legislators,
standing so completely within the institution, never distinctly and
nakedly behold it. They speak of moving society, but have no
resting-place without it. They may be men of a certain experience and
discrimination, and have no doubt invented ingenious and even useful
systems, for which we sincerely thank them; but all their wit and
usefulness lie within certain not very wide limits. They are wont to
forget that the world is not governed by policy and expediency. Webster
never goes behind government, and so cannot speak with authority about
it. His words are wisdom to those legislators who contemplate no
essential reform in the existing government; but for thinkers, and
those who legislate for all time, he never once glances at the subject.
I know of those whose serene and wise speculations on this theme would
soon reveal the limits of his mind’s range and hospitality. Yet,
compared with the cheap professions of most reformers, and the still
cheaper wisdom and eloquence of politicians in general, his are almost
the only sensible and valuable words, and we thank Heaven for him.
Comparatively, he is always strong, original, and, above all,
practical. Still his quality is not wisdom, but prudence. The lawyer’s
truth is not Truth, but consistency or a consistent expediency. Truth
is always in harmony with herself, and is not concerned chiefly to
reveal the justice that may consist with wrong-doing. He well deserves
to be called, as he has been called, the Defender of the Constitution.
There are really no blows to be given by him but defensive ones. He is
not a leader, but a follower. His leaders are the men of ’87. “I have
never made an effort,” he says, “and never propose to make an effort; I
have never countenanced an effort, and never mean to countenance an
effort, to disturb the arrangement as originally made, by which the
various States came into the Union.” Still thinking of the sanction
which the Constitution gives to slavery, he says, “Because it was part
of the original compact,—let it stand.” Notwithstanding his special
acuteness and ability, he is unable to take a fact out of its merely
political relations, and behold it as it lies absolutely to be disposed
of by the intellect,—what, for instance, it behoves a man to do here in
America today with regard to slavery, but ventures, or is driven, to
make some such desperate answer as the following, while professing to
speak absolutely, and as a private man,—from which what new and
singular code of social duties might be inferred?—“The manner,” says
he, “in which the governments of those States where slavery exists are
to regulate it, is for their own consideration, under the
responsibility to their constituents, to the general laws of propriety,
humanity, and justice, and to God. Associations formed elsewhere,
springing from a feeling of humanity, or any other cause, have nothing
whatever to do with it. They have never received any encouragement from
me and they never will.”

They who know of no purer sources of truth, who have traced up its
stream no higher, stand, and wisely stand, by the Bible and the
Constitution, and drink at it there with reverence and humanity; but
they who behold where it comes trickling into this lake or that pool,
gird up their loins once more, and continue their pilgrimage toward its
fountain-head.

No man with a genius for legislation has appeared in America. They are
rare in the history of the world. There are orators, politicians, and
eloquent men, by the thousand; but the speaker has not yet opened his
mouth to speak who is capable of settling the much-vexed questions of
the day. We love eloquence for its own sake, and not for any truth
which it may utter, or any heroism it may inspire. Our legislators have
not yet learned the comparative value of free-trade and of freedom, of
union, and of rectitude, to a nation. They have no genius or talent for
comparatively humble questions of taxation and finance, commerce and
manufactures and agriculture. If we were left solely to the wordy wit
of legislators in Congress for our guidance, uncorrected by the
seasonable experience and the effectual complaints of the people,
America would not long retain her rank among the nations. For eighteen
hundred years, though perchance I have no right to say it, the New
Testament has been written; yet where is the legislator who has wisdom
and practical talent enough to avail himself of the light which it
sheds on the science of legislation.

The authority of government, even such as I am willing to submit
to,—for I will cheerfully obey those who know and can do better than I,
and in many things even those who neither know nor can do so well,—is
still an impure one: to be strictly just, it must have the sanction and
consent of the governed. It can have no pure right over my person and
property but what I concede to it. The progress from an absolute to a
limited monarchy, from a limited monarchy to a democracy, is a progress
toward a true respect for the individual. Even the Chinese philosopher
was wise enough to regard the individual as the basis of the empire. Is
a democracy, such as we know it, the last improvement possible in
government? Is it not possible to take a step further towards
recognizing and organizing the rights of man? There will never be a
really free and enlightened State, until the State comes to recognize
the individual as a higher and independent power, from which all its
own power and authority are derived, and treats him accordingly. I
please myself with imagining a State at last which can afford to be
just to all men, and to treat the individual with respect as a
neighbor; which even would not think it inconsistent with its own
repose, if a few were to live aloof from it, not meddling with it, nor
embraced by it, who fulfilled all the duties of neighbors and
fellow-men. A State which bore this kind of fruit, and suffered it to
drop off as fast as it ripened, would prepare the way for a still more
perfect and glorious State, which also I have imagined, but not yet
anywhere seen.



