\usemodule[tikz]
\usemodule[pgfplots]

\mainlanguage[sv]

\setuppapersize[A4][A4]
\setuppagenumbering [location=bottom]

\setupbodyfont[schola,10pt]

\setuplayout[backspace=1.2in,
             width=middle,
             topspace=1.2in]

%\showframe

\setupindenting[yes,medium,next]

	\setuphead[chapter]
		[page=yes,
			number=no,
			style=\tfa,
			deeptextcommand=\WORD,
			textstyle={\kerncharacters[.1]},
%			sectionsegments=chapter,
%			align=nothyphenated,
			header=empty,
%			numbercommand={},
%			align=middle,
			before=,
			after={\blank\hrule\blank[14pt]}]
            
	\setuphead[section]
		[page=no,
			number=no,
			header=,
%			sectionsegments=section,
			%numbercommand={\DAN},
			alternative=,
			before=\blank,
			after=\blank,
			style={\sc},
			textstyle={\kerncharacters[.1]},
			inbetween=,]
            
%\showframe

\setuplayout[header=0pt,headerdistance=0pt]

\starttext
\chapter{Ett mått på bemanning}

Fysioterapeuter har ansvar för olika stora patientgrupper och jobbar olika många timmar i veckan på olika ställen. Av ren nyfikenhet ville jag se om detta kunde jämföras på ett enkelt sätt. Med parametrarna antal patienter och antal jobbade timmar per vecka kan man, möjligen naivt, ställa frågan: om en given fysioterapeut besökte alla sina patienter lika mycket under en vecka, hur många minuter skulle varje patient få? Vi kan kalla detta mått "Fysiominuter per patient per vecka": 

\startformula
\frac{(\text{Jobbade timmar per vecka} \times 60\ \text{minuter})}{\text{Antal boende}}
\stopformula

Exempelvis, jag jobbar heltid, två dagar i veckan på Lars Kristoffers väg, som har 32 patienter. Detta boende får alltså (40 timmar $\times \frac{2}{5} \text{dagar} = 16$) 16 timmar $\times$ 60 minuter, delat på 32 patienter. 


\startsetups booktabs
  \setupTABLE[frame=off, rulethickness=0.4pt, loffset=1em]
  \setupTABLE[column][first][loffset=0em]
  \setupTABLE[row][first][topframe=on, rulethickness=0.8pt]
  \setupTABLE[row][last] [bottomframe=on, rulethickness=0.8pt]
\stopsetups

\placetable[right]{Fysiominuter per patient per vecka}{
\bTABLE[setups=booktabs]
  \setupTABLE[column][2][align=flushright]
  \bTR
\bTD Boende \eTD
\bTD Minuter\eTD
  \eTR
  \bTR[topframe=on]
\bTD Lars Kristoffers väg \eTD
\bTD 32 \eTD
  \eTR
  \bTR
\bTD Vevrehemmet \eTD
\bTD 34 \eTD
  \eTR
  \bTR
\bTD Norrdala \eTD
\bTD 26 \eTD
  \eTR
  \bTR
\bTD Reimersdal \eTD
\bTD 26 \eTD
  \eTR
  \bTR
\bTD Linegården \eTD
\bTD 40 \eTD
  \eTR
  \bTR
\bTD Linelyckan \eTD
\bTD 48 \eTD
  \eTR
  \bTR
\bTD Fästan \eTD
\bTD 34 \eTD
  \eTR
  \bTR
\bTD \it Medelvärde \eTD
\bTD \it 34 \eTD
  \eTR
  

\eTABLE}

\section{Mått på bemanning}

Hur?

\section{Data}

(pgfplot stuff)


\starttikzpicture
        \startaxis[
%nodes near coords,
xmin=0.5, xmax=7.5,
xtick={1,2,3,4,5,6,7},
axis lines=left,
xticklabels={
Lars Kristoffers väg, Vevrehemmet, Norrdala, Reimersdal, Linegården, Linelyckan, Fästan},
x tick label style={rotate=45, anchor=north east},
ylabel={myylabel},
ymin=0,
ymax=50,
]

\addplot[ybar,bar width=0.6cm,fill=gray,draw=black,nodes near coords] plot coordinates{
(1,32)
(2,34)
(3,26)
(4,26)
(5,40)
(6,48)
(7,34)}
;


%% This is just to add a zero line for clarity
%\addplot [
%color=black,
%dashed,
%forget plot
%]
%table{
%0 30
%5 30
%};
%\draw [dashed] 
%        (axis cs: 0,34) -- (axis cs: 8,34)
%        node[right] {Medel};

        \stopaxis
\stoptikzpicture

\stoptext
